\section{繫辭上}
  天尊地卑,乾坤定矣。卑高以陳,貴賤位矣。動靜有常,剛柔斷矣。方以類聚,物以群分,吉凶生矣。在天成象,在地成形,變化見矣。是故,剛柔相摩,八卦相盪。鼓之以雷霆,潤之以風雨,日月運行,一寒一暑,乾道成男,坤道成女。乾知大始,坤作成物。乾以易知,坤以簡能。易則易知,簡則易從。易知則有親,易從則有功。有親則可久,有功則可大。可久則賢人之德,可大則賢人之業。易簡,而天下之理得矣;天下之理得,而成位乎其中矣。
\\\\
  聖人設卦觀象,繫辭焉而明吉凶,剛柔相推而生變化。是故,吉凶者,失得之象也。悔吝者,懮虞之象也。變化者,進退之象也。剛柔者,晝夜之象也。六爻之動,三極之道也。是故,君子所居而安者,易之序也。所樂而玩者,爻之辭也。是故,君子居則觀其象,而玩其辭;動則觀其變,而玩其占。是以自天祐之,吉无不利。
\\\\
  彖者,言乎象者也。爻者,言乎變者也。吉凶者,言乎其失得也。悔吝者,言乎其小疵也。无咎者,善補過也。是故,列貴賤者存乎位。齊小大者,存乎卦。辯吉凶者,存乎辭。懮悔吝者,存乎介。震无咎者,存乎悔。是故,卦有小大,辭有險易。辭也者,各指其所之。
\\\\
  易與天地準,故能彌綸天地之道。仰以觀於天文,俯以察於地理,是故知幽明之故。原始反終,故知死生之說。精氣為物,遊魂為變,是故知鬼神之情狀。與天地相似,故不違。知周乎萬物,而道濟天下,故不過。旁行而不流,樂天知命,故不懮。安土敦乎仁,故能愛。範圍天地之化而不過,曲成萬物而不遺,通乎晝夜之道而知,故神无方而易无體。
\\\\
  一陰一陽之謂道,繼之者善也,成之者性也。仁者見之謂之仁,知者見之謂之知。百姓日用而不知,故君子之道鮮矣。顯諸仁,藏諸用,鼓萬物而不與聖人同懮,盛德大業至矣哉。富有之謂大業,日新之謂盛德。生生之謂易,成象之謂乾,效法之為坤,極數知來之謂占,通變之謂事,陰陽不測之謂神。
\\\\
  夫易,廣矣大矣,以言乎遠,則不禦;以言乎邇,則靜而正;以言乎天地之間,則備矣。夫乾,其靜也專,其動也直,是以大生焉。夫坤,其靜也翕,其動也闢,是以廣生焉。廣大配天地,變通配四時,陰陽之義配日月,易簡之善配至德。
\\\\
  子曰:「易其至矣乎!」,夫易,聖人所以崇德而廣業也。知崇禮卑,崇效天,卑法地。天地設位,而易行乎其中矣,成性存存,道義之門。
\\\\
  聖人有以見天下之賾,而擬諸其形容,象其物宜,是故謂之象。聖人有以見天下之動,而觀其會通,以行其典禮。繫辭焉,以斷其吉凶,是故謂之爻。言天下之至賾,而不可惡也。言天下之至動,而不可亂也。擬之而後言,議之而後動,擬議以成其變化。「鳴鶴在陰,其子和之,我有好爵,吾與爾靡之。」子曰:「君子居其室,出其言,善則千里之外應之,況其邇者乎,居其室,出其言不善,則千里之外違之,況其邇者乎,言出乎身,加乎民,行發乎邇,見乎遠。言行君子之樞機,樞機之發,榮辱之主也。言行,君子之所以動天地也,可不慎乎。」「同人,先號咷而後笑。」子曰:「君子之道,或出或處,或默或語,二人同心,其利斷金。同心之言,其臭如蘭。」「初六,藉用白茅,无咎。」子曰:「苟錯諸地而可矣。藉之用茅,何咎之有?慎之至也。夫茅之為物薄,而用可重也。慎斯術也以往,其无所失矣。」「勞謙君子,有終吉。」子曰:「勞而不伐,有功而不德,厚之至也,語以其功下人者也。德言盛,禮言恭,謙也者,致恭以存其位者也。」「亢龍有悔」,子曰:「貴而无位,高而无民,賢人在下位而无輔,是以動而有悔也。」「不出戶庭,无咎。」子曰:「亂之所生也,則言語以為階。君不密,則失臣;臣不密,則失身;幾事不密,則害成。是以君子慎密而不出也。」子曰:「作易者其知盜乎?易曰:負且乘,致寇至。負也者,小人之事也。乘也者,君子之器也。小人而乘君子之器,盜思奪之矣!上慢下暴,盜思伐之矣!慢藏誨盜,冶容誨淫,易曰:「負且乘,致寇至,盜之招也。」
\\\\
  天一地二,天三地四,天五地六,天七地八,天九地十。天數五,地數五,五位相得而各有合。天數二十有五,地數三十,凡天地之數,五十有五,此所以成變化,而行鬼神也。大衍之數五十,其用四十有九。分而為二以象兩,掛一以象三,揲之以四以象四時,歸奇於扐以象閏。五歲再閏,故再扐而後掛。乾之策,二百一十有六;坤之策,百四十有四,凡三百有六十,當期之日。二篇之策,萬有一千五百二十,當萬物之數也。是故,四營而成易,十有八變而成卦。八卦而小成,引而伸之,觸類而長之,天下之能事畢矣。顯道神德行,是故可與酬酢,可與祐神矣。子曰:「知變化之道者,其知神之所為乎。」
\\\\
  易有聖人之道四焉;以言者尚其辭,以動者尚其變,以制器者尚其象,以卜筮者尚其占。以君子將有為也,將有行也,問焉而以言,其受命也如響,无有遠近幽深,遂知來物。非天下之至精,其孰能與於此。參伍以變,錯綜其數,通其變,遂成天下之文。極其數,遂定天下之象。非天下之至變,其孰能與於此。易无思也,无為也,寂然不動,感而遂通天下之故。非天下之至神,其孰能與於此。夫易,聖人之所以極深而研幾也。唯深也,故能通天下之志。唯幾也,故能成天下之務。唯神也,故不疾而速,不行而至。子曰:「易有聖人之道四焉」者,此之謂也。
\\\\
  子曰:「夫易,何為者也?夫易開物成務,冒天下之道,如斯而已者也。是故,聖人以通天下之志,以定天下之業,以斷天下之疑。」是故,蓍之德,圓而神;卦之德,方以知;六爻之義,易以貢。聖人以此洗心,退藏於密,吉凶與民同患。神以知來,知以藏往,其孰能與此哉!古之聰明叡知神武而不殺者夫?是以,明於天之道,而察於民之故,是興神物以前民用。聖人以此齊戒,以神明其德夫!是故,闔戶謂之坤;闢戶謂之乾;一闔一闢謂之變;往來不窮謂之通;見乃謂之象;形乃謂之器;制而用之,謂之法;利用出入,民咸用之,謂之神。是故,易有太極,是生兩儀,兩儀生四象,四象生八卦,八卦定吉凶,吉凶生大業。是故,法象莫大乎天地,變通莫大乎四時,縣象著明莫大乎日月,崇高莫大乎富貴;備物致用,立成器以為天下利,莫大乎聖人;探賾索隱,鉤深致遠,以定天下之吉凶,成天下之亹亹者,莫大乎蓍龜。是故,天生神物,聖人則之;天地變化,聖人效之;天垂象,見吉凶,聖人象之。河出圖,洛出書,聖人則之。易有四象,所以示也。繫辭焉,所以告也。定之以吉凶,所以斷也。
\\\\
  易曰:「自天祐之,吉无不利。」子曰:「祐者,助也。天之所助者,順也;人之所助者,信也。履信思乎順,又以尚賢也。是以自天祐之,吉无不利也。」子曰:「書不盡言,言不盡意。然則聖人之意,其不可見乎。」子曰:「聖人立象以盡意,設卦以盡情偽,繫辭以盡其言,變而通之以盡利,鼓之舞之以盡神。」乾坤其易之縕邪?乾坤成列,而易立乎其中矣。乾坤毀,則无以見易,易不可見,則乾坤或幾乎息矣。是故,形而上者謂之道,形而下者謂之器。化而裁之謂之變,推而行之謂之通,舉而錯之天下之民,謂之事業。是故,夫象,聖人有以見天下之賾,而擬諸其形容,象其物宜,是故謂之象。聖人有以見天下之動,而觀其會通,以行其典禮,繫辭焉,以斷其吉凶,是故謂之爻。極天下之賾者,存乎卦;鼓天下之動者,存乎辭;化而裁之,存乎變;推而行之,存乎通;神而明之,存乎其人;默而成之,不言而信,存乎德行。

\section{繫辭下}
  八卦成列,象在其中矣。因而重之,爻在其中矣。剛柔相推,變在其中矣。繫辭焉而命之,動在其中矣。 吉凶悔吝者,生乎動者也。剛柔者,立本者也。變通者,趣時者也。 吉凶者,貞勝者也。天地之道,貞觀者也。日月之道,貞明者也,天下之動,貞夫一者也。 夫乾,確然示人易矣。夫坤,隤然示人簡矣。爻也者,效此者也。象也者,像此者也。 爻象動乎內,吉凶見乎外,功業見乎變,聖人之情見乎辭。 天地之大德曰生,聖人之大寶曰位。何以守位曰仁,何以聚人曰財。理財正辭,禁民為非曰義。 
\\\\
  古者包犧氏之王天下也,仰則觀象於天,俯則觀法於地,觀鳥獸之文,與地之宜,近取諸身,遠取諸物,於是始作八卦,以通神明之德,以類萬物之情。 作結繩而為罔罟,以佃以漁,蓋取諸離。 包犧氏沒,神農氏作,斲木為耜,揉木為耒,耒耨之利,以教天下,蓋取諸益。 日中為市,致天下之民,聚天下之貨,交易而退,各得其所,蓋取諸噬嗑。 神農氏沒,黃帝、堯、舜氏作,通其變,使民不倦,神而化之,使民宜之。易窮則變,變則通,通則久。是以自天祐之,吉无不利,黃帝、堯、舜垂衣裳而天下治,蓋取諸乾坤。 刳木為舟,剡木為楫,舟楫之利,以濟不通,致遠以利天下,蓋取諸渙。 服牛乘馬,引重致遠,以利天下,蓋取諸隨。 重門擊柝,以待暴客,蓋取諸豫。 斷木為杵,掘地為臼,臼杵之利,萬民以濟,蓋取諸小過。 弦木為弧,剡木為矢,弧矢之利,以威天下,蓋取諸睽。 上古穴居而野處,後世聖人易之以宮室,上棟下宇,以待風雨,蓋取諸大壯。 古之葬者,厚衣之以薪,葬之中野,不封不樹,喪期无數。後世聖人易之以棺槨,蓋取諸大過。 上古結繩而治,後世聖人易之以書契,百官以治,萬民以察,蓋取諸夬。 
\\\\
  是故,易者,象也,象也者像也。彖者,材也,爻也者,效天下之動者也。是故,吉凶生,而悔吝著也。 
\\\\
  陽卦多陰,陰卦多陽,其故何也?陽卦奇,陰卦偶。其德行何也?陽一君而二民,君子之道也。陰二君而一民,小人之道也。 
\\\\
  易曰:「憧憧往來,朋從爾思。」子曰:「天下何思何慮?天下同歸而殊塗,一致而百慮,天下何思何慮?」 「日往則月來,月往則日來,日月相推而明生焉。寒往則暑來,暑往則寒來,寒暑相推而歲成焉。往者屈也,來者信也,屈信相感而利生焉。」 「尺蠖之屈,以求信也。龍蛇之蟄,以存身也。精義入神,以致用也。利用安身,以崇德也。過此以往,未之或知也。窮神知化,德之盛也。」 易曰:「困于石,據于蒺蔾,入于其宮,不見其妻,凶。」子曰:「非所困而困焉,名必辱。非所據而據焉,身必危。既辱且危,死期將至,妻其可得見耶?」 易曰:「公用射隼,于高墉之上,獲之无不利。」子曰:「隼者禽也,弓矢者器也,射之者人也。君子藏器於身,待時而動,何不利之有?動而不括,是以出而有獲,語成器而動者也。」 子曰:「小人不恥不仁,不畏不義,不見利不勸,不威不懲,小懲而大誡,此小人之福也。易曰:『履校滅趾无咎,此之謂也』。」 「善不積,不足以成名;惡不積,不足以滅身。小人以小善為无益,而弗為也,以小惡為无傷,而弗去也,故惡積而不可掩,罪大而不可解。易曰:『何校滅耳凶』。」 子曰:「危者,安其位者也;亡者,保其存者也;亂者,有其治者也。是故,君子安而不忘危,存而不忘亡,治而不忘亂;是以身安而國家可保也。易曰:『其亡其亡,繫于苞桑』。」 子曰:「德薄而位尊,知小而謀大,力小而任重,鮮不及矣,易曰:『鼎折足,覆公餗,其形渥,凶。』言不勝其任也。」 子曰:「知幾其神乎?君子上交不諂,下交不瀆,其知幾乎,幾者動之微,吉之先見者也,君子見幾而作,不俟終日。易曰:『介于石,不終日,貞吉。』介如石焉,寧用終日,斷可識矣,君子知微知彰,知柔知剛,萬夫之望。」 子曰:「顏氏之子,其殆庶幾乎?有不善未嘗不知,知之未嘗復行也。易曰:『不遠復,无祇悔,元吉。』」 天地絪縕,萬物化醇,男女構精,萬物化生,易曰:『三人行,則損一人;一人行,則得其友。』言致一也。 子曰:「君子安其身而後動,易其心而後語,定其交而後求,君子脩此三者,故全也,危以動,則民不與也,懼以語,則民不應也,无交而求,則民不與也,莫之與,則傷之者至矣。易曰:『莫益之,或擊之,立心勿恆,凶。』。」 
\\\\
  子曰:「乾坤其易之門邪?乾,陽物也;坤,陰物也;陰陽合德,而剛柔有體,以體天地之撰,以通神明之德,其稱名也雜而不越,於稽其類,其衰世之意邪?」夫易,彰往而察來,而微顯闡幽,開而當名,辨物正言,斷辭則備矣,其稱名也小,其取類也大,其旨遠,其辭文,其言曲而中,其事肆而隱,因貳以濟民行,以明失得之報。 
\\\\
  易之興也,其於中古乎,作易者,其有憂患乎。 是故,履,德之基也;謙,德之柄也;復,德之本也;恆,德之固也;損德之脩也;益,德之裕也;困,德之辨也;井,德之地也;巽,德之制也。 履,和而至;謙,尊而光;復,小而辨於物;恆,雜而不厭;損,先難而後易;益,長裕而不設;困,窮而通;井,居其所而遷,巽,稱而隱。 履以和行,謙以制禮,復以自知,恆以一德,損以遠害,益以興利,困以寡怨,井以辯義,巽以行權。 
\\\\
  易之為書也不可遠,為道也屢遷,變動不居,周流六虛,上下无常,剛柔相易,不可為典要,唯變所適,其出入以度,外內使知懼,又明於憂患與故,无有師保,如臨父母,初率其辭,而揆其方,既有典常,苟非其人,道不虛行。 
\\\\
  易之為書也,原始要終,以為質也,六爻相雜,唯其時物也,其初難知,其上易知,本末也,初辭擬之,卒成之終,若夫雜物撰德,辨是與非,則非其中爻不備。 噫,亦要存亡吉凶,則居可知矣,知者觀其彖辭,則思過半矣。 二與四同功,而異位,其善不同,二多譽,四多懼,近也,柔之為道,不利遠者,其要无咎,其用柔中也,三與五同功,而異位,三多凶,五多功,貴賤之等也,其柔危,其剛勝邪? 
\\\\
  易之為書也,廣大悉備,有天道焉,有人道焉,有地道焉。兼三材而兩之,故六六者,非它也,三材之道也,道有變動,故曰爻,爻有等,故曰物,物相雜,故曰文,文不當,故吉凶生焉。 
\\\\
  易之興也,其當殷之末世,周之盛德邪,當文王與紂之事邪,是故其辭危,危者使平,易者使傾,其道甚大,百物不廢,懼以終始,其要无咎,此之謂易之道也。 
\\\\
  夫乾,天下之至健也,德行恆易以知險,夫坤,天下之至順也,德行恆簡以知阻。 能說諸心,能研諸侯之慮,定天下之吉凶,成天下之亹亹者,是故,變化云為,吉事有祥,象事知器,占事知來。天地設位,聖人成能。人謀鬼謀,百姓與能。 八卦以象告,爻彖以情言,剛柔雜居,而吉凶可見矣。 變動以利言,吉凶以情遷。是故愛惡相攻而吉凶生,遠近相取而悔吝生,情偽相感而利害生。凡易之情,近而不相得則凶,或害之,悔且吝。 將叛者其辭慚,中心疑者其辭枝,吉人之辭寡,躁人之辭多,誣善之人其辭游,失其守者其辭屈。
