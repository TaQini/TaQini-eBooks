\section{文言·乾}

  《文言》曰:「元」者,善之長也;「亨」者,嘉之會也;「利」者,義之和也;「貞」者,事之幹也。君子體仁足以長人,嘉會足以合禮,利物足以和義,貞固足以幹事。君子行此四德者,故曰「乾、元、亨、利、貞」。
\\\\
  初九曰、「潛龍勿用」,何謂也?子曰:「龍、德而隱者也。不易乎世,不成乎名,遯世无悶,不見是而无悶。樂則行之,憂則違之,確乎其不可拔,潛龍也。」
\\\\
  九二曰:「見龍在田,利見大人」,何謂也?子曰:「龍德而正中者也。庸言之信,庸行之謹,閑邪存其誠,善世而不伐,德博而化。《易》曰:『見龍在田,利見大人』,君德也。」
\\\\
  九三曰:「君子終日乾乾、夕惕若、厲、无咎」。何謂也?子曰:「君子進德脩業,忠信,所以進德也,脩辭立其誠,所以居業也。知至至之,可與幾也,知終終之,可與存義也。是故居上位而不驕,在下位而不憂,故乾乾因其時而惕,雖危无咎矣。」
\\\\
  九四曰:「或躍在淵,无咎。」何謂也?子曰:「上下无常,非為邪也。進退无恆,非離群也。君子進德脩業,欲及時也,故无咎。」
\\\\
  九五曰:「飛龍在天,利見大人」。何謂也?子曰:「同聲相應,同氣相求。水流濕,火就燥,雲從龍,風從虎,聖人作而萬物覩。本乎天者親上,本乎地者親下,則各從其類也。」
\\\\
  上九曰:「亢龍有悔」,何謂也?子曰:「貴而无位,高而无民,賢人在下位而无輔,是以動而有悔也。」
\\\\
  「潛龍勿用」,下也;「見龍在田」,時舍也;「終日乾乾」,行事也;「或躍在淵」,自試也;「飛龍在天」,上治也;「亢龍有悔」、窮之災也。乾元「用九」,天下治也。
\\\\
  「潛龍勿用」,陽氣潛藏。「見龍在田」,天下文明。「終日乾乾」,與時偕行。「或躍在淵」,乾道乃革。「飛龍在天」,乃位乎天德。「亢龍有悔」,與時偕極。乾元「用九」,乃見天則。
\\\\
  《乾》「元」者,始而亨者也。「利貞」者,性情也。乾始能以美利利天下,不言所利,大矣哉!
\\\\
  大哉乾乎!剛健中正,純粹精也。六爻發揮,旁通情也。「時乘六龍」、以「御天」也,「雲行雨施」、天下平也。君子以成德為行,日可見之行也。「潛」之為言也,隱而未見,行而未成,是以君子「弗用」也。
\\\\
  君子學以聚之,問以辯之,寬以居之,仁以行之。《易》曰「見龍在田、利見大人」,君德也。
\\\\
  九三重剛而不中,上不在天,下不在田,故「乾乾」因其時而「惕」,雖危「无咎」矣。
\\\\
  九四重剛而不中,上不在天,下不在田,中不在人,故「或」之。「或」之者、疑之也,故「无咎」。
\\\\
  夫「大人」者、與天地合其德,與日月合其明,與四時合其序,與鬼神合其吉凶,先天而天弗違,後天而奉天時。天且弗違,而況於人乎?況於鬼神乎?
\\\\
  「亢」之為言也,知進而不知退,知存而不知亡,知得而不知喪。其唯聖人乎!知進退存亡而不失其正者,其唯聖人乎!

\section{文言·坤}

  《文言》曰:《坤》至柔而動也剛,至靜而德方,後得主而有常,含萬物而化光。坤道其順乎,承天而時行。
\\\\
  積善之家,必有餘慶;積不善之家,必有餘殃。臣弒其君,子弒其父,非一朝一夕之故,其所由來者漸矣,由辯之不早辯也。《易》曰「履霜、堅冰至」,蓋言順也。
\\\\
  「直」其正也,「方」其義也。君子敬以直內,義以方外,敬義立而德不孤。「直、方、大、不習无不利」,則不疑其所行也。
\\\\
  陰雖有美「含」之以從王事,弗敢成也。地道也,妻道也,臣道也。地道「无成」而代「有終」也。天地變化,草木蕃。天地閉,賢人隱。《易》曰「括囊、无咎无譽」,蓋言謹也。
\\\\
  君子「黃」中通理,正位居體,美在其中而暢於四支,發於事業,美之至也。
\\\\
  陰疑於陽必「戰」,為其嫌於无陽也,故稱「龍」焉。猶未離其類也,故稱「血」焉。夫「玄黃」者、天地之雜也。天玄而地黃。