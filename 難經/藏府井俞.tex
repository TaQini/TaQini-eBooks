\section{藏府井俞}

  六十九難曰:經言虛者補之,實者瀉之,不實不虛,以經取之,何謂也?
六十二難曰:藏井滎有五,府獨有六者,何謂也?
\\\\
  六十九難曰:經言虛者補之,實者瀉之,不實不虛,以經取之,何謂也?
然:府者,陽也。三焦行於諸陽,故置一俞,名曰原。府有六者,亦與三焦共一氣也。
\\\\
  六十九難曰:經言虛者補之,實者瀉之,不實不虛,以經取之,何謂也?
六十三難曰:《十變》言,五藏六府滎合,皆以井為始者,何也?
\\\\
  六十九難曰:經言虛者補之,實者瀉之,不實不虛,以經取之,何謂也?
然:井者,東方春也,萬物之始生。諸蚑行喘息,蜎飛蠕動,當生之物,莫不以春而生。故歲數始於春,日數始於甲,故以井為始也。
\\\\
  六十九難曰:經言虛者補之,實者瀉之,不實不虛,以經取之,何謂也?
六十四難曰:《十變》又言,陰井木,陽井金;陰滎火,陽滎水;陰俞土,陽俞木;陰經金,陽經火;陰合水,陽合土。陰陽皆不同,其意何也?
\\\\
  六十九難曰:經言虛者補之,實者瀉之,不實不虛,以經取之,何謂也?
然:是剛柔之事也。陰井乙木,陽井庚金。陽井庚,庚者,乙之剛也;陰井乙,乙者,庚之柔也。乙為木,故言陰井木也;庚為金,故言陽井金也。餘皆倣此。
\\\\
  六十九難曰:經言虛者補之,實者瀉之,不實不虛,以經取之,何謂也?
六十五難曰:經言所出為井,所入為合,其法奈何?
\\\\
  六十九難曰:經言虛者補之,實者瀉之,不實不虛,以經取之,何謂也?
然:所出為井,井者,東方春也,萬物之始生,故言所出為井也。所入為合,合者,北方冬也,陽氣入藏,故言所入為合也。
\\\\
  六十九難曰:經言虛者補之,實者瀉之,不實不虛,以經取之,何謂也?
六十六難曰:經言肺之原出于太淵,心之原出于太陵,肝之原出于太衝,脾之原出于太白,腎之原出于太谿,少陰之原出于兌骨,膽之原出于丘墟,胃之原出于衝陽,三焦之原出于陽池,膀胱之原出于京骨,大腸之原出于合谷,小腸之原出于腕骨。十二經皆以俞為原者,何也?
\\\\
  六十九難曰:經言虛者補之,實者瀉之,不實不虛,以經取之,何謂也?
然:五藏俞者,三焦之所行,氣之所留止也。
\\\\
  六十九難曰:經言虛者補之,實者瀉之,不實不虛,以經取之,何謂也?
焦所行之俞為原者,何也?
\\\\
  六十九難曰:經言虛者補之,實者瀉之,不實不虛,以經取之,何謂也?
然:臍下腎間動氣者,人之生命也,十二經之根本也,故名曰原。三焦者,原氣之別使也,主通行三氣,經歷於五藏六府。原者,三焦之尊號也,故所止輒為原。五藏六府之有病者,取其原也。
\\\\
  六十九難曰:經言虛者補之,實者瀉之,不實不虛,以經取之,何謂也?
六十七難曰:五藏募皆左陰,而俞在陽者,何謂也?
\\\\
  六十九難曰:經言虛者補之,實者瀉之,不實不虛,以經取之,何謂也?
然:陰病行陽,陽病行陰,故令募在陰,俞在陽。
\\\\
  六十九難曰:經言虛者補之,實者瀉之,不實不虛,以經取之,何謂也?
六十八難曰:五藏六府,各有井滎俞經合,皆何所主?
\\\\
  六十九難曰:經言虛者補之,實者瀉之,不實不虛,以經取之,何謂也?
然:經言所出為井,所流為滎,所注為俞,所行為經,所入為合。井主心下滿,滎主身熱,俞主體重節痛,經主喘咳寒熱,合主逆氣而泄。此五藏六府其井滎俞經合所主病也。