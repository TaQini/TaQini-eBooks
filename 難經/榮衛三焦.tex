\section{榮衛三焦}

  三十難曰:榮氣之行,常與衛氣相隨不?
\\\\
  然:經言人受氣於穀。穀入於胃,乃傳與五藏六府,五藏六府皆受於氣。其清者為榮,濁者為衛,榮行脈中,衛行脈外,榮周不息,五十而復大會。陰陽相貫,如環之無端,故知榮衛相隨也。
\\\\
  三十一難曰:三焦者,何稟何生?何始何終?其治常在何許?可曉以不?
\\\\
  然:三焦者,水穀之道路,氣之所終始也。上焦者,在心下,下膈,在胃上口,主內而不出。其治在膻中,玉堂下一寸六分,直兩乳間陷者是。中焦者,在胃中脘,不上不下,主腐熟水穀。其治有齊傍。下焦者,當膀胱上口,主分別清濁,主出而不內,以傳導也。其治在齊下一寸。故名曰三焦,其府在氣街。一本曰衝。
