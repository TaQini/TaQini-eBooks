\section{藏府積聚}

  六十九難曰:經言虛者補之,實者瀉之,不實不虛,以經取之,何謂也?
五十五難曰:病有積、有聚,何以別之?
\\\\
  六十九難曰:經言虛者補之,實者瀉之,不實不虛,以經取之,何謂也?
然:積者,陰氣也;聚者,陽氣也,故陰沉而伏,陽浮而動。氣之所積名曰積,氣之所聚名曰聚。故積者,五藏所生;聚者,六府所成也。積者,陰氣也,其始發有常處,其痛不離其部,上下有所終始,左右有所窮處;聚者,陽氣也,其始發無根本,上下無所留止,其痛無常處,謂之聚。故以是別知積聚也。
\\\\
  六十九難曰:經言虛者補之,實者瀉之,不實不虛,以經取之,何謂也?
五十六難曰:五藏之積,各有名乎?以何月何日得之?
\\\\
  六十九難曰:經言虛者補之,實者瀉之,不實不虛,以經取之,何謂也?
然:肝之積名曰肥氣,在左脅下,如覆杯,有頭足。久不愈,令人發咳逆,痎瘧,連歲不已。以季夏戊己日得之。何以言之?肺病傳於肝,肝當傳脾,脾季夏適王,王者不受邪,肝復欲還肺,肺不肯受,故留結為積。故知肥氣以季夏戊己日得之。
\\\\
  六十九難曰:經言虛者補之,實者瀉之,不實不虛,以經取之,何謂也?
心之積名曰伏梁,起齊上,大如臂,上至心下。久不愈,令人病煩心,以秋庚辛日得之。何以言之?腎病傳心,心當傳肺,肺以秋適王,王者不受邪,心復欲還腎,腎不肯受,故留結為積。故知伏梁以秋庚辛日得之。
\\\\
  六十九難曰:經言虛者補之,實者瀉之,不實不虛,以經取之,何謂也?
脾之積名曰痞氣,在胃脘,覆大如盤。久不愈,令人四肢不收,發黃疸,飲食不為肌膚。以冬壬癸日得之。何以言之?肝病傳脾,脾當傳腎,腎以冬適王,王者不受邪,脾復欲還肝,肝不肯受,故留結為積。故知痞氣以冬壬癸日得之。
\\\\
  六十九難曰:經言虛者補之,實者瀉之,不實不虛,以經取之,何謂也?
肺之積名曰息賁,在右脅下,覆大如杯。久不已,令人洒淅寒熱,喘咳,發肺壅。以春甲乙日得之。何以言之?心病傳肺,肺當傳肝,肝以春適王,王者不受邪,肺復欲還心,心不肯受,故留結為積。故知息賁以春甲乙日得之。
\\\\
  六十九難曰:經言虛者補之,實者瀉之,不實不虛,以經取之,何謂也?
腎之積名曰賁豚,發於少腹,上至心下,若豚狀,或上或下無時,久不已,令人喘逆,骨痿少氣,以夏丙丁日得之。何以言之?脾病傳腎,腎當傳心,心以夏適王,王者不受邪,腎復欲還脾,脾不肯受,故留結為積。故知賁豚以夏丙丁日得之。此是五積之要法也。