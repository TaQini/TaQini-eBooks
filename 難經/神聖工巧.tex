\section{神聖工巧}

  六十九難曰:經言虛者補之,實者瀉之,不實不虛,以經取之,何謂也?
六十一難曰:經言望而知之謂之神,聞而知之謂之聖,問而知之謂之工,切脈而知之謂之巧,何謂也?
\\\\
  六十九難曰:經言虛者補之,實者瀉之,不實不虛,以經取之,何謂也?
然:望而知之者,望見其五色,以知其病。聞而知之者,聞其五音,以別其病。問而知之者,聞其所欲五味,以知其病所起所在也。切脈而知之者,診其寸口,視其虛實,以知其病,病在何藏府也。經言以外知之曰聖,以內知之曰神,此之謂也。