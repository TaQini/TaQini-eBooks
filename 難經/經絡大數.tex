\section{經絡大數}

  六十九難曰:經言虛者補之,實者瀉之,不實不虛,以經取之,何謂也?
二十五難曰:有十二經,五藏六府十一耳,其一經者,何等經也?
\\\\
  六十九難曰:經言虛者補之,實者瀉之,不實不虛,以經取之,何謂也?
然:一經者,手少陰與心主別脈也。心主與三焦為表裏,俱有名而無形,故言經有十二也。
\\\\
  六十九難曰:經言虛者補之,實者瀉之,不實不虛,以經取之,何謂也?
二十六難曰:經有十二,絡有十五,餘三絡者,是何等絡也?
\\\\
  六十九難曰:經言虛者補之,實者瀉之,不實不虛,以經取之,何謂也?
然:有陽絡,有陰絡,有脾之大絡。陽絡者,陽蹻之絡也;陰絡者,陰蹻之絡也。故絡有十五焉。
