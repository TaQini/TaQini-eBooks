%!TEX program = xelatex
%!TEX encoding = UTF-8

%% if you want chinese vertical, then pass argument 'landscape' to document class,
%% and uncomment the rotation,
%% and change font settings to vertical.

\documentclass[UTF8, nofont, landscape]{ctexbook} %% landscape, landscape

%% %% landscape------------------
\usepackage{atbegshi}
\AtBeginShipout{%
  \global\setbox\AtBeginShipoutBox\vbox{%
    \special{pdf: put @thispage <</Rotate 90>>}%
    \box\AtBeginShipoutBox
  }%
}%
%% %% font
\defaultCJKfontfeatures{RawFeature={vertical:+vert}}
\makeatletter
\newcommand*{\shifttext}[2]{%
  \settowidth{\@tempdima}{#2}%
  \makebox[\@tempdima]{\hspace*{#1}#2}%
}%
\makeatother
\newcommand\ytza[1]{\shifttext{-3.5pt}{\ytzfz #1}}

%% no landscape------------
% \newcommand\ytza[1]{\raisebox{2pt}{\ytzfz #1}}


%% common font settings
\setCJKmainfont[BoldFont=Adobe Heiti Std,ItalicFont=Adobe Kaiti Std]{Adobe Song Std}
\setCJKsansfont{Adobe Heiti Std}
\setCJKmonofont{Adobe Fangsong Std}
\newCJKfontfamily{\fzsongbig}{FZSongS-Extended}
\newCJKfontfamily{\arpluming}{AR PL UMing CN} %這個明細體和新明細體补充顯示Adobe缺失的漢字。但它的编码库可能比Adobe还小些。
\newCJKfontfamily{\simsun}{SimSun}
\newCJKfontfamily{\newsimsun}{NSimSun}
\DeclareTextFontCommand{\yt}{\arpluming}
\DeclareTextFontCommand{\ytz}{\simsun}
\DeclareTextFontCommand{\ytzn}{\newsimsun}
\DeclareTextFontCommand{\ytzfz}{\fzsongbig}


\renewcommand{\thepage}{\Chinese{page}}

% \usepackage{wallpaper}
% \ThisULCornerWallPaper{1}{竹.jpg}

\usepackage{geometry}
%% \newgeometry{
%%   top=50pt, bottom=50pt, left=86pt, right=86pt,
%%   headsep=5pt,
%% }
\newgeometry{
  top=50pt, bottom=50pt, left=46pt, right=46pt,
  headsep=25pt,
}
\savegeometry{mdGeo}
\loadgeometry{mdGeo}


%% \usepackage{titlesec}        % this can not be used when vertical mode
%% %% \titleformat{\chapter}[display]
%% %%             {\Huge\bfseries\centering}
%% %%             {}{0pt}{#1}
%% %% \titleformat{\chapter}[display]
%% %%   {\centering\xingkai}
%% %%   {}{0pt}{}
%% \titlespacing*{\section}{0pt}{9pt}{0pt}
%% \titlespacing*{\chapter}{0pt}{0pt}{-16pt}


\newcommand{\pageCN}{第\thepage 页}

\usepackage{fancyhdr}
%% \usepackage{ifthen}
\fancypagestyle{plain}{ % first page style
    \fancyhf{}
    %% \fancyfoot[LE,RO]{
    %%   {\fangsong \pageCN}%偶数页
    %% }
}
%% % 从第二页开始的style
%% \fancyhf{}
%% \fancyhead[LE]{\textit\pageCN\leftmark}  %\lishu 北航明德养生社\qquad 晨读本
%% \fancyfoot[LE]{}
%% \fancyfoot[RO]{\textit\pageCN\rightmark}
%% \renewcommand{\headrulewidth}{0.5bp} % 页眉线宽度
%% \pagestyle{fancy}



%% \RequirePackage{titletoc}

%% \titlecontents{chapter}[0pt]{\heiti\zihao{-4}}{\thecontentslabel\ }{}
%%               {\hspace{.5em}\titlerule*[4pt]{$\cdot$}\contentspage}
%%               \titlecontents{section}[2em]{\vspace{0.1\baselineskip}\songti\zihao{-4}}{\thecontentslabel\ }{}
%%                             {\hspace{.5em}\titlerule*[4pt]{$\cdot$}\contentspage}
%%                             \titlecontents{subsection}[4em]{\vspace{0.1\baselineskip}\songti\zihao{-4}}{\thecontentslabel\ }{}
%% {\hspace{.5em}\titlerule*[4pt]{$\cdot$}\contentspage}


\ctexset{
  %% fontset = adobe,
  today = big,
  punct = quanjiao, %% banjiao,
  autoindent = 0pt,
  section = {
    numbering = false, % 标题不显示编号
    titleformat = \Huge,
    format = \texttt
  },
  chapter = {
    numbering = false, % 标题不显示编号
    titleformat = \centering\Huge, % 小四号字
    format = \textbf % 楷体
  }
}

%% \setCJKmainfont{AdobeSongStd-Light.otf} 
%% \setCJKmainfont{AdobeSongStd-Regular.otf} 
%% \setCJKmainfont{AdobeFangsongStd-Regular.otf}
% \setCJKmainfont{方正宋刻本秀楷繁体.ttf}
% \setCJKmainfont{方正清刻本悦宋繁.ttf}
% \setCJKmainfont{WenYue-GuDianMingChaoTi-NC-W5.otf}

\title{\zihao{0}\textbf{《黃帝八十一難經》}}

\author{\normalsize 由 \textit{樂行} 編輯/整理}
\date{\normalsize\today 版}

\begin{document}
\maketitle
\tableofcontents

\part{經脈診候}
\LARGE
\section{經脈診候}

  一難曰:十二經皆有動脈,獨取寸口,以決五藏六府死生吉凶之法,何謂也?
\\\\
  然:寸口者,脈之大會,手太陰之脈動也。人一呼脈行三寸,一吸脈行三寸,呼吸定息,脈行六寸。人一日一夜,凡一萬三千五百息,脈行五十度,周於身。漏水下百刻,榮衛行陽二十五度,行陰亦二十五度,為一周也,故五十度復會於手太陰。寸口者,五藏六府之所終始,故法取於寸口也。
\\\\
  二難曰:脈有尺寸,何謂也?
\\\\
  然:尺寸者,脈之大要會也。從關至尺是尺內,陰之所治也;從關至魚際是寸內,陽之所治也。故分寸為尺,分尺為寸。故陰得尺內一寸,陽得寸內九分,尺寸終始一寸九分,故曰尺寸也。
\\\\
  三難曰:脈有大過,有不及,有陰陽相乘,有覆有溢,有關有格,何謂也?
\\\\
  然:關之前者,陽之動,脈當見九分而浮。過者,法曰大過;減者,法曰不及。遂上魚為溢,為外關內格,此陰乘之脈也。
\\\\
  關以後者,陰之動也,脈當見一寸而沉。過者,法曰大過;減者,法曰不及。遂入尺為覆,為內關外格,此陽乘之脈也。故曰覆溢,是其真藏之脈,人不病而死也。
\\\\
  四難曰:脈有陰陽之法,何謂也?
\\\\
  然:呼出心與肺,吸入腎與肝,呼吸之間,脾受穀味也,其脈在中。浮者陽也,沉者陰也,故曰陰陽也。
\\\\
  心肺俱浮,何以別之?
\\\\
  然:浮而大散者,心也;浮而短濇者,肺也。
\\\\
  腎、肝俱沉,何以別之?
\\\\
  然:牢而長者,肝也;按之濡,舉指來實者,腎也。脾者中州,故其脈在中,是陰陽之法也。
\\\\
  脈有一陰一陽,一陰二陽,一陰三陽;有一陽一陰,一陽二陰,一陽三陰。如此之言,寸口有六脈俱動耶?
\\\\
  然:此言者,非有六脈俱動也,謂浮、沉、長、短、滑、濇也。浮者陽也,滑者陽也,長者陽也;沉者陰也,短者陰也,濇者陰也。所謂一陰一陽者,謂脈來沉而滑也;一陰二陽者,謂脈來沉滑而長也;一陰三陽者,謂脈來沉滑而長,時一沉也。所言一陽一陰者,謂脈來浮而濇也;一陽二陰者,謂脈來長而沉濇也;一陽三陰者,謂脈來沉濇而短,時一浮也。各以其經所在,名病逆順也。
\\\\
  五難曰:脈有輕重,何謂也?
\\\\
  然:初持脈,如三菽之重,與皮毛相得者,肺部也。如六菽之重,與血脈相得者,心部也。如九菽之重,與肌肉相得者,脾部也。如十二菽之重,與筋平者,肝部也。按之至骨,舉指來疾者,腎也。故曰輕重也。
\\\\
  六難曰:脈有陰盛陽虛,陽盛陰虛,何謂也?
\\\\
  然:浮之損小,沉之實大,故曰陰盛陽虛。沉之損小,浮之實大,故曰陽盛陰虛,是陰陽虛實意也。
\\\\
  七難曰:經言少陽之至,乍小乍大,乍短乍長;陽明之至,浮大而短;太陽之至,洪大而長;太陰之至,緊大而長;少陰之至,緊細而微;厥陰之至,沉短而敦。此六者,是平脈邪?將病脈邪?
\\\\
  然:皆王脈也。
\\\\
  其氣以何月,各王幾日?
\\\\
  然:冬至之後,得甲子少陽王,復得甲子陽明王,復得甲子太陽王,復得甲子太陰王,復得甲子少陰王,復得甲子厥陰王。王各六十日,六六三百六十日,以成一歲。此三陽三陰之王時日大要也。
\\\\
  八難曰:寸口脈平而死者,何謂也?
\\\\
  然:諸十二經脈者,皆係於生氣之原。所謂生氣之原者,謂十二經之根本也,謂腎間動氣也。此五藏六府之本,十二經脈之根,呼吸之門,三焦之原。一名守邪之神。故氣者,人之根本也,根絕則莖葉枯矣。寸口脈平而死者,生氣獨絕於內也。
\\\\
  九難曰:何以別知藏府之病耶?
\\\\
  然:數者府也,遲者藏也。數則為熱,遲則為寒。諸陽為熱,諸陰為寒。故以別知藏府之病也。
\\\\
  十難曰:一脈為十變者,何謂也?
\\\\
  然:五邪剛柔相逢之意也。假令心脈急甚者,肝邪干心也;心脈微急者,膽邪干小腸也;心脈大甚者,心邪自干心也;心脈微大者,小腸邪自干小腸也;心脈緩甚者,脾邪干心也;心脈微緩者,胃邪干小腸也;心脈濇甚者,肺邪干心也;心脈微濇者,大腸邪干小腸也;心脈沉甚者,腎邪干心也;心脈微沉者,膀胱邪干小腸也。五藏各有剛柔邪,故令一脈輒變為十也。
\\\\
  十一難曰:經言脈不滿五十動而一止,一藏無氣者,何藏也?
\\\\
  然:人吸者隨陰入,呼者因陽出。今吸不能至腎,至肝而還,故知一藏無氣者,腎氣先盡也。
\\\\
  十二難曰:經言五藏脈已絕於內,用鍼者反實其外;五藏脈已絕於外,用鍼者反實其內。內外之絕,何以別之?
\\\\
  然:五藏脈已絕於內者,腎肝氣已絕於內也,而醫反補其心肺;五藏脈已絕於外者,其心肺脈已絕於外也,而醫反補其腎肝。陽絕補陰,陰絕補陽,是謂實實虛虛,損不足益有餘。如此死者,醫殺之耳。
\\\\
  十三難曰:經言見其色而不得其脈,反得相勝之脈者即死,得相生之脈者,病即自已。色之與脈當參相應,為之奈何?
\\\\
  然:五藏有五色,皆見於面,亦當與寸口、尺內相應。假令色青,其脈當弦而急;色赤,其脈浮大而散;色黃,其脈中緩而大;色白,其脈浮濇而短;色黑,其脈沉濇而滑。此所謂五色之與脈,當參相應也。
\\\\
  脈數,尺之皮膚亦數;脈急,尺之皮膚亦急;脈緩,尺之皮膚亦緩;脈濇,尺之皮膚亦濇;脈滑,尺之皮膚亦滑。
\\\\
  藏各有聲、色、臭、味,當與寸口、尺內相應,其不相應者病也。假令色青,其脈浮濇而短,若大而緩為相勝;浮大而散,若小而滑為相生也。經言知一為下工,知二為中工,知三為上工。上工者十全九,中工者十全八,下工者十全六,此之謂也。
\\\\
  十四難曰:脈有損至,何謂也?
\\\\
  然:至之脈,一呼再至曰平,三至曰離經,四至曰奪精,五至曰死,六至曰命絕,此死之脈。何謂損?一呼一至曰離經,二呼一至曰奪精,三呼一至曰死,四呼一至曰命絕。此謂損之脈也。至脈從下上,損脈從上下也。
\\\\
  損脈之為病柰何?
\\\\
  然:一損損於皮毛,皮聚而毛落;二損損於血脈,血脈虛少,不能榮於五藏六府也;三損損於肌肉,肌肉消瘦,飲食不為肌膚;四損損於筋,筋緩不能自收持;五損損於骨,骨痿不能起於床。反此者,至於收病也。從上下者,骨痿不能起於床者死;從下上者,皮聚而毛落者死。
\\\\
  治損之法奈何?
\\\\
  然:損其肺者,益其氣;損其心者,調其榮衛;損其脾者,調其飲食,適寒溫;損其肝者,緩其中;損其腎者,益其精,此治損之法也。
\\\\
  脈有一呼再至,一吸再至;有一呼三至,一吸三至;有一呼四至,一吸四至;有一呼五至,一吸五至;有一呼六至,一吸六至;有一呼一至,一吸一至;有再呼一至,再吸一至;有呼吸再至。脈來如此,何以別知其病也?
\\\\
  然:脈來一呼再至,一吸再至,不大不小曰平。一呼三至,一吸三至,為適得病,前大後小,即頭痛目眩,前小後大,即胸滿短氣。一呼四至,一吸四至,病欲甚,脈洪大者,苦煩滿,沉細者,胸中痛,滑者傷熱,濇者中霧露。一呼五至,一吸五至,其人當困,沉細夜加,浮大晝加,不大不小,雖困可治,其有大小者為難治。一呼六至,一吸六至,為死脈也,沉細夜死,浮大晝死。一呼一至,一吸一至,名曰損,人雖能行,猶當著床,所以然者,血氣皆不足故也。再呼一至,呼吸再至,名曰無魂,無魂者當死也,人雖能行,名曰行尸。
\\\\
  上部有脈,下部無脈,其人當吐,不吐者死。上部無脈,下部有脈,雖困無能為害也。所以然者,譬如人之有尺,樹之有根,枝葉雖枯槁,根本將自生。脈有根本,人有元氣,故知不死。
\\\\
  十五難曰:經言春脈弦,夏脈鉤,秋脈毛,冬脈石,是王脈耶?將病脈也?
\\\\
  然:弦、鉤、毛、石者,四時之脈也。春脈弦者,肝,東方木也,萬物始生,未有枝葉。故其脈之來,濡弱而長,故曰弦。夏脈鉤者,心,南方火也,萬物之所盛,垂枝布葉,皆下曲如鉤。故其脈之,來疾去遲,故曰鉤。秋脈毛者,肺,西方金也,萬物之所終,草木華葉,皆秋而落,其枝獨在,若毫毛也。故其脈之來,輕虛以浮,故曰毛。冬脈石者,腎,北方水也,萬物之所藏也,盛冬之時,水凝如石。故其脈之來,沉濡而滑,故曰石。此四時之脈也。
\\\\
  如有變柰何?
\\\\
  然:春脈弦,反者為病。何謂反?
\\\\
  然:其氣來實強,是謂太過,病在外;氣來虛微,是謂不及,病在內。氣來厭厭聶聶,如循榆葉曰平;益實而滑,如循長竿曰病;急而勁益強,如新張弓弦曰死。春脈微弦曰平,弦多胃氣少曰病,但弦無胃氣曰死,春以胃氣為本。
\\\\
  夏脈鉤,反者為病。何謂反?
\\\\
  然:其氣來實強,是謂太過,病在外;氣來虛微,是謂不及,病在內。其脈來累累如環,如循琅玕曰平;來而益數,如雞舉足者曰病;前曲後居,如操帶鉤曰死。夏脈微鉤曰平,鉤多胃氣少曰病,但鉤無胃氣曰死,夏以胃氣為本。
\\\\
  秋脈微毛,反者為病。何謂反?
\\\\
  然:氣來實強,是謂太過,病在外;氣來虛微,是謂不及,病在內。其脈來藹藹如車蓋,按之益大曰平;不上不下,如循雞羽曰病;按之消索,如風吹毛曰死。秋脈微毛為平,毛多胃氣少曰病,但毛無胃氣曰死,秋以胃氣為本。
\\\\
  冬脈石,反者為病。何謂反?
\\\\
  然:其氣來實強,是謂太過,病在外;氣來虛微,是謂不及,病在內。脈來上大下兌,濡滑如雀之啄曰平;啄啄連屬,其中微曲曰病;來如解索,去如彈石曰死。冬脈微石曰平,石多胃氣少曰病,但石無胃氣曰死,冬以胃氣為本。
\\\\
  胃者,水穀之海也,主稟四時。故皆以胃氣為本,是謂四時之變病,死生之要會也。脾者,中州也,其平和不可得見,衰乃見耳。來如雀之,如水之下漏,是脾之衰見也。
\\\\
  十六難曰:脈有三部九候,有陰陽,有輕重,有六十首,一脈變為四時,離聖久遠,各自是其法,何以別之?
\\\\
  然:是其病,有內外證。
\\\\
  其病為之柰何?
\\\\
  然:假令得肝脈,其外證:善潔,面青,善怒;其內證:齊左有動氣,按之牢若痛;其病:四肢滿,閉癃,溲便難,轉節。有是者肝也,無是者非也。
\\\\
  假令得心脈,其外證:面赤,口乾,喜笑;其內證:齊上有動氣,按之牢若痛;其病:煩心,心痛,掌中熱而啘。有是者心也,無是者非也。
\\\\
  假令得脾脈,其外證:面黃,善噫,善思,善味;其內證:當齊有動氣,按之牢若痛;其病:腹脹滿,食不消,體重節痛,怠墮嗜臥,四肢不收。有是者脾也,無是者非也。
\\\\
  假令得肺脈,其外證:面白,善嚏,悲愁不樂,欲哭;其內證:齊右有動氣,按之牢若痛;其病:喘欬,洒淅寒熱。有是者肺也,無是者非也。
\\\\
  假令得腎脈,其外證:面黑,喜恐欠;其內證:齊下有動氣,按之牢若痛;其病:逆氣,少腹急痛,泄如下重,足脛寒而逆。有是者腎也,無是者非也。
\\\\
  十七難曰:經言病或有死,或有不治自愈,或連年月不已。其死生存亡,可切脈而知之耶?
\\\\
  然:可盡知也。診病若閉目不欲見人者,脈當得肝脈強急而長,而反得肺脈浮短而濇者,死也。
\\\\
  病若開目而渴,心下牢者,脈當得緊實而數,反得沉濡而微者,死也。
\\\\
  病若吐血,復鼽衂血者,脈當沉細,而反浮大而牢者,死也。
\\\\
  病若譫言妄語,身當有熱,脈當洪大,而手足厥逆,脈沉細而微者,死也。
\\\\
  病若大腹而洩者,脈當微細而濇,反緊大而滑者,死也。
\\\\
  十八難曰:脈有三部,部有四經,手有太陰、陽明,足有太陽、少陰,為上下部,何謂也?
\\\\
  然:手太陰,陽明金也;足少陰,太陽水也。金生水,水流下行而不能上,故在下部也。足厥陰,少陽木也,生手太陽、少陰火,火炎上行而不能下,故為上部。手心主,少陽火,生足太陰、陽明土,土主中宮,故在中部也。此皆五行子母更相生養者也。
\\\\
  脈有三部九候,各何所主之?
\\\\
  然:三部者,寸、關、尺也。九候者,浮、中、沉也。上部法天,主胸以上至頭之有疾也;中部法人,主膈以下至齊之有疾也;下部法地,主齊以下至足之有疾也。審而刺之者也。
\\\\
  人病有沉滯久積聚,可切脈而知之耶?
\\\\
  然:診在右脅有積氣,得肺脈結,脈結甚則積甚,結微則氣微。
\\\\
  診不得肺脈,而右脅有積氣者,何也?
\\\\
  然:肺脈雖不見,右手脈當沉伏。
\\\\
  其外痼疾同法耶?將異也?
\\\\
  然:結者,脈來去時一止,無常數,名曰結也。伏者,脈行筋下也。浮者,脈在肉上行也。左右表裏,法皆如此。假令脈結伏者,內無積聚;脈浮結者,外無痼疾;有積聚脈不結伏,有痼疾脈不浮結。為脈不應病,病不應脈,是為死病也。
\\\\
  十九難曰:經言脈有逆順,男女有常,而反者,何謂也?
\\\\
  然:男子生於寅,寅為木,陽也;女子生於申,申為金,陰也。故男脈在關上,女脈在關下。是以男子尺脈恆弱,女子尺脈恆盛,是其常也。反者,男得女脈,女得男脈也。
\\\\
  其為病何如?
\\\\
  然:男得女脈為不足,病在內;左得之,病則在左;右得之,病則在右,隨脈言之也。女得男脈為太過,病在四肢;左得之,病則在左;右得之,病則在右,隨脈言之,此之謂也。
\\\\
  二十難曰:經言脈有伏匿,伏匿於何藏而言伏匿耶?
\\\\
  然:謂陰陽更相乘,更相伏也。脈居陰部而反陽脈見者,為陽乘陰也;脈雖時沉濇而短,此謂陽中伏陰也;脈居陽部而反陰脈見者,為陰乘陽也;脈雖時浮滑而長,此謂陰中伏陽也。重陽者狂,重陰者癲。脫陽者見鬼,脫陰者目盲。
\\\\
  二十一難曰:經言人形病、脈不病曰生,脈病、形不病曰死,何謂也?
\\\\
  然:人形病,脈不病,非有不病者也,謂息數不應脈數也。此大法。
\\\\
  二十二難曰:經言脈有是動,有所生病。一脈輒變為二病者,何也?
\\\\
  然:經言是動者,氣也;所生病者,血也。邪在氣,氣為是動;邪在血,血為所生病。氣主呴之,血主濡之。氣留而不行者,為氣先病也;血壅而不濡者,為血後病也。故先為是動,後所生病也。
\\\\
  二十三難曰:手足三陰三陽,脈之度數,可曉以不?
\\\\
  然:手三陽之脈,從手至頭,長五尺,五六合三丈。
\\\\
  手三陰之脈,從手至胸中,長三尺五寸,三六一丈八尺,五六三尺,合二丈一尺。
\\\\
  足三陽之脈,從足至頭,長八尺,六八四丈八尺。
\\\\
  足三陰之脈,從足至胸,長六尺五寸,六六三丈六尺,五六三尺,合三丈九尺。
\\\\
  人兩足蹻脈,從足至目,長七尺五寸,二七一丈四尺,二五一尺,合一丈五尺。
\\\\
  督脈、任脈,各長四尺五寸,二四八尺,二五一尺,合九尺。
\\\\
  凡脈長一十六丈二尺,此所謂十二經脈長短之數也。
\\\\
  經脈十二,絡脈十五,何始何窮也?
\\\\
  然:經脈者,行血氣,通陰陽,以榮於身者也。其始從中焦,注手太陰、陽明;陽明注足陽明、太陰;太陰注手少陰、太陽;太陽注足太陽、少陰;少陰注手心主、少陽;少陽注足少陽、厥陰;厥陰復還注手太陰。
\\\\
  別絡十五,皆因其原,如環無端,轉相溉灌,朝於寸口、人迎,以處百病,而決死生也。
\\\\
  經曰:明知終始,陰陽定矣,何謂也?
\\\\
  然:終始者,脈之紀也。寸口、人迎,陰陽之氣通於朝使,如環無端,故曰始也。終者,三陰三陽之脈絕,絕則死。死各有形,故曰終也。
\\\\
  二十四難曰:手足三陰三陽氣已絕,何以為候?可知其吉凶不?
\\\\
  然:足少陰氣絕,即骨枯。少陰者,冬脈也,伏行而溫於骨髓。故骨髓不溫,即肉不著骨;骨肉不相親,即肉濡而卻;肉濡而卻,故齒長而枯,髮無潤澤;者,骨先死。戊日篤,己日死。
\\\\
  足太陰氣絕,則脈不榮其口唇。口唇者,肌肉之本也。脈不榮,則肌肉不滑澤;肌肉不滑澤,則肉滿;肉滿,則唇反;唇反,則肉先死。甲日篤,乙日死。
\\\\
  足厥陰氣絕,即筋縮引卵與舌卷。厥陰者,肝脈也。肝者,筋之合也。筋者,聚於陰器而絡於舌本。故脈不榮,則筋縮急;即引卵與舌;故舌卷卵縮,此筋先死。庚日篤,辛日死。筋縮急。
\\\\
  手太陰氣絕,即皮毛焦。太陰者,肺也,行氣溫於皮毛者也。氣弗榮則皮毛焦,皮毛焦則津液去,津液去即皮節傷,皮節傷則皮枯毛折,毛折者則毛先死。丙日篤,丁日死。
\\\\
  手少陰氣絕則脈不通,脈不通則血不流,血不流則色澤去,故面黑如梨,此血先死。壬日篤,癸日死。
\\\\
  陰氣俱絕者,則目眩轉目瞑。目瞑者,為失志;失志者,則志先死,死即目瞑也。
\\\\
  陽氣俱絕者,則陰與陽相離,陰陽相離,則腠理泄,絕汗乃出,大如貫珠,轉出不流,即氣先死。旦占夕死,夕占旦死。





\part{經絡大數}
\LARGE
\section{經絡大數}

  六十九難曰:經言虛者補之,實者瀉之,不實不虛,以經取之,何謂也?
二十五難曰:有十二經,五藏六府十一耳,其一經者,何等經也?
\\\\
  六十九難曰:經言虛者補之,實者瀉之,不實不虛,以經取之,何謂也?
然:一經者,手少陰與心主別脈也。心主與三焦為表裏,俱有名而無形,故言經有十二也。
\\\\
  六十九難曰:經言虛者補之,實者瀉之,不實不虛,以經取之,何謂也?
二十六難曰:經有十二,絡有十五,餘三絡者,是何等絡也?
\\\\
  六十九難曰:經言虛者補之,實者瀉之,不實不虛,以經取之,何謂也?
然:有陽絡,有陰絡,有脾之大絡。陽絡者,陽蹻之絡也;陰絡者,陰蹻之絡也。故絡有十五焉。


\part{奇經八脈}
\LARGE
\section{奇經八脈}

  二十七難曰:脈有奇經八脈者,不扚於十二經,何謂也?
\\\\
  然:有陽維,有陰維,有陽蹻,有陰蹻,有衝,有督,有任,有帶之脈。凡此八脈者,皆不拘於經,故曰奇經八脈也。
\\\\
  經有十二,絡有十五,凡二十七氣,相隨上下,何獨不拘於經也?
\\\\
  然:聖人圖設溝渠,通利水道,以備不然。天雨降下,溝渠溢滿,當此之時,霶霈妄行,聖人不能復圖也。此絡脈滿溢,諸經不能復拘也。
\\\\
  二十八難曰:其奇經八脈者,既不拘於十二經,皆何起何繼也?
\\\\
  然:督脈者,起於下極之俞,並於脊裏,上至風府,入於腦。
\\\\
  任脈者,起於中極之下,以上毛際,循腹裏,上關元,至喉咽。
\\\\
  衝脈者,起於氣衝,並足陽明之經,夾齊上行,至胸中而散也。
\\\\
  帶脈者,起於季脅,廻身一周。
\\\\
  陽蹻脈者,起於跟中,循外踝上行,入風池。
\\\\
  陰蹻脈者,亦起於跟中,循內踝上行,至咽喉,交貫衝脈。
\\\\
  陽維、陰維者,維絡于身,溢畜不能環流灌溉諸經者也。故陽維起於諸陽會也,陰維起於諸陰交也。
\\\\
  比于聖人圖設溝渠,溝渠滿溢,流于深湖,故聖人不能拘通也。而人脈隆聖,入於八脈而不環周,故十二經亦不能拘之。其受邪氣,畜則腫熱,砭射之也。
\\\\
  二十九難曰:奇經之為病何如?
\\\\
  然:陽維維於陽,陰維維於陰,陰陽不能自相維,則悵然失志,溶溶不能自收持。陰蹻為病,陽緩而陰急。陽蹻為病,陰緩而陽急。衝之為病,逆氣而裏急。督之為病,脊強而厥。任之為病,其內苦結,男子為七疝,女子為瘕聚。帶之為病,腹滿,腰溶溶若坐水中。陽維為病苦寒熱,陰維為病苦心痛。此奇經八脈之為病也。

\part{榮衛三焦}
\LARGE
\section{榮衛三焦}

  三十難曰:榮氣之行,常與衛氣相隨不?
\\\\
  然:經言人受氣於穀。穀入於胃,乃傳與五藏六府,五藏六府皆受於氣。其清者為榮,濁者為衛,榮行脈中,衛行脈外,榮周不息,五十而復大會。陰陽相貫,如環之無端,故知榮衛相隨也。
\\\\
  三十一難曰:三焦者,何稟何生?何始何終?其治常在何許?可曉以不?
\\\\
  然:三焦者,水穀之道路,氣之所終始也。上焦者,在心下,下膈,在胃上口,主內而不出。其治在膻中,玉堂下一寸六分,直兩乳間陷者是。中焦者,在胃中脘,不上不下,主腐熟水穀。其治有齊傍。下焦者,當膀胱上口,主分別清濁,主出而不內,以傳導也。其治在齊下一寸。故名曰三焦,其府在氣街。一本曰衝。


\part{藏府配像}
\LARGE
\section{藏府配像}

  三十二難曰:五藏俱等,而心肺獨在膈上者,何也?
\\\\
  然:心者血,肺者氣,血為榮,氣為衛,相隨上下,謂之榮衛,通行經絡,營周於外,故令心肺在膈上也。
\\\\
  三十三難曰:肝青象木,肺白象金;肝得水而沉,木得水而浮;肺得水而浮,金得水而沉。其意何也?
\\\\
  然:肝者,非為純木也,乙角也,庚之柔。大言陰與陽,小言夫與婦。釋其微陽,而吸其微陰之氣,其意樂金,又行陰道多,故令肝得水而沉也。肺者,非為純金也,辛商也,丙之柔。大言陰與陽,小言夫與婦。釋其微陰,婚而就火,其意樂火,又行陽道多,故令肺得水而浮也。
\\\\
  肺熟而復沉,肝熟而復浮者,何也?故知辛當歸庚,乙當歸甲也。
\\\\
  三十四難曰:五藏各有聲、色、臭、味、,可曉知以不?
\\\\
  然:《十變》言,肝色青,其臭臊,其味酸,其聲呼,其液泣;心色赤,其臭焦,其味苦,其聲言,其液汗;脾色黃,其臭香,其味甘,其聲歌,其液涎;肺色白,其臭腥,其味辛,其聲哭,其液涕;腎色黑,其臭腐,其味鹹,其聲呻,其液唾。是五藏聲、色、臭、味、也。
\\\\
  藏有七神,各何所藏耶?
\\\\
  然:藏者,人之神氣所舍藏也。故肝藏魂,肺藏魄,心藏神,脾藏意與智,腎藏精與志也。
\\\\
  三十五難曰:五藏各有所,府皆相近,而心肺獨去大腸、小腸遠者,何謂也?
\\\\
  經言心榮肺衛,通行陽氣,故居有上;大腸、小腸傳陰氣而下,故居在下。所以相去而遠也。
\\\\
  又諸府者,皆陽也,清淨之處。今大腸、小腸、胃與膀胱,皆受不淨,其意何也?
\\\\
  然:諸府者,謂是非也。經言:小腸者,受盛之府也;大腸者,傳瀉行道之府也;膽者,清淨之府也;胃者,水穀之府也;膀胱者,津液之府也。一府猶無兩名,故知非也。
\\\\
  小腸者,心之府;大腸者,肺之府;胃者,脾之府;膽者,肝之府;膀胱者,腎之府。小腸謂赤腸,大腸謂白腸,膽者謂青腸,胃者謂黃腸,膀胱者謂黑腸。下焦所治也。
\\\\
  三十六難曰:藏各有一耳,腎獨有兩者,何也?
\\\\
  然:腎兩者,非皆腎也。其左者為腎,右者為命門。命門者,諸神精之所舍,原氣之所繫也。故男子以藏精,女子以繫胞,故知腎有一也。
\\\\
  三十七難曰:五藏之氣,於何發起,通於何許,可曉以不?
\\\\
  然:五藏者,當上關於九竅也。故肺氣通於鼻,鼻和則知香臭矣;肝氣通於目,目和則知白黑矣;脾氣通於口,口和則知穀味矣;心氣通於舌,舌和則知五味矣;腎氣通於耳,耳和則知五音矣。五藏不和,則九竅不通;六府不和,則留結為癰。
\\\\
  邪在六府,則陽脈不和;陽脈不和,則氣留之;氣留之,則陽脈盛矣。邪在五藏,則陰脈不和;陰脈不和,則血留之;血留之,則陰脈盛矣。陰氣太盛,則陽氣不得相營也,故曰格。陽氣太盛,則陰氣不得相營也,故曰關。陰陽俱盛,不得相營也,故曰關格。關格者,不得盡其命而死矣。
\\\\
  經言氣獨行於五藏,不營於六府者,何也?
\\\\
  然:氣之所行也,如水之流,不得息也。故陰脈營於五藏,陽脈營於六府,如環之無端,莫知其紀,終而復始,其不覆溢,人氣內溫於藏府,外濡於腠理。

\part{藏府度數}
\LARGE
\section{藏府度數}

  六十九難曰:經言虛者補之,實者瀉之,不實不虛,以經取之,何謂也?
三十八難曰:藏唯有五,府獨有六者,何也?
\\\\
  六十九難曰:經言虛者補之,實者瀉之,不實不虛,以經取之,何謂也?
然:所以府有六者,謂三焦也。有原氣之別焉,主持諸氣,有名而無形,其經屬手少陽,此外府也,故言府有六焉。
\\\\
  六十九難曰:經言虛者補之,實者瀉之,不實不虛,以經取之,何謂也?
三十九難曰:經言府有五,藏有六者,何也?
\\\\
  六十九難曰:經言虛者補之,實者瀉之,不實不虛,以經取之,何謂也?
然:六府者,正有五府也。然五藏亦有六藏者,謂腎有兩藏也。其左為腎,右為命門。命門者,謂精神之所舍也;男子以藏精,女子以繫胞,其氣與腎通,故言藏有六也。
\\\\
  六十九難曰:經言虛者補之,實者瀉之,不實不虛,以經取之,何謂也?
府有五者,何也?
\\\\
  六十九難曰:經言虛者補之,實者瀉之,不實不虛,以經取之,何謂也?
然:五藏各一府,三焦亦是一府,然不屬於五藏,故言府有五焉。
\\\\
  六十九難曰:經言虛者補之,實者瀉之,不實不虛,以經取之,何謂也?
四十難曰:經言肝主色,心主臭,脾主味,肺主聲,腎主液。鼻者,肺之候,而反知香臭;耳者,腎之候,而反聞聲,其意何也?
\\\\
  六十九難曰:經言虛者補之,實者瀉之,不實不虛,以經取之,何謂也?
然:肺者,西方金也,金生於巳,巳者南方火也,火者心,心主臭,故令鼻知香臭;腎者,北方水也,水生於申,申者西方金,金者肺,肺主聲,故令耳聞聲。
\\\\
  六十九難曰:經言虛者補之,實者瀉之,不實不虛,以經取之,何謂也?
四十一難曰:肝獨有兩葉,以何應也?
\\\\
  六十九難曰:經言虛者補之,實者瀉之,不實不虛,以經取之,何謂也?
然:肝者,東方木也。木者,春也。萬物始生,其尚幼小,意無所親,去太陰尚近,離太陽不遠,猶有兩心,故有兩葉,亦應木葉也。
\\\\
  六十九難曰:經言虛者補之,實者瀉之,不實不虛,以經取之,何謂也?
四十二難曰:人腸胃長短,受水穀多少,各幾何?
\\\\
  六十九難曰:經言虛者補之,實者瀉之,不實不虛,以經取之,何謂也?
然:胃大一尺五寸,徑五寸,長二尺六寸,橫屈受水穀三斗五升,其中常留穀二斗,水一斗五升。小腸大二寸半,徑八分分之少半,長三丈二尺,受穀二斗四升,水六升合合之太半。廻腸大四寸,徑一寸半,長二丈一尺,受穀一斗,水七升半。廣腸大八寸,徑二寸半,長二尺八寸,受穀九升三合八分合之一。故腸胃凡長五丈八尺四寸,合受水穀八斗七升六合八分合之一,此腸胃長短,受水穀之數也。
\\\\
  六十九難曰:經言虛者補之,實者瀉之,不實不虛,以經取之,何謂也?
肝重四斤四兩,左三葉,右四葉,凡七葉,主藏魂。心重十二兩,中有七孔三毛,盛精汁三合,主藏神。脾重二斤三兩,扁廣三寸,長五寸,有散膏半斤,主裹血,溫五藏,主藏意。肺重三兩三兩,六葉兩耳,凡八葉,主藏魂。腎有兩枚,重一斤一兩,主藏志。
\\\\
  六十九難曰:經言虛者補之,實者瀉之,不實不虛,以經取之,何謂也?
膽在肝之短葉間,重三兩三銖,盛精汁三合。胃重二斤二兩,紆曲屈伸,長二尺六寸,大一尺五寸,徑五寸,盛穀二斗,水一斗五升。小腸重二斤十四兩,長三丈二尺,廣二寸半,徑八分分之少半,左廻疊積十六曲,盛穀二斗四升,水六升三合合之太半。大腸重二斤十二兩,長二丈一尺,廣四寸,徑一寸,當齊右廻十六曲,盛穀一斗,水七升半。膀胱重九兩二銖,縱廣九寸,盛溺九升九合。
\\\\
  六十九難曰:經言虛者補之,實者瀉之,不實不虛,以經取之,何謂也?
口廣二寸半,唇至齒長九分,齒以後至會厭,深三寸半,大容五合。舌重十兩,長七寸,廣二寸半。咽門重十兩,廣二寸半,至胃長一尺六寸。喉嚨重十二兩,廣二寸,長一尺二寸,九節。肛門重十二兩,大八寸,徑二寸大半,長二尺八寸,受穀九升三合八分合之一。
\\\\
  六十九難曰:經言虛者補之,實者瀉之,不實不虛,以經取之,何謂也?
四十三難曰:人不食飲,七日而死者,何也?
\\\\
  六十九難曰:經言虛者補之,實者瀉之,不實不虛,以經取之,何謂也?
然:人胃中常有留穀二斗,水一斗五升。故平人日再至圊,一行二升半,日中五升,七日五七三斗五升,而水穀盡矣。故平人不食飲七日而死者,水穀津液俱盡,即死矣。
\\\\
  六十九難曰:經言虛者補之,實者瀉之,不實不虛,以經取之,何謂也?
四十四難曰:七衝門何在?
\\\\
  六十九難曰:經言虛者補之,實者瀉之,不實不虛,以經取之,何謂也?
然:唇為飛門,齒為戶門,會厭為吸門,胃為賁門,太倉下口為幽門,太腸小腸會為闌門,下極為魄門,故曰七衝門也。
\\\\
  六十九難曰:經言虛者補之,實者瀉之,不實不虛,以經取之,何謂也?
四十五難曰:經言八會者,何也?
\\\\
  六十九難曰:經言虛者補之,實者瀉之,不實不虛,以經取之,何謂也?
然:府會大倉,藏會季脅,筋會陽陵泉,髓會絕骨,血會鬲俞,骨會大抒,脈太淵,氣會三焦外一筋直兩乳內也。熱病在內者,取其會之氣穴也。
\\\\
  六十九難曰:經言虛者補之,實者瀉之,不實不虛,以經取之,何謂也?
四十六難曰:老人臥而不寐,少壯寐而不寤者,何也?
\\\\
  六十九難曰:經言虛者補之,實者瀉之,不實不虛,以經取之,何謂也?
然:經言少壯者,血氣盛,肌肉滑,氣道通,榮衛之行不失於常,故晝日精,夜不寤。老人血氣衰,氣肉不滑,榮衛之道濇,故晝日不能精,夜不得寐也,故知老人不得寐也。
\\\\
  六十九難曰:經言虛者補之,實者瀉之,不實不虛,以經取之,何謂也?
四十七難曰:人面獨能耐寒者,何也?
\\\\
  六十九難曰:經言虛者補之,實者瀉之,不實不虛,以經取之,何謂也?
然:人頭者,諸陽之會也。諸陰脈皆至頸胸中而還,獨諸陽脈皆上至頭耳,故令面耐寒也。

\part{虛實邪正}
\LARGE
\section{虛實邪正}

  四十八難曰:人有三虛三實,何謂也?
\\\\
  然:有脈之虛實,有病之虛實,有診之虛實也。脈之虛實者,濡者為虛,緊牢者為實。病之虛實者,出者為虛,入者為實;言者為虛,不言者為實;緩者為虛,急者為實。診之虛實者,濡者為虛,牢者為實;癢者為虛,痛者為實;外痛內快,為外實內虛;內痛外快,為內實外虛。故曰虛實也。
\\\\
  四十九難曰:有正經自病,有五邪所傷,何以別之?
\\\\
  然:經言憂愁思慮則傷心;形寒飲冷則傷肺;恚怒氣逆,上而不下則傷肝;飲食勞倦則傷脾;久坐濕地,強力入水則傷腎。是正經之自病也。
\\\\
  何謂五邪?
\\\\
  然:有中風,有傷暑,有飲食勞倦,有傷寒,有中濕,此之謂五邪。
\\\\
  假令心病,何以知中風得之?
\\\\
  然:其色當赤。何以言之?肝主色,自入為青,入心為赤,入脾為黃,入肺為白,入腎為黑。肝為心邪,故知當赤色也。其病身熱,脅下滿痛,其脈浮大而絃。
\\\\
  何以知傷暑得之?
\\\\
  然:當惡臭。何以言之?心主臭,自入為焦臭,入脾為香臭,入肝為臊臭,入腎為腐臭,入肺為腥臭。故知心病傷暑得之也,當惡臭。其病身熱而煩,心痛,其脈浮大而散。
\\\\
  何以知飲食勞倦得之?
\\\\
  然:當喜苦味也。虛為不欲食,實為欲食。何以言之?脾主味,入肝為酸,入心為苦,入肺為辛,入腎為鹹,自入為甘。故知脾邪入心,為喜苦味也。其病身熱而體重嗜臥,四肢不收,其脈浮大而緩。
\\\\
  何以知傷寒得之?
\\\\
  然:當譫言妄語。何以言之?肺主聲,入肝為呼,入心為言,入脾為歌,入腎為呻,自入為哭,故知肺邪入心為譫言妄語也。其病身熱,洒洒惡寒,甚則喘咳,其脈浮大而濇。
\\\\
  何以知中濕得之?
\\\\
  然:當喜汗出不可止。何以言之?腎主濕,入肝為泣,入心為汗,入脾為液,入肺為涕,自入為唾。故知腎邪入心,為汗出不可止也。其病身熱而小腹痛,足脛寒而逆,其脈沉濡而大。此五邪之法也。
\\\\
  五十難曰:病有虛邪,有實邪,有賊邪,有微邪,有正邪,何以別之?
\\\\
  然:從後來者為虛邪,從前來者為實邪,從所不勝來者為賊邪,從所勝來者為微邪,自病者為正邪。何以言之?假令心病,中風得之為虛邪,傷暑得之為正邪,飲食勞倦得之為實邪,傷寒得之為微邪,中濕得之為賊邪。
\\\\
  五十一難曰:病有欲得溫者,有欲得寒者,有欲得見人者,有不欲得見人者,而各不同,病在何藏府也?
\\\\
  然:病欲得寒,而欲見人者,病在府也;病欲得溫,而不欲得見人者,病在藏也。何以言之?府者陽也,陽病欲得寒,又欲見人;藏者陰也,陰病欲得溫,又欲閉戶獨處,惡聞人聲,故以別知藏府之病也。
\\\\
  五十二難曰:府藏發病,根本等不?
\\\\
  然:不等也。
\\\\
  其不等奈何?
\\\\
  然:藏病者,止而不移,其病不離其處;府病者,彷彿賁嚮,上下行流,居處無常。故以此知藏府根本不同也。

\part{藏府傳病}
\LARGE
\section{藏府傳病}

  五十三難曰:經言七傳者死,間藏者生,何謂也?
\\\\
  然:七傳者,傳其所勝也。間藏者,傳其子也。何以言之?假令心病傳肺,肺傳肝,肝傳脾,脾傳腎,腎傳心,一藏不再傷,故言七傳者死也。間藏者,傳其所生也。假令心病傳脾,脾傳肺,肺傳腎,腎傳肝,肝傳心,是母子相傳,竟而復始,如環之無端,故言生也。
\\\\
  五十四難曰:藏病難治,府病易治,何謂也?
\\\\
  然:藏病所以難治者,傳其所勝也;府病易治者,傳其子也。與七傳間藏同法也。

\part{藏府積聚}
\LARGE
\section{藏府積聚}

  六十九難曰:經言虛者補之,實者瀉之,不實不虛,以經取之,何謂也?
五十五難曰:病有積、有聚,何以別之?
\\\\
  六十九難曰:經言虛者補之,實者瀉之,不實不虛,以經取之,何謂也?
然:積者,陰氣也;聚者,陽氣也,故陰沉而伏,陽浮而動。氣之所積名曰積,氣之所聚名曰聚。故積者,五藏所生;聚者,六府所成也。積者,陰氣也,其始發有常處,其痛不離其部,上下有所終始,左右有所窮處;聚者,陽氣也,其始發無根本,上下無所留止,其痛無常處,謂之聚。故以是別知積聚也。
\\\\
  六十九難曰:經言虛者補之,實者瀉之,不實不虛,以經取之,何謂也?
五十六難曰:五藏之積,各有名乎?以何月何日得之?
\\\\
  六十九難曰:經言虛者補之,實者瀉之,不實不虛,以經取之,何謂也?
然:肝之積名曰肥氣,在左脅下,如覆杯,有頭足。久不愈,令人發咳逆,痎瘧,連歲不已。以季夏戊己日得之。何以言之?肺病傳於肝,肝當傳脾,脾季夏適王,王者不受邪,肝復欲還肺,肺不肯受,故留結為積。故知肥氣以季夏戊己日得之。
\\\\
  六十九難曰:經言虛者補之,實者瀉之,不實不虛,以經取之,何謂也?
心之積名曰伏梁,起齊上,大如臂,上至心下。久不愈,令人病煩心,以秋庚辛日得之。何以言之?腎病傳心,心當傳肺,肺以秋適王,王者不受邪,心復欲還腎,腎不肯受,故留結為積。故知伏梁以秋庚辛日得之。
\\\\
  六十九難曰:經言虛者補之,實者瀉之,不實不虛,以經取之,何謂也?
脾之積名曰痞氣,在胃脘,覆大如盤。久不愈,令人四肢不收,發黃疸,飲食不為肌膚。以冬壬癸日得之。何以言之?肝病傳脾,脾當傳腎,腎以冬適王,王者不受邪,脾復欲還肝,肝不肯受,故留結為積。故知痞氣以冬壬癸日得之。
\\\\
  六十九難曰:經言虛者補之,實者瀉之,不實不虛,以經取之,何謂也?
肺之積名曰息賁,在右脅下,覆大如杯。久不已,令人洒淅寒熱,喘咳,發肺壅。以春甲乙日得之。何以言之?心病傳肺,肺當傳肝,肝以春適王,王者不受邪,肺復欲還心,心不肯受,故留結為積。故知息賁以春甲乙日得之。
\\\\
  六十九難曰:經言虛者補之,實者瀉之,不實不虛,以經取之,何謂也?
腎之積名曰賁豚,發於少腹,上至心下,若豚狀,或上或下無時,久不已,令人喘逆,骨痿少氣,以夏丙丁日得之。何以言之?脾病傳腎,腎當傳心,心以夏適王,王者不受邪,腎復欲還脾,脾不肯受,故留結為積。故知賁豚以夏丙丁日得之。此是五積之要法也。

\part{泄傷寒}
\LARGE
\section{泄傷寒}

  五十七難曰:泄凡有幾?皆有名不?
\\\\
  然:泄凡有五,其名不同。有胃泄,有脾泄,有大腸泄,有小腸泄,有大瘕泄,名曰後重。胃泄者,飲食不化,色黃。脾泄者,腹脹滿,泄注,食即嘔吐逆。大腸泄者,食已窘迫,大便色白,腸鳴切痛。小腸泄者,溲而便膿血,少腹痛。大瘕泄者,裏急後重,數至圊而不能便,莖中痛。此五泄之法也。
\\\\
  五十八難曰:傷寒有幾?其脈有變不?
\\\\
  然:傷寒有五,有中風,有傷寒,有濕溫,有熱病,有溫病,其所苦各不同。中風之脈,陽浮而滑,陰濡而弱;濕溫之脈,陽濡而弱,陰小而急;傷寒之脈,陰陽俱盛而緊濇;熱病之脈,陰陽俱浮,浮之滑,沉之散濇;溫病之脈,行在諸經,不知何經之動也,各隨其經所在而取之。
\\\\
  腸寒有汗出而愈,下之而死者;有汗出而死,下之而愈者,何也?
\\\\
  然:陽虛陰盛,汗出而愈,下之即死;陽盛陰虛,汗出而死,下之而愈。
\\\\
  寒熱之病,候之如何也?
\\\\
  然:皮寒熱者,皮不可近席,毛髮焦,鼻槀,不得汗;肌寒熱者,皮膚痛,唇舌槀,無汗;骨寒熱者,病無所安,汗注不休,齒本槀痛。
\\\\
  五十九難曰:狂癲之病,何以別之?
\\\\
  然:狂之始發,少臥而不饑,自高賢也,自辨智也,自貴倨也,妄笑好歌樂,妄行不休是也。癲疾始發,意不樂,直視僵仆。其脈三部陰陽俱盛是也。
\\\\
  六十難曰:頭心之病,有厥痛,有真痛,何謂也?
\\\\
  然:手三陽之脈,受風寒,伏留而不去者,則名厥頭痛;入連在腦者,名真頭痛。其五藏氣相干,名厥心痛;其痛甚,但在心,手足青者,即名真心痛。其真心痛者,旦發夕死,夕發旦死。

\part{神聖工巧}
\LARGE
\section{神聖工巧}

  六十九難曰:經言虛者補之,實者瀉之,不實不虛,以經取之,何謂也?
六十一難曰:經言望而知之謂之神,聞而知之謂之聖,問而知之謂之工,切脈而知之謂之巧,何謂也?
\\\\
  六十九難曰:經言虛者補之,實者瀉之,不實不虛,以經取之,何謂也?
然:望而知之者,望見其五色,以知其病。聞而知之者,聞其五音,以別其病。問而知之者,聞其所欲五味,以知其病所起所在也。切脈而知之者,診其寸口,視其虛實,以知其病,病在何藏府也。經言以外知之曰聖,以內知之曰神,此之謂也。

\part{藏府井俞}
\LARGE
\section{藏府井俞}

  六十九難曰:經言虛者補之,實者瀉之,不實不虛,以經取之,何謂也?
六十二難曰:藏井滎有五,府獨有六者,何謂也?
\\\\
  六十九難曰:經言虛者補之,實者瀉之,不實不虛,以經取之,何謂也?
然:府者,陽也。三焦行於諸陽,故置一俞,名曰原。府有六者,亦與三焦共一氣也。
\\\\
  六十九難曰:經言虛者補之,實者瀉之,不實不虛,以經取之,何謂也?
六十三難曰:《十變》言,五藏六府滎合,皆以井為始者,何也?
\\\\
  六十九難曰:經言虛者補之,實者瀉之,不實不虛,以經取之,何謂也?
然:井者,東方春也,萬物之始生。諸蚑行喘息,蜎飛蠕動,當生之物,莫不以春而生。故歲數始於春,日數始於甲,故以井為始也。
\\\\
  六十九難曰:經言虛者補之,實者瀉之,不實不虛,以經取之,何謂也?
六十四難曰:《十變》又言,陰井木,陽井金;陰滎火,陽滎水;陰俞土,陽俞木;陰經金,陽經火;陰合水,陽合土。陰陽皆不同,其意何也?
\\\\
  六十九難曰:經言虛者補之,實者瀉之,不實不虛,以經取之,何謂也?
然:是剛柔之事也。陰井乙木,陽井庚金。陽井庚,庚者,乙之剛也;陰井乙,乙者,庚之柔也。乙為木,故言陰井木也;庚為金,故言陽井金也。餘皆倣此。
\\\\
  六十九難曰:經言虛者補之,實者瀉之,不實不虛,以經取之,何謂也?
六十五難曰:經言所出為井,所入為合,其法奈何?
\\\\
  六十九難曰:經言虛者補之,實者瀉之,不實不虛,以經取之,何謂也?
然:所出為井,井者,東方春也,萬物之始生,故言所出為井也。所入為合,合者,北方冬也,陽氣入藏,故言所入為合也。
\\\\
  六十九難曰:經言虛者補之,實者瀉之,不實不虛,以經取之,何謂也?
六十六難曰:經言肺之原出于太淵,心之原出于太陵,肝之原出于太衝,脾之原出于太白,腎之原出于太谿,少陰之原出于兌骨,膽之原出于丘墟,胃之原出于衝陽,三焦之原出于陽池,膀胱之原出于京骨,大腸之原出于合谷,小腸之原出于腕骨。十二經皆以俞為原者,何也?
\\\\
  六十九難曰:經言虛者補之,實者瀉之,不實不虛,以經取之,何謂也?
然:五藏俞者,三焦之所行,氣之所留止也。
\\\\
  六十九難曰:經言虛者補之,實者瀉之,不實不虛,以經取之,何謂也?
焦所行之俞為原者,何也?
\\\\
  六十九難曰:經言虛者補之,實者瀉之,不實不虛,以經取之,何謂也?
然:臍下腎間動氣者,人之生命也,十二經之根本也,故名曰原。三焦者,原氣之別使也,主通行三氣,經歷於五藏六府。原者,三焦之尊號也,故所止輒為原。五藏六府之有病者,取其原也。
\\\\
  六十九難曰:經言虛者補之,實者瀉之,不實不虛,以經取之,何謂也?
六十七難曰:五藏募皆左陰,而俞在陽者,何謂也?
\\\\
  六十九難曰:經言虛者補之,實者瀉之,不實不虛,以經取之,何謂也?
然:陰病行陽,陽病行陰,故令募在陰,俞在陽。
\\\\
  六十九難曰:經言虛者補之,實者瀉之,不實不虛,以經取之,何謂也?
六十八難曰:五藏六府,各有井滎俞經合,皆何所主?
\\\\
  六十九難曰:經言虛者補之,實者瀉之,不實不虛,以經取之,何謂也?
然:經言所出為井,所流為滎,所注為俞,所行為經,所入為合。井主心下滿,滎主身熱,俞主體重節痛,經主喘咳寒熱,合主逆氣而泄。此五藏六府其井滎俞經合所主病也。

\part{用鍼補瀉}
\LARGE
\section{用鍼補瀉}

  六十九難曰:經言虛者補之,實者瀉之,不實不虛,以經取之,何謂也?
\\\\
  然:虛者補其母,實者瀉其子,當先補之,然後瀉之。不實不虛,以經取之者,是正經自生病,不中他邪也,當自取其經,故言以經取之。
\\\\
  七十難曰:經言春夏刺淺,秋冬刺深者,何謂也?
\\\\
  然:春夏者,陽氣在上,人氣亦在上,故當淺取之;秋冬者,陽氣在下,人氣亦在下,故當深取之。
\\\\
  春夏各致一陰,秋冬各致一陽者,何謂也?
\\\\
  然:春夏溫,必致一陰者,初下針,沉之至腎肝之部,得氣,引持之陰也;秋冬寒,必致一陽者,初內針,淺而浮之至心肺之部,得氣,推內之陽也。是謂春夏必致一陰,秋冬必致一陽。
\\\\
  七十一難曰:經言刺榮無傷衛,刺衛無傷榮,何謂也?
\\\\
  然:鍼陽者,臥鍼而刺之;刺陰者,先以左手攝按所鍼榮俞之處,氣散乃內針。是謂刺榮無傷衛,刺衛無傷榮也。
\\\\
  七十二難曰:經言能知迎隨之氣,可令調之;調氣之方,必在陰陽,何謂也?
\\\\
  然:所謂迎隨者,知榮衛之流行,經脈之往來也。隨其逆順而取之,故曰迎隨。調氣之方,必在陰陽者,知其內外表裏,隨其陰陽而調之,故曰調氣之方,必在陰陽。
\\\\
  七十三難曰:諸井者,肌肉淺薄,氣少,不足使也,刺之奈何?
\\\\
  然:諸井者,木也;滎者,火也。火者,木之子,當刺井者,以滎瀉之。故經言補者不可以為瀉,瀉者不可以為補,此之謂也。
\\\\
  七十四難曰:經言春刺井,夏刺滎,季夏刺俞,秋刺經,冬刺合者,何謂也?
\\\\
  然:春刺井者,邪在肝;夏刺滎者,邪在心;季夏刺俞者,邪在脾;秋刺經者,邪在肺;冬刺合者,邪在腎。
\\\\
  其肝、心、脾、肺、腎而繫於春、夏、秋、冬者,何也?
\\\\
  然:五藏一病,輒有五也。假令肝病,色青者肝也,臊臭者肝也,喜酸者肝也,喜呼者肝也,喜泣者肝也。其病眾多,不可盡言也。四時有數,而並繫於春夏秋冬者也。鍼之要妙,在於秋毫者。
\\\\
  七十五難曰:經言東方實,西方虛,瀉南方,補北方,何謂也?
\\\\
  然:金木水火土,當更相平。東方木也,西方金也。木欲實,金當平之;火欲實,水當平之;土欲實,木當平之;金欲實,火當平之;水欲實,土當平之。東方肝也,則知肝實;西方肺也,則知肺虛。瀉南方火,補北方水。南方火,火者,木之子也;北方水,水者,木之母也。水勝火,子能令母實,母能令子虛,故瀉火補水,欲令金不得平木也。經曰:不能治其虛,何問其餘。此之謂也。
\\\\
  七十六難曰:何謂補瀉?當補之時,何所取氣?當瀉之時,何所置氣?
\\\\
  然:當補之時,從衛取氣;當瀉之時,從榮置氣。其陽氣不足,陰氣有餘,當先補其陽,而後瀉其陰;陰氣不足,陽氣有餘,當先補其陰,而後瀉其陽。榮衛通行,此其要也。
\\\\
  七十七難曰:經言上工治未病,中工治已病者,何謂也?
\\\\
  然:所謂治未病者,見肝之病,則知肝當傳之與脾,故先實其脾氣,無令得受肝之邪,故曰治未病焉。中工治已病者,見肝之病,不曉相傳,但一心治肝,故曰治已病也。
\\\\
  七十八難曰:針有補瀉,何謂也?
\\\\
  然:補瀉之法,非必呼吸出內針也。
\\\\
  然:知為鍼者,信其左;不知為針者,信其右。當刺之時,必先以左手厭按所鍼滎俞之處,彈而努之,爪而下之,其氣之來,如動脈之狀,順鍼而刺之。得氣因推而內之,是謂補;動而伸之,是謂瀉。不得氣,乃與男外女內;不得氣,是謂十死不治也。
\\\\
  七十九難曰:經言迎而奪之,安得無虛?隨而濟之,安得無實?虛之與實,若得若失;實之與虛,若有若無,何謂也?
\\\\
  然:迎而奪之者,瀉其子也;隨而濟之者,補其母也。假令心病,瀉手心主俞,是謂迎而奪之者也。補手心主井,是謂隨而濟之者也。所謂實之與虛者,牢濡之意也。氣來實者為得,濡虛者為失,故曰若得若失也。
\\\\
  八十難曰:經言有見如入,有見如出者,何謂也?
\\\\
  然:所謂有見如入者,謂左手見氣來至,乃內針;針入見氣盡,乃出針。是謂有見如入,有見如出也。
\\\\
  八十一難曰:經言無實實虛虛,損不足而益有餘,是寸口脈耶?將病自有虛實耶?其損益柰何?
\\\\
  然:是病,非謂寸口脈也。謂病自有實虛也。假令肝實而肺虛,肝者木也,肺者金也,金木當更相平,當知金平木。假令肺實而肝虛,微少氣,用針不瀉其肝,而反重實其肺,故曰實實虛虛,損不足而益有餘,此者中工之所害也。

\end{document}
