\section{藏府配像}

  三十二難曰:五藏俱等,而心肺獨在膈上者,何也?
\\\\
  然:心者血,肺者氣,血為榮,氣為衛,相隨上下,謂之榮衛,通行經絡,營周於外,故令心肺在膈上也。
\\\\
  三十三難曰:肝青象木,肺白象金;肝得水而沉,木得水而浮;肺得水而浮,金得水而沉。其意何也?
\\\\
  然:肝者,非為純木也,乙角也,庚之柔。大言陰與陽,小言夫與婦。釋其微陽,而吸其微陰之氣,其意樂金,又行陰道多,故令肝得水而沉也。肺者,非為純金也,辛商也,丙之柔。大言陰與陽,小言夫與婦。釋其微陰,婚而就火,其意樂火,又行陽道多,故令肺得水而浮也。
\\\\
  肺熟而復沉,肝熟而復浮者,何也?故知辛當歸庚,乙當歸甲也。
\\\\
  三十四難曰:五藏各有聲、色、臭、味、,可曉知以不?
\\\\
  然:《十變》言,肝色青,其臭臊,其味酸,其聲呼,其液泣;心色赤,其臭焦,其味苦,其聲言,其液汗;脾色黃,其臭香,其味甘,其聲歌,其液涎;肺色白,其臭腥,其味辛,其聲哭,其液涕;腎色黑,其臭腐,其味鹹,其聲呻,其液唾。是五藏聲、色、臭、味、也。
\\\\
  藏有七神,各何所藏耶?
\\\\
  然:藏者,人之神氣所舍藏也。故肝藏魂,肺藏魄,心藏神,脾藏意與智,腎藏精與志也。
\\\\
  三十五難曰:五藏各有所,府皆相近,而心肺獨去大腸、小腸遠者,何謂也?
\\\\
  經言心榮肺衛,通行陽氣,故居有上;大腸、小腸傳陰氣而下,故居在下。所以相去而遠也。
\\\\
  又諸府者,皆陽也,清淨之處。今大腸、小腸、胃與膀胱,皆受不淨,其意何也?
\\\\
  然:諸府者,謂是非也。經言:小腸者,受盛之府也;大腸者,傳瀉行道之府也;膽者,清淨之府也;胃者,水穀之府也;膀胱者,津液之府也。一府猶無兩名,故知非也。
\\\\
  小腸者,心之府;大腸者,肺之府;胃者,脾之府;膽者,肝之府;膀胱者,腎之府。小腸謂赤腸,大腸謂白腸,膽者謂青腸,胃者謂黃腸,膀胱者謂黑腸。下焦所治也。
\\\\
  三十六難曰:藏各有一耳,腎獨有兩者,何也?
\\\\
  然:腎兩者,非皆腎也。其左者為腎,右者為命門。命門者,諸神精之所舍,原氣之所繫也。故男子以藏精,女子以繫胞,故知腎有一也。
\\\\
  三十七難曰:五藏之氣,於何發起,通於何許,可曉以不?
\\\\
  然:五藏者,當上關於九竅也。故肺氣通於鼻,鼻和則知香臭矣;肝氣通於目,目和則知白黑矣;脾氣通於口,口和則知穀味矣;心氣通於舌,舌和則知五味矣;腎氣通於耳,耳和則知五音矣。五藏不和,則九竅不通;六府不和,則留結為癰。
\\\\
  邪在六府,則陽脈不和;陽脈不和,則氣留之;氣留之,則陽脈盛矣。邪在五藏,則陰脈不和;陰脈不和,則血留之;血留之,則陰脈盛矣。陰氣太盛,則陽氣不得相營也,故曰格。陽氣太盛,則陰氣不得相營也,故曰關。陰陽俱盛,不得相營也,故曰關格。關格者,不得盡其命而死矣。
\\\\
  經言氣獨行於五藏,不營於六府者,何也?
\\\\
  然:氣之所行也,如水之流,不得息也。故陰脈營於五藏,陽脈營於六府,如環之無端,莫知其紀,終而復始,其不覆溢,人氣內溫於藏府,外濡於腠理。