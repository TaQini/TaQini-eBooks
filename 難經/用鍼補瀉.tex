\section{用鍼補瀉}

  六十九難曰:經言虛者補之,實者瀉之,不實不虛,以經取之,何謂也?
\\\\
  然:虛者補其母,實者瀉其子,當先補之,然後瀉之。不實不虛,以經取之者,是正經自生病,不中他邪也,當自取其經,故言以經取之。
\\\\
  七十難曰:經言春夏刺淺,秋冬刺深者,何謂也?
\\\\
  然:春夏者,陽氣在上,人氣亦在上,故當淺取之;秋冬者,陽氣在下,人氣亦在下,故當深取之。
\\\\
  春夏各致一陰,秋冬各致一陽者,何謂也?
\\\\
  然:春夏溫,必致一陰者,初下針,沉之至腎肝之部,得氣,引持之陰也;秋冬寒,必致一陽者,初內針,淺而浮之至心肺之部,得氣,推內之陽也。是謂春夏必致一陰,秋冬必致一陽。
\\\\
  七十一難曰:經言刺榮無傷衛,刺衛無傷榮,何謂也?
\\\\
  然:鍼陽者,臥鍼而刺之;刺陰者,先以左手攝按所鍼榮俞之處,氣散乃內針。是謂刺榮無傷衛,刺衛無傷榮也。
\\\\
  七十二難曰:經言能知迎隨之氣,可令調之;調氣之方,必在陰陽,何謂也?
\\\\
  然:所謂迎隨者,知榮衛之流行,經脈之往來也。隨其逆順而取之,故曰迎隨。調氣之方,必在陰陽者,知其內外表裏,隨其陰陽而調之,故曰調氣之方,必在陰陽。
\\\\
  七十三難曰:諸井者,肌肉淺薄,氣少,不足使也,刺之奈何?
\\\\
  然:諸井者,木也;滎者,火也。火者,木之子,當刺井者,以滎瀉之。故經言補者不可以為瀉,瀉者不可以為補,此之謂也。
\\\\
  七十四難曰:經言春刺井,夏刺滎,季夏刺俞,秋刺經,冬刺合者,何謂也?
\\\\
  然:春刺井者,邪在肝;夏刺滎者,邪在心;季夏刺俞者,邪在脾;秋刺經者,邪在肺;冬刺合者,邪在腎。
\\\\
  其肝、心、脾、肺、腎而繫於春、夏、秋、冬者,何也?
\\\\
  然:五藏一病,輒有五也。假令肝病,色青者肝也,臊臭者肝也,喜酸者肝也,喜呼者肝也,喜泣者肝也。其病眾多,不可盡言也。四時有數,而並繫於春夏秋冬者也。鍼之要妙,在於秋毫者。
\\\\
  七十五難曰:經言東方實,西方虛,瀉南方,補北方,何謂也?
\\\\
  然:金木水火土,當更相平。東方木也,西方金也。木欲實,金當平之;火欲實,水當平之;土欲實,木當平之;金欲實,火當平之;水欲實,土當平之。東方肝也,則知肝實;西方肺也,則知肺虛。瀉南方火,補北方水。南方火,火者,木之子也;北方水,水者,木之母也。水勝火,子能令母實,母能令子虛,故瀉火補水,欲令金不得平木也。經曰:不能治其虛,何問其餘。此之謂也。
\\\\
  七十六難曰:何謂補瀉?當補之時,何所取氣?當瀉之時,何所置氣?
\\\\
  然:當補之時,從衛取氣;當瀉之時,從榮置氣。其陽氣不足,陰氣有餘,當先補其陽,而後瀉其陰;陰氣不足,陽氣有餘,當先補其陰,而後瀉其陽。榮衛通行,此其要也。
\\\\
  七十七難曰:經言上工治未病,中工治已病者,何謂也?
\\\\
  然:所謂治未病者,見肝之病,則知肝當傳之與脾,故先實其脾氣,無令得受肝之邪,故曰治未病焉。中工治已病者,見肝之病,不曉相傳,但一心治肝,故曰治已病也。
\\\\
  七十八難曰:針有補瀉,何謂也?
\\\\
  然:補瀉之法,非必呼吸出內針也。
\\\\
  然:知為鍼者,信其左;不知為針者,信其右。當刺之時,必先以左手厭按所鍼滎俞之處,彈而努之,爪而下之,其氣之來,如動脈之狀,順鍼而刺之。得氣因推而內之,是謂補;動而伸之,是謂瀉。不得氣,乃與男外女內;不得氣,是謂十死不治也。
\\\\
  七十九難曰:經言迎而奪之,安得無虛?隨而濟之,安得無實?虛之與實,若得若失;實之與虛,若有若無,何謂也?
\\\\
  然:迎而奪之者,瀉其子也;隨而濟之者,補其母也。假令心病,瀉手心主俞,是謂迎而奪之者也。補手心主井,是謂隨而濟之者也。所謂實之與虛者,牢濡之意也。氣來實者為得,濡虛者為失,故曰若得若失也。
\\\\
  八十難曰:經言有見如入,有見如出者,何謂也?
\\\\
  然:所謂有見如入者,謂左手見氣來至,乃內針;針入見氣盡,乃出針。是謂有見如入,有見如出也。
\\\\
  八十一難曰:經言無實實虛虛,損不足而益有餘,是寸口脈耶?將病自有虛實耶?其損益柰何?
\\\\
  然:是病,非謂寸口脈也。謂病自有實虛也。假令肝實而肺虛,肝者木也,肺者金也,金木當更相平,當知金平木。假令肺實而肝虛,微少氣,用針不瀉其肝,而反重實其肺,故曰實實虛虛,損不足而益有餘,此者中工之所害也。