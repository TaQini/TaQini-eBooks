\section{奇經八脈}

  二十七難曰:脈有奇經八脈者,不扚於十二經,何謂也?
\\\\
  然:有陽維,有陰維,有陽蹻,有陰蹻,有衝,有督,有任,有帶之脈。凡此八脈者,皆不拘於經,故曰奇經八脈也。
\\\\
  經有十二,絡有十五,凡二十七氣,相隨上下,何獨不拘於經也?
\\\\
  然:聖人圖設溝渠,通利水道,以備不然。天雨降下,溝渠溢滿,當此之時,霶霈妄行,聖人不能復圖也。此絡脈滿溢,諸經不能復拘也。
\\\\
  二十八難曰:其奇經八脈者,既不拘於十二經,皆何起何繼也?
\\\\
  然:督脈者,起於下極之俞,並於脊裏,上至風府,入於腦。
\\\\
  任脈者,起於中極之下,以上毛際,循腹裏,上關元,至喉咽。
\\\\
  衝脈者,起於氣衝,並足陽明之經,夾齊上行,至胸中而散也。
\\\\
  帶脈者,起於季脅,廻身一周。
\\\\
  陽蹻脈者,起於跟中,循外踝上行,入風池。
\\\\
  陰蹻脈者,亦起於跟中,循內踝上行,至咽喉,交貫衝脈。
\\\\
  陽維、陰維者,維絡于身,溢畜不能環流灌溉諸經者也。故陽維起於諸陽會也,陰維起於諸陰交也。
\\\\
  比于聖人圖設溝渠,溝渠滿溢,流于深湖,故聖人不能拘通也。而人脈隆聖,入於八脈而不環周,故十二經亦不能拘之。其受邪氣,畜則腫熱,砭射之也。
\\\\
  二十九難曰:奇經之為病何如?
\\\\
  然:陽維維於陽,陰維維於陰,陰陽不能自相維,則悵然失志,溶溶不能自收持。陰蹻為病,陽緩而陰急。陽蹻為病,陰緩而陽急。衝之為病,逆氣而裏急。督之為病,脊強而厥。任之為病,其內苦結,男子為七疝,女子為瘕聚。帶之為病,腹滿,腰溶溶若坐水中。陽維為病苦寒熱,陰維為病苦心痛。此奇經八脈之為病也。