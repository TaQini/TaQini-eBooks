\section{藏府度數}

  六十九難曰:經言虛者補之,實者瀉之,不實不虛,以經取之,何謂也?
三十八難曰:藏唯有五,府獨有六者,何也?
\\\\
  六十九難曰:經言虛者補之,實者瀉之,不實不虛,以經取之,何謂也?
然:所以府有六者,謂三焦也。有原氣之別焉,主持諸氣,有名而無形,其經屬手少陽,此外府也,故言府有六焉。
\\\\
  六十九難曰:經言虛者補之,實者瀉之,不實不虛,以經取之,何謂也?
三十九難曰:經言府有五,藏有六者,何也?
\\\\
  六十九難曰:經言虛者補之,實者瀉之,不實不虛,以經取之,何謂也?
然:六府者,正有五府也。然五藏亦有六藏者,謂腎有兩藏也。其左為腎,右為命門。命門者,謂精神之所舍也;男子以藏精,女子以繫胞,其氣與腎通,故言藏有六也。
\\\\
  六十九難曰:經言虛者補之,實者瀉之,不實不虛,以經取之,何謂也?
府有五者,何也?
\\\\
  六十九難曰:經言虛者補之,實者瀉之,不實不虛,以經取之,何謂也?
然:五藏各一府,三焦亦是一府,然不屬於五藏,故言府有五焉。
\\\\
  六十九難曰:經言虛者補之,實者瀉之,不實不虛,以經取之,何謂也?
四十難曰:經言肝主色,心主臭,脾主味,肺主聲,腎主液。鼻者,肺之候,而反知香臭;耳者,腎之候,而反聞聲,其意何也?
\\\\
  六十九難曰:經言虛者補之,實者瀉之,不實不虛,以經取之,何謂也?
然:肺者,西方金也,金生於巳,巳者南方火也,火者心,心主臭,故令鼻知香臭;腎者,北方水也,水生於申,申者西方金,金者肺,肺主聲,故令耳聞聲。
\\\\
  六十九難曰:經言虛者補之,實者瀉之,不實不虛,以經取之,何謂也?
四十一難曰:肝獨有兩葉,以何應也?
\\\\
  六十九難曰:經言虛者補之,實者瀉之,不實不虛,以經取之,何謂也?
然:肝者,東方木也。木者,春也。萬物始生,其尚幼小,意無所親,去太陰尚近,離太陽不遠,猶有兩心,故有兩葉,亦應木葉也。
\\\\
  六十九難曰:經言虛者補之,實者瀉之,不實不虛,以經取之,何謂也?
四十二難曰:人腸胃長短,受水穀多少,各幾何?
\\\\
  六十九難曰:經言虛者補之,實者瀉之,不實不虛,以經取之,何謂也?
然:胃大一尺五寸,徑五寸,長二尺六寸,橫屈受水穀三斗五升,其中常留穀二斗,水一斗五升。小腸大二寸半,徑八分分之少半,長三丈二尺,受穀二斗四升,水六升合合之太半。廻腸大四寸,徑一寸半,長二丈一尺,受穀一斗,水七升半。廣腸大八寸,徑二寸半,長二尺八寸,受穀九升三合八分合之一。故腸胃凡長五丈八尺四寸,合受水穀八斗七升六合八分合之一,此腸胃長短,受水穀之數也。
\\\\
  六十九難曰:經言虛者補之,實者瀉之,不實不虛,以經取之,何謂也?
肝重四斤四兩,左三葉,右四葉,凡七葉,主藏魂。心重十二兩,中有七孔三毛,盛精汁三合,主藏神。脾重二斤三兩,扁廣三寸,長五寸,有散膏半斤,主裹血,溫五藏,主藏意。肺重三兩三兩,六葉兩耳,凡八葉,主藏魂。腎有兩枚,重一斤一兩,主藏志。
\\\\
  六十九難曰:經言虛者補之,實者瀉之,不實不虛,以經取之,何謂也?
膽在肝之短葉間,重三兩三銖,盛精汁三合。胃重二斤二兩,紆曲屈伸,長二尺六寸,大一尺五寸,徑五寸,盛穀二斗,水一斗五升。小腸重二斤十四兩,長三丈二尺,廣二寸半,徑八分分之少半,左廻疊積十六曲,盛穀二斗四升,水六升三合合之太半。大腸重二斤十二兩,長二丈一尺,廣四寸,徑一寸,當齊右廻十六曲,盛穀一斗,水七升半。膀胱重九兩二銖,縱廣九寸,盛溺九升九合。
\\\\
  六十九難曰:經言虛者補之,實者瀉之,不實不虛,以經取之,何謂也?
口廣二寸半,唇至齒長九分,齒以後至會厭,深三寸半,大容五合。舌重十兩,長七寸,廣二寸半。咽門重十兩,廣二寸半,至胃長一尺六寸。喉嚨重十二兩,廣二寸,長一尺二寸,九節。肛門重十二兩,大八寸,徑二寸大半,長二尺八寸,受穀九升三合八分合之一。
\\\\
  六十九難曰:經言虛者補之,實者瀉之,不實不虛,以經取之,何謂也?
四十三難曰:人不食飲,七日而死者,何也?
\\\\
  六十九難曰:經言虛者補之,實者瀉之,不實不虛,以經取之,何謂也?
然:人胃中常有留穀二斗,水一斗五升。故平人日再至圊,一行二升半,日中五升,七日五七三斗五升,而水穀盡矣。故平人不食飲七日而死者,水穀津液俱盡,即死矣。
\\\\
  六十九難曰:經言虛者補之,實者瀉之,不實不虛,以經取之,何謂也?
四十四難曰:七衝門何在?
\\\\
  六十九難曰:經言虛者補之,實者瀉之,不實不虛,以經取之,何謂也?
然:唇為飛門,齒為戶門,會厭為吸門,胃為賁門,太倉下口為幽門,太腸小腸會為闌門,下極為魄門,故曰七衝門也。
\\\\
  六十九難曰:經言虛者補之,實者瀉之,不實不虛,以經取之,何謂也?
四十五難曰:經言八會者,何也?
\\\\
  六十九難曰:經言虛者補之,實者瀉之,不實不虛,以經取之,何謂也?
然:府會大倉,藏會季脅,筋會陽陵泉,髓會絕骨,血會鬲俞,骨會大抒,脈太淵,氣會三焦外一筋直兩乳內也。熱病在內者,取其會之氣穴也。
\\\\
  六十九難曰:經言虛者補之,實者瀉之,不實不虛,以經取之,何謂也?
四十六難曰:老人臥而不寐,少壯寐而不寤者,何也?
\\\\
  六十九難曰:經言虛者補之,實者瀉之,不實不虛,以經取之,何謂也?
然:經言少壯者,血氣盛,肌肉滑,氣道通,榮衛之行不失於常,故晝日精,夜不寤。老人血氣衰,氣肉不滑,榮衛之道濇,故晝日不能精,夜不得寐也,故知老人不得寐也。
\\\\
  六十九難曰:經言虛者補之,實者瀉之,不實不虛,以經取之,何謂也?
四十七難曰:人面獨能耐寒者,何也?
\\\\
  六十九難曰:經言虛者補之,實者瀉之,不實不虛,以經取之,何謂也?
然:人頭者,諸陽之會也。諸陰脈皆至頸胸中而還,獨諸陽脈皆上至頭耳,故令面耐寒也。