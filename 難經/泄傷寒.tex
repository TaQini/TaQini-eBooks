\section{泄傷寒}

  五十七難曰:泄凡有幾?皆有名不?
\\\\
  然:泄凡有五,其名不同。有胃泄,有脾泄,有大腸泄,有小腸泄,有大瘕泄,名曰後重。胃泄者,飲食不化,色黃。脾泄者,腹脹滿,泄注,食即嘔吐逆。大腸泄者,食已窘迫,大便色白,腸鳴切痛。小腸泄者,溲而便膿血,少腹痛。大瘕泄者,裏急後重,數至圊而不能便,莖中痛。此五泄之法也。
\\\\
  五十八難曰:傷寒有幾?其脈有變不?
\\\\
  然:傷寒有五,有中風,有傷寒,有濕溫,有熱病,有溫病,其所苦各不同。中風之脈,陽浮而滑,陰濡而弱;濕溫之脈,陽濡而弱,陰小而急;傷寒之脈,陰陽俱盛而緊濇;熱病之脈,陰陽俱浮,浮之滑,沉之散濇;溫病之脈,行在諸經,不知何經之動也,各隨其經所在而取之。
\\\\
  腸寒有汗出而愈,下之而死者;有汗出而死,下之而愈者,何也?
\\\\
  然:陽虛陰盛,汗出而愈,下之即死;陽盛陰虛,汗出而死,下之而愈。
\\\\
  寒熱之病,候之如何也?
\\\\
  然:皮寒熱者,皮不可近席,毛髮焦,鼻槀,不得汗;肌寒熱者,皮膚痛,唇舌槀,無汗;骨寒熱者,病無所安,汗注不休,齒本槀痛。
\\\\
  五十九難曰:狂癲之病,何以別之?
\\\\
  然:狂之始發,少臥而不饑,自高賢也,自辨智也,自貴倨也,妄笑好歌樂,妄行不休是也。癲疾始發,意不樂,直視僵仆。其脈三部陰陽俱盛是也。
\\\\
  六十難曰:頭心之病,有厥痛,有真痛,何謂也?
\\\\
  然:手三陽之脈,受風寒,伏留而不去者,則名厥頭痛;入連在腦者,名真頭痛。其五藏氣相干,名厥心痛;其痛甚,但在心,手足青者,即名真心痛。其真心痛者,旦發夕死,夕發旦死。