\section{經脈診候}

  一難曰:十二經皆有動脈,獨取寸口,以決五藏六府死生吉凶之法,何謂也?
\\\\
  然:寸口者,脈之大會,手太陰之脈動也。人一呼脈行三寸,一吸脈行三寸,呼吸定息,脈行六寸。人一日一夜,凡一萬三千五百息,脈行五十度,周於身。漏水下百刻,榮衛行陽二十五度,行陰亦二十五度,為一周也,故五十度復會於手太陰。寸口者,五藏六府之所終始,故法取於寸口也。
\\\\
  二難曰:脈有尺寸,何謂也?
\\\\
  然:尺寸者,脈之大要會也。從關至尺是尺內,陰之所治也;從關至魚際是寸內,陽之所治也。故分寸為尺,分尺為寸。故陰得尺內一寸,陽得寸內九分,尺寸終始一寸九分,故曰尺寸也。
\\\\
  三難曰:脈有大過,有不及,有陰陽相乘,有覆有溢,有關有格,何謂也?
\\\\
  然:關之前者,陽之動,脈當見九分而浮。過者,法曰大過;減者,法曰不及。遂上魚為溢,為外關內格,此陰乘之脈也。
\\\\
  關以後者,陰之動也,脈當見一寸而沉。過者,法曰大過;減者,法曰不及。遂入尺為覆,為內關外格,此陽乘之脈也。故曰覆溢,是其真藏之脈,人不病而死也。
\\\\
  四難曰:脈有陰陽之法,何謂也?
\\\\
  然:呼出心與肺,吸入腎與肝,呼吸之間,脾受穀味也,其脈在中。浮者陽也,沉者陰也,故曰陰陽也。
\\\\
  心肺俱浮,何以別之?
\\\\
  然:浮而大散者,心也;浮而短濇者,肺也。
\\\\
  腎、肝俱沉,何以別之?
\\\\
  然:牢而長者,肝也;按之濡,舉指來實者,腎也。脾者中州,故其脈在中,是陰陽之法也。
\\\\
  脈有一陰一陽,一陰二陽,一陰三陽;有一陽一陰,一陽二陰,一陽三陰。如此之言,寸口有六脈俱動耶?
\\\\
  然:此言者,非有六脈俱動也,謂浮、沉、長、短、滑、濇也。浮者陽也,滑者陽也,長者陽也;沉者陰也,短者陰也,濇者陰也。所謂一陰一陽者,謂脈來沉而滑也;一陰二陽者,謂脈來沉滑而長也;一陰三陽者,謂脈來沉滑而長,時一沉也。所言一陽一陰者,謂脈來浮而濇也;一陽二陰者,謂脈來長而沉濇也;一陽三陰者,謂脈來沉濇而短,時一浮也。各以其經所在,名病逆順也。
\\\\
  五難曰:脈有輕重,何謂也?
\\\\
  然:初持脈,如三菽之重,與皮毛相得者,肺部也。如六菽之重,與血脈相得者,心部也。如九菽之重,與肌肉相得者,脾部也。如十二菽之重,與筋平者,肝部也。按之至骨,舉指來疾者,腎也。故曰輕重也。
\\\\
  六難曰:脈有陰盛陽虛,陽盛陰虛,何謂也?
\\\\
  然:浮之損小,沉之實大,故曰陰盛陽虛。沉之損小,浮之實大,故曰陽盛陰虛,是陰陽虛實意也。
\\\\
  七難曰:經言少陽之至,乍小乍大,乍短乍長;陽明之至,浮大而短;太陽之至,洪大而長;太陰之至,緊大而長;少陰之至,緊細而微;厥陰之至,沉短而敦。此六者,是平脈邪?將病脈邪?
\\\\
  然:皆王脈也。
\\\\
  其氣以何月,各王幾日?
\\\\
  然:冬至之後,得甲子少陽王,復得甲子陽明王,復得甲子太陽王,復得甲子太陰王,復得甲子少陰王,復得甲子厥陰王。王各六十日,六六三百六十日,以成一歲。此三陽三陰之王時日大要也。
\\\\
  八難曰:寸口脈平而死者,何謂也?
\\\\
  然:諸十二經脈者,皆係於生氣之原。所謂生氣之原者,謂十二經之根本也,謂腎間動氣也。此五藏六府之本,十二經脈之根,呼吸之門,三焦之原。一名守邪之神。故氣者,人之根本也,根絕則莖葉枯矣。寸口脈平而死者,生氣獨絕於內也。
\\\\
  九難曰:何以別知藏府之病耶?
\\\\
  然:數者府也,遲者藏也。數則為熱,遲則為寒。諸陽為熱,諸陰為寒。故以別知藏府之病也。
\\\\
  十難曰:一脈為十變者,何謂也?
\\\\
  然:五邪剛柔相逢之意也。假令心脈急甚者,肝邪干心也;心脈微急者,膽邪干小腸也;心脈大甚者,心邪自干心也;心脈微大者,小腸邪自干小腸也;心脈緩甚者,脾邪干心也;心脈微緩者,胃邪干小腸也;心脈濇甚者,肺邪干心也;心脈微濇者,大腸邪干小腸也;心脈沉甚者,腎邪干心也;心脈微沉者,膀胱邪干小腸也。五藏各有剛柔邪,故令一脈輒變為十也。
\\\\
  十一難曰:經言脈不滿五十動而一止,一藏無氣者,何藏也?
\\\\
  然:人吸者隨陰入,呼者因陽出。今吸不能至腎,至肝而還,故知一藏無氣者,腎氣先盡也。
\\\\
  十二難曰:經言五藏脈已絕於內,用鍼者反實其外;五藏脈已絕於外,用鍼者反實其內。內外之絕,何以別之?
\\\\
  然:五藏脈已絕於內者,腎肝氣已絕於內也,而醫反補其心肺;五藏脈已絕於外者,其心肺脈已絕於外也,而醫反補其腎肝。陽絕補陰,陰絕補陽,是謂實實虛虛,損不足益有餘。如此死者,醫殺之耳。
\\\\
  十三難曰:經言見其色而不得其脈,反得相勝之脈者即死,得相生之脈者,病即自已。色之與脈當參相應,為之奈何?
\\\\
  然:五藏有五色,皆見於面,亦當與寸口、尺內相應。假令色青,其脈當弦而急;色赤,其脈浮大而散;色黃,其脈中緩而大;色白,其脈浮濇而短;色黑,其脈沉濇而滑。此所謂五色之與脈,當參相應也。
\\\\
  脈數,尺之皮膚亦數;脈急,尺之皮膚亦急;脈緩,尺之皮膚亦緩;脈濇,尺之皮膚亦濇;脈滑,尺之皮膚亦滑。
\\\\
  藏各有聲、色、臭、味,當與寸口、尺內相應,其不相應者病也。假令色青,其脈浮濇而短,若大而緩為相勝;浮大而散,若小而滑為相生也。經言知一為下工,知二為中工,知三為上工。上工者十全九,中工者十全八,下工者十全六,此之謂也。
\\\\
  十四難曰:脈有損至,何謂也?
\\\\
  然:至之脈,一呼再至曰平,三至曰離經,四至曰奪精,五至曰死,六至曰命絕,此死之脈。何謂損?一呼一至曰離經,二呼一至曰奪精,三呼一至曰死,四呼一至曰命絕。此謂損之脈也。至脈從下上,損脈從上下也。
\\\\
  損脈之為病柰何?
\\\\
  然:一損損於皮毛,皮聚而毛落;二損損於血脈,血脈虛少,不能榮於五藏六府也;三損損於肌肉,肌肉消瘦,飲食不為肌膚;四損損於筋,筋緩不能自收持;五損損於骨,骨痿不能起於床。反此者,至於收病也。從上下者,骨痿不能起於床者死;從下上者,皮聚而毛落者死。
\\\\
  治損之法奈何?
\\\\
  然:損其肺者,益其氣;損其心者,調其榮衛;損其脾者,調其飲食,適寒溫;損其肝者,緩其中;損其腎者,益其精,此治損之法也。
\\\\
  脈有一呼再至,一吸再至;有一呼三至,一吸三至;有一呼四至,一吸四至;有一呼五至,一吸五至;有一呼六至,一吸六至;有一呼一至,一吸一至;有再呼一至,再吸一至;有呼吸再至。脈來如此,何以別知其病也?
\\\\
  然:脈來一呼再至,一吸再至,不大不小曰平。一呼三至,一吸三至,為適得病,前大後小,即頭痛目眩,前小後大,即胸滿短氣。一呼四至,一吸四至,病欲甚,脈洪大者,苦煩滿,沉細者,胸中痛,滑者傷熱,濇者中霧露。一呼五至,一吸五至,其人當困,沉細夜加,浮大晝加,不大不小,雖困可治,其有大小者為難治。一呼六至,一吸六至,為死脈也,沉細夜死,浮大晝死。一呼一至,一吸一至,名曰損,人雖能行,猶當著床,所以然者,血氣皆不足故也。再呼一至,呼吸再至,名曰無魂,無魂者當死也,人雖能行,名曰行尸。
\\\\
  上部有脈,下部無脈,其人當吐,不吐者死。上部無脈,下部有脈,雖困無能為害也。所以然者,譬如人之有尺,樹之有根,枝葉雖枯槁,根本將自生。脈有根本,人有元氣,故知不死。
\\\\
  十五難曰:經言春脈弦,夏脈鉤,秋脈毛,冬脈石,是王脈耶?將病脈也?
\\\\
  然:弦、鉤、毛、石者,四時之脈也。春脈弦者,肝,東方木也,萬物始生,未有枝葉。故其脈之來,濡弱而長,故曰弦。夏脈鉤者,心,南方火也,萬物之所盛,垂枝布葉,皆下曲如鉤。故其脈之,來疾去遲,故曰鉤。秋脈毛者,肺,西方金也,萬物之所終,草木華葉,皆秋而落,其枝獨在,若毫毛也。故其脈之來,輕虛以浮,故曰毛。冬脈石者,腎,北方水也,萬物之所藏也,盛冬之時,水凝如石。故其脈之來,沉濡而滑,故曰石。此四時之脈也。
\\\\
  如有變柰何?
\\\\
  然:春脈弦,反者為病。何謂反?
\\\\
  然:其氣來實強,是謂太過,病在外;氣來虛微,是謂不及,病在內。氣來厭厭聶聶,如循榆葉曰平;益實而滑,如循長竿曰病;急而勁益強,如新張弓弦曰死。春脈微弦曰平,弦多胃氣少曰病,但弦無胃氣曰死,春以胃氣為本。
\\\\
  夏脈鉤,反者為病。何謂反?
\\\\
  然:其氣來實強,是謂太過,病在外;氣來虛微,是謂不及,病在內。其脈來累累如環,如循琅玕曰平;來而益數,如雞舉足者曰病;前曲後居,如操帶鉤曰死。夏脈微鉤曰平,鉤多胃氣少曰病,但鉤無胃氣曰死,夏以胃氣為本。
\\\\
  秋脈微毛,反者為病。何謂反?
\\\\
  然:氣來實強,是謂太過,病在外;氣來虛微,是謂不及,病在內。其脈來藹藹如車蓋,按之益大曰平;不上不下,如循雞羽曰病;按之消索,如風吹毛曰死。秋脈微毛為平,毛多胃氣少曰病,但毛無胃氣曰死,秋以胃氣為本。
\\\\
  冬脈石,反者為病。何謂反?
\\\\
  然:其氣來實強,是謂太過,病在外;氣來虛微,是謂不及,病在內。脈來上大下兌,濡滑如雀之啄曰平;啄啄連屬,其中微曲曰病;來如解索,去如彈石曰死。冬脈微石曰平,石多胃氣少曰病,但石無胃氣曰死,冬以胃氣為本。
\\\\
  胃者,水穀之海也,主稟四時。故皆以胃氣為本,是謂四時之變病,死生之要會也。脾者,中州也,其平和不可得見,衰乃見耳。來如雀之,如水之下漏,是脾之衰見也。
\\\\
  十六難曰:脈有三部九候,有陰陽,有輕重,有六十首,一脈變為四時,離聖久遠,各自是其法,何以別之?
\\\\
  然:是其病,有內外證。
\\\\
  其病為之柰何?
\\\\
  然:假令得肝脈,其外證:善潔,面青,善怒;其內證:齊左有動氣,按之牢若痛;其病:四肢滿,閉癃,溲便難,轉節。有是者肝也,無是者非也。
\\\\
  假令得心脈,其外證:面赤,口乾,喜笑;其內證:齊上有動氣,按之牢若痛;其病:煩心,心痛,掌中熱而啘。有是者心也,無是者非也。
\\\\
  假令得脾脈,其外證:面黃,善噫,善思,善味;其內證:當齊有動氣,按之牢若痛;其病:腹脹滿,食不消,體重節痛,怠墮嗜臥,四肢不收。有是者脾也,無是者非也。
\\\\
  假令得肺脈,其外證:面白,善嚏,悲愁不樂,欲哭;其內證:齊右有動氣,按之牢若痛;其病:喘欬,洒淅寒熱。有是者肺也,無是者非也。
\\\\
  假令得腎脈,其外證:面黑,喜恐欠;其內證:齊下有動氣,按之牢若痛;其病:逆氣,少腹急痛,泄如下重,足脛寒而逆。有是者腎也,無是者非也。
\\\\
  十七難曰:經言病或有死,或有不治自愈,或連年月不已。其死生存亡,可切脈而知之耶?
\\\\
  然:可盡知也。診病若閉目不欲見人者,脈當得肝脈強急而長,而反得肺脈浮短而濇者,死也。
\\\\
  病若開目而渴,心下牢者,脈當得緊實而數,反得沉濡而微者,死也。
\\\\
  病若吐血,復鼽衂血者,脈當沉細,而反浮大而牢者,死也。
\\\\
  病若譫言妄語,身當有熱,脈當洪大,而手足厥逆,脈沉細而微者,死也。
\\\\
  病若大腹而洩者,脈當微細而濇,反緊大而滑者,死也。
\\\\
  十八難曰:脈有三部,部有四經,手有太陰、陽明,足有太陽、少陰,為上下部,何謂也?
\\\\
  然:手太陰,陽明金也;足少陰,太陽水也。金生水,水流下行而不能上,故在下部也。足厥陰,少陽木也,生手太陽、少陰火,火炎上行而不能下,故為上部。手心主,少陽火,生足太陰、陽明土,土主中宮,故在中部也。此皆五行子母更相生養者也。
\\\\
  脈有三部九候,各何所主之?
\\\\
  然:三部者,寸、關、尺也。九候者,浮、中、沉也。上部法天,主胸以上至頭之有疾也;中部法人,主膈以下至齊之有疾也;下部法地,主齊以下至足之有疾也。審而刺之者也。
\\\\
  人病有沉滯久積聚,可切脈而知之耶?
\\\\
  然:診在右脅有積氣,得肺脈結,脈結甚則積甚,結微則氣微。
\\\\
  診不得肺脈,而右脅有積氣者,何也?
\\\\
  然:肺脈雖不見,右手脈當沉伏。
\\\\
  其外痼疾同法耶?將異也?
\\\\
  然:結者,脈來去時一止,無常數,名曰結也。伏者,脈行筋下也。浮者,脈在肉上行也。左右表裏,法皆如此。假令脈結伏者,內無積聚;脈浮結者,外無痼疾;有積聚脈不結伏,有痼疾脈不浮結。為脈不應病,病不應脈,是為死病也。
\\\\
  十九難曰:經言脈有逆順,男女有常,而反者,何謂也?
\\\\
  然:男子生於寅,寅為木,陽也;女子生於申,申為金,陰也。故男脈在關上,女脈在關下。是以男子尺脈恆弱,女子尺脈恆盛,是其常也。反者,男得女脈,女得男脈也。
\\\\
  其為病何如?
\\\\
  然:男得女脈為不足,病在內;左得之,病則在左;右得之,病則在右,隨脈言之也。女得男脈為太過,病在四肢;左得之,病則在左;右得之,病則在右,隨脈言之,此之謂也。
\\\\
  二十難曰:經言脈有伏匿,伏匿於何藏而言伏匿耶?
\\\\
  然:謂陰陽更相乘,更相伏也。脈居陰部而反陽脈見者,為陽乘陰也;脈雖時沉濇而短,此謂陽中伏陰也;脈居陽部而反陰脈見者,為陰乘陽也;脈雖時浮滑而長,此謂陰中伏陽也。重陽者狂,重陰者癲。脫陽者見鬼,脫陰者目盲。
\\\\
  二十一難曰:經言人形病、脈不病曰生,脈病、形不病曰死,何謂也?
\\\\
  然:人形病,脈不病,非有不病者也,謂息數不應脈數也。此大法。
\\\\
  二十二難曰:經言脈有是動,有所生病。一脈輒變為二病者,何也?
\\\\
  然:經言是動者,氣也;所生病者,血也。邪在氣,氣為是動;邪在血,血為所生病。氣主呴之,血主濡之。氣留而不行者,為氣先病也;血壅而不濡者,為血後病也。故先為是動,後所生病也。
\\\\
  二十三難曰:手足三陰三陽,脈之度數,可曉以不?
\\\\
  然:手三陽之脈,從手至頭,長五尺,五六合三丈。
\\\\
  手三陰之脈,從手至胸中,長三尺五寸,三六一丈八尺,五六三尺,合二丈一尺。
\\\\
  足三陽之脈,從足至頭,長八尺,六八四丈八尺。
\\\\
  足三陰之脈,從足至胸,長六尺五寸,六六三丈六尺,五六三尺,合三丈九尺。
\\\\
  人兩足蹻脈,從足至目,長七尺五寸,二七一丈四尺,二五一尺,合一丈五尺。
\\\\
  督脈、任脈,各長四尺五寸,二四八尺,二五一尺,合九尺。
\\\\
  凡脈長一十六丈二尺,此所謂十二經脈長短之數也。
\\\\
  經脈十二,絡脈十五,何始何窮也?
\\\\
  然:經脈者,行血氣,通陰陽,以榮於身者也。其始從中焦,注手太陰、陽明;陽明注足陽明、太陰;太陰注手少陰、太陽;太陽注足太陽、少陰;少陰注手心主、少陽;少陽注足少陽、厥陰;厥陰復還注手太陰。
\\\\
  別絡十五,皆因其原,如環無端,轉相溉灌,朝於寸口、人迎,以處百病,而決死生也。
\\\\
  經曰:明知終始,陰陽定矣,何謂也?
\\\\
  然:終始者,脈之紀也。寸口、人迎,陰陽之氣通於朝使,如環無端,故曰始也。終者,三陰三陽之脈絕,絕則死。死各有形,故曰終也。
\\\\
  二十四難曰:手足三陰三陽氣已絕,何以為候?可知其吉凶不?
\\\\
  然:足少陰氣絕,即骨枯。少陰者,冬脈也,伏行而溫於骨髓。故骨髓不溫,即肉不著骨;骨肉不相親,即肉濡而卻;肉濡而卻,故齒長而枯,髮無潤澤;者,骨先死。戊日篤,己日死。
\\\\
  足太陰氣絕,則脈不榮其口唇。口唇者,肌肉之本也。脈不榮,則肌肉不滑澤;肌肉不滑澤,則肉滿;肉滿,則唇反;唇反,則肉先死。甲日篤,乙日死。
\\\\
  足厥陰氣絕,即筋縮引卵與舌卷。厥陰者,肝脈也。肝者,筋之合也。筋者,聚於陰器而絡於舌本。故脈不榮,則筋縮急;即引卵與舌;故舌卷卵縮,此筋先死。庚日篤,辛日死。筋縮急。
\\\\
  手太陰氣絕,即皮毛焦。太陰者,肺也,行氣溫於皮毛者也。氣弗榮則皮毛焦,皮毛焦則津液去,津液去即皮節傷,皮節傷則皮枯毛折,毛折者則毛先死。丙日篤,丁日死。
\\\\
  手少陰氣絕則脈不通,脈不通則血不流,血不流則色澤去,故面黑如梨,此血先死。壬日篤,癸日死。
\\\\
  陰氣俱絕者,則目眩轉目瞑。目瞑者,為失志;失志者,則志先死,死即目瞑也。
\\\\
  陽氣俱絕者,則陰與陽相離,陰陽相離,則腠理泄,絕汗乃出,大如貫珠,轉出不流,即氣先死。旦占夕死,夕占旦死。



