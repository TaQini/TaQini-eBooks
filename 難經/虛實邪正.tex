\section{虛實邪正}

  四十八難曰:人有三虛三實,何謂也?
\\\\
  然:有脈之虛實,有病之虛實,有診之虛實也。脈之虛實者,濡者為虛,緊牢者為實。病之虛實者,出者為虛,入者為實;言者為虛,不言者為實;緩者為虛,急者為實。診之虛實者,濡者為虛,牢者為實;癢者為虛,痛者為實;外痛內快,為外實內虛;內痛外快,為內實外虛。故曰虛實也。
\\\\
  四十九難曰:有正經自病,有五邪所傷,何以別之?
\\\\
  然:經言憂愁思慮則傷心;形寒飲冷則傷肺;恚怒氣逆,上而不下則傷肝;飲食勞倦則傷脾;久坐濕地,強力入水則傷腎。是正經之自病也。
\\\\
  何謂五邪?
\\\\
  然:有中風,有傷暑,有飲食勞倦,有傷寒,有中濕,此之謂五邪。
\\\\
  假令心病,何以知中風得之?
\\\\
  然:其色當赤。何以言之?肝主色,自入為青,入心為赤,入脾為黃,入肺為白,入腎為黑。肝為心邪,故知當赤色也。其病身熱,脅下滿痛,其脈浮大而絃。
\\\\
  何以知傷暑得之?
\\\\
  然:當惡臭。何以言之?心主臭,自入為焦臭,入脾為香臭,入肝為臊臭,入腎為腐臭,入肺為腥臭。故知心病傷暑得之也,當惡臭。其病身熱而煩,心痛,其脈浮大而散。
\\\\
  何以知飲食勞倦得之?
\\\\
  然:當喜苦味也。虛為不欲食,實為欲食。何以言之?脾主味,入肝為酸,入心為苦,入肺為辛,入腎為鹹,自入為甘。故知脾邪入心,為喜苦味也。其病身熱而體重嗜臥,四肢不收,其脈浮大而緩。
\\\\
  何以知傷寒得之?
\\\\
  然:當譫言妄語。何以言之?肺主聲,入肝為呼,入心為言,入脾為歌,入腎為呻,自入為哭,故知肺邪入心為譫言妄語也。其病身熱,洒洒惡寒,甚則喘咳,其脈浮大而濇。
\\\\
  何以知中濕得之?
\\\\
  然:當喜汗出不可止。何以言之?腎主濕,入肝為泣,入心為汗,入脾為液,入肺為涕,自入為唾。故知腎邪入心,為汗出不可止也。其病身熱而小腹痛,足脛寒而逆,其脈沉濡而大。此五邪之法也。
\\\\
  五十難曰:病有虛邪,有實邪,有賊邪,有微邪,有正邪,何以別之?
\\\\
  然:從後來者為虛邪,從前來者為實邪,從所不勝來者為賊邪,從所勝來者為微邪,自病者為正邪。何以言之?假令心病,中風得之為虛邪,傷暑得之為正邪,飲食勞倦得之為實邪,傷寒得之為微邪,中濕得之為賊邪。
\\\\
  五十一難曰:病有欲得溫者,有欲得寒者,有欲得見人者,有不欲得見人者,而各不同,病在何藏府也?
\\\\
  然:病欲得寒,而欲見人者,病在府也;病欲得溫,而不欲得見人者,病在藏也。何以言之?府者陽也,陽病欲得寒,又欲見人;藏者陰也,陰病欲得溫,又欲閉戶獨處,惡聞人聲,故以別知藏府之病也。
\\\\
  五十二難曰:府藏發病,根本等不?
\\\\
  然:不等也。
\\\\
  其不等奈何?
\\\\
  然:藏病者,止而不移,其病不離其處;府病者,彷彿賁嚮,上下行流,居處無常。故以此知藏府根本不同也。