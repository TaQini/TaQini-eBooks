
\section{項羽本紀}
  項籍者,下相人也,字羽。初起時,年二十四。其季父項梁,梁父即楚將項燕,為秦將王翦所戮者也。項氏世世為楚將,封於項,故姓項氏。
\\\\
  項籍少時,學書不成,去學劍,又不成。項梁怒之。籍曰:「書足以記名姓而已。劍一人敵,不足學,學萬人敵。」於是項梁乃教籍兵法,籍大喜,略知其意,又不肯竟學。項梁嘗有櫟陽逮,乃請蘄獄掾曹咎書抵櫟陽獄掾司馬欣,以故事得已。項梁殺人,與籍避仇於吳中。吳中賢士大夫皆出項梁下。每吳中有大繇役及喪,項梁常為主辦,陰以兵法部勒賓客及子弟,以是知其能。秦始皇帝游會稽,渡浙江,梁與籍俱觀。籍曰:「彼可取而代也。」梁掩其口,曰:「毋妄言,族矣!」梁以此奇籍。籍長八尺餘,力能扛鼎,才氣過人,雖吳中子弟皆已憚籍矣。
\\\\
  秦二世元年七月,陳涉等起大澤中。其九月,會稽守通謂梁曰:「江西皆反,此亦天亡秦之時也。吾聞先即制人,後則為人所制。吾欲發兵,使公及桓楚將。」是時桓楚亡在澤中。梁曰:「桓楚亡,人莫知其處,獨籍知之耳。」梁乃出,誡籍持劍居外待。梁復入,與守坐,曰:「請召籍,使受命召桓楚。」守曰:「諾。」梁召籍入。須臾,梁眴籍曰:「可行矣!」於是籍遂拔劍斬守頭。項梁持守頭,佩其印綬。門下大驚,擾亂,籍所擊殺數十百人。一府中皆慴伏,莫敢起。梁乃召故所知豪吏,諭以所為起大事,遂舉吳中兵。使人收下縣,得精兵八千人。梁部署吳中豪傑為校尉、候、司馬。有一人不得用,自言於梁。梁曰:「前時某喪使公主某事,不能辦,以此不任用公。」眾乃皆伏。於是梁為會稽守,籍為裨將,徇下縣。
\\\\
  廣陵人召平於是為陳王徇廣陵,未能下。聞陳王敗走,秦兵又且至,乃渡江矯陳王命,拜梁為楚王上柱國。曰:「江東已定,急引兵西擊秦。」項梁乃以八千人渡江而西。聞陳嬰已下東陽,使使欲與連和俱西。陳嬰者,故東陽令史,居縣中,素信謹,稱為長者。東陽少年殺其令,相聚數千人,欲置長,無適用,乃請陳嬰。嬰謝不能,遂彊立嬰為長,縣中從者得二萬人。少年欲立嬰便為王,異軍蒼頭特起。陳嬰母謂嬰曰:「自我為汝家婦,未嘗聞汝先古之有貴者。今暴得大名,不祥。不如有所屬,事成猶得封侯,事敗易以亡,非世所指名也。」嬰乃不敢為王。謂其軍吏曰:「項氏世世將家,有名於楚。今欲舉大事,將非其人,不可。我倚名族,亡秦必矣。」於是眾從其言,以兵屬項梁。項梁渡淮,黥布、蒲將軍亦以兵屬焉。凡六七萬人,軍下邳。
\\\\
  當是時,秦嘉已立景駒為楚王,軍彭城東,欲距項梁。項梁謂軍吏曰:「陳王先首事,戰不利,未聞所在。今秦嘉倍陳王而立景駒,逆無道。」乃進兵擊秦嘉。秦嘉軍敗走,追之至胡陵。嘉還戰一日,嘉死,軍降。景駒走死梁地。項梁已并秦嘉軍,軍胡陵,將引軍而西。章邯軍至栗,項梁使別將朱雞石、餘樊君與戰。餘樊君死。朱雞石軍敗,亡走胡陵。項梁乃引兵入薛,誅雞石。項梁前使項羽別攻襄城,襄城堅守不下。已拔,皆阬之。還報項梁。項梁聞陳王定死,召諸別將會薛計事。此時沛公亦起沛,往焉。
\\\\
  居鄛人范增,年七十,素居家,好奇計,往說項梁曰:「陳勝敗碧當。夫秦滅六國,楚最無罪。自懷王入秦不反,楚人憐之至今,故楚南公曰『楚雖三戶,亡秦必楚』也。今陳勝首事,不立楚後而自立,其勢不長。今君起江東,楚蜂午之將皆爭附君者,以君世世楚將,為能復立楚之後也。」於是項梁然其言,乃求楚懷王孫心民閒,為人牧羊,立以為楚懷王,從民所望也。陳嬰為楚上柱國,封五縣,與懷王都盱臺。項梁自號為武信君。
\\\\
  居數月,引兵攻亢父,與齊田榮、司馬龍且軍救東阿,大破秦軍於東阿。田榮即引兵歸,逐其王假。假亡走楚。假相田角亡走趙。角弟田閒故齊將,居趙不敢歸。田榮立田儋子市為齊王。項梁已破東阿下軍,遂追秦軍。數使使趣齊兵,欲與俱西。田榮曰:「楚殺田假,趙殺田角、田閒,乃發兵。」項梁曰:「田假為與國之王,窮來從我,不忍殺之。」趙亦不殺田角、田閒以市於齊。齊遂不肯發兵助楚。項梁使沛公及項羽別攻城陽,屠之。西破秦軍濮陽東,秦兵收入濮陽。沛公、項羽乃攻定陶。定陶未下,去,西略地至雝丘,大破秦軍,斬李由。還攻外黃,外黃未下。
\\\\
  項梁起東阿,西,(北)[比]至定陶,再破秦軍,項羽等又斬李由,益輕秦,有驕色。宋義乃諫項梁曰:「戰勝而將驕卒惰者敗。今卒少惰矣,秦兵日益,臣為君畏之。」項梁弗聽。乃使宋義使於齊。道遇齊使者高陵君顯,曰:「公將見武信君乎?」曰:「然。」曰:「臣論武信君軍必敗。公徐行即免死,疾行則及禍。」秦果悉起兵益章邯,擊楚軍,大破之定陶,項梁死。沛公、項羽去外黃攻陳留,陳留堅守不能下。沛公、項羽相與謀曰:「今項梁軍破,士卒恐。」乃與呂臣軍俱引兵而東。呂臣軍彭城東,項羽軍彭城西,沛公軍碭。
\\\\
  章邯已破項梁軍,則以為楚地兵不足憂,乃渡河擊趙,大破之。當此時,趙歇為王,陳餘為將,張耳為相,皆走入鉅鹿城。章邯令王離、涉閒圍鉅鹿,章邯軍其南,筑甬道而輸之粟。陳餘為將,將卒數萬人而軍鉅鹿之北,此所謂河北之軍也。
\\\\
  楚兵已破於定陶,懷王恐,從盱臺之彭城,并項羽、呂臣軍自將之。以呂臣為司徒,以其父呂青為令尹。以沛公為碭郡長,封為武安侯,將碭郡兵。
\\\\
  初,宋義所遇齊使者高陵君顯在楚軍,見楚王曰:「宋義論武信君之軍必敗,居數日,軍果敗。兵未戰而先見敗徵,此可謂知兵矣。」王召宋義與計事而大說之,因置以為上將軍,項羽為魯公,為次將,范增為末將,救趙。諸別將皆屬宋義,號為卿子冠軍。行至安陽,留四十六日不進。項羽曰:「吾聞秦軍圍趙王鉅鹿,疾引兵渡河,楚擊其外,趙應其內,破秦軍必矣。」宋義曰:「不然。夫搏牛之虻不可以破蟣虱。今秦攻趙,戰勝則兵罷,我承其敝;不勝,則我引兵鼓行而西,必舉秦矣。故不如先鬬秦趙。夫被堅執銳,義不如公;坐而運策,公不如義。」因下令軍中曰:「猛如虎,很如羊,貪如狼,彊不可使者,皆斬之。」乃遣其子宋襄相齊,身送之至無鹽,飲酒高會。天寒大雨,士卒凍饑。項羽曰:「將戮力而攻秦,久留不行。今歲饑民貧,士卒食芋菽,軍無見糧,乃飲酒高會,不引兵渡河因趙食,與趙并力攻秦,乃曰『承其敝』。夫以秦之彊,攻新造之趙,其勢必舉趙。趙舉而秦彊,何敝之承!且國兵新破,王坐不安席,埽境內而專屬於將軍,國家安危,在此一舉。今不恤士卒而徇其私,非社稷之臣。」項羽晨朝上將軍宋義,即其帳中斬宋義頭,出令軍中曰:「宋義與齊謀反楚,楚王陰令羽誅之。」當是時,諸將皆慴服,莫敢枝梧。皆曰:「首立楚者,將軍家也。今將軍誅亂。」乃相與共立羽為假上將軍。使人追宋義子,及之齊,殺之。使桓楚報命於懷王。懷王因使項羽為上將軍,當陽君、蒲將軍皆屬項羽。
\\\\
  項羽已殺卿子冠軍,威震楚國,名聞諸侯。乃遣當陽君、蒲將軍將卒二萬渡河,救鉅鹿。戰少利,陳餘復請兵。項羽乃悉引兵渡河,皆沈船,破釜甑,燒廬舍,持三日糧,以示士卒必死,無一還心。於是至則圍王離,與秦軍遇,九戰,絕其甬道,大破之,殺蘇角,虜王離。涉閒不降楚,自燒殺。當是時,楚兵冠諸侯。諸侯軍救鉅鹿下者十餘壁,莫敢縱兵。及楚擊秦,諸將皆從壁上觀。楚戰士無不一以當十,楚兵呼聲動天,諸侯軍無不人人惴恐。於是已破秦軍,項羽召見諸侯將,入轅門,無不膝行而前,莫敢仰視。項羽由是始為諸侯上將軍,諸侯皆屬焉。
\\\\
  章邯軍棘原,項羽軍漳南,相持未戰。秦軍數卻,二世使人讓章邯。章邯恐,使長史欣請事。至咸陽,留司馬門三日,趙高不見,有不信之心。長史欣恐,還走其軍,不敢出故道,趙高果使人追之,不及。欣至軍,報曰:「趙高用事於中,下無可為者。今戰能勝,高必疾妒吾功;戰不能勝,不免於死。願將軍孰計之。」陳餘亦遺章邯書曰:「白起為秦將,南征鄢郢,北阬馬服,攻城略地,不可勝計,而竟賜死。蒙恬為秦將,北逐戎人,開榆中地數千里,竟斬陽周。何者?功多,秦不能盡封,因以法誅之。今將軍為秦將三歲矣,所亡失以十萬數,而諸侯并起滋益多。彼趙高素諛日久,今事急,亦恐二世誅之,故欲以法誅將軍以塞責,使人更代將軍以脫其禍。夫將軍居外久,多內卻,有功亦誅,無功亦誅。且天之亡秦,無愚智皆知之。今將軍內不能直諫,外為亡國將,孤特獨立而欲常存,豈不哀哉!將軍何不還兵與諸侯為從,約共攻秦,分王其地,南面稱孤;此孰與身伏鈇質,妻子為僇乎?」章邯狐疑,陰使候始成使項羽,欲約。約未成,項羽使蒲將軍日夜引兵度三戶,軍漳南,與秦戰,再破之。項羽悉引兵擊秦軍汙水上,大破之。
\\\\
  章邯使人見項羽,欲約。項羽召軍吏謀曰:「糧少,欲聽其約。」軍吏皆曰:「善。」項羽乃與期洹水南殷虛上。已盟,章邯見項羽而流涕,為言趙高。項羽乃立章邯為雍王,置楚軍中。使長史欣為上將軍,將秦軍為前行。
\\\\
  到新安。諸侯吏卒異時故繇使屯戍過秦中,秦中吏卒遇之多無狀,及秦軍降諸侯,諸侯吏卒乘勝多奴虜使之,輕折辱秦吏卒。秦吏卒多竊言曰:「章將軍等詐吾屬降諸侯,今能入關破秦,大善;即不能,諸侯虜吾屬而東,秦必盡誅吾父母妻子。」諸侯微聞其計,以告項羽。項羽乃召黥布、蒲將軍計曰:「秦吏卒尚眾,其心不服,至關中不聽,事必危,不如擊殺之,而獨與章邯、長史欣、都尉翳入秦。」於是楚軍夜擊阬秦卒二十餘萬人新安城南。
\\\\
  行略定秦地。函谷關有兵守關,不得入。又聞沛公已破咸陽,項羽大怒,使當陽君等擊關。項羽遂入,至于戲西。沛公軍霸上,未得與項羽相見。沛公左司馬曹無傷使人言於項羽曰:「沛公欲王關中,使子嬰為相,珍寶盡有之。」項羽大怒,曰:「旦日饗士卒,為擊破沛公軍!」當是時,項羽兵四十萬,在新豐鴻門,沛公兵十萬,在霸上。范增說項羽曰:「沛公居山東時,貪於財貨,好美姬。今入關,財物無所取,婦女無所幸,此其志不在小。吾令人望其氣,皆為龍虎,成五采,此天子氣也。急擊勿失。」
\\\\
  楚左尹項伯者,項羽季父也,素善留侯張良。張良是時從沛公,項伯乃夜馳之沛公軍,私見張良,具告以事,欲呼張良與俱去。曰:「毋從俱死也。」張良曰:「臣為韓王送沛公,沛公今事有急,亡去不義,不可不語。」良乃入,具告沛公。沛公大驚,曰:「為之柰何?」張良曰:「誰為大王為此計者?」曰:「鯫生說我曰『距關,毋內諸侯,秦地可盡王也』。故聽之。」良曰:「料大王士卒足以當項王乎?」沛公默然,曰:「固不如也,且為之柰何?」張良曰:「請往謂項伯,言沛公不敢背項王也。」沛公曰:「君安與項伯有故?」張良曰:「秦時與臣游,項伯殺人,臣活之。今事有急,故幸來告良。」沛公曰「孰與君少長?」良曰:「長於臣。」沛公曰「君為我呼入,吾得兄事之。」張良出,要項伯。項伯即入見沛公。沛公奉卮酒為壽,約為婚姻,曰:「吾入關,秋豪不敢有所近,籍吏民,封府庫,而待將軍。所以遣將守關者,備他盜之出入與非常也。日夜望將軍至,豈敢反乎!願伯具言臣之不敢倍德也。」項伯許諾。謂沛公曰:「旦日不可不蚤自來謝項王。」沛公曰:「諾。」於是項伯復夜去,至軍中,具以沛公言報項王。因言曰:「沛公不先破關中,公豈敢入乎?今人有大功而擊之,不義也,不如因善遇之。」項王許諾。
\\\\
  沛公旦日從百餘騎來見項王,至鴻門,謝曰:「臣與將軍戮力而攻秦,將軍戰河北,臣戰河南,然不自意能先入關破秦,得復見將軍於此。今者有小人之言,令將軍與臣有郤。」項王曰:「此沛公左司馬曹無傷言之;不然,籍何以至此。」項王即日因留沛公與飲。項王、項伯東向坐。亞父南向坐。亞父者,范增也。沛公北向坐,張良西向侍。范增數目項王,舉所佩玉珪以示之者三,項王默然不應。范增起,出召項莊,謂曰:「君王為人不忍,若入前為壽,壽畢,請以劍舞,因擊沛公於坐,殺之。不者,若屬皆且為所虜。」莊則入為壽,壽畢,曰:「君王與沛公飲,軍中無以為樂,請以劍舞。」項王曰:「諾。」項莊拔劍起舞,項伯亦拔劍起舞,常以身翼蔽沛公,莊不得擊。於是張良至軍門,見樊噲。樊噲曰:「今日之事何如?」良曰:「甚急。今者項莊拔劍舞,其意常在沛公也。」噲曰:「此迫矣,臣請入,與之同命。」噲即帶劍擁盾入軍門。交戟之衛士欲止不內,樊噲側其盾以撞,衛士仆地,噲遂入,披帷西向立,瞋目視項王,頭髪上指,目眥盡裂。項王按劍而跽曰:「客何為者?」張良曰:「沛公之參乘樊噲者也。」項王曰:「壯士,賜之卮酒。」則與斗卮酒。噲拜謝,起,立而飲之。項王曰:「賜之彘肩。」則與一生彘肩。樊噲覆其盾於地,加彘肩上,拔劍切而啗之。項王曰:「壯士,能復飲乎?」樊噲曰:「臣死且不避,卮酒安足辭!夫秦王有虎狼之心,殺人如不能舉,刑人如恐不勝,天下皆叛之。懷王與諸將約曰『先破秦入咸陽者王之』。今沛公先破秦入咸陽,豪毛不敢有所近,封閉宮室,還軍霸上,以待大王來。故遣將守關者,備他盜出入與非常也。勞苦而功高如此,未有封侯之賞,而聽細說,欲誅有功之人。此亡秦之續耳,竊為大王不取也。」項王未有以應,曰:「坐。」樊噲從良坐。坐須臾,沛公起如廁,因招樊噲出。
\\\\
  沛公已出,項王使都尉陳平召沛公。沛公曰:「今者出,未辭也,為之柰何?」樊噲曰:「大行不顧細謹,大禮不辭小讓。如今人方為刀俎,我為魚肉,何辭為。」於是遂去。乃令張良留謝。良問曰:「大王來何操?」曰:「我持白璧一雙,欲獻項王,玉斗一雙,欲與亞父,會其怒,不敢獻。公為我獻之」張良曰:「謹諾。」當是時,項王軍在鴻門下,沛公軍在霸上,相去四十里。沛公則置車騎,脫身獨騎,與樊噲、夏侯嬰、靳彊、紀信等四人持劍盾步走,從酈山下,道芷陽閒行。沛公謂張良曰:「從此道至吾軍,不過二十里耳。度我至軍中,公乃入。」沛公已去,閒至軍中,張良入謝,曰:「沛公不勝桮杓,不能辭。謹使臣良奉白璧一雙,再拜獻大王足下;玉斗一雙,再拜奉大將軍足下。」項王曰:「沛公安在?」良曰:「聞大王有意督過之,脫身獨去,已至軍矣。」項王則受璧,置之坐上。亞父受玉斗,置之地,拔劍撞而破之,曰:「唉!豎子不足與謀。奪項王天下者,必沛公也,吾屬今為之虜矣。」沛公至軍,立誅殺曹無傷。
\\\\
  居數日,項羽引兵西屠咸陽,殺秦降王子嬰,燒秦宮室,火三月不滅;收其貨寶婦女而東。人或說項王曰:「關中阻山河四塞,地肥饒,可都以霸。」項王見秦宮皆以燒殘破,又心懷思欲東歸,曰:「富貴不歸故鄉,如衣繡夜行,誰知之者!」說者曰:「人言楚人沐猴而冠耳,果然。」項王聞之,烹說者。
\\\\
  項王使人致命懷王。懷王曰:「如約。」乃尊懷王為義帝。項王欲自王,先王諸將相。謂曰:「天下初發難時,假立諸侯後以伐秦。然身被堅執銳首事,暴露於野三年,滅秦定天下者,皆將相諸君與籍之力也。義帝雖無功,故當分其地而王之。」諸將皆曰:「善。」乃分天下,立諸將為侯王。項王、范增疑沛公之有天下,業已講解,又惡負約,恐諸侯叛之,乃陰謀曰:「巴、蜀道險,秦之遷人皆居蜀。」乃曰:「巴、蜀亦關中地也。」故立沛公為漢王,王巴、蜀、漢中,都南鄭。而三分關中,王秦降將以距塞漢王。項王乃立章邯為雍王,王咸陽以西,都廢丘。長史欣者,故為櫟陽獄掾,嘗有德於項梁;都尉董翳者,本勸章邯降楚。故立司馬欣為塞王,王咸陽以東至河,都櫟陽;立董翳為翟王,王上郡,都高奴。徙魏王豹為西魏王,王河東,都平陽。瑕丘申陽者,張耳嬖臣也,先下河南(郡),迎楚河上,故立申陽為河南王,都雒陽。韓王成因故都,都陽翟。趙將司馬卬定河內,數有功,故立卬為殷王,王河內,都朝歌。徙趙王歇為代王。趙相張耳素賢,又從入關,故立耳為常山王,王趙地,都襄國。當陽君黥布為楚將,常冠軍,故立布為九江王,都六。鄱君吳芮率百越佐諸侯,又從入關,故立芮為衡山王,都邾。義帝柱國共敖將兵擊南郡,功多,因立敖為臨江王,都江陵。徙燕王韓廣為遼東王。燕將臧荼從楚救趙,因從入關,故立荼為燕王,都薊。徙齊王田市為膠東王。齊將田都從共救趙,因從入關,故立都為齊王,都臨菑。故秦所滅齊王建孫田安,項羽方渡河救趙,田安下濟北數城,引其兵降項羽,故立安為濟北王,都博陽。田榮者,數負項梁,又不肯將兵從楚擊秦,以故不封。成安君陳餘棄將印去,不從入關,然素聞其賢,有功於趙,聞其在南皮,故因環封三縣。番君將梅鋗功多,故封十萬戶侯。項王自立為西楚霸王,王九郡,都彭城。
\\\\
  漢之元年四月,諸侯罷戲下,各就國。項王出之國,使人徙義帝,曰:「古之帝者地方千里,必居上游。」乃使使徙義帝長沙郴縣。趣義帝行,其群臣稍稍背叛之,乃陰令衡山、臨江王擊殺之江中。韓王成無軍功,項王不使之國,與俱至彭城,廢以為侯,已又殺之。臧荼之國,因逐韓廣之遼東,廣弗聽,荼擊殺廣無終,
\\\\
  田榮聞項羽徙齊王市膠東,而立齊將田都為齊王,乃大怒,不肯遣齊王之膠東,因以齊反,迎擊田都。田都走楚。齊王市畏項王,乃亡之膠東就國。田榮怒,追擊殺之即墨。榮因自立為齊王,而西殺擊濟北王田安,并王三齊。榮與彭越將軍印,令反梁地。陳餘陰使張同、夏說說齊王田榮曰:「項羽為天下宰,不平。今盡王故王於醜地,而王其群臣諸將善地,逐其故主趙王,乃北居代,餘以為不可。聞大王起兵,且不聽不義,願大王資餘兵,請以擊常山,以復趙王,請以國為捍蔽。」齊王許之,因遣兵之趙。陳餘悉發三縣兵,與齊并力擊常山,大破之。張耳走歸漢。陳餘迎故趙王歇於代,反之趙。趙王因立陳餘為代王。
\\\\
  是時,漢還定三秦。項羽聞漢王皆已并關中,且東,齊、趙叛之:大怒。乃以故吳令鄭昌為韓王,以距漢。令蕭公角等擊彭越。彭越敗蕭公角等。漢使張良徇韓,乃遺項王書曰:「漢王失職,欲得關中,如約即止,不敢東。」又以齊、梁反書遺項王曰:「齊欲與趙并滅楚。」楚以此故無西意,而北擊齊。徵兵九江王布。布稱疾不往,使將將數千人行。項王由此怨布也。漢之二年冬,項羽遂北至城陽,田榮亦將兵會戰。田榮不勝,走至平原,平原民殺之。遂北燒夷齊城郭室屋,皆阬田榮降卒,系虜其老弱婦女。徇齊至北海,多所殘滅。齊人相聚而叛之。於是田榮弟田橫收齊亡卒得數萬人,反城陽。項王因留,連戰未能下。
\\\\
  春,漢王部五諸侯兵,凡五十六萬人,東伐楚。項王聞之,即令諸將擊齊,而自以精兵三萬人南從魯出胡陵。四月,漢皆已入彭城,收其貨寶美人,日置酒高會。項王乃西從蕭,晨擊漢軍而東,至彭城,日中,大破漢軍。漢軍皆走,相隨入穀、泗水,殺漢卒十餘萬人。漢卒皆南走山,楚又追擊至靈壁東睢水上。漢軍卻,為楚所擠,多殺,漢卒十餘萬人皆入睢水,睢水為之不流。圍漢王三匝。於是大風從西北而起,折木發屋,揚沙石,窈冥晝晦,逢迎楚軍。楚軍大亂,壞散,而漢王乃得與數十騎遁去,欲過沛,收家室而西;楚亦使人追之沛,取漢王家:家皆亡,不與漢王相見。漢王道逢得孝惠、魯元,乃載行。楚騎追漢王,漢王急,推墮孝惠、魯元車下,滕公常下收載之。如是者三。曰:「雖急不可以驅,柰何棄之?」於是遂得脫。求太公、呂后不相遇。審食其從太公、呂后閒行,求漢王,反遇楚軍。楚軍遂與歸,報項王,項王常置軍中。
\\\\
  是時呂后兄周呂侯為漢將兵居下邑,漢王閒往從之,稍稍收其士卒。至滎陽,諸敗軍皆會,蕭何亦發關中老弱未傅悉詣滎陽,復大振。楚起於彭城,常乘勝逐北,與漢戰滎陽南京、索閒,漢敗楚,楚以故不能過滎陽而西。
\\\\
  項王之救彭城,追漢王至滎陽,田橫亦得收齊,立田榮子廣為齊王。漢王之敗彭城,諸侯皆復與楚而背漢。漢軍滎陽,筑甬道屬之河,以取敖倉粟。漢之三年,項王數侵奪漢甬道,漢王食乏,恐,請和,割滎陽以西為漢。
\\\\
  項王欲聽之。歷陽侯范增曰:「漢易與耳,今釋弗取,後必悔之。」項王乃與范增急圍滎陽。漢王患之,乃用陳平計閒項王。項王使者來,為太牢具,舉欲進之。見使者,詳驚愕曰:「吾以為亞父使者,乃反項王使者。」更持去,以惡食食項王使者。使者歸報項王,項王乃疑范增與漢有私,稍奪之權。范增大怒,曰:「天下事大定矣,君王自為之。願賜骸骨歸卒伍。」項王許之。行未至彭城,疽發背而死。
\\\\
  漢將紀信說漢王曰:「事已急矣,請為王誑楚為王,王可以閒出。」於是漢王夜出女子滎陽東門被甲二千人,楚兵四面擊之。紀信乘黃屋車,傅左纛,曰:「城中食盡,漢王降。」楚軍皆呼萬歲。漢王亦與數十騎從城西門出,走成皋。項王見紀信,問:「漢王安在?」曰:「漢王已出矣。」項王燒殺紀信。
\\\\
  漢王使御史大夫周苛、樅公、魏豹守滎陽。周苛、樅公謀曰:「反國之王,難與守城。」乃共殺魏豹。楚下滎陽城,生得周苛。項王謂周苛曰:「為我將,我以公為上將軍,封三萬戶。」周苛罵曰:「若不趣降漢,漢今虜若,若非漢敵也。」項王怒,烹周苛,井殺樅公。
\\\\
  漢王之出滎陽,南走宛、葉,得九江王布,行收兵,復入保成皋。漢之四年,項王進兵圍成皋。漢王逃,獨與滕公出成皋北門,渡河走修武,從張耳、韓信軍。諸將稍稍得出成皋,從漢王。楚遂拔成皋,欲西。漢使兵距之鞏,令其不得西。
\\\\
  是時,彭越渡河擊楚東阿,殺楚將軍薛公。項王乃自東擊彭越。漢王得淮陰侯兵,欲渡河南。鄭忠說漢王,乃止壁河內。使劉賈將兵佐彭越,燒楚積聚。項王東擊破之,走彭越。漢王則引兵渡河,復取成皋,軍廣武,就敖倉食。項王已定東海來,西,與漢俱臨廣武而軍,相守數月。
\\\\
  當此時,彭越數反梁地,絕楚糧食,項王患之。為高俎,置太公其上,告漢王曰:「今不急下,吾烹太公。」漢王曰:「吾與項羽俱北面受命懷王,曰『約為兄弟』,吾翁即若翁,必欲烹而翁,則幸分我一桮羹。」項王怒,欲殺之。項伯曰:「天下事未可知,且為天下者不顧家,雖殺之無益,只益禍耳。」項王從之。
\\\\
  楚漢久相持未決,丁壯苦軍旅,老弱罷轉漕。項王謂漢王曰:「天下匈匈數歲者,徒以吾兩人耳,願與漢王挑戰決雌雄,毋徒苦天下之民父子為也。」漢王笑謝曰:「吾寧鬬智,不能鬬力。」項王令壯士出挑戰。漢有善騎射者樓煩,楚挑戰三合,樓煩輒射殺之。項王大怒,乃自被甲持戟挑戰。樓煩欲射之,項王瞋目叱之,樓煩目不敢視,手不敢發,遂走還入壁,不敢復出。漢王使人閒問之,乃項王也。漢王大驚。於是項王乃即漢王相與臨廣武閒而語。漢王數之,項王怒,欲一戰。漢王不聽,項王伏弩射中漢王。漢王傷,走入成皋。
\\\\
  項王聞淮陰侯已舉河北,破齊、趙,且欲擊楚,乃使龍且往擊之。淮陰侯與戰,騎將灌嬰擊之,大破楚軍,殺龍且。韓信因自立為齊王。項王聞龍且軍破,則恐,使盱臺人武涉往說淮陰侯。淮陰侯弗聽。是時,彭越復反,下梁地,絕楚糧。項王乃謂海春侯大司馬曹咎等曰:「謹守成皋,則漢欲挑戰,慎勿與戰,毋令得東而已。我十五日必誅彭越,定梁地,復從將軍。」乃東,行擊陳留、外黃。
\\\\
  外黃不下。數日,已降,項王怒,悉令男子年十五已上詣城東,欲阬之。外黃令舍人兒年十三,往說項王曰:「彭越彊劫外黃,外黃恐,故且降,待大王。大王至,又皆阬之,百姓豈有歸心?從此以東,梁地十餘城皆恐,莫肯下矣。」項王然其言,乃赦外黃當阬者。東至睢陽,聞之皆爭下項王。
\\\\
  漢果數挑楚軍戰,楚軍不出。使人辱之,五六日,大司馬怒,渡兵汜水。士卒半渡,漢擊之,大破楚軍,盡得楚國貨賂。大司馬咎、長史翳、塞王欣皆自剄汜水上。大司馬咎者,故蘄獄掾,長史欣亦故櫟陽獄吏,兩人嘗有德於項梁,是以項王信任之。當是時,項王在睢陽,聞海春侯軍敗,則引兵還。漢軍方圍鐘離眛於滎陽東,項王至,漢軍畏楚,盡走險阻。
\\\\
  是時,漢兵盛食多,項王兵罷食絕。漢遣陸賈說項王,請太公,項王弗聽。漢王復使侯公往說項王,項王乃與漢約,中分天下,割鴻溝以西者為漢,鴻溝而東者為楚。項王許之,即歸漢王父母妻子。軍皆呼萬歲。漢王乃封侯公為平國君。匿弗肯復見。曰:「此天下辯士,所居傾國,故號為平國君。」項王已約,乃引兵解而東歸。
\\\\
  漢欲西歸,張良、陳平說曰:「漢有天下太半,而諸侯皆附之。楚兵罷食盡,此天亡楚之時也,不如因其機而遂取之。今釋弗擊,此所謂『養虎自遺患』也。」漢王聽之。漢五年,漢王乃追項王至陽夏南,止軍,與淮陰侯韓信、建成侯彭越期會而擊楚軍。至固陵,而信、越之兵不會。楚擊漢軍,大破之。漢王復入壁,深塹而自守。謂張子房曰:「諸侯不從約,為之柰何?」對曰:「楚兵且破,信、越未有分地,其不至固宜。君王能與共分天下,今可立致也。即不能,事未可知也。君王能自陳以東傅海,盡與韓信;睢陽以北至穀城,以與彭越:使各自為戰,則楚易敗也。」漢王曰:「善。」於是乃發使者告韓信、彭越曰:「并力擊楚。楚破,自陳以東傅海與齊王,睢陽以北至穀城與彭相國。」使者至,韓信、彭越皆報曰:「請今進兵。」韓信乃從齊往,劉賈軍從壽春并行,屠城父,至垓下。大司馬周殷叛楚,以舒屠六,舉九江兵,隨劉賈、彭越皆會垓下,詣項王。
\\\\
  項王軍壁垓下,兵少食盡,漢軍及諸侯兵圍之數重。夜聞漢軍四面皆楚歌,項王乃大驚曰:「漢皆已得楚乎?是何楚人之多也!」項王則夜起,飲帳中。有美人名虞,常幸從;駿馬名騅,常騎之。於是項王乃悲歌慨,自為詩曰:「力拔山兮氣蓋世,時不利兮騅不逝。騅不逝兮可柰何,虞兮虞兮柰若何!」歌數闋,美人和之。項王泣數行下,左右皆泣,莫能仰視。
\\\\
  於是項王乃上馬騎,麾下壯士騎從者八百餘人,直夜潰圍南出,馳走。平明,漢軍乃覺之,令騎將灌嬰以五千騎追之。項王渡淮,騎能屬者百餘人耳。項王至陰陵,迷失道,問一田父,田父紿曰「左」。左,乃陷大澤中。以故漢追及之。項王乃復引兵而東,至東城,乃有二十八騎。漢騎追者數千人。項王自度不得脫。謂其騎曰:「吾起兵至今八歲矣,身七十餘戰,所當者破,所擊者服,未嘗敗北,遂霸有天下。然今卒困於此,此天之亡我,非戰之罪也。今日固決死,願為諸君快戰,必三勝之,為諸君潰圍,斬將,刈旗,令諸君知天亡我,非戰之罪也。」乃分其騎以為四隊,四向。漢軍圍之數重。項王謂其騎曰:「吾為公取彼一將。」令四面騎馳下,期山東為三處。於是項王大呼馳下,漢軍皆披靡,遂斬漢一將。是時,赤泉侯為騎將,追項王,項王瞋目而叱之,赤泉侯人馬俱驚,辟易數里與其騎會為三處。漢軍不知項王所在,乃分軍為三,復圍之。項王乃馳,復斬漢一都尉,殺數十百人,復聚其騎,亡其兩騎耳。乃謂其騎曰:「何如?」騎皆伏曰:「如大王言。」
\\\\
  於是項王乃欲東渡烏江。烏江亭長檥船待,謂項王曰:「江東雖小,地方千里,眾數十萬人,亦足王也。願大王急渡。今獨臣有船,漢軍至,無以渡。」項王笑曰:「天之亡我,我何渡為!且籍與江東子弟八千人渡江而西,今無一人還,縱江東父兄憐而王我,我何面目見之?縱彼不言,籍獨不愧於心乎?」乃謂亭長曰:「吾知公長者。吾騎此馬五歲,所當無敵,嘗一日行千里,不忍殺之,以賜公。」乃令騎皆下馬步行,持短兵接戰。獨籍所殺漢軍數百人。項王身亦被十餘創。顧見漢騎司馬呂馬童,曰:「若非吾故人乎?」馬童面之,指王翳曰:「此項王也。」項王乃曰:「吾聞漢購我頭千金,邑萬戶,吾為若德。」乃自刎而死。王翳取其頭,餘騎相蹂踐爭項王,相殺者數十人。最其後,郎中騎楊喜,騎司馬呂馬童,郎中呂勝、楊武各得其一體。五人共會其體,皆是。故分其地為五:封呂馬童為中水侯,封王翳為杜衍侯,封楊喜為赤泉侯,封楊武為吳防侯,封呂勝為涅陽侯。
\\\\
  項王已死,楚地皆降漢,獨魯不下。漢乃引天下兵欲屠之,為其守禮義,為主死節,乃持項王頭視魯,魯父兄乃降。始,楚懷王初封項籍為魯公,及其死,魯最後下,故以魯公禮葬項王穀城。漢王為發哀,泣之而去。
\\\\
  諸項氏枝屬,漢王皆不誅。乃封項伯為射陽侯。桃侯、平皋侯、玄武侯皆項氏,賜姓劉。
\\\\
  太史公曰:吾聞之周生曰「舜目蓋重瞳子」,又聞項羽亦重瞳子。羽豈其苗裔邪?何興之暴也!夫秦失其政,陳涉首難,豪傑蜂起,相與并爭,不可勝數。然羽非有尺寸,乘執起隴畝之中,三年,遂將五諸侯滅秦,分裂天下,而封王侯,政由羽出,號為「霸王」,位雖不終,近古以來未嘗有也。及羽背關懷楚,放逐義帝而自立,怨王侯叛己,難矣。自矜功伐,奮其私智而不師古,謂霸王之業,欲以力征經營天下,五年卒亡其國,身死東城,尚不覺寤而不自責,過矣。乃引「天亡我,非用兵之罪也」,豈不謬哉!
\section{高祖本紀}
  高祖,沛豐邑中陽裏人,姓劉氏,字季。父曰太公,母曰劉媼。其先劉媼嘗息大澤之陂,夢與神遇。是時雷電晦冥,太公往視,則見蛟龍於其上。已而有身,遂產高祖。
\\\\
  高祖為人,隆準而龍顏,美須髯,左股有七十二黑子。仁而愛人,喜施,意豁如也。常有大度,不事家人生產作業。及壯,試為吏,為泗水亭長,廷中吏無所不狎侮,好酒及色。常從王媼、武負貰酒,醉臥,武負、王媼見其上常有龍,怪之。高祖每酤留飲,酒讎數倍。及見怪,歲竟,此兩家常折券棄責。
\\\\
  高祖常繇咸陽,縱觀,觀秦皇帝,喟然太息曰:「嗟乎,大丈夫當如此也!」
\\\\
  單父人呂公善沛令,避仇從之客,因家沛焉。沛中豪桀吏聞令有重客,皆往賀。蕭何為主吏,主進,令諸大夫曰:「進不滿千錢,坐之堂下。」高祖為亭長,素易諸吏,乃紿為謁曰「賀錢萬」,實不持一錢。謁入,呂公大驚,起,迎之門。呂公者,好相人,見高祖狀貌,因重敬之,引入坐。蕭何曰:「劉季固多大言,少成事。」高祖因狎侮諸客,遂坐上坐,無所詘。酒闌,呂公因目固留高祖。高祖竟酒,後。呂公曰:「臣少好相人,相人多矣,無如季相,願季自愛。臣有息女,願為季箕帚妾。」酒罷,呂媼怒呂公曰:「公始常欲奇此女,與貴人。沛令善公,求之不與,何自妄許與劉季?」呂公曰:「此非兒女子所知也。」卒與劉季。呂公女乃呂后也,生孝惠帝、魯元公主。
\\\\
  高祖為亭長時,常告歸之田。呂后與兩子居田中耨,有一老父過請飲,呂后因餔之。老父相呂后曰:「夫人天下貴人。」令相兩子,見孝惠,曰:「夫人所以貴者,乃此男也。」相魯元,亦皆貴。老父已去,高祖適從旁舍來,呂后具言客有過,相我子母皆大貴。高祖問,曰:「未遠。」乃追及,問老父。老父曰:「鄉者夫人嬰兒皆似君,君相貴不可言。」高祖乃謝曰:「誠如父言,不敢忘德。」及高祖貴,遂不知老父處。
\\\\
  高祖為亭長,乃以竹皮為冠,令求盜之薛治之,時時冠之,及貴常冠,所謂「劉氏冠」乃是也。
\\\\
  高祖以亭長為縣送徒酈山,徒多道亡。自度比至皆亡之,到豐西澤中,止飲,夜乃解縱所送徒。曰:「公等皆去,吾亦從此逝矣!」徒中壯士願從者十餘人。高祖被酒,夜徑澤中,令一人行前。行前者還報曰:「前有大蛇當徑,願還。」高祖醉,曰:「壯士行,何畏!」乃前,拔劍擊斬蛇。蛇遂分為兩,徑開。行數里,醉,因臥。後人來至蛇所,有一老嫗夜哭。人問何哭,嫗曰:「人殺吾子,故哭之。」人曰:「嫗子何為見殺?」嫗曰:「吾,白帝子也,化為蛇,當道,今為赤帝子斬之,故哭。」人乃以嫗為不誠,欲告之,嫗因忽不見。後人至,高祖覺。後人告高祖,高祖乃心獨喜,自負。諸從者日益畏之。
\\\\
  秦始皇帝常曰「東南有天子氣」,於是因東游以厭之。高祖即自疑,亡匿,隱於芒、碭山澤巖石之閒。呂后與人俱求,常得之。高祖怪問之。呂后曰:「季所居上常有雲氣,故從往常得季。」高祖心喜。沛中子弟或聞之,多欲附者矣。
\\\\
  秦二世元年秋,陳勝等起蘄,至陳而王,號為「張楚」。諸郡縣皆多殺其長吏以應陳涉。沛令恐,欲以沛應涉。掾、主吏蕭何、曹參乃曰:「君為秦吏,今欲背之,率沛子弟,恐不聽。願君召諸亡在外者,可得數百人,因劫眾,眾不敢不聽。」乃令樊噲召劉季。劉季之眾已數十百人矣。
\\\\
  於是樊噲從劉季來。沛令後悔,恐其有變,乃閉城城守,欲誅蕭、曹。蕭、曹恐,踰城保劉季。劉季乃書帛射城上,謂沛父老曰:「天下苦秦久矣。今父老雖為沛令守,諸侯并起,今屠沛。沛今共誅令,擇子弟可立者立之,以應諸侯,則家室完。不然,父子俱屠,無為也。」父老乃率子弟共殺沛令,開城門迎劉季,欲以為沛令。劉季曰:「天下方擾,諸侯并起,今置將不善,壹敗涂地。吾非敢自愛,恐能薄,不能完父兄子弟。此大事,願更相推擇可者。」蕭、曹等皆文吏,自愛,恐事不就,後秦種族其家,盡讓劉季。諸父老皆曰:「平生所聞劉季諸珍怪,當貴,且卜筮之,莫如劉季最吉。」於是劉季數讓。眾莫敢為,乃立季為沛公。祠黃帝,祭蚩尤於沛庭,而釁鼓旗,幟皆赤。由所殺蛇白帝子,殺者赤帝子,故上赤。於是少年豪吏如蕭、曹、樊噲等皆為收沛子弟二三千人,攻胡陵、方與,還守豐。
\\\\
  秦二世二年,陳涉之將周章軍西至戲而還。燕、趙、齊、魏皆自立為王。項氏起吳。秦泗川監平將兵圍豐,二日,出與戰,破之。命雍齒守豐,引兵之薛。泗州守壯敗於薛,走至戚,沛公左司馬得泗川守壯,殺之。沛公還軍亢父,至方與,(周市來攻方與)未戰。陳王使魏人周市略地。周市使人謂雍齒曰:「豐,故梁徙也。今魏地已定者數十城。齒今下魏,魏以齒為侯守豐。不下,且屠豐。」雍齒雅不欲屬沛公,及魏招之,即反為魏守豐。沛公引兵攻豐,不能取。沛公病,還之沛。沛公怨雍齒與豐子弟叛之,聞東陽甯君、秦嘉立景駒為假王,在留,乃往從之,欲請兵以攻豐。是時秦將章邯從陳,別將司馬夷將兵北定楚地,屠相,至碭。東陽甯君、沛公引兵西,與戰蕭西,不利。還收兵聚留,引兵攻碭,三日乃取碭。因收碭兵,得五六千人。攻下邑,拔之。還軍豐。聞項梁在薛,從騎百餘往見之。項梁益沛公卒五千人,五大夫將十人。沛公還,引兵攻豐。
\\\\
  從項梁月餘,項羽已拔襄城還。項梁盡召別將居薛。聞陳王定死,因立楚後懷王孫心為楚王,治盱臺。項梁號武信君。居數月,北攻亢父,救東阿,破秦軍。齊軍歸,楚獨追北,使沛公、項羽別攻城陽,屠之。軍濮陽之東,與秦軍戰,破之。
\\\\
  秦軍復振,守濮陽,環水。楚軍去而攻定陶,定陶未下。沛公與項羽西略地至雍丘之下,與秦軍戰,大破之,斬李由。還攻外黃,外黃未下。
\\\\
  項梁再破秦軍,有驕色。宋義諫,不聽。秦益章邯兵,夜銜枚擊項梁,大破之定陶,項梁死。沛公與項羽方攻陳留,聞項梁死,引兵與呂將軍俱東。呂臣軍彭城東,項羽軍彭城西,沛公軍碭。
\\\\
  章邯已破項梁軍,則以為楚地兵不足憂,乃渡河,北擊趙,大破之。當是之時,趙歇為王,秦將王離圍之鉅鹿城,此所謂河北之軍也。
\\\\
  秦二世三年,楚懷王見項梁軍破,恐,徙盱臺都彭城,并呂臣、項羽軍自將之。以沛公為碭郡長,封為武安侯,將碭郡兵。封項羽為長安侯,號為魯公。呂臣為司徒,其父呂青為令尹。
\\\\
  趙數請救,懷王乃以宋義為上將軍,項羽為次將,范增為末將,北救趙。令沛公西略地入關。與諸將約,先入定關中者王之。
\\\\
  當是時,秦兵彊,常乘勝逐北,諸將莫利先入關。獨項羽怨秦破項梁軍,奮,願與沛公西入關。懷王諸老將皆曰:「項羽為人彊悍猾賊。項羽嘗攻襄城,襄城無遺類,皆阬之,諸所過無不殘滅。且楚數進取,前陳王、項梁皆敗。不如更遣長者扶義而西,告諭秦父兄。秦父兄苦其主久矣,今誠得長者往,毋侵暴,宜可下。今項羽彊悍,今不可遣。獨沛公素寬大長者,可遣。」卒不許項羽,而遣沛公西略地,收陳王、項梁散卒。乃道碭至成陽,與杠裏秦軍夾壁,破(魏)[秦]二軍。楚軍出兵擊王離,大破之。
\\\\
  沛公引兵西,遇彭越昌邑,因與俱攻秦軍,戰不利。還至栗,遇剛武侯,奪其軍,可四千餘人,并之。與魏將皇欣、魏申徒武蒲之軍并攻昌邑,昌邑未拔。西過高陽。酈食其(謂)[為]監門,曰:「諸將過此者多,吾視沛公大人長者。」乃求見說沛公。沛公方踞床,使兩女子洗足。酈生不拜,長揖,曰:「足下必欲誅無道秦,不宜踞見長者。」於是沛公起,攝衣謝之,延上坐。食其說沛公襲陳留,得秦積粟。乃以酈食其為廣野君,酈商為將,將陳留兵,與偕攻開封,開封未拔。西與秦將楊熊戰白馬,又戰曲遇東,大破之。楊熊走之滎陽,二世使使者斬以徇。南攻潁陽,屠之。因張良遂略韓地轘轅。
\\\\
  當是時,趙別將司馬卬方欲渡河入關,沛公乃北攻平陰,絕河津。南,戰雒陽東,軍不利,還至陽城,收軍中馬騎,與南陽守齮戰犨東,破之。略南陽郡,南陽守齮走,保城守宛。沛公引兵過而西。張良諫曰:「沛公雖欲急入關,秦兵尚眾,距險。今不下宛,宛從後擊,彊秦在前,此危道也。」於是沛公乃夜引兵從他道還,更旗幟,黎明,圍宛城三匝。南陽守欲自剄。其舍人陳恢曰:「死未晚也。」乃踰城見沛公,曰:「臣聞足下約,先入咸陽者王之。今足下留守宛。宛,大郡之都也,連城數十,人民眾,積蓄多,吏人自以為降必死,故皆堅守乘城。今足下盡日止攻,士死傷者必多;引兵去宛,宛必隨足下後:足下前則失咸陽之約,後又有彊宛之患。為足下計,莫若約降,封其守,因使止守,引其甲卒與之西。諸城未下者,聞聲爭開門而待,足下通行無所累。」沛公曰:「善。」乃以宛守為殷侯,封陳恢千戶。引兵西,無不下者。至丹水,高武侯鰓、襄侯王陵降西陵。還攻胡陽,遇番君別將梅鋗,與皆,降析、酈。遣魏人甯昌使秦,使者未來。是時章邯已以軍降項羽於趙矣。
\\\\
  初,項羽與宋義北救趙,及項羽殺宋義,代為上將軍,諸將黥布皆屬,破秦將王離軍,降章邯,諸侯皆附。及趙高已殺二世,使人來,欲約分王關中。沛公以為詐,乃用張良計,使酈生、陸賈往說秦將,啗以利,因襲攻武關,破之。又與秦軍戰於藍田南,益張疑兵旗幟,諸所過毋得掠鹵,秦人喜,秦軍解,因大破之。又戰其北,大破之。乘勝,遂破之。
\\\\
  漢元年十月,沛公兵遂先諸侯至霸上。秦王子嬰素車白馬,系頸以組,封皇帝璽符節,降軹道旁。諸將或言誅秦王。沛公曰:「始懷王遣我,固以能寬容;且人已服降,又殺之,不祥。」乃以秦王屬吏,遂西入咸陽。欲止宮休舍,樊噲、張良諫,乃封秦重寶財物府庫,還軍霸上。召諸縣父老豪桀曰:「父老苦秦苛法久矣,誹謗者族,偶語者棄市。吾與諸侯約,先入關者王之,吾當王關中。與父老約,法三章耳:殺人者死,傷人及盜抵罪。餘悉除去秦法。諸吏人皆案堵如故。凡吾所以來,為父老除害,非有所侵暴,無恐!且吾所以還軍霸上,待諸侯至而定約束耳。」乃使人與秦吏行縣鄉邑,告諭之。秦人大喜,爭持牛羊酒食獻饗軍士。沛公又讓不受,曰:「倉粟多,非乏,不欲費人。」人又益喜,唯恐沛公不為秦王。
\\\\
  或說沛公曰:「秦富十倍天下,地形彊。今聞章邯降項羽,項羽乃號為雍王,王關中。今則來,沛公恐不得有此。可急使兵守函谷關,無內諸侯軍,稍徵關中兵以自益,距之。」沛公然其計,從之。十一月中,項羽果率諸侯兵西,欲入關,關門閉。聞沛公已定關中,大怒,使黥布等攻破函谷關。十二月中,遂至戲。沛公左司馬曹無傷聞項王怒,欲攻沛公,使人言項羽曰:「沛公欲王關中,令子嬰為相,珍寶盡有之。」欲以求封。亞父勸項羽擊沛公。方饗士,旦日合戰。是時項羽兵四十萬,號百萬。沛公兵十萬,號二十萬,力不敵。會項伯欲活張良,夜往見良,因以文諭項羽,項羽乃止。沛公從百餘騎,驅之鴻門,見謝項羽。項羽曰:「此沛公左司馬曹無傷言之。不然,籍何以生此!」沛公以樊噲、張良故,得解歸。歸,立誅曹無傷。
\\\\
  項羽遂西,屠燒咸陽秦宮室,所過無不殘破。秦人大失望,然恐,不敢不服耳。
\\\\
  項羽使人還報懷王。懷王曰:「如約。」項羽怨懷王不肯令與沛公俱西入關,而北救趙,後天下約。乃曰:「懷王者,吾家項梁所立耳,非有功伐,何以得主約!本定天下,諸將及籍也。」乃詳尊懷王為義帝,實不用其命。
\\\\
  正月,項羽自立為西楚霸王,王梁、楚地九郡,都彭城。負約,更立沛公為漢王,王巴、蜀、漢中,都南鄭。三分關中,立秦三將:章邯為雍王,都廢丘;司馬欣為塞王,都櫟陽;董翳為翟王,都高奴。楚將瑕丘申陽為河南王,都洛陽。趙將司馬卬為殷王,都朝歌。趙王歇徙王代。趙相張耳為常山王,都襄國。當陽君黥布為九江王,都六。懷王柱國共敖為臨江王,都江陵。番君吳芮為衡山王,都邾。燕將臧荼為燕王,都薊。故燕王韓廣徙王遼東。廣不聽,臧荼攻殺之無終。封成安君陳餘河閒三縣,居南皮。封梅鋗十萬戶。
\\\\
  四月,兵罷戲下,諸侯各就國。漢王之國,項王使卒三萬人從,楚與諸侯之慕從者數萬人,從杜南入蝕中。去輒燒絕棧道,以備諸侯盜兵襲之,亦示項羽無東意。至南鄭,諸將及士卒多道亡歸,士卒皆歌思東歸。韓信說漢王曰:「項羽王諸將之有功者,而王獨居南鄭,是遷也。軍吏士卒皆山東之人也,日夜跂而望歸,及其鋒而用之,可以有大功。天下已定,人皆自寧,不可復用。不如決策東鄉,爭權天下。」
\\\\
  項羽出關,使人徙義帝。曰:「古之帝者地方千里,必居上游。」乃使使徙義帝長沙郴縣,趣義帝行,群臣稍倍叛之,乃陰令衡山王、臨江王擊之,殺義帝江南。項羽怨田榮,立齊將田都為齊王。田榮怒,因自立為齊王,殺田都而反楚;予彭越將軍印,令反梁地。楚令蕭公角擊彭越,彭越大破之。陳餘怨項羽之弗王己也,令夏說說田榮,請兵擊張耳。齊予陳餘兵,擊破常山王張耳,張耳亡歸漢。迎趙王歇於代,復立為趙王。趙王因立陳餘為代王。項羽大怒,北擊齊。
\\\\
  八月,漢王用韓信之計,從故道還,襲雍王章邯。邯迎擊漢陳倉,雍兵敗,還走;止戰好畤,又復敗,走廢丘。漢王遂定雍地。東至咸陽,引兵圍雍王廢丘,而遣諸將略定隴西、北地、上郡。令將軍薛歐、王吸出武關,因王陵兵南陽,以迎太公、呂後於沛。楚聞之,發兵距之陽夏,不得前。令故吳令鄭昌為韓王,距漢兵。
\\\\
  二年,漢王東略地,塞王欣、翟王翳、河南王申陽皆降。韓王昌不聽,使韓信擊破之。於是置隴西、北地、上郡、渭南、河上、中地郡;關外置河南郡。更立韓太尉信為韓王。諸將以萬人若以一郡降者,封萬戶。繕治河上塞。諸故秦苑囿園池,皆令人得田之,正月,虜雍王弟章平。大赦罪人。
\\\\
  漢王之出關至陜,撫關外父老,還,張耳來見,漢王厚遇之。
\\\\
  二月,令除秦社稷,更立漢社稷。
\\\\
  三月,漢王從臨晉渡,魏王豹將兵從。下河內,虜殷王,置河內郡。南渡平陰津,至雒陽。新城三老董公遮說漢王以義帝死故。漢王聞之,袒而大哭。遂為義帝發喪,臨三日。發使者告諸侯曰:「天下共立義帝,北面事之。今項羽放殺義帝於江南,大逆無道。寡人親為發喪,諸侯皆縞素。悉發關內兵,收三河士,南浮江漢以下,願從諸侯王擊楚之殺義帝者。」
\\\\
  是時項王北擊齊,田榮與戰城陽。田榮敗,走平原,平原民殺之。齊皆降楚。楚因焚燒其城郭,系虜其子女。齊人叛之。田榮弟橫立榮子廣為齊王,齊王反楚城陽。項羽雖聞漢東,既已連齊兵,欲遂破之而擊漢。漢王以故得劫五諸侯兵,遂入彭城。項羽聞之,乃引兵去齊,從魯出胡陵,至蕭,與漢大戰彭城靈壁東睢水上,大破漢軍,多殺士卒,睢水為之不流。乃取漢王父母妻子於沛,置之軍中以為質。當是時,諸侯見楚彊漢敗,還皆去漢復為楚。塞王欣亡入楚。
\\\\
  呂后兄周呂侯為漢將兵,居下邑。漢王從之,稍收士卒,軍碭。漢王乃西過梁地,至虞。使謁者隨何之九江王布所,曰:「公能令布舉兵叛楚,項羽必留擊之。得留數月,吾取天下必矣。」隨何往說九江王布,布果背楚。楚使龍且往擊之。
\\\\
  漢王之敗彭城而西,行使人求家室,家室亦亡,不相得。敗後乃獨得孝惠,六月,立為太子,大赦罪人。令太子守櫟陽,諸侯子在關中者皆集櫟陽為衛。引水灌廢丘,廢丘降,章邯自殺。更名廢丘為槐里。於是令祠官祀天地四方上帝山川,以時祀之。興關內卒乘塞。
\\\\
  是時九江王布與龍且戰,不勝,與隨何閒行歸漢。漢王稍收士卒,與諸將及關中卒益出,是以兵大振滎陽,破楚京、索閒。
\\\\
  三年,魏王豹謁歸視親疾,至即絕河津,反為楚。漢王使酈生說豹,豹不聽。漢王遣將軍韓信擊,大破之,虜豹。遂定魏地,置三郡,曰河東、太原、上黨。漢王乃令張耳與韓信遂東下井陘擊趙,斬陳餘、趙王歇。其明年,立張耳為趙王。
\\\\
  漢王軍滎陽南,筑甬道屬之河,以取敖倉。與項羽相距歲餘。項羽數侵奪漢甬道,漢軍乏食,遂圍漢王。漢王請和,割滎陽以西者為漢。項王不聽。漢王患之,乃用陳平之計,予陳平金四萬斤,以閒疏楚君臣。於是項羽乃疑亞父。亞父是時勸項羽遂下滎陽,及其見疑,乃怒,辭老,願賜骸骨歸卒伍,未至彭城而死。
\\\\
  漢軍絕食,乃夜出女子東門二千餘人,被甲,楚因四面擊之。將軍紀信乃乘王駕,詐為漢王,誑楚,楚皆呼萬歲,之城東觀,以故漢王得與數十騎出西門遁。令御史大夫周苛、魏豹、樅公守滎陽。諸將卒不能從者,盡在城中。周苛、樅公相謂曰:「反國之王,難與守城。」因殺魏豹。
\\\\
  漢王之出滎陽入關,收兵欲復東。袁生說漢王曰:「漢與楚相距滎陽數歲,漢常困。願君王出武關,項羽必引兵南走,王深壁,令滎陽成皋閒且得休。使韓信等輯河北趙地,連燕齊,君王乃復走滎陽,未晚也。如此,則楚所備者多,力分,漢得休,復與之戰,破楚必矣。」漢王從其計,出軍宛葉閒,與黥布行收兵。
\\\\
  項羽聞漢王在宛,果引兵南。漢王堅壁不與戰。是時彭越渡睢水,與項聲、薛公戰下邳,彭越大破楚軍。項羽乃引兵東擊彭越。漢王亦引兵北軍成皋。項羽已破走彭越,聞漢王復軍成皋,乃復引兵西,拔滎陽,誅周苛、樅公,而虜韓王信,遂圍成皋。
\\\\
  漢王跳,獨與滕公共車出成皋玉門,北渡河,馳宿修武。自稱使者,晨馳入張耳、韓信壁,而奪之軍。乃使張耳北益收兵趙地,使韓信東擊齊。漢王得韓信軍,則復振。引兵臨河,南饗軍小修武南,欲復戰。郎中鄭忠乃說止漢王,使高壘深塹,勿與戰。漢王聽其計,使盧綰、劉賈將卒二萬人,騎數百,渡白馬津,入楚地,與彭越復擊破楚軍燕郭西,遂復下梁地十餘城。
\\\\
  淮陰已受命東,未渡平原。漢王使酈生往說齊王田廣,廣叛楚,與漢和,共擊項羽。韓信用蒯通計,遂襲破齊。齊王烹酈生,東走高密。項羽聞韓信已舉河北兵破齊、趙,且欲擊楚,則使龍且、周蘭往擊之。韓信與戰,騎將灌嬰擊,大破楚軍,殺龍且。齊王廣奔彭越。當此時,彭越將兵居梁地,往來苦楚兵,絕其糧食。
\\\\
  四年,項羽乃謂海春侯大司馬曹咎曰:「謹守成皋。若漢挑戰,慎勿與戰,無令得東而已。我十五日必定梁地,復從將軍。」乃行擊陳留、外黃、睢陽,下之。漢果數挑楚軍,楚軍不出,使人辱之五六日,大司馬怒,度兵汜水。士卒半渡,漢擊之,大破楚軍,盡得楚國金玉貨賂。大司馬咎、長史欣皆自剄汜水上。項羽至睢陽,聞海春侯破,乃引兵還。漢軍方圍鐘離眛於滎陽東,項羽至,盡走險阻。
\\\\
  韓信已破齊,使人言曰:「齊邊楚,權輕,不為假王,恐不能安齊。」漢王欲攻之。留侯曰:「不如因而立之,使自為守。」乃遣張良操印綬立韓信為齊王。
\\\\
  項羽聞龍且軍破,則恐,使盱臺人武涉往說韓信。韓信不聽。
\\\\
  楚漢久相持未決,丁壯苦軍旅,老弱罷轉馕。漢王項羽相與臨廣武之閒而語。項羽欲與漢王獨身挑戰。漢王數項羽曰:「始與項羽俱受命懷王,曰先入定關中者王之,項羽負約,王我於蜀漢,罪一。秦項羽矯殺卿子冠軍而自尊,罪二。項羽已救趙,當還報,而擅劫諸侯兵入關,罪三。懷王約入秦無暴掠,項羽燒秦宮室,掘始皇帝冢,私收其財物,罪四。又彊殺秦降王子嬰,罪五。詐阬秦子弟新安二十萬,王其將,罪六。項羽皆王諸將善地,而徙逐故主,令臣下爭叛逆,罪七。項羽出逐義帝彭城,自都之,奪韓王地,并王梁楚,多自予,罪八。項羽使人陰弒義帝江南,罪九。夫為人臣而弒其主,殺已降,為政不平,主約不信,天下所不容,大逆無道,罪十也。吾以義兵從諸侯誅殘賊,使刑餘罪人擊殺項羽,何苦乃與公挑戰!」項羽大怒,伏弩射中漢王。漢王傷匈,乃捫足曰:「虜中吾指!」漢王病創臥,張良彊請漢王起行勞軍,以安士卒,毋令楚乘勝於漢。漢王出行軍,病甚,因馳入成皋。
\\\\
  病愈,西入關,至櫟陽,存問父老,置酒,梟故塞王欣頭櫟陽市。留四日,復如軍,軍廣武。關中兵益出。
\\\\
  當此時,彭越將兵居梁地,往來苦楚兵,絕其糧食。田橫往從之。項羽數擊彭越等,齊王信又進擊楚。項羽恐,乃與漢王約,中分天下,割鴻溝而西者為漢,鴻溝而東者為楚。項王歸漢王父母妻子,軍中皆呼萬歲,乃歸而別去。
\\\\
  項羽解而東歸。漢王欲引而西歸,用留侯、陳平計,乃進兵追項羽,至陽夏南止軍,與齊王信、建成侯彭越期會而擊楚軍。至固陵,不會。楚擊漢軍,大破之。漢王復入壁,深塹而守之。用張良計,於是韓信、彭越皆往。及劉賈入楚地,圍壽春,漢王敗碧陵,乃使使者召大司馬周殷舉九江兵而迎(之)武王,行屠城父,隨(何)劉賈、齊梁諸侯皆大會垓下。立武王布為淮南王。
\\\\
  五年,高祖與諸侯兵共擊楚軍,與項羽決勝垓下。淮陰侯將三十萬自當之,孔將軍居左,費將軍居右,皇帝在後,絳侯、柴將軍在皇帝後。項羽之卒可十萬。淮陰先合,不利,卻。孔將軍、費將軍縱,楚兵不利,淮陰侯復乘之,大敗垓下。項羽卒聞漢軍之楚歌,以為漢盡得楚地,項羽乃敗而走,是以兵大敗。使騎將灌嬰追殺項羽東城,斬首八萬,遂略定楚地。魯為楚堅守不下。漢王引諸侯兵北,示魯父老項羽頭,魯乃降。遂以魯公號葬項羽穀城。還至定陶,馳入齊王壁,奪其軍。
\\\\
  正月,諸侯及將相相與共請尊漢王為皇帝。漢王曰:「吾聞帝賢者有也,空言虛語,非所守也,吾不敢當帝位。」群臣皆曰:「大王起微細,誅暴逆,平定四海,有功者輒裂地而封為王侯。大王不尊號,皆疑不信。臣等以死守之。」漢王三讓,不得已,曰:「諸君必以為便,便國家。」甲午,乃即皇帝位汜水之陽。
\\\\
  皇帝曰義帝無後。齊王韓信習楚風俗,徙為楚王,都下邳。立建成侯彭越為梁王,都定陶。故韓王信為韓王,都陽翟。徙衡山王吳芮為長沙王,都臨湘。番君之將梅鋗有功,從入武關,故德番君。淮南王布、燕王臧荼、趙王敖皆如故。
\\\\
  天下大定。高祖都雒陽,諸侯皆臣屬。故臨江王驩為項羽叛漢,令盧綰、劉賈圍之,不下。數月而降,殺之雒陽。
\\\\
  五月,兵皆罷歸家。諸侯子在關中者復之十二歲,其歸者復之六歲,食之一歲。
\\\\
  高祖置酒雒陽南宮。高祖曰:「列侯諸將無敢隱朕,皆言其情。吾所以有天下者何?項氏之所以失天下者何?」高起、王陵對曰:「陛下慢而侮人,項羽仁而愛人。然陛下使人攻城略地,所降下者因以予之,與天下同利也。項羽妒賢嫉能,有功者害之,賢者疑之,戰勝而不予人功,得地而不予人利,此所以失天下也。」高祖曰:「公知其一,未知其二。夫運籌策帷帳之中,決勝於千里之外,吾不如子房。鎮國家,撫百姓,給餽馕,不絕糧道,吾不如蕭何。連百萬之軍,戰必勝,攻必取,吾不如韓信。此三者,皆人傑也,吾能用之,此吾所以取天下也。項羽有一范增而不能用,此其所以為我擒也。」
\\\\
  高祖欲長都雒陽,齊人劉敬說,乃留侯勸上入都關中,高祖是日駕,入都關中。六月,大赦天下。
\\\\
  十月,燕王臧荼反,攻下代地。高祖自將擊之,得燕王臧荼。即立太尉盧綰為燕王。使丞相噲將兵攻代。
\\\\
  其秋,利幾反,高祖自將兵擊之,利幾走。利幾者,項氏之將。項氏敗,利幾為陳公,不隨項羽,亡降高祖,高祖侯之潁川。高祖至雒陽,舉通侯籍召之,而利幾恐,故反。
\\\\
  六年,高祖五日一朝太公,如家人父子禮。太公家令說太公曰:「天無二日,土無二王。今高祖雖子,人主也;太公雖父,人臣也。柰何令人主拜人臣!如此,則威重不行。」後高祖朝,太公擁篲,迎門卻行。高祖大驚,下扶太公。太公曰:「帝,人主也,柰何以我亂天下法!」於是高祖乃尊太公為太上皇。心善家令言,賜金五百斤。
\\\\
  十二月,人有上變事告楚王信謀反,上問左右,左右爭欲擊之。用陳平計,乃偽遊雲夢,會諸侯於陳,楚王信迎,即因執之。是日,大赦天下。田肯賀,因說高祖曰:「陛下得韓信,又治秦中。秦,形勝之國,帶河山之險,縣隔千里,持戟百萬,秦得百二焉。地勢便利,其以下兵於諸侯,譬猶居高屋之上建瓴水也。夫齊,東有瑯邪、即墨之饒,南有泰山之固,西有濁河之限,北有勃海之利。地方二千里,持戟百萬,縣隔千里之外,齊得十二焉。故此東西秦也。非親子弟,莫可使王齊矣。」高祖曰:「善。」賜黃金五百斤。
\\\\
  後十餘日,封韓信為淮陰侯,分其地為二國。高祖曰將軍劉賈數有功,以為荊王,王淮東。弟交為楚王,王淮西。子肥為齊王,王七十餘城,民能齊言者皆屬齊。乃論功,與諸列侯剖符行封。徙韓王信太原。
\\\\
  七年,匈奴攻韓王信馬邑,信因與謀反太原。白土曼丘臣、王黃立故趙將趙利為王以反,高祖自往擊之。會天寒,士卒墮指者什二三,遂至平城。匈奴圍我平城,七日而後罷去。令樊噲止定代地。立兄劉仲為代王。
\\\\
  二月,高祖自平城過趙、雒陽,至長安。長樂宮成,丞相已下徙治長安。
\\\\
  八年,高祖東擊韓王信餘反寇於東垣。
\\\\
  蕭丞相營作未央宮,立東闕、北闕、前殿、武庫、太倉。高祖還,見宮闕壯甚,怒,謂蕭何曰:「天下匈匈苦戰數歲,成敗未可知,是何治宮室過度也?」蕭何曰:「天下方未定,故可因遂就宮室。且夫天子四海為家,非壯麗無以重威,且無令後世有以加也。」高祖乃說。
\\\\
  高祖之東垣,過柏人,趙相貫高等謀弒高祖,高祖心動,因不留。代王劉仲棄國亡,自歸雒陽,廢以為合陽侯。
\\\\
  九年,趙相貫高等事發覺,夷三族。廢趙王敖為宣平侯。是歲,徙貴族楚昭、屈、景、懷、齊田氏關中。
\\\\
  未央宮成。高祖大朝諸侯群臣,置酒未央前殿。高祖奉玉卮,起為太上皇壽,曰:「始大人常以臣無賴,不能治產業,不如仲力。今某之業所就孰與仲多?」殿上群臣皆呼萬歲,大笑為樂。
\\\\
  十年十月,淮南王黥布、梁王彭越、燕王盧綰、荊王劉賈、楚王劉交、齊王劉肥、長沙王吳芮皆來朝長樂宮。春夏無事。
\\\\
  七月,太上皇崩櫟陽宮。楚王、梁王皆來送葬。赦櫟陽囚。更命酈邑曰新豐。
\\\\
  八月,趙相國陳豨反代地。上曰:「豨嘗為吾使,甚有信。代地吾所急也,故封豨為列侯,以相國守代,今乃與王黃等劫掠代地!代地吏民非有罪也。其赦代吏民。」九月,上自東往擊之。至邯鄲,上喜曰:「豨不南據邯鄲而阻漳水,吾知其無能為也。」聞豨將皆故賈人也,上曰:「吾知所以與之。」乃多以金啗豨將,豨將多降者。
\\\\
  十一年,高祖在邯鄲誅豨等未畢,豨將侯敞將萬餘人游行,王黃軍曲逆,張春渡河擊聊城。漢使將軍郭蒙與齊將擊,大破之。太尉周勃道太原入,定代地。至馬邑,馬邑不下,即攻殘之。
\\\\
  豨將趙利守東垣,高祖攻之,不下。月餘,卒罵高祖,高祖怒。城降,令出罵者斬之,不罵者原之。於是乃分趙山北,立子恒以為代王,都晉陽。
\\\\
  春,淮陰侯韓信謀反關中,夷三族。
\\\\
  夏,梁王彭越謀反,廢遷蜀;復欲反,遂夷三族。立子恢為梁王,子友為淮陽王。
\\\\
  秋七月,淮南王黥布反,東并荊王劉賈地,北渡淮,楚王交走入薛。高祖自往擊之。立子長為淮南王。
\\\\
  十二年,十月,高祖已擊布軍會甀,布走,令別將追之。
\\\\
  高祖還歸,過沛,留。置酒沛宮,悉召故人父老子弟縱酒,發沛中兒得百二十人,教之歌。酒酣,高祖擊筑,自為歌詩曰:「大風起兮雲飛揚,威加海內兮歸故鄉,安得猛士兮守四方!」令兒皆和習之。高祖乃起舞,慷慨傷懷,泣數行下。謂沛父兄曰:「游子悲故鄉。吾雖都關中,萬歲後吾魂魄猶樂思沛。且朕自沛公以誅暴逆,遂有天下,其以沛為朕湯沐邑,復其民,世世無有所與。」沛父兄諸母故人日樂飲極驩,道舊故為笑樂。十餘日,高祖欲去,沛父兄固請留高祖。高祖曰:「吾人眾多,父兄不能給。」乃去。沛中空縣皆之邑西獻。高祖復留止,張飲三日。沛父兄皆頓首曰:「沛幸得復,豐未復,唯陛下哀憐之。」高祖曰:「豐吾所生長,極不忘耳,吾特為其以雍齒故反我為魏。」沛父兄固請,乃并復豐,比沛。於是拜沛侯劉濞為吳王。
\\\\
  漢將別擊布軍洮水南北,皆大破之,追得斬布鄱陽。
\\\\
  樊噲別將兵定代,斬陳豨當城。
\\\\
  十一月,高祖自布軍至長安。十二月,高祖曰:「秦始皇帝、楚隱王陳涉、魏安釐王、齊緡王、趙悼襄王皆絕無後,予守冢各十家,秦皇帝二十家,魏公子無忌五家。」赦代地吏民為陳豨、趙利所劫掠者,皆赦之。陳豨降將言豨反時,燕王盧綰使人之豨所,與陰謀。上使辟陽侯迎綰,綰稱病。辟陽侯歸,具言綰反有端矣。二月,使樊噲、周勃將兵擊燕王綰,赦燕吏民與反者。立皇子建為燕王。
\\\\
  高祖擊布時,為流矢所中,行道病。病甚,呂后迎良醫,醫入見,高祖問醫,醫曰:「病可治。」於是高祖嫚罵之曰:「吾以布衣提三尺劍取天下,此非天命乎?命乃在天,雖扁鵲何益!」遂不使治病,賜金五十斤罷之。已而呂后問:「陛下百歲後,蕭相國即死,令誰代之?」上曰:「曹參可。」問其次,上曰:「王陵可。然陵少憨,陳平可以助之。陳平智有餘,然難以獨任。周勃重厚少文,然安劉氏者必勃也,可令為太尉。」呂后復問其次,上曰:「此後亦非而所知也。」
\\\\
  盧綰與數千騎居塞下候伺,幸上病愈自入謝。
\\\\
  四月甲辰,高祖崩長樂宮。四日不發喪。呂后與審食其謀曰:「諸將與帝為編戶民,今北面為臣,此常怏怏,今乃事少主,非盡族是,天下不安。」人或聞之,語酈將軍。酈將軍往見審食其,曰:「吾聞帝已崩,四日不發喪,欲誅諸將。誠如此,天下危矣。陳平、灌嬰將十萬守滎陽,樊噲、周勃將二十萬定燕、代,此聞帝崩,諸將皆誅,必連兵還鄉以攻關中。大臣內叛,諸侯外反,亡可翹足而待也。」審食其入言之,乃以丁未發喪,大赦天下。
\\\\
  盧綰聞高祖崩,遂亡入匈奴。
\\\\
  丙寅,葬。己巳,立太子,至太上皇廟。群臣皆曰:「高祖起微細,撥亂世反之正,平定天下,為漢太祖,功最高。」上尊號為高皇帝。太子襲號為皇帝,孝惠帝也。令郡國諸侯各立高祖廟,以歲時祠。
\\\\
  及孝惠五年,思高祖之悲樂沛,以沛宮為高祖原廟。高祖所教歌兒百二十人,皆令為吹樂,後有缺,輒補之。
\\\\
  高帝八男:長庶齊悼惠王肥;次孝惠,呂后子;次戚夫人子趙隱王如意;次代王恒,已立為孝文帝,薄太后子;次梁王恢,呂太后時徙為趙共王;次淮陽王友,呂太后時徙為趙幽王;次淮南厲王長;次燕王建。
\\\\
  太史公曰:夏之政忠。忠之敝,小人以野,故殷人承之以敬。敬之敝,小人以鬼,故周人承之以文。文之敝,小人以僿,故救僿莫若以忠。三王之道若循環,終而復始。周秦之閒,可謂文敝矣。秦政不改,反酷刑法,豈不繆乎?故漢興,承敝易變,使人不倦,得天統矣。朝以十月。車服黃屋左纛。葬長陵。