\section{孔子世家}
  孔子生魯昌平鄉陬邑。其先宋人也,曰孔防叔。防叔生伯夏,伯夏生叔梁紇。紇與顏氏女野合而生孔子,禱於尼丘得孔子。魯襄公二十二年而孔子生。生而首上圩頂,故因名曰丘云。字仲尼,姓孔氏。
\\\\
  丘生而叔梁紇死,葬於防山。防山在魯東,由是孔子疑其父墓處,母諱之也。孔子為兒嬉戲,常陳俎豆,設禮容。孔子母死,乃殯五父之衢,蓋其慎也。郰人輓父之母誨孔子父墓,然後往合葬於防焉。
\\\\
  孔子要絰,季氏饗士,孔子與往。陽虎絀曰:「季氏饗士,非敢饗子也。」孔子由是退。
\\\\
  孔子年十七,魯大夫孟釐子病且死,誡其嗣懿子曰:「孔丘,聖人之後,滅於宋。其祖弗父何始有宋而嗣讓厲公。及正考父佐戴、武、宣公,三命茲益恭,故鼎銘云:『一命而僂,再命而傴,三命而俯,循墻而走,亦莫敢余侮。饘於是,粥於是,以餬余口。』其恭如是。吾聞聖人之後,雖不當世,必有達者。今孔丘年少好禮,其達者歟?吾即沒,若必師之。」及釐子卒,懿子與魯人南宮敬叔往學禮焉。是歲,季武子卒,平子代立。
\\\\
  孔子貧且賤。及長,嘗為季氏史,料量平;嘗為司職吏而畜蕃息。由是為司空。已而去魯,斥乎齊,逐乎宋、衛,困於陳蔡之間,於是反魯。孔子長九尺有六寸,人皆謂之「長人」而異之。魯復善待,由是反魯。
\\\\
  魯南宮敬叔言魯君曰:「請與孔子適周。」魯君與之一乘車,兩馬,一豎子俱,適周問禮,蓋見老子云。辭去,而老子送之曰:「吾聞富貴者送人以財,仁人者送人以言。吾不能富貴,竊仁人之號,送子以言,曰:『聰明深察而近於死者,好議人者也。博辯廣大危其身者,發人之惡者也。為人子者毋以有己,為人臣者毋以有己。』」孔子自周反于魯,弟子稍益進焉。
\\\\
  是時也,晉平公淫,六卿擅權,東伐諸侯;楚靈王兵彊,陵轢中國;齊大而近於魯。魯小弱,附於楚則晉怒;附於晉則楚來伐;不備於齊,齊師侵魯。
\\\\
  魯昭公之二十年,而孔子蓋年三十矣。齊景公與晏嬰來適魯,景公問孔子曰:「昔秦穆公國小處辟,其霸何也?」對曰:「秦,國雖小,其志大;處雖辟,行中正。身舉五羖,爵之大夫,起纍紲之中,與語三日,授之以政。以此取之,雖王可也,其霸小矣。」景公說。
\\\\
  孔子年三十五,而季平子與郈昭伯以鬬雞故得罪魯昭公,昭公率師擊平子,平子與孟氏、叔孫氏三家共攻昭公,昭公師敗,奔於齊,齊處昭公乾侯。其後頃之,魯亂。孔子適齊,為高昭子家臣,欲以通乎景公。與齊太師語樂,聞韶音,學之,三月不知肉味,齊人稱之。
\\\\
  景公問政孔子,孔子曰:「君君,臣臣,父父,子子。」景公曰:「善哉!信如君不君,臣不臣,父不父,子不子,雖有粟,吾豈得而食諸!」他日又復問政於孔子,孔子曰:「政在節財。」景公說,將欲以尼谿田封孔子。晏嬰進曰:「夫儒者滑稽而不可軌法;倨傲自順,不可以為下;崇喪遂哀,破產厚葬,不可以為俗;游說乞貸,不可以為國。自大賢之息,周室既衰,禮樂缺有間。今孔子盛容飾,繁登降之禮,趨詳之節,累世不能殫其學,當年不能究其禮。君欲用之以移齊俗,非所以先細民也。」後景公敬見孔子,不問其禮。異日,景公止孔子曰:「奉子以季氏,吾不能。」以季孟之間待之。齊大夫欲害孔子,孔子聞之。景公曰:「吾老矣,弗能用也。」孔子遂行,反乎魯。
\\\\
  孔子年四十二,魯昭公卒於乾侯,定公立。定公立五年,夏,季平子卒,桓子嗣立。季桓子穿井得土缶,中若羊,問仲尼云「得狗」。仲尼曰:「以丘所聞,羊也。丘聞之,木石之怪夔、罔閬,水之怪龍、罔象,土之怪墳羊。」
\\\\
  吳伐越,墮會稽,得骨節專車。吳使使問仲尼:「骨何者最大?」仲尼曰:「禹致群神於會稽山,防風氏後至,禹殺而戮之,其節專車,此為大矣。」吳客曰:「誰為神?」仲尼曰:「山川之神足以綱紀天下,其守為神,社稷為公侯,皆屬於王者。」客曰:「防風何守?」仲尼曰:「汪罔氏之君守封、禺之山,為釐姓。在虞、夏、商為汪罔,於周為長翟,今謂之大人。」客曰:「人長幾何?」仲尼曰:「僬僥氏三尺,短之至也。長者不過十之,數之極也。」於是吳客曰:「善哉聖人!」
\\\\
  桓子嬖臣曰仲梁懷,與陽虎有隙。陽虎欲逐懷,公山不狃止之。其秋,懷益驕,陽虎執懷。桓子怒,陽虎因囚桓子,與盟而醳之。陽虎由此益輕季氏。季氏亦僭於公室,陪臣執國政,是以魯自大夫以下皆僭離於正道。故孔子不仕,退而脩詩書禮樂,弟子彌眾,至自遠方,莫不受業焉。
\\\\
  定公八年,公山不狃不得意於季氏,因陽虎為亂,欲廢三桓之適,更立其庶孽陽虎素所善者,遂執季桓子。桓子詐之,得脫。定公九年,陽虎不勝,奔于齊。是時孔子年五十。
\\\\
  公山不狃以費畔季氏,使人召孔子。孔子循道彌久,溫溫無所試,莫能己用,曰:「蓋周文武起豐鎬而王,今費雖小,儻庶幾乎!」欲往。子路不說,止孔子。孔子曰:「夫召我者豈徒哉?如用我,其為東周乎!」然亦卒不行。
\\\\
  其後定公以孔子為中都宰,一年,四方皆則之。由中都宰為司空,由司空為大司寇。
\\\\
  定公十年春,及齊平。夏,齊大夫黎鉏言於景公曰:「魯用孔丘,其勢危齊。」乃使使告魯為好會,會於夾谷。魯定公且以乘車好往。孔子攝相事,曰:「臣聞有文事者必有武備,有武事者必有文備。古者諸侯出疆,必具官以從。請具左右司馬。」定公曰:「諾。」具左右司馬。會齊侯夾谷,為壇位,土階三等,以會遇之禮相見,揖讓而登。獻酬之禮畢,齊有司趨而進曰:「請奏四方之樂。」景公曰:「諾。」於是旍旄羽袚矛戟劍撥鼓噪而至。孔子趨而進,歷階而登,不盡一等,舉袂而言曰:「吾兩君為好會,夷狄之樂何為於此!請命有司!」有司卻之,不去,則左右視晏子與景公。景公心怍,麾而去之。有頃,齊有司趨而進曰:「請奏宮中之樂。」景公曰:「諾。」優倡侏儒為戲而前。孔子趨而進,歷階而登,不盡一等,曰:「匹夫而營惑諸侯者罪當誅!請命有司!」有司加法焉,手足異處。景公懼而動,知義不若,歸而大恐,告其群臣曰:「魯以君子之道輔其君,而子獨以夷狄之道教寡人,使得罪於魯君,為之奈何?」有司進對曰:「君子有過則謝以質,小人有過則謝以文。君若悼之,則謝以質。」於是齊侯乃歸所侵魯之鄆、汶陽、龜陰之田以謝過。
\\\\
  定公十三年夏,孔子言於定公曰:「臣無藏甲,大夫毋百雉之城。」使仲由為季氏宰,將墮三都。於是叔孫氏先墮郈。季氏將墮費,公山不狃、叔孫輒率費人襲魯。公與三子入于季氏之宮,登武子之臺。費人攻之,弗克,入及公側。孔子命申句須、樂頎下伐之,費人北。國人追之,敗諸姑蔑。二子奔齊,遂墮費。將墮成,公斂處父謂孟孫曰:「墮成,齊人必至于北門。且成,孟氏之保鄣,無成是無孟氏也。我將弗墮。」十二月,公圍成,弗克。
\\\\
  定公十四年,孔子年五十六,由大司寇行攝相事,有喜色。門人曰:「聞君子禍至不懼,福至不喜。」孔子曰:「有是言也。不曰『樂其以貴下人』乎?」於是誅魯大夫亂政者少正卯。與聞國政三月,粥羔豚者弗飾賈;男女行者別於塗;塗不拾遺;四方之客至乎邑者不求有司,皆予之以歸。
\\\\
  齊人聞而懼,曰:「孔子為政必霸,霸則吾地近焉,我之為先并矣。盍致地焉?」黎鉏曰:「請先嘗沮之;沮之而不可則致地,庸遲乎!」於是選齊國中女子好者八十人,皆衣文衣而舞康樂,文馬三十駟,遺魯君。陳女樂文馬於魯城南高門外,季桓子微服往觀再三,將受,乃語魯君為周道游,往觀終日,怠於政事。子路曰:「夫子可以行矣。」孔子曰:「魯今且郊,如致膰乎大夫,則吾猶可以止。」桓子卒受齊女樂,三日不聽政;郊,又不致膰俎於大夫。孔子遂行,宿乎屯。而師己送,曰:「夫子則非罪。」孔子曰:「吾歌可夫?」歌曰:「彼婦之口,可以出走;彼婦之謁,可以死敗。蓋優哉游哉,維以卒歲!」師己反,桓子曰:「孔子亦何言?」師己以實告。桓子喟然嘆曰:「夫子罪我以群婢故也夫!」
\\\\
  孔子遂適衛,主於子路妻兄顏濁鄒家。衛靈公問孔子:「居魯得祿幾何?」對曰:「奉粟六萬。」衛人亦致粟六萬。居頃之,或譖孔子於衛靈公。靈公使公孫余假一出一入。孔子恐獲罪焉,居十月,去衛。
\\\\
  將適陳,過匡,顏刻為僕,以其策指之曰:「昔吾入此,由彼缺也。」匡人聞之,以為魯之陽虎。陽虎嘗暴匡人,匡人於是遂止孔子。孔子狀類陽虎,拘焉五日,顏淵後,子曰:「吾以汝為死矣。」顏淵曰:「子在,回何敢死!」匡人拘孔子益急,弟子懼。孔子曰:「文王既沒,文不在茲乎?天之將喪斯文也,後死者不得與于斯文也。天之未喪斯文也,匡人其如予何!」孔子使從者為甯武子臣於衛,然後得去。
\\\\
  去即過蒲。月餘,反乎衛,主蘧伯玉家。靈公夫人有南子者,使人謂孔子曰:「四方之君子不辱欲與寡君為兄弟者,必見寡小君。寡小君願見。」孔子辭謝,不得已而見之。夫人在絺帷中。孔子入門,北面稽首。夫人自帷中再拜,環珮玉聲璆然。孔子曰:「吾鄉為弗見,見之禮答焉。」子路不說。孔子矢之曰:「予所不者,天厭之!天厭之!」居衛月餘,靈公與夫人同車,宦者雍渠參乘,出,使孔子為次乘,招搖市過之。孔子曰:「吾未見好德如好色者也。」於是醜之,去衛,過曹。是歲,魯定公卒。
\\\\
  孔子去曹適宋,與弟子習禮大樹下。宋司馬桓魋欲殺孔子,拔其樹。孔子去。弟子曰:「可以速矣。」孔子曰:「天生德於予,桓魋其如予何!」
\\\\
  孔子適鄭,與弟子相失,孔子獨立郭東門。鄭人或謂子貢曰:「東門有人,其顙似堯,其項類皋陶,其肩類子產,然自要以下不及禹三寸。纍纍若喪家之狗。」子貢以實告孔子。孔子欣然笑曰:「形狀,末也。而謂似喪家之狗,然哉!然哉!」
\\\\
  孔子遂至陳,主於司城貞子家。歲餘,吳王夫差伐陳,取三邑而去。趙鞅伐朝歌。楚圍蔡,蔡遷于吳。吳敗越王句踐會稽。
\\\\
  有隼集于陳廷而死,楛矢貫之,石砮,矢長尺有咫。陳湣公使使問仲尼。仲尼曰:「隼來遠矣,此肅慎之矢也。昔武王克商,通道九夷百蠻,使各以其方賄來貢,使無忘職業。於是肅慎貢楛矢石砮,長尺有咫。先王欲昭其令德,以肅慎矢分大姬,配虞胡公而封諸陳。分同姓以珍玉,展親;分異姓以遠職,使無忘服。故分陳以肅慎矢。」試求之故府,果得之。
\\\\
  孔子居陳三歲,會晉楚爭彊,更伐陳,及吳侵陳,陳常被寇。孔子曰:「歸與!歸與!吾黨之小子狂簡,進取不忘其初。」於是孔子去陳。
\\\\
  過蒲,會公叔氏以蒲畔,蒲人止孔子。弟子有公良孺者,以私車五乘從孔子。其為人長賢,有勇力,謂曰:「吾昔從夫子遇難於匡,今又遇難於此,命也已。吾與夫子再罹難,寧鬬而死。」鬬甚疾。蒲人懼,謂孔子曰:「苟毋適衛,吾出子。」與之盟,出孔子東門。孔子遂適衛。子貢曰:「盟可負耶?」孔子曰:「要盟也,神不聽。」
\\\\
  衛靈公聞孔子來,喜,郊迎。問曰:「蒲可伐乎?」對曰:「可。」靈公曰:「吾大夫以為不可。今蒲,衛之所以待晉楚也,以衛伐之,無乃不可乎?」孔子曰:「其男子有死之志,婦人有保西河之志。吾所伐者不過四五人。」靈公曰:「善。」然不伐蒲。
\\\\
  靈公老,怠於政,不用孔子。孔子喟然歎曰:「苟有用我者,朞月而已,三年有成。」孔子行。
\\\\
  佛肸為中牟宰。趙簡子攻范、中行,伐中牟。佛肸畔,使人召孔子。孔子欲往。子路曰:「由聞諸夫子,『其身親為不善者,君子不入也』。今佛肸親以中牟畔,子欲往,如之何?」孔子曰:「有是言也。不曰堅乎,磨而不磷;不曰白乎,涅而不淄。我豈匏瓜也哉,焉能系而不食?」
\\\\
  孔子擊磬。有荷蕢而過門者,曰:「有心哉,擊磬乎!硜硜乎,莫己知也夫而已矣!」
\\\\
  孔子學鼓琴師襄子,十日不進。師襄子曰:「可以益矣。」孔子曰:「丘已習其曲矣,未得其數也。」有間,曰:「已習其數,可以益矣。」孔子曰:「丘未得其志也。」有間,曰:「已習其志,可以益矣。」孔子曰:「丘未得其為人也。」有間,[曰]有所穆然深思焉,有所怡然高望而遠志焉。曰:「丘得其為人,黯然而黑,幾然而長,眼如望羊,如王四國,非文王其誰能為此也!」師襄子辟席再拜,曰:「師蓋云文王操也。」
\\\\
  孔子既不得用於衛,將西見趙簡子。至於河而聞竇鳴犢、舜華之死也,臨河而嘆曰:「美哉水,洋洋乎!丘之不濟此,命也夫!」子貢趨而進曰:「敢問何謂也?」孔子曰:「竇鳴犢,舜華,晉國之賢大夫也。趙簡子未得志之時,須此兩人而后從政;及其已得志,殺之乃從政。丘聞之也:刳胎殺夭,則麒麟不至郊;竭澤涸漁,則蛟龍不合陰陽;覆巢毀卵,則鳳皇不翔。何則?君子諱傷其類也。夫鳥獸之於不義也尚知辟之,而況乎丘哉!」乃還息乎陬鄉,作為陬操以哀之。而反乎衛,入主蘧伯玉家。
\\\\
  他日,靈公問兵陳。孔子曰:「俎豆之事則嘗聞之,軍旅之事未之學也。」明日,與孔子語,見蜚鴈,仰視之,色不在孔子。孔子遂行,復如陳。
\\\\
  夏,衛靈公卒,立孫輒,是為衛出公。六月,趙鞅內太子蒯聵于戚。陽虎使太子絻,八人衰絰,偽自衛迎者,哭而入,遂居焉。冬,蔡遷于州來。是歲魯哀公三年,而孔子年六十矣。齊助衛圍戚,以衛太子蒯聵在故也。
\\\\
  夏,魯桓釐廟燔,南宮敬叔救火。孔子在陳,聞之,曰:「災必於桓釐廟乎?」已而果然。
\\\\
  秋,季桓子病,輦而見魯城,喟然嘆曰:「昔此國幾興矣,以吾獲罪於孔子,故不興也。」顧謂其嗣康子曰:「我即死,若必相魯;相魯,必召仲尼。」後數日,桓子卒,康子代立。已葬,欲召仲尼。公之魚曰:「昔吾先君用之不終,終為諸侯笑。今又用之,不能終,是再為諸侯笑。」康子曰:「則誰召而可?」曰:「必召冉求。」於是使使召冉求。冉求將行,孔子曰:「魯人召求,非小用之,將大用之也。」是日,孔子曰:「歸乎歸乎!吾黨之小子狂簡,斐然成章,吾不知所以裁之。」子貢知孔子思歸,送冉求,因誡曰「即用,以孔子為招」云。
\\\\
  冉求既去,明年,孔子自陳遷于蔡。蔡昭公將如吳,吳召之也。前昭公欺其臣遷州來,後將往,大夫懼復遷,公孫翩射殺昭公。楚侵蔡。秋,齊景公卒。
\\\\
  明年,孔子自蔡如葉。葉公問政,孔子曰:「政在來遠附邇。」他日,葉公問孔子於子路,子路不對。孔子聞之,曰:「由,爾何不對曰『其為人也,學道不倦,誨人不厭,發憤忘食,樂以忘憂,不知老之將至』云爾。」
\\\\
  去葉,反于蔡。長沮、桀溺耦而耕,孔子以為隱者,使子路問津焉。長沮曰:「彼執輿者為誰?」子路曰:「為孔丘。」曰:「是魯孔丘與?」曰:「然。」曰:「是知津矣。」桀溺謂子路曰:「子為誰?」曰:「為仲由。」曰:「子,孔丘之徒與?」曰:「然。」桀溺曰:「悠悠者天下皆是也,而誰以易之?且與其從辟人之士,豈若從辟世之士哉!」耰而不輟。子路以告孔子,孔子憮然曰:「鳥獸不可與同群。天下有道,丘不與易也。」
\\\\
  他日,子路行,遇荷蓧丈人,曰:「子見夫子乎?」丈人曰:「四體不勤,五穀不分,孰為夫子!」植其杖而芸。子路以告,孔子曰:「隱者也。」復往,則亡。
\\\\
  孔子遷于蔡三歲,吳伐陳。楚救陳,軍于城父。聞孔子在陳蔡之間,楚使人聘孔子。孔子將往拜禮,陳蔡大夫謀曰:「孔子賢者,所刺譏皆中諸侯之疾。今者久留陳蔡之間,諸大夫所設行皆非仲尼之意。今楚,大國也,來聘孔子。孔子用於楚,則陳蔡用事大夫危矣。」於是乃相與發徒役圍孔子於野。不得行,絕糧。從者病,莫能興。孔子講誦弦歌不衰。子路慍見曰:「君子亦有窮乎?」孔子曰:「君子固窮,小人窮斯濫矣。」
\\\\
  子貢色作。孔子曰:「賜,爾以予為多學而識之者與?」曰:「然。非與?」孔子曰:「非也。予一以貫之。」
\\\\
  孔子知弟子有慍心,乃召子路而問曰:「《詩》云『匪兕匪虎,率彼曠野』。吾道非邪?吾何為於此?」子路曰:「意者吾未仁邪?人之不我信也。意者吾未知邪?人之不我行也。」孔子曰:「有是乎!由,譬使仁者而必信,安有伯夷、叔齊?使知者而必行,安有王子比干?」
\\\\
  子路出,子貢入見。孔子曰:「賜,《詩》云『匪兕匪虎,率彼曠野』。吾道非邪?吾何為於此?」子貢曰:「夫子之道至大也,故天下莫能容夫子。夫子蓋少貶焉?」孔子曰:「賜,良農能稼而不能為穡,良工能巧而不能為順。君子能脩其道,綱而紀之,統而理之,而不能為容。今爾不脩爾道而求為容。賜,而志不遠矣!」
\\\\
  子貢出,顏回入見。孔子曰:「回,《詩》云『匪兕匪虎,率彼曠野』。吾道非邪?吾何為於此?」顏回曰:「夫子之道至大,故天下莫能容。雖然,夫子推而行之,不容何病,不容然後見君子!夫道之不修也,是吾醜也。夫道既已大修而不用,是有國者之醜也。不容何病,不容然後見君子!」孔子欣然而笑曰:「有是哉顏氏之子!使爾多財,吾為爾宰。」
\\\\
  於是使子貢至楚。楚昭王興師迎孔子,然後得免。
\\\\
  昭王將以書社地七百里封孔子。楚令尹子西曰:「王之使使諸侯有如子貢者乎?」曰:「無有。」「王之輔相有如顏回者乎?」曰:「無有。」「王之將率有如子路者乎?」曰:「無有。」「王之官尹有如宰予者乎?」曰:「無有。」「且楚之祖封於周,號為子男五十里。今孔丘述三五之法,明周召之業,王若用之,則楚安得世世堂堂方數千里乎?夫文王在豐,武王在鎬,百里之君卒王天下。今孔丘得據土壤,賢弟子為佐,非楚之福也。」昭王乃止。其秋,楚昭王卒于城父。
\\\\
  楚狂接輿歌而過孔子曰:「鳳兮!鳳兮!何德之衰?往者不可諫兮,來者猶可追也!已而,已而!今之從政者殆而!」孔子下,欲與之言。趨而去,弗得與之言。
\\\\
  於是孔子自楚反乎衛。是歲也,孔子年六十三,而魯哀公六年也。
\\\\
  其明年,吳與魯會繒,徵百牢。太宰嚭召季康子。康子使子貢往,然後得已。
\\\\
  孔子曰:「魯衛之政,兄弟也。」是時,衛君輒父不得立,在外,諸侯數以為讓。而孔子弟子多仕於衛,衛君欲得孔子為政。子路曰:「衛君待子而為政,子將奚先?」孔子曰:「必也正名乎!」子路曰:「有是哉,子之迂也!何其正也?」孔子曰:「野哉由也!夫名不正則言不順,言不順則事不成,事不成則禮樂不興,禮樂不興則刑罰不中,刑罰不中則民無所錯手足矣。夫君子為之必可名,言之必可行。君子於其言,無所苟而已矣。」
\\\\
  其明年,冉有為季氏將師,與齊戰於郎,克之。季康子曰:「子之於軍旅,學之乎?性之乎?」冉有曰:「學之於孔子。」季康子曰:「孔子何如人哉?」對曰:「用之有名;播之百姓,質諸鬼神而無憾。求之至於此道,雖累千社,夫子不利也。」康子曰:「我欲召之,可乎?」對曰:「欲召之,則毋以小人固之,則可矣。」而衛孔文子將攻太叔,問策於仲尼。仲尼辭不知,退而命載而行,曰:「鳥能擇木,木豈能擇鳥乎!」文子固止。會季康子逐公華、公賓、公林,以幣迎孔子,孔子歸魯。
\\\\
  孔子之去魯凡十四歲而反乎魯。
\\\\
  魯哀公問政,對曰:「政在選臣。」季康子問政,曰:「舉直錯諸枉,則枉者直。」康子患盜,孔子曰:「苟子之不欲,雖賞之不竊。」然魯終不能用孔子,孔子亦不求仕。
\\\\
  孔子之時,周室微而禮樂廢,詩書缺。追跡三代之禮,序書傳,上紀唐虞之際,下至秦繆,編次其事。曰:「夏禮吾能言之,杞不足徵也。殷禮吾能言之,宋不足徵也。足,則吾能徵之矣。」觀殷夏所損益,曰:「後雖百世可知也,以一文一質。周監二代,郁郁乎文哉。吾從周。」故《書傳》、《禮記》自孔氏。
\\\\
  孔子語魯大師:「樂其可知也。始作翕如,縱之純如,皦如,繹如也,以成。」「吾自衛反魯,然後樂正,雅頌各得其所。」
\\\\
  古者詩三千餘篇,及至孔子,去其重,取可施於禮義,上采契后稷,中述殷周之盛,至幽厲之缺,始於衽席,故曰「關雎之亂以為風始,鹿鳴為小雅始,文王為大雅始,清廟為頌始」。三百五篇孔子皆弦歌之,以求合韶武雅頌之音。禮樂自此可得而述,以備王道,成六藝。
\\\\
  孔子晚而喜易,序彖、繫、象、說卦、文言。讀易,韋編三絕。曰:「假我數年,若是,我於易則彬彬矣。」
\\\\
  孔子以詩書禮樂教,弟子蓋三千焉,身通六藝者七十有二人。如顏濁鄒之徒,頗受業者甚眾。
\\\\
  孔子以四教:文,行,忠,信。絕四:毋意,毋必,毋固,毋我。所慎:齊,戰,疾。子罕言利與命與仁。不憤不啟,舉一隅不以三隅反,則弗復也。
\\\\
  其於鄉黨,恂恂似不能言者。其於宗廟朝廷,辯辯言,唯謹爾。朝,與上大夫言,誾誾如也;與下大夫言,侃侃如也。
\\\\
  入公門,鞠躬如也;趨進,翼如也。君召使儐,色勃如也。君命召,不俟駕行矣。
\\\\
  魚餒,肉敗,割不正,不食。席不正,不坐。食於有喪者之側,未嘗飽也。
\\\\
  是日哭,則不歌。見齊衰、瞽者,雖童子必變。
\\\\
  「三人行,必得我師。」「德之不脩,學之不講,聞義不能徙,不善不能改,是吾憂也。」使人歌,善,則使復之,然后和之。
\\\\
  子不語:怪,力,亂,神。
\\\\
  子貢曰:「夫子之文章,可得聞也。夫子言天道與性命,弗可得聞也已。」顏淵喟然嘆曰:「仰之彌高,鑽之彌堅。瞻之在前,忽焉在後。夫子循循然善誘人,博我以文,約我以禮,欲罷不能。既竭我才,如有所立,卓爾。雖欲從之,蔑由也已。」達巷黨人[童子]曰:「大哉孔子,博學而無所成名。」子聞之曰:「我何執?執御乎?執射乎?我執御矣。」牢曰:「子云『不試,故藝』。」
\\\\
  魯哀公十四年春,狩大野。叔孫氏車子鉏商獲獸,以為不祥。仲尼視之,曰:「麟也。」取之。曰:「河不出圖,雒不出書,吾已矣夫!」顏淵死,孔子曰:「天喪予!」及西狩見麟,曰:「吾道窮矣!」喟然嘆曰:「莫知我夫!」子貢曰:「何為莫知子?」子曰:「不怨天,不尤人,下學而上達,知我者其天乎!」
\\\\
  「不降其志,不辱其身,伯夷、叔齊乎!」謂「柳下惠、少連降志辱身矣」。謂「虞仲、夷逸隱居放言,行中清,廢中權」。「我則異於是,無可無不可。」
\\\\
  子曰:「弗乎弗乎,君子病沒世而名不稱焉。吾道不行矣,吾何以自見於後世哉?」乃因史記作春秋,上至隱公,下訖哀公十四年,十二公。據魯,親周,故殷,運之三代。約其文辭而指博。故吳楚之君自稱王,而春秋貶之曰「子」;踐土之會實召周天子,而春秋諱之曰「天王狩於河陽」:推此類以繩當世。貶損之義,後有王者舉而開之。春秋之義行,則天下亂臣賊子懼焉。
\\\\
  孔子在位聽訟,文辭有可與人共者,弗獨有也。至於為春秋,筆則筆,削則削,子夏之徒不能贊一辭。弟子受春秋,孔子曰:「後世知丘者以春秋,而罪丘者亦以春秋。」
\\\\
  明歲,子路死於衛。孔子病,子貢請見。孔子方負杖逍遙於門,曰:「賜,汝來何其晚也?」孔子因歎,歌曰:「太山壞乎!梁柱摧乎!哲人萎乎!」因以涕下。謂子貢曰:「天下無道久矣,莫能宗予。夏人殯於東階,周人於西階,殷人兩柱閒。昨暮予夢坐奠兩柱之閒,予始殷人也。」後七日卒。
\\\\
  孔子年七十三,以魯哀公十六年四月己丑卒。
\\\\
  哀公誄之曰:「旻天不弔,不愸遺一老,俾屏余一人以在位,煢煢余在疚。嗚呼哀哉!尼父,毋自律!」子貢曰:「君其不沒於魯乎!夫子之言曰:『禮失則昬,名失則愆。失志為昬,失所為愆。』生不能用,死而誄之,非禮也。稱『余一人』,非名也。」
\\\\
  孔子葬魯城北泗上,弟子皆服三年。三年心喪畢,相訣而去,則哭,各復盡哀;或復留。唯子貢廬於冢上,凡六年,然後去。弟子及魯人往從冢而家者百有餘室,因命曰孔里。魯世世相傳以歲時奉祠孔子冢,而諸儒亦講禮鄉飲大射於孔子冢。孔子冢大一頃。故所居堂弟子內,後世因廟藏孔子衣冠琴車書,至于漢二百餘年不絕。高皇帝過魯,以太牢祠焉。諸侯卿相至,常先謁然後從政。
\\\\
  孔子生鯉,字伯魚。伯魚年五十,先孔子死。
\\\\
  伯魚生伋,字子思,年六十二。嘗困於宋。子思作中庸。
\\\\
  子思生白,字子上,年四十七。子上生求,字子家,年四十五。子家生箕,字子京,年四十六。子京生穿,字子高,年五十一。子高生子慎,年五十七,嘗為魏相。
\\\\
  子慎生鮒,年五十七,為陳王涉博士,死於陳下。
\\\\
  鮒弟子襄,年五十七。嘗為孝惠皇帝博士,遷為長沙太守。長九尺六寸。
\\\\
  子襄生忠,年五十七。忠生武,武生延年及安國。安國為今皇帝博士,至臨淮太守,蚤卒。安國生卬,卬生驩。
\\\\
  太史公曰:《詩》有之:「高山仰止,景行行止。」雖不能至,然心鄉往之。余讀孔氏書,想見其為人。適魯,觀仲尼廟堂車服禮器,諸生以時習禮其家,余祗回留之不能去云。天下君王至於賢人眾矣,當時則榮,沒則已焉。孔子布衣,傳十餘世,學者宗之。自天子王侯,中國言六藝者折中於夫子,可謂至聖矣!


\section{留侯世家}
  留侯張良者,其先韓人也。大父開地,相韓昭侯、宣惠王、襄哀王。父平,相釐王、悼惠王。悼惠王二十三年,平卒。卒二十歲,秦滅韓。良年少,未宦事韓。韓破,良家僮三百人,弟死不葬,悉以家財求客刺秦王,為韓報仇,以大父、父五世相韓故。
\\\\
  良嘗學禮淮陽。東見倉海君。得力士,為鐵椎重百二十斤。秦皇帝東游,良與客狙擊秦皇帝博浪沙中,誤中副車。秦皇帝大怒,大索天下,求賊甚急,為張良故也。良乃更名姓,亡匿下邳。
\\\\
  良嘗閒從容步游下邳圯上,有一老父,衣褐,至良所,直墮其履圯下,顧謂良曰:「孺子,下取履!」良鄂然,欲毆之。為其老,彊忍,下取履。父曰:「履我!」良業為取履,因長跪履之。父以足受,笑而去。良殊大驚,隨目之。父去里所,復還,曰:「孺子可教矣。後五日平明,與我會此。」良因怪之,跪曰:「諾。」五日平明,良往。父已先在,怒曰:「與老人期,後,何也?」去,曰:「後五日早會。」五日雞鳴,良往。父又先在,復怒曰:「後,何也?」去,曰:「後五日復早來。」五日,良夜未半往。有頃,父亦來,喜曰:「當如是。」出一編書,曰:「讀此則為王者師矣。後十年興。十三年孺子見我濟北,穀城山下黃石即我矣。」遂去,無他言,不復見。旦日視其書,乃太公兵法也。良因異之,常習誦讀之。
\\\\
  居下邳,為任俠。項伯常殺人,從良匿。
\\\\
  後十年,陳涉等起兵,良亦聚少年百餘人。景駒自立為楚假王,在留。良欲往從之,道還沛公。沛公將數千人,略地下邳西,遂屬焉。沛公拜良為廄將。良數以太公兵法說沛公,沛公善之,常用其策。良為他人者,皆不省。良曰:「沛公殆天授。」故遂從之,不去見景駒。
\\\\
  及沛公之薛,見項梁。項梁立楚懷王。良乃說項梁曰:「君已立楚後,而韓諸公子橫陽君成賢,可立為王,益樹黨。」項梁使良求韓成,立以為韓王。以良為韓申徒,與韓王將千餘人西略韓地,得數城,秦輒復取之,往來為游兵潁川。
\\\\
  沛公之從雒陽南出轘轅,良引兵從沛公,下韓十餘城,擊破楊熊軍。沛公乃令韓王成留守陽翟,與良俱南,攻下宛,西入武關。沛公欲以兵二萬人擊秦嶢下軍,良說曰:「秦兵尚彊,未可輕。臣聞其將屠者子,賈豎易動以利。願沛公且留壁,使人先行,為五萬人具食,益為張旗幟諸山上,為疑兵,令酈食其持重寶啗秦將。」秦將果畔,欲連和俱西襲咸陽,沛公欲聽之。良曰:「此獨其將欲叛耳,恐士卒不從。不從必危,不如因其解擊之。」沛公乃引兵擊秦軍,大破之。(遂)[逐]北至藍田,再戰,秦兵竟敗。遂至咸陽,秦王子嬰降沛公。
\\\\
  沛公入秦宮,宮室帷帳狗馬重寶婦女以千數,意欲留居之。樊噲諫沛公出舍,沛公不聽。良曰:「夫秦為無道,故沛公得至此。夫為天下除殘賊,宜縞素為資。今始入秦,即安其樂,此所謂『助桀為虐』。且『忠言逆耳利於行,毒藥苦口利於病』,願沛公聽樊噲言。」沛公乃還軍霸上。
\\\\
  項羽至鴻門下,欲擊沛公,項伯乃夜馳入沛公軍,私見張良,欲與俱去。良曰:「臣為韓王送沛公,今事有急,亡去不義。」乃具以語沛公。沛公大驚,曰:「為將奈何?」良曰:「沛公誠欲倍項羽邪?」沛公曰:「鯫生教我距關無內諸侯,秦地可盡王,故聽之。」良曰:「沛公自度能卻項羽乎?」沛公默然良久,曰:「固不能也。今為奈何?」良乃固要項伯。項伯見沛公。沛公與飲為壽,結賓婚。令項伯具言沛公不敢倍項羽,所以距關者,備他盜也。及見項羽後解,語在項羽事中。
\\\\
  漢元年正月,沛公為漢王,王巴蜀。漢王賜良金百鎰,珠二斗,良具以獻項伯。漢王亦因令良厚遺項伯,使請漢中地。項王乃許之,遂得漢中地。漢王之國,良送至襃中,遣良歸韓。良因說漢王曰:「王何不燒絕所過棧道,示天下無還心,以固項王意。」乃使良還。行,燒絕棧道。
\\\\
  良至韓,韓王成以良從漢王故,項王不遣成之國,從與俱東。良說項王曰:「漢王燒絕棧道,無還心矣。」乃以齊王田榮反,書告項王。項王以此無西憂漢心,而發兵北擊齊。
\\\\
  項王竟不肯遣韓王,乃以為侯,又殺之彭城。良亡,間行歸漢王,漢王亦已還定三秦矣。復以良為成信侯,從東擊楚。至彭城,漢敗而還。至下邑,漢王下馬踞鞍而問曰:「吾欲捐關以東等棄之,誰可與共功者?」良進曰:「九江王黥布,楚梟將,與項王有郄;彭越與齊王田榮反梁地:此兩人可急使。而漢王之將獨韓信可屬大事,當一面。即欲捐之,捐之此三人,則楚可破也。」漢王乃遣隨何說九江王布,而使人連彭越。及魏王豹反,使韓信將兵擊之,因舉燕、代、齊、趙。然卒破楚者,此三人力也。
\\\\
  張良多病,未嘗特將也,常為畫策,時時從漢王。
\\\\
  漢三年,項羽急圍漢王滎陽,漢王恐憂,與酈食其謀橈楚權。食其曰:「昔湯伐桀,封其後於杞。武王伐紂,封其後於宋。今秦失德棄義,侵伐諸侯社稷,滅六國之後,使無立錐之地。陛下誠能復立六國後世,畢已受印,此其君臣百姓必皆戴陛下之德,莫不鄉風慕義,願為臣妾。德義已行,陛下南鄉稱霸,楚必斂衽而朝。」漢王曰:「善。趣刻印,先生因行佩之矣。」
\\\\
  食其未行,張良從外來謁。漢王方食,曰:「子房前!客有為我計橈楚權者。」其以酈生語告,曰:「於子房何如?」良曰:「誰為陛下畫此計者?陛下事去矣。」漢王曰:「何哉?」張良對曰:「臣請藉前箸為大王籌之。」曰:「昔者湯伐桀而封其後於杞者,度能制桀之死命也。今陛下能制項籍之死命乎?」曰:「未能也。」「其不可一也。武王伐紂封其後於宋者,度能得紂之頭也。今陛下能得項籍之頭乎?」曰:「未能也。」「其不可二也。武王入殷,表商容之閭,釋箕子之拘,封比干之墓。今陛下能封聖人之墓,表賢者之閭,式智者之門乎?」曰:「未能也。」「其不可三也。發鉅橋之粟,散鹿臺之錢,以賜貧窮。今陛下能散府庫以賜貧窮乎?」曰:「未能也。」「其不可四矣。殷事已畢,偃革為軒,倒置干戈,覆以虎皮,以示天下不復用兵。今陛下能偃武行文,不復用兵乎?」曰:「未能也。」「其不可五矣。休馬華山之陽,示以無所為。今陛下能休馬無所用乎?」曰:「未能也。」「其不可六矣。放牛桃林之陰,以示不復輸積。今陛下能放牛不復輸積乎?」曰:「未能也。」「其不可七矣。且天下游士離其親戚,棄墳墓,去故舊,從陛下游者,徒欲日夜望咫尺之地。今復六國,立韓、魏、燕、趙、齊、楚之後,天下游士各歸事其主,從其親戚,反其故舊墳墓,陛下與誰取天下乎?其不可八矣。且夫楚唯無彊,六國立者復橈而從之,陛下焉得而臣之?誠用客之謀,陛下事去矣。」漢王輟食吐哺,罵曰:「豎儒,幾敗而公事!」令趣銷印。
\\\\
  漢四年,韓信破齊而欲自立為齊王,漢王怒。張良說漢王,漢王使良授齊王信印,語在淮陰事中。
\\\\
  其秋,漢王追楚至陽夏南,戰不利而壁固陵,諸侯期不至。良說漢王,漢王用其計,諸侯皆至。語在項籍事中。
\\\\
  漢六年正月,封功臣。良未嘗有戰鬬功,高帝曰:「運籌策帷帳中,決勝千里外,子房功也。自擇齊三萬戶。」良曰:「始臣起下邳,與上會留,此天以臣授陛下。陛下用臣計,幸而時中,臣願封留足矣,不敢當三萬戶。」乃封張良為留侯,與蕭何等俱封。
\\\\
  [六年]上已封大功臣二十餘人,其餘日夜爭功不決,未得行封。上在雒陽南宮,從複道望見諸將往往相與坐沙中語。上曰:「此何語?」留侯曰:「陛下不知乎?此謀反耳。」上曰:「天下屬安定,何故反乎?」留侯曰:「陛下起布衣,以此屬取天下,今陛下為天子,而所封皆蕭、曹故人所親愛,而所誅者皆生平所仇怨。今軍吏計功,以天下不足遍封,此屬畏陛下不能盡封,恐又見疑平生過失及誅,故即相聚謀反耳。」上乃憂曰:「為之奈何?」留侯曰:「上平生所憎,群臣所共知,誰最甚者?」上曰:「雍齒與我故,數嘗窘辱我。我欲殺之,為其功多,故不忍。」留侯曰:「今急先封雍齒以示群臣,群臣見雍齒封,則人人自堅矣。」於是上乃置酒,封雍齒為什方侯,而急趣丞相、御史定功行封。群臣罷酒,皆喜曰:「雍齒尚為侯,我屬無患矣。」
\\\\
  劉敬說高帝曰:「都關中。」上疑之。左右大臣皆山東人,多勸上都雒陽:「雒陽東有成皋,西有殽黽,倍河,向伊雒,其固亦足恃。」留侯曰:「雒陽雖有此固,其中小,不過數百里,田地薄,四面受敵,此非用武之國也。夫關中左殽函,右隴蜀,沃野千里,南有巴蜀之饒,北有胡苑之利,阻三面而守,獨以一面東制諸侯。諸侯安定,河渭漕輓天下,西給京師;諸侯有變,順流而下,足以委輸。此所謂金城千里,天府之國也,劉敬說是也。」於是高帝即日駕,西都關中。
\\\\
  留侯從入關。留侯性多病,即道引不食穀,杜門不出歲餘。
\\\\
  上欲廢太子,立戚夫人子趙王如意。大臣多諫爭,未能得堅決者也。呂后恐,不知所為。人或謂呂后曰:「留侯善畫計筴,上信用之。」呂后乃使建成侯呂澤劫留侯,曰:「君常為上謀臣,今上欲易太子,君安得高枕而臥乎?」留侯曰:「始上數在困急之中,幸用臣筴。今天下安定,以愛欲易太子,骨肉之間,雖臣等百餘人何益。」呂澤彊要曰:「為我畫計。」留侯曰:「此難以口舌爭也。顧上有不能致者,天下有四人。四人者年老矣,皆以為上慢侮人,故逃匿山中,義不為漢臣。然上高此四人。今公誠能無愛金玉璧帛,令太子為書,卑辭安車,因使辯士固請,宜來。來,以為客,時時從入朝,令上見之,則必異而問之。問之,上知此四人賢,則一助也。」於是呂后令呂澤使人奉太子書,卑辭厚禮,迎此四人。四人至,客建成侯所。
\\\\
  漢十一年,黥布反,上病,欲使太子將,往擊之。四人相謂曰:「凡來者,將以存太子。太子將兵,事危矣。」乃說建成侯曰:「太子將兵,有功則位不益太子;無功還,則從此受禍矣。且太子所與俱諸將,皆嘗與上定天下梟將也,今使太子將之,此無異使羊將狼也,皆不肯為盡力,其無功必矣。臣聞『母愛者子抱』,今戚夫人日夜待御,趙王如意常抱居前,上曰『終不使不肖子居愛子之上』,明乎其代太子位必矣。君何不急請呂后承間為上泣言:『黥布,天下猛將也,善用兵,今諸將皆陛下故等夷,乃令太子將此屬,無異使羊將狼,莫肯為用,且使布聞之,則鼓行而西耳。上雖病,彊載輜車,臥而護之,諸將不敢不盡力。上雖苦,為妻子自彊。』」於是呂澤立夜見呂后,呂后承間為上泣涕而言,如四人意。上曰:「吾惟豎子固不足遣,而公自行耳。」於是上自將兵而東,群臣居守,皆送至灞上。留侯病,自彊起,至曲郵,見上曰:「臣宜從,病甚。楚人剽疾,願上無與楚人爭鋒。」因說上曰:「令太子為將軍,監關中兵。」上曰:「子房雖病,彊臥而傅太子。」是時叔孫通為太傅,留侯行少傅事。
\\\\
  漢十二年,上從擊破布軍歸,疾益甚,愈欲易太子。留侯諫,不聽,因疾不視事。叔孫太傅稱說引古今,以死爭太子。上詳許之,猶欲易之。及燕,置酒,太子侍。四人從太子,年皆八十有餘,鬚眉皓白,衣冠甚偉。上怪之,問曰:「彼何為者?」四人前對,各言名姓,曰東園公,角里先生,綺里季,夏黃公。上乃大驚,曰:「吾求公數歲,公辟逃我,今公何自從吾兒游乎?」四人皆曰:「陛下輕士善罵,臣等義不受辱,故恐而亡匿。竊聞太子為人仁孝,恭敬愛士,天下莫不延頸欲為太子死者,故臣等來耳。」上曰:「煩公幸卒調護太子。」
\\\\
  四人為壽已畢,趨去。上目送之,召戚夫人指示四人者曰:「我欲易之,彼四人輔之,羽翼已成,難動矣。呂后真而主矣。」戚夫人泣,上曰:「為我楚舞,吾為若楚歌。」歌曰:「鴻鴈高飛,一舉千里。羽翮已就,橫絕四海。橫絕四海,當可奈何!雖有矰繳,尚安所施!」歌數闋,戚夫人噓唏流涕,上起去,罷酒。竟不易太子者,留侯本招此四人之力也。
\\\\
  留侯從上擊代,出奇計馬邑下,及立蕭何相國,所與上從容言天下事甚眾,非天下所以存亡,故不著。留侯乃稱曰:「家世相韓,及韓滅,不愛萬金之資,為韓報讐彊秦,天下振動。今以三寸舌為帝者師,封萬戶,位列侯,此布衣之極,於良足矣。願棄人間事,欲從赤松子游耳。」乃學辟穀,道引輕身。會高帝崩,呂后德留侯,乃彊食之,曰:「人生一世間,如白駒過隙,何至自苦如此乎!」留侯不得已,彊聽而食。
\\\\
  後八年卒,謚為文成侯。子不疑代侯。
\\\\
  子房始所見下邳圯上老父與太公書者,後十三年從高帝過濟北,果見穀城山下黃石,取而葆祠之。留侯死,并葬黃石(冢)。每上冢伏臘,祠黃石。
\\\\
  留侯不疑,孝文帝五年坐不敬,國除。
\\\\
  太史公曰:學者多言無鬼神,然言有物。至如留侯所見老父予書,亦可怪矣。高祖離困者數矣,而留侯常有功力焉,豈可謂非天乎?上曰:「夫運籌筴帷帳之中,決勝千里外,吾不如子房。」余以為其人計魁梧奇偉,至見其圖,狀貌如婦人好女。蓋孔子曰:「以貌取人,失之子羽。」留侯亦云。


\section{陳丞相世家}

  陳丞相平者,陽武戶牖鄉人也。少時家貧,好讀書,有田三十畝,獨與兄伯居。伯常耕田,縱平使游學。平為人長[大]美色。人或謂陳平曰:「貧何食而肥若是?」其嫂嫉平之不視家生產,曰:「亦食糠覈耳。有叔如此,不如無有。」伯聞之,逐其婦而棄之。
\\\\
  及平長,可娶妻,富人莫肯與者,貧者平亦恥之。久之,戶牖富人有張負,張負女孫五嫁而夫輒死,人莫敢娶。平欲得之。邑中有喪,平貧,侍喪,以先往後罷為助。張負既見之喪所,獨視偉平,平亦以故後去。負隨平至其家,家乃負郭窮巷,以獘席為門,然門外多有長者車轍。張負歸,謂其子仲曰:「吾欲以女孫予陳平。」張仲曰:「平貧不事事,一縣中盡笑其所為,獨柰何予女乎?」負曰:「人固有好美如陳平而長貧賤者乎?」卒與女。為平貧,乃假貸幣以聘,予酒肉之資以內婦。負誡其孫曰:「毋以貧故,事人不謹。事兄伯如事父,事嫂如母。」平既娶張氏女,齎用益饒,游道日廣。
\\\\
  裏中社,平為宰,分肉食甚均。父老曰:「善,陳孺子之為宰!」平曰:「嗟乎,使平得宰天下,亦如是肉矣!」
\\\\
  陳涉起而王陳,使周市略定魏地,立魏咎為魏王,與秦軍相攻於臨濟。陳平固已前謝其兄伯,從少年往事魏王咎於臨濟。魏王以為太仆。說魏王不聽,人或讒之,陳平亡去。
\\\\
  久之,項羽略地至河上,陳平往歸之,從入破秦,賜平爵卿。項羽之東王彭城也,漢王還定三秦而東,殷王反楚。項羽乃以平為信武君,將魏王咎客在楚者以往,擊降殷王而還。項王使項悍拜平為都尉,賜金二十溢。居無何,漢王攻下殷(王)。項王怒,將誅定殷者將吏。陳平懼誅,乃封其金與印,使使歸項王,而平身閒行杖劍亡。渡河,船人見其美丈夫獨行,疑其亡將,要中當有金玉寶器,目之,欲殺平。平恐,乃解衣躶而佐刺船,船人知其無有,乃止。
\\\\
  平遂至修武降漢,因魏無知求見漢王,漢王召入。是時萬石君奮為漢王中涓,受平謁,入見平。平等七人俱進,賜食。王曰:「罷,就舍矣。」平曰:「臣為事來,所言不可以過今日。」於是漢王與語而說之,問曰:「子之居楚何官?」曰:「為都尉。」是日乃拜平為都尉,使為參乘,典護軍。諸將盡讙,曰:「大王一日得楚之亡卒,未知其高下,而即與同載,反使監護軍長者!」漢王聞之,愈益幸平。遂與東伐項王。至彭城,為楚所敗。引而還,收散兵至滎陽,以平為亞將,屬於韓王信,軍廣武。
\\\\
  絳侯、灌嬰等咸讒陳平曰:「平雖美丈夫,如冠玉耳,其中未必有也。臣聞平居家時,盜其嫂;事魏不容,亡歸楚;歸楚不中,又亡歸漢。今日大王尊官之,令護軍。臣聞平受諸將金,金多者得善處,金少者得惡處。平,反覆亂臣也,願王察之。」漢王疑之,召讓魏無知。無知曰:「臣所言者,能也;陛下所問者,行也。今有尾生、孝己之行而無益處於勝負之數,陛下何暇用之乎?楚漢相距,臣進奇謀之士,顧其計誠足以利國家不耳。且盜嫂受金又何足疑乎?」漢王召讓平曰:「先生事魏不中,遂事楚而去,今又從吾游,信者固多心乎?」平曰:「臣事魏王,魏王不能用臣說,故去事項王。項王不能信人,其所任愛,非諸項即妻之昆弟,雖有奇士不能用,平乃去楚。聞漢王之能用人,故歸大王。臣躶身來,不受金無以為資。誠臣計畫有可采者,(顧)[願]大王用之;使無可用者,金具在,請封輸官,得請骸骨。」漢王乃謝,厚賜,拜為護軍中尉,盡護諸將。諸將乃不敢復言。
\\\\
  其後,楚急攻,絕漢甬道,圍漢王於滎陽城。久之,漢王患之,請割滎陽以西以和。項王不聽。漢王謂陳平曰:「天下紛紛,何時定乎?」陳平曰:「項王為人,恭敬愛人,士之廉節好禮者多歸之。至於行功爵邑,重之,士亦以此不附。今大王慢而少禮,士廉節者不來;然大王能饒人以爵邑,士之頑鈍嗜利無恥者亦多歸漢。誠各去其兩短,襲其兩長,天下指麾則定矣。然大王恣侮人,不能得廉節之士。顧楚有可亂者,彼項王骨鯁之臣亞父、鐘離眛、龍且、周殷之屬,不過數人耳。大王誠能出捐數萬斤金,行反閒,閒其君臣,以疑其心,項王為人意忌信讒,必內相誅。漢因舉兵而攻之,破楚必矣。」漢王以為然,乃出黃金四萬斤,與陳平,恣所為,不問其出入。
\\\\
  陳平既多以金縱反閒於楚軍,宣言諸將鐘離眛等為項王將,功多矣,然而終不得裂地而王,欲與漢為一,以滅項氏而分王其地。項羽果意不信鐘離眛等。項王既疑之,使使至漢。漢王為太牢具,舉進。見楚使,即詳驚曰:「吾以為亞父使,乃項王使!」復持去,更以惡草具進楚使。楚使歸,具以報項王。項王果大疑亞父。亞父欲急攻下滎陽城,項王不信,不肯聽。亞父聞項王疑之,乃怒曰:「天下事大定矣,君王自為之!願請骸骨歸!」歸未至彭城,疽發背而死。陳平乃夜出女子二千人滎陽城東門,楚因擊之,陳平乃與漢王從城西門夜出去。遂入關,收散兵復東。
\\\\
  其明年,淮陰侯破齊,自立為齊王,使使言之漢王。漢王大怒而罵,陳平躡漢王。漢王亦悟,乃厚遇齊使,使張子房卒立信為齊王。封平以戶牖鄉。用其奇計策,卒滅楚。常以護軍中尉從定燕王臧荼。
\\\\
  漢六年,人有上書告楚王韓信反。高帝問諸將,諸將曰:「亟發兵阬豎子耳。」高帝默然。問陳平,平固辭謝,曰:「諸將云何?」上具告之。陳平曰:「人之上書言信反,有知之者乎?」曰:「未有。」曰:「信知之乎?」曰:「不知。」陳平曰:「陛下精兵孰與楚?」上曰:「不能過。」平曰:「陛下將用兵有能過韓信者乎?」上曰:「莫及也。」平曰:「今兵不如楚精,而將不能及,而舉兵攻之,是趣之戰也,竊為陛下危之。」上曰:「為之柰何?」平曰:「古者天子巡狩,會諸侯。南方有雲夢,陛下弟出偽游雲夢,會諸侯於陳。陳,楚之西界,信聞天子以好出游,其勢必無事而郊迎謁。謁,而陛下因禽之,此特一力士之事耳。」高帝以為然,乃發使告諸侯會陳,「吾將南游雲夢」。上因隨以行。行未至陳,楚王信果郊迎道中。高帝豫具武士,見信至,即執縛之,載後車。信呼曰:「天下已定,我固當烹!」高帝顧謂信曰:「若毋聲!而反,明矣!」武士反接之。遂會諸侯于陳,盡定楚地。還至雒陽,赦信以為淮陰侯,而與功臣剖符定封。
\\\\
  於是與平剖符,世世勿絕,為戶牖侯。平辭曰:「此非臣之功也。」上曰:「吾用先生謀計,戰勝剋敵,非功而何?」平曰:「非魏無知臣安得進?」上曰;「若子可謂不背本矣。」乃復賞魏無知。其明年,以護軍中尉從攻反者韓王信於代。卒至平城,為匈奴所圍,七日不得食。高帝用陳平奇計,使單于閼氏,圍以得開。高帝既出,其計祕,世莫得聞。
\\\\
  高帝南過曲逆,上其城,望見其屋室甚大,曰:「壯哉縣!吾行天下,獨見洛陽與是耳。」顧問御史曰:「曲逆戶口幾何?」對曰:「始秦時三萬餘戶,閒者兵數起,多亡匿,今見五千戶。」於是乃詔御史,更以陳平為曲逆侯,盡食之,除前所食戶牖。
\\\\
  其後常以護軍中尉從攻陳豨及黥布。凡六出奇計,輒益邑,凡六益封。奇計或頗祕,世莫能聞也。
\\\\
  高帝從破布軍還,病創,徐行至長安。燕王盧綰反,上使樊噲以相國將兵攻之。既行,人有短惡噲者。高帝怒曰:「噲見吾病,乃冀我死也。」用陳平謀而召絳侯周勃受詔床下,曰:「陳平亟馳傳載勃代噲將,平至軍中即斬噲頭!」二人既受詔,馳傳未至軍,行計之曰:「樊噲,帝之故人也,功多,且又乃呂后弟呂媭之夫,有親且貴,帝以忿怒故,欲斬之,則恐後悔。寧囚而致上,上自誅之。」未至軍,為壇,以節召樊噲。噲受詔,即反接載檻車,傳詣長安,而令絳侯勃代將,將兵定燕反縣。
\\\\
  平行聞高帝崩,平恐呂太后及呂媭讒怒,乃馳傳先去。逢使者詔平與灌嬰屯於滎陽。平受詔,立復馳至宮,哭甚哀,因奏事喪前。呂太后哀之,曰:「君勞,出休矣。」平畏讒之就,因固請得宿衛中。太后乃以為郎中令,曰:「傅教孝惠。」是後呂媭讒乃不得行。樊噲至,則赦復爵邑。
\\\\
  孝惠帝六年,相國曹參卒,以安國侯王陵為右丞相,陳平為左丞相。
\\\\
  王陵者,故沛人,始為縣豪,高祖微時,兄事陵。陵少文,任氣,好直言。及高祖起沛,入至咸陽,陵亦自聚黨數千人,居南陽,不肯從沛公。及漢王之還攻項籍,陵乃以兵屬漢。項羽取陵母置軍中,陵使至,則東鄉坐陵母,欲以招陵。陵母既私送使者,泣曰:「為老妾語陵,謹事漢王。漢王,長者也,無以老妾故,持二心。妾以死送使者。」遂伏劍而死。項王怒,烹陵母。陵卒從漢王定天下。以善雍齒,雍齒,高帝之仇,而陵本無意從高帝,以故晚封,為安國侯。
\\\\
  安國侯既為右丞相,二歲,孝惠帝崩。高后欲立諸呂為王,問王陵,王陵曰:「不可。」問陳平,陳平曰:「可。」呂太后怒,乃詳遷陵為帝太傅,實不用陵。陵怒,謝疾免,杜門竟不朝請,七年而卒。
\\\\
  陵之免丞相,呂太后乃徙平為右丞相,以辟陽侯審食其為左丞相。左丞相不治,常給事於中。
\\\\
  食其亦沛人。漢王之敗彭城西,楚取太上皇、呂后為質,食其以舍人侍呂后。其後從破項籍為侯,幸於呂太后。及為相,居中,百官皆因決事。
\\\\
  呂媭常以前陳平為高帝謀執樊噲,數讒曰:「陳平為相非治事,日飲醇酒,戲婦女。」陳平聞,日益甚。呂太后聞之,私獨喜。面質呂媭於陳平曰:「鄙語曰『兒婦人口不可用』,顧君與我何如耳。無畏呂媭之讒也。」
\\\\
  呂太后立諸呂為王,陳平偽聽之。及呂太后崩,平與太尉勃合謀,卒誅諸呂,立孝文皇帝,陳平本謀也。審食其免相。
\\\\
  孝文帝立,以為太尉勃親以兵誅呂氏,功多;陳平欲讓勃尊位,乃謝病。孝文帝初立,怪平病,問之。平曰:「高祖時,勃功不如臣平。及誅諸呂,臣功亦不如勃。願以右丞相讓勃。」於是孝文帝乃以絳侯勃為右丞相,位次第一;平徙為左丞相,位次第二。賜平金千斤,益封三千戶。
\\\\
  居頃之,孝文皇帝既益明習國家事,朝而問右丞相勃曰:「天下一歲決獄幾何?」勃謝曰:「不知。」問:「天下一歲錢穀出入幾何?」勃又謝不知,汗出沾背,愧不能對。於是上亦問左丞相平。平曰:「有主者。」上曰:「主者謂誰?」平曰:「陛下即問決獄,責廷尉;問錢穀,責治粟內史。」上曰:「茍各有主者,而君所主者何事也?」平謝曰:「主臣!陛下不知其駑下,使待罪宰相。宰相者,上佐天子理陰陽,順四時,下育萬物之宜,外鎮撫四夷諸侯,內親附百姓,使卿大夫各得任其職焉。」孝文帝乃稱善。右丞相大慚,出而讓陳平曰:「君獨不素教我對!」陳平笑曰:「君居其位,不知其任邪?且陛下即問長安中盜賊數,君欲彊對邪?」於是絳侯自知其能不如平遠矣。居頃之,絳侯謝病請免相,陳平專為一丞相。
\\\\
  孝文帝二年,丞相陳平卒,謚為獻侯。子共侯買代侯。二年卒,子簡侯恢代侯。二十三年卒,子何代侯。二十三年,何坐略人妻,棄市,國除。
\\\\
  始陳平曰:「我多陰謀,是道家之所禁。吾世即廢,亦已矣,終不能復起,以吾多陰禍也。」然其後曾孫陳掌以衛氏親貴戚,願得續封陳氏,然終不得。
\\\\
  太史公曰:陳丞相平少時,本好黃帝、老子之術。方其割肉俎上之時,其意固已遠矣。傾側擾攘楚魏之閒,卒歸高帝。常出奇計,救紛糾之難,振國家之患。及呂后時,事多故矣,然平竟自脫,定宗廟,以榮名終,稱賢相,豈不善始善終哉!非知謀孰能當此者乎?
