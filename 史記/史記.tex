%!TEX program = xelatex
%!TEX encoding = UTF-8

%% if you want chinese vertical, then pass argument 'landscape' to document class,
%% and uncomment the rotation,
%% and change font settings to vertical.

\documentclass[UTF8, nofont, landscape]{ctexbook} %% landscape, landscape

%% %% landscape------------------
\usepackage{atbegshi}
\AtBeginShipout{%
  \global\setbox\AtBeginShipoutBox\vbox{%
    \special{pdf: put @thispage <</Rotate 90>>}%
    \box\AtBeginShipoutBox
  }%
}%
%% %% font
\defaultCJKfontfeatures{RawFeature={vertical:+vert}}
\makeatletter
\newcommand*{\shifttext}[2]{%
  \settowidth{\@tempdima}{#2}%
  \makebox[\@tempdima]{\hspace*{#1}#2}%
}%
\makeatother
\newcommand\ytza[1]{\shifttext{-3.5pt}{\ytzfz #1}}

%% no landscape------------
% \newcommand\ytza[1]{\raisebox{2pt}{\ytzfz #1}}


%% common font settings
\setCJKmainfont[BoldFont=Adobe Heiti Std,ItalicFont=Adobe Kaiti Std]{Adobe Song Std}
\setCJKsansfont{Adobe Heiti Std}
\setCJKmonofont{Adobe Fangsong Std}
\newCJKfontfamily{\fzsongbig}{FZSongS-Extended}
\newCJKfontfamily{\arpluming}{AR PL UMing CN} %這個明細體和新明細體补充顯示Adobe缺失的漢字。但它的编码库可能比Adobe还小些。
\newCJKfontfamily{\simsun}{SimSun}
\newCJKfontfamily{\newsimsun}{NSimSun}
\DeclareTextFontCommand{\yt}{\arpluming}
\DeclareTextFontCommand{\ytz}{\simsun}
\DeclareTextFontCommand{\ytzn}{\newsimsun}
\DeclareTextFontCommand{\ytzfz}{\fzsongbig}


\renewcommand{\thepage}{\Chinese{page}}

% \usepackage{wallpaper}
% \ThisULCornerWallPaper{1}{竹.jpg}

\usepackage{geometry}
%% \newgeometry{
%%   top=50pt, bottom=50pt, left=86pt, right=86pt,
%%   headsep=5pt,
%% }
\newgeometry{
  top=50pt, bottom=50pt, left=46pt, right=46pt,
  headsep=25pt,
}
\savegeometry{mdGeo}
\loadgeometry{mdGeo}


%% \usepackage{titlesec}        % this can not be used when vertical mode
%% %% \titleformat{\chapter}[display]
%% %%             {\Huge\bfseries\centering}
%% %%             {}{0pt}{#1}
%% %% \titleformat{\chapter}[display]
%% %%   {\centering\xingkai}
%% %%   {}{0pt}{}
%% \titlespacing*{\section}{0pt}{9pt}{0pt}
%% \titlespacing*{\chapter}{0pt}{0pt}{-16pt}


\newcommand{\pageCN}{第\thepage 页}

\usepackage{fancyhdr}
%% \usepackage{ifthen}
\fancypagestyle{plain}{ % first page style
    \fancyhf{}
    %% \fancyfoot[LE,RO]{
    %%   {\fangsong \pageCN}%偶数页
    %% }
}
%% % 从第二页开始的style
%% \fancyhf{}
%% \fancyhead[LE]{\textit\pageCN\leftmark}  %\lishu 北航明德养生社\qquad 晨读本
%% \fancyfoot[LE]{}
%% \fancyfoot[RO]{\textit\pageCN\rightmark}
%% \renewcommand{\headrulewidth}{0.5bp} % 页眉线宽度
%% \pagestyle{fancy}



%% \RequirePackage{titletoc}

%% \titlecontents{chapter}[0pt]{\heiti\zihao{-4}}{\thecontentslabel\ }{}
%%               {\hspace{.5em}\titlerule*[4pt]{$\cdot$}\contentspage}
%%               \titlecontents{section}[2em]{\vspace{0.1\baselineskip}\songti\zihao{-4}}{\thecontentslabel\ }{}
%%                             {\hspace{.5em}\titlerule*[4pt]{$\cdot$}\contentspage}
%%                             \titlecontents{subsection}[4em]{\vspace{0.1\baselineskip}\songti\zihao{-4}}{\thecontentslabel\ }{}
%% {\hspace{.5em}\titlerule*[4pt]{$\cdot$}\contentspage}


\ctexset{
  %% fontset = adobe,
  today = big,
  punct = quanjiao, %% banjiao,
  autoindent = 0pt,
  section = {
    numbering = false, % 标题不显示编号
    titleformat = \Huge,
    format = \texttt
  },
  chapter = {
    numbering = false, % 标题不显示编号
    titleformat = \centering\Huge, % 小四号字
    format = \textbf % 楷体
  }
}

%% \setCJKmainfont{AdobeSongStd-Light.otf} 
%% \setCJKmainfont{AdobeSongStd-Regular.otf} 
%% \setCJKmainfont{AdobeFangsongStd-Regular.otf}
% \setCJKmainfont{方正宋刻本秀楷繁体.ttf}
% \setCJKmainfont{方正清刻本悦宋繁.ttf}
% \setCJKmainfont{WenYue-GuDianMingChaoTi-NC-W5.otf}

\title{\zihao{0}\textbf{《史記》選}}

\author{\normalsize 太史公 \textit{司馬遷} 著 \\ \normalsize 由 \textit{樂行} 編輯/整理}
\date{\normalsize\today 版}

\begin{document}
\maketitle
\tableofcontents

\part{本紀}
\LARGE

\section{項羽本紀}
  項籍者,下相人也,字羽。初起時,年二十四。其季父項梁,梁父即楚將項燕,為秦將王翦所戮者也。項氏世世為楚將,封於項,故姓項氏。
\\\\
  項籍少時,學書不成,去學劍,又不成。項梁怒之。籍曰:「書足以記名姓而已。劍一人敵,不足學,學萬人敵。」於是項梁乃教籍兵法,籍大喜,略知其意,又不肯竟學。項梁嘗有櫟陽逮,乃請蘄獄掾曹咎書抵櫟陽獄掾司馬欣,以故事得已。項梁殺人,與籍避仇於吳中。吳中賢士大夫皆出項梁下。每吳中有大繇役及喪,項梁常為主辦,陰以兵法部勒賓客及子弟,以是知其能。秦始皇帝游會稽,渡浙江,梁與籍俱觀。籍曰:「彼可取而代也。」梁掩其口,曰:「毋妄言,族矣!」梁以此奇籍。籍長八尺餘,力能扛鼎,才氣過人,雖吳中子弟皆已憚籍矣。
\\\\
  秦二世元年七月,陳涉等起大澤中。其九月,會稽守通謂梁曰:「江西皆反,此亦天亡秦之時也。吾聞先即制人,後則為人所制。吾欲發兵,使公及桓楚將。」是時桓楚亡在澤中。梁曰:「桓楚亡,人莫知其處,獨籍知之耳。」梁乃出,誡籍持劍居外待。梁復入,與守坐,曰:「請召籍,使受命召桓楚。」守曰:「諾。」梁召籍入。須臾,梁眴籍曰:「可行矣!」於是籍遂拔劍斬守頭。項梁持守頭,佩其印綬。門下大驚,擾亂,籍所擊殺數十百人。一府中皆慴伏,莫敢起。梁乃召故所知豪吏,諭以所為起大事,遂舉吳中兵。使人收下縣,得精兵八千人。梁部署吳中豪傑為校尉、候、司馬。有一人不得用,自言於梁。梁曰:「前時某喪使公主某事,不能辦,以此不任用公。」眾乃皆伏。於是梁為會稽守,籍為裨將,徇下縣。
\\\\
  廣陵人召平於是為陳王徇廣陵,未能下。聞陳王敗走,秦兵又且至,乃渡江矯陳王命,拜梁為楚王上柱國。曰:「江東已定,急引兵西擊秦。」項梁乃以八千人渡江而西。聞陳嬰已下東陽,使使欲與連和俱西。陳嬰者,故東陽令史,居縣中,素信謹,稱為長者。東陽少年殺其令,相聚數千人,欲置長,無適用,乃請陳嬰。嬰謝不能,遂彊立嬰為長,縣中從者得二萬人。少年欲立嬰便為王,異軍蒼頭特起。陳嬰母謂嬰曰:「自我為汝家婦,未嘗聞汝先古之有貴者。今暴得大名,不祥。不如有所屬,事成猶得封侯,事敗易以亡,非世所指名也。」嬰乃不敢為王。謂其軍吏曰:「項氏世世將家,有名於楚。今欲舉大事,將非其人,不可。我倚名族,亡秦必矣。」於是眾從其言,以兵屬項梁。項梁渡淮,黥布、蒲將軍亦以兵屬焉。凡六七萬人,軍下邳。
\\\\
  當是時,秦嘉已立景駒為楚王,軍彭城東,欲距項梁。項梁謂軍吏曰:「陳王先首事,戰不利,未聞所在。今秦嘉倍陳王而立景駒,逆無道。」乃進兵擊秦嘉。秦嘉軍敗走,追之至胡陵。嘉還戰一日,嘉死,軍降。景駒走死梁地。項梁已并秦嘉軍,軍胡陵,將引軍而西。章邯軍至栗,項梁使別將朱雞石、餘樊君與戰。餘樊君死。朱雞石軍敗,亡走胡陵。項梁乃引兵入薛,誅雞石。項梁前使項羽別攻襄城,襄城堅守不下。已拔,皆阬之。還報項梁。項梁聞陳王定死,召諸別將會薛計事。此時沛公亦起沛,往焉。
\\\\
  居鄛人范增,年七十,素居家,好奇計,往說項梁曰:「陳勝敗碧當。夫秦滅六國,楚最無罪。自懷王入秦不反,楚人憐之至今,故楚南公曰『楚雖三戶,亡秦必楚』也。今陳勝首事,不立楚後而自立,其勢不長。今君起江東,楚蜂午之將皆爭附君者,以君世世楚將,為能復立楚之後也。」於是項梁然其言,乃求楚懷王孫心民閒,為人牧羊,立以為楚懷王,從民所望也。陳嬰為楚上柱國,封五縣,與懷王都盱臺。項梁自號為武信君。
\\\\
  居數月,引兵攻亢父,與齊田榮、司馬龍且軍救東阿,大破秦軍於東阿。田榮即引兵歸,逐其王假。假亡走楚。假相田角亡走趙。角弟田閒故齊將,居趙不敢歸。田榮立田儋子市為齊王。項梁已破東阿下軍,遂追秦軍。數使使趣齊兵,欲與俱西。田榮曰:「楚殺田假,趙殺田角、田閒,乃發兵。」項梁曰:「田假為與國之王,窮來從我,不忍殺之。」趙亦不殺田角、田閒以市於齊。齊遂不肯發兵助楚。項梁使沛公及項羽別攻城陽,屠之。西破秦軍濮陽東,秦兵收入濮陽。沛公、項羽乃攻定陶。定陶未下,去,西略地至雝丘,大破秦軍,斬李由。還攻外黃,外黃未下。
\\\\
  項梁起東阿,西,(北)[比]至定陶,再破秦軍,項羽等又斬李由,益輕秦,有驕色。宋義乃諫項梁曰:「戰勝而將驕卒惰者敗。今卒少惰矣,秦兵日益,臣為君畏之。」項梁弗聽。乃使宋義使於齊。道遇齊使者高陵君顯,曰:「公將見武信君乎?」曰:「然。」曰:「臣論武信君軍必敗。公徐行即免死,疾行則及禍。」秦果悉起兵益章邯,擊楚軍,大破之定陶,項梁死。沛公、項羽去外黃攻陳留,陳留堅守不能下。沛公、項羽相與謀曰:「今項梁軍破,士卒恐。」乃與呂臣軍俱引兵而東。呂臣軍彭城東,項羽軍彭城西,沛公軍碭。
\\\\
  章邯已破項梁軍,則以為楚地兵不足憂,乃渡河擊趙,大破之。當此時,趙歇為王,陳餘為將,張耳為相,皆走入鉅鹿城。章邯令王離、涉閒圍鉅鹿,章邯軍其南,筑甬道而輸之粟。陳餘為將,將卒數萬人而軍鉅鹿之北,此所謂河北之軍也。
\\\\
  楚兵已破於定陶,懷王恐,從盱臺之彭城,并項羽、呂臣軍自將之。以呂臣為司徒,以其父呂青為令尹。以沛公為碭郡長,封為武安侯,將碭郡兵。
\\\\
  初,宋義所遇齊使者高陵君顯在楚軍,見楚王曰:「宋義論武信君之軍必敗,居數日,軍果敗。兵未戰而先見敗徵,此可謂知兵矣。」王召宋義與計事而大說之,因置以為上將軍,項羽為魯公,為次將,范增為末將,救趙。諸別將皆屬宋義,號為卿子冠軍。行至安陽,留四十六日不進。項羽曰:「吾聞秦軍圍趙王鉅鹿,疾引兵渡河,楚擊其外,趙應其內,破秦軍必矣。」宋義曰:「不然。夫搏牛之虻不可以破蟣虱。今秦攻趙,戰勝則兵罷,我承其敝;不勝,則我引兵鼓行而西,必舉秦矣。故不如先鬬秦趙。夫被堅執銳,義不如公;坐而運策,公不如義。」因下令軍中曰:「猛如虎,很如羊,貪如狼,彊不可使者,皆斬之。」乃遣其子宋襄相齊,身送之至無鹽,飲酒高會。天寒大雨,士卒凍饑。項羽曰:「將戮力而攻秦,久留不行。今歲饑民貧,士卒食芋菽,軍無見糧,乃飲酒高會,不引兵渡河因趙食,與趙并力攻秦,乃曰『承其敝』。夫以秦之彊,攻新造之趙,其勢必舉趙。趙舉而秦彊,何敝之承!且國兵新破,王坐不安席,埽境內而專屬於將軍,國家安危,在此一舉。今不恤士卒而徇其私,非社稷之臣。」項羽晨朝上將軍宋義,即其帳中斬宋義頭,出令軍中曰:「宋義與齊謀反楚,楚王陰令羽誅之。」當是時,諸將皆慴服,莫敢枝梧。皆曰:「首立楚者,將軍家也。今將軍誅亂。」乃相與共立羽為假上將軍。使人追宋義子,及之齊,殺之。使桓楚報命於懷王。懷王因使項羽為上將軍,當陽君、蒲將軍皆屬項羽。
\\\\
  項羽已殺卿子冠軍,威震楚國,名聞諸侯。乃遣當陽君、蒲將軍將卒二萬渡河,救鉅鹿。戰少利,陳餘復請兵。項羽乃悉引兵渡河,皆沈船,破釜甑,燒廬舍,持三日糧,以示士卒必死,無一還心。於是至則圍王離,與秦軍遇,九戰,絕其甬道,大破之,殺蘇角,虜王離。涉閒不降楚,自燒殺。當是時,楚兵冠諸侯。諸侯軍救鉅鹿下者十餘壁,莫敢縱兵。及楚擊秦,諸將皆從壁上觀。楚戰士無不一以當十,楚兵呼聲動天,諸侯軍無不人人惴恐。於是已破秦軍,項羽召見諸侯將,入轅門,無不膝行而前,莫敢仰視。項羽由是始為諸侯上將軍,諸侯皆屬焉。
\\\\
  章邯軍棘原,項羽軍漳南,相持未戰。秦軍數卻,二世使人讓章邯。章邯恐,使長史欣請事。至咸陽,留司馬門三日,趙高不見,有不信之心。長史欣恐,還走其軍,不敢出故道,趙高果使人追之,不及。欣至軍,報曰:「趙高用事於中,下無可為者。今戰能勝,高必疾妒吾功;戰不能勝,不免於死。願將軍孰計之。」陳餘亦遺章邯書曰:「白起為秦將,南征鄢郢,北阬馬服,攻城略地,不可勝計,而竟賜死。蒙恬為秦將,北逐戎人,開榆中地數千里,竟斬陽周。何者?功多,秦不能盡封,因以法誅之。今將軍為秦將三歲矣,所亡失以十萬數,而諸侯并起滋益多。彼趙高素諛日久,今事急,亦恐二世誅之,故欲以法誅將軍以塞責,使人更代將軍以脫其禍。夫將軍居外久,多內卻,有功亦誅,無功亦誅。且天之亡秦,無愚智皆知之。今將軍內不能直諫,外為亡國將,孤特獨立而欲常存,豈不哀哉!將軍何不還兵與諸侯為從,約共攻秦,分王其地,南面稱孤;此孰與身伏鈇質,妻子為僇乎?」章邯狐疑,陰使候始成使項羽,欲約。約未成,項羽使蒲將軍日夜引兵度三戶,軍漳南,與秦戰,再破之。項羽悉引兵擊秦軍汙水上,大破之。
\\\\
  章邯使人見項羽,欲約。項羽召軍吏謀曰:「糧少,欲聽其約。」軍吏皆曰:「善。」項羽乃與期洹水南殷虛上。已盟,章邯見項羽而流涕,為言趙高。項羽乃立章邯為雍王,置楚軍中。使長史欣為上將軍,將秦軍為前行。
\\\\
  到新安。諸侯吏卒異時故繇使屯戍過秦中,秦中吏卒遇之多無狀,及秦軍降諸侯,諸侯吏卒乘勝多奴虜使之,輕折辱秦吏卒。秦吏卒多竊言曰:「章將軍等詐吾屬降諸侯,今能入關破秦,大善;即不能,諸侯虜吾屬而東,秦必盡誅吾父母妻子。」諸侯微聞其計,以告項羽。項羽乃召黥布、蒲將軍計曰:「秦吏卒尚眾,其心不服,至關中不聽,事必危,不如擊殺之,而獨與章邯、長史欣、都尉翳入秦。」於是楚軍夜擊阬秦卒二十餘萬人新安城南。
\\\\
  行略定秦地。函谷關有兵守關,不得入。又聞沛公已破咸陽,項羽大怒,使當陽君等擊關。項羽遂入,至于戲西。沛公軍霸上,未得與項羽相見。沛公左司馬曹無傷使人言於項羽曰:「沛公欲王關中,使子嬰為相,珍寶盡有之。」項羽大怒,曰:「旦日饗士卒,為擊破沛公軍!」當是時,項羽兵四十萬,在新豐鴻門,沛公兵十萬,在霸上。范增說項羽曰:「沛公居山東時,貪於財貨,好美姬。今入關,財物無所取,婦女無所幸,此其志不在小。吾令人望其氣,皆為龍虎,成五采,此天子氣也。急擊勿失。」
\\\\
  楚左尹項伯者,項羽季父也,素善留侯張良。張良是時從沛公,項伯乃夜馳之沛公軍,私見張良,具告以事,欲呼張良與俱去。曰:「毋從俱死也。」張良曰:「臣為韓王送沛公,沛公今事有急,亡去不義,不可不語。」良乃入,具告沛公。沛公大驚,曰:「為之柰何?」張良曰:「誰為大王為此計者?」曰:「鯫生說我曰『距關,毋內諸侯,秦地可盡王也』。故聽之。」良曰:「料大王士卒足以當項王乎?」沛公默然,曰:「固不如也,且為之柰何?」張良曰:「請往謂項伯,言沛公不敢背項王也。」沛公曰:「君安與項伯有故?」張良曰:「秦時與臣游,項伯殺人,臣活之。今事有急,故幸來告良。」沛公曰「孰與君少長?」良曰:「長於臣。」沛公曰「君為我呼入,吾得兄事之。」張良出,要項伯。項伯即入見沛公。沛公奉卮酒為壽,約為婚姻,曰:「吾入關,秋豪不敢有所近,籍吏民,封府庫,而待將軍。所以遣將守關者,備他盜之出入與非常也。日夜望將軍至,豈敢反乎!願伯具言臣之不敢倍德也。」項伯許諾。謂沛公曰:「旦日不可不蚤自來謝項王。」沛公曰:「諾。」於是項伯復夜去,至軍中,具以沛公言報項王。因言曰:「沛公不先破關中,公豈敢入乎?今人有大功而擊之,不義也,不如因善遇之。」項王許諾。
\\\\
  沛公旦日從百餘騎來見項王,至鴻門,謝曰:「臣與將軍戮力而攻秦,將軍戰河北,臣戰河南,然不自意能先入關破秦,得復見將軍於此。今者有小人之言,令將軍與臣有郤。」項王曰:「此沛公左司馬曹無傷言之;不然,籍何以至此。」項王即日因留沛公與飲。項王、項伯東向坐。亞父南向坐。亞父者,范增也。沛公北向坐,張良西向侍。范增數目項王,舉所佩玉珪以示之者三,項王默然不應。范增起,出召項莊,謂曰:「君王為人不忍,若入前為壽,壽畢,請以劍舞,因擊沛公於坐,殺之。不者,若屬皆且為所虜。」莊則入為壽,壽畢,曰:「君王與沛公飲,軍中無以為樂,請以劍舞。」項王曰:「諾。」項莊拔劍起舞,項伯亦拔劍起舞,常以身翼蔽沛公,莊不得擊。於是張良至軍門,見樊噲。樊噲曰:「今日之事何如?」良曰:「甚急。今者項莊拔劍舞,其意常在沛公也。」噲曰:「此迫矣,臣請入,與之同命。」噲即帶劍擁盾入軍門。交戟之衛士欲止不內,樊噲側其盾以撞,衛士仆地,噲遂入,披帷西向立,瞋目視項王,頭髪上指,目眥盡裂。項王按劍而跽曰:「客何為者?」張良曰:「沛公之參乘樊噲者也。」項王曰:「壯士,賜之卮酒。」則與斗卮酒。噲拜謝,起,立而飲之。項王曰:「賜之彘肩。」則與一生彘肩。樊噲覆其盾於地,加彘肩上,拔劍切而啗之。項王曰:「壯士,能復飲乎?」樊噲曰:「臣死且不避,卮酒安足辭!夫秦王有虎狼之心,殺人如不能舉,刑人如恐不勝,天下皆叛之。懷王與諸將約曰『先破秦入咸陽者王之』。今沛公先破秦入咸陽,豪毛不敢有所近,封閉宮室,還軍霸上,以待大王來。故遣將守關者,備他盜出入與非常也。勞苦而功高如此,未有封侯之賞,而聽細說,欲誅有功之人。此亡秦之續耳,竊為大王不取也。」項王未有以應,曰:「坐。」樊噲從良坐。坐須臾,沛公起如廁,因招樊噲出。
\\\\
  沛公已出,項王使都尉陳平召沛公。沛公曰:「今者出,未辭也,為之柰何?」樊噲曰:「大行不顧細謹,大禮不辭小讓。如今人方為刀俎,我為魚肉,何辭為。」於是遂去。乃令張良留謝。良問曰:「大王來何操?」曰:「我持白璧一雙,欲獻項王,玉斗一雙,欲與亞父,會其怒,不敢獻。公為我獻之」張良曰:「謹諾。」當是時,項王軍在鴻門下,沛公軍在霸上,相去四十里。沛公則置車騎,脫身獨騎,與樊噲、夏侯嬰、靳彊、紀信等四人持劍盾步走,從酈山下,道芷陽閒行。沛公謂張良曰:「從此道至吾軍,不過二十里耳。度我至軍中,公乃入。」沛公已去,閒至軍中,張良入謝,曰:「沛公不勝桮杓,不能辭。謹使臣良奉白璧一雙,再拜獻大王足下;玉斗一雙,再拜奉大將軍足下。」項王曰:「沛公安在?」良曰:「聞大王有意督過之,脫身獨去,已至軍矣。」項王則受璧,置之坐上。亞父受玉斗,置之地,拔劍撞而破之,曰:「唉!豎子不足與謀。奪項王天下者,必沛公也,吾屬今為之虜矣。」沛公至軍,立誅殺曹無傷。
\\\\
  居數日,項羽引兵西屠咸陽,殺秦降王子嬰,燒秦宮室,火三月不滅;收其貨寶婦女而東。人或說項王曰:「關中阻山河四塞,地肥饒,可都以霸。」項王見秦宮皆以燒殘破,又心懷思欲東歸,曰:「富貴不歸故鄉,如衣繡夜行,誰知之者!」說者曰:「人言楚人沐猴而冠耳,果然。」項王聞之,烹說者。
\\\\
  項王使人致命懷王。懷王曰:「如約。」乃尊懷王為義帝。項王欲自王,先王諸將相。謂曰:「天下初發難時,假立諸侯後以伐秦。然身被堅執銳首事,暴露於野三年,滅秦定天下者,皆將相諸君與籍之力也。義帝雖無功,故當分其地而王之。」諸將皆曰:「善。」乃分天下,立諸將為侯王。項王、范增疑沛公之有天下,業已講解,又惡負約,恐諸侯叛之,乃陰謀曰:「巴、蜀道險,秦之遷人皆居蜀。」乃曰:「巴、蜀亦關中地也。」故立沛公為漢王,王巴、蜀、漢中,都南鄭。而三分關中,王秦降將以距塞漢王。項王乃立章邯為雍王,王咸陽以西,都廢丘。長史欣者,故為櫟陽獄掾,嘗有德於項梁;都尉董翳者,本勸章邯降楚。故立司馬欣為塞王,王咸陽以東至河,都櫟陽;立董翳為翟王,王上郡,都高奴。徙魏王豹為西魏王,王河東,都平陽。瑕丘申陽者,張耳嬖臣也,先下河南(郡),迎楚河上,故立申陽為河南王,都雒陽。韓王成因故都,都陽翟。趙將司馬卬定河內,數有功,故立卬為殷王,王河內,都朝歌。徙趙王歇為代王。趙相張耳素賢,又從入關,故立耳為常山王,王趙地,都襄國。當陽君黥布為楚將,常冠軍,故立布為九江王,都六。鄱君吳芮率百越佐諸侯,又從入關,故立芮為衡山王,都邾。義帝柱國共敖將兵擊南郡,功多,因立敖為臨江王,都江陵。徙燕王韓廣為遼東王。燕將臧荼從楚救趙,因從入關,故立荼為燕王,都薊。徙齊王田市為膠東王。齊將田都從共救趙,因從入關,故立都為齊王,都臨菑。故秦所滅齊王建孫田安,項羽方渡河救趙,田安下濟北數城,引其兵降項羽,故立安為濟北王,都博陽。田榮者,數負項梁,又不肯將兵從楚擊秦,以故不封。成安君陳餘棄將印去,不從入關,然素聞其賢,有功於趙,聞其在南皮,故因環封三縣。番君將梅鋗功多,故封十萬戶侯。項王自立為西楚霸王,王九郡,都彭城。
\\\\
  漢之元年四月,諸侯罷戲下,各就國。項王出之國,使人徙義帝,曰:「古之帝者地方千里,必居上游。」乃使使徙義帝長沙郴縣。趣義帝行,其群臣稍稍背叛之,乃陰令衡山、臨江王擊殺之江中。韓王成無軍功,項王不使之國,與俱至彭城,廢以為侯,已又殺之。臧荼之國,因逐韓廣之遼東,廣弗聽,荼擊殺廣無終,
\\\\
  田榮聞項羽徙齊王市膠東,而立齊將田都為齊王,乃大怒,不肯遣齊王之膠東,因以齊反,迎擊田都。田都走楚。齊王市畏項王,乃亡之膠東就國。田榮怒,追擊殺之即墨。榮因自立為齊王,而西殺擊濟北王田安,并王三齊。榮與彭越將軍印,令反梁地。陳餘陰使張同、夏說說齊王田榮曰:「項羽為天下宰,不平。今盡王故王於醜地,而王其群臣諸將善地,逐其故主趙王,乃北居代,餘以為不可。聞大王起兵,且不聽不義,願大王資餘兵,請以擊常山,以復趙王,請以國為捍蔽。」齊王許之,因遣兵之趙。陳餘悉發三縣兵,與齊并力擊常山,大破之。張耳走歸漢。陳餘迎故趙王歇於代,反之趙。趙王因立陳餘為代王。
\\\\
  是時,漢還定三秦。項羽聞漢王皆已并關中,且東,齊、趙叛之:大怒。乃以故吳令鄭昌為韓王,以距漢。令蕭公角等擊彭越。彭越敗蕭公角等。漢使張良徇韓,乃遺項王書曰:「漢王失職,欲得關中,如約即止,不敢東。」又以齊、梁反書遺項王曰:「齊欲與趙并滅楚。」楚以此故無西意,而北擊齊。徵兵九江王布。布稱疾不往,使將將數千人行。項王由此怨布也。漢之二年冬,項羽遂北至城陽,田榮亦將兵會戰。田榮不勝,走至平原,平原民殺之。遂北燒夷齊城郭室屋,皆阬田榮降卒,系虜其老弱婦女。徇齊至北海,多所殘滅。齊人相聚而叛之。於是田榮弟田橫收齊亡卒得數萬人,反城陽。項王因留,連戰未能下。
\\\\
  春,漢王部五諸侯兵,凡五十六萬人,東伐楚。項王聞之,即令諸將擊齊,而自以精兵三萬人南從魯出胡陵。四月,漢皆已入彭城,收其貨寶美人,日置酒高會。項王乃西從蕭,晨擊漢軍而東,至彭城,日中,大破漢軍。漢軍皆走,相隨入穀、泗水,殺漢卒十餘萬人。漢卒皆南走山,楚又追擊至靈壁東睢水上。漢軍卻,為楚所擠,多殺,漢卒十餘萬人皆入睢水,睢水為之不流。圍漢王三匝。於是大風從西北而起,折木發屋,揚沙石,窈冥晝晦,逢迎楚軍。楚軍大亂,壞散,而漢王乃得與數十騎遁去,欲過沛,收家室而西;楚亦使人追之沛,取漢王家:家皆亡,不與漢王相見。漢王道逢得孝惠、魯元,乃載行。楚騎追漢王,漢王急,推墮孝惠、魯元車下,滕公常下收載之。如是者三。曰:「雖急不可以驅,柰何棄之?」於是遂得脫。求太公、呂后不相遇。審食其從太公、呂后閒行,求漢王,反遇楚軍。楚軍遂與歸,報項王,項王常置軍中。
\\\\
  是時呂后兄周呂侯為漢將兵居下邑,漢王閒往從之,稍稍收其士卒。至滎陽,諸敗軍皆會,蕭何亦發關中老弱未傅悉詣滎陽,復大振。楚起於彭城,常乘勝逐北,與漢戰滎陽南京、索閒,漢敗楚,楚以故不能過滎陽而西。
\\\\
  項王之救彭城,追漢王至滎陽,田橫亦得收齊,立田榮子廣為齊王。漢王之敗彭城,諸侯皆復與楚而背漢。漢軍滎陽,筑甬道屬之河,以取敖倉粟。漢之三年,項王數侵奪漢甬道,漢王食乏,恐,請和,割滎陽以西為漢。
\\\\
  項王欲聽之。歷陽侯范增曰:「漢易與耳,今釋弗取,後必悔之。」項王乃與范增急圍滎陽。漢王患之,乃用陳平計閒項王。項王使者來,為太牢具,舉欲進之。見使者,詳驚愕曰:「吾以為亞父使者,乃反項王使者。」更持去,以惡食食項王使者。使者歸報項王,項王乃疑范增與漢有私,稍奪之權。范增大怒,曰:「天下事大定矣,君王自為之。願賜骸骨歸卒伍。」項王許之。行未至彭城,疽發背而死。
\\\\
  漢將紀信說漢王曰:「事已急矣,請為王誑楚為王,王可以閒出。」於是漢王夜出女子滎陽東門被甲二千人,楚兵四面擊之。紀信乘黃屋車,傅左纛,曰:「城中食盡,漢王降。」楚軍皆呼萬歲。漢王亦與數十騎從城西門出,走成皋。項王見紀信,問:「漢王安在?」曰:「漢王已出矣。」項王燒殺紀信。
\\\\
  漢王使御史大夫周苛、樅公、魏豹守滎陽。周苛、樅公謀曰:「反國之王,難與守城。」乃共殺魏豹。楚下滎陽城,生得周苛。項王謂周苛曰:「為我將,我以公為上將軍,封三萬戶。」周苛罵曰:「若不趣降漢,漢今虜若,若非漢敵也。」項王怒,烹周苛,井殺樅公。
\\\\
  漢王之出滎陽,南走宛、葉,得九江王布,行收兵,復入保成皋。漢之四年,項王進兵圍成皋。漢王逃,獨與滕公出成皋北門,渡河走修武,從張耳、韓信軍。諸將稍稍得出成皋,從漢王。楚遂拔成皋,欲西。漢使兵距之鞏,令其不得西。
\\\\
  是時,彭越渡河擊楚東阿,殺楚將軍薛公。項王乃自東擊彭越。漢王得淮陰侯兵,欲渡河南。鄭忠說漢王,乃止壁河內。使劉賈將兵佐彭越,燒楚積聚。項王東擊破之,走彭越。漢王則引兵渡河,復取成皋,軍廣武,就敖倉食。項王已定東海來,西,與漢俱臨廣武而軍,相守數月。
\\\\
  當此時,彭越數反梁地,絕楚糧食,項王患之。為高俎,置太公其上,告漢王曰:「今不急下,吾烹太公。」漢王曰:「吾與項羽俱北面受命懷王,曰『約為兄弟』,吾翁即若翁,必欲烹而翁,則幸分我一桮羹。」項王怒,欲殺之。項伯曰:「天下事未可知,且為天下者不顧家,雖殺之無益,只益禍耳。」項王從之。
\\\\
  楚漢久相持未決,丁壯苦軍旅,老弱罷轉漕。項王謂漢王曰:「天下匈匈數歲者,徒以吾兩人耳,願與漢王挑戰決雌雄,毋徒苦天下之民父子為也。」漢王笑謝曰:「吾寧鬬智,不能鬬力。」項王令壯士出挑戰。漢有善騎射者樓煩,楚挑戰三合,樓煩輒射殺之。項王大怒,乃自被甲持戟挑戰。樓煩欲射之,項王瞋目叱之,樓煩目不敢視,手不敢發,遂走還入壁,不敢復出。漢王使人閒問之,乃項王也。漢王大驚。於是項王乃即漢王相與臨廣武閒而語。漢王數之,項王怒,欲一戰。漢王不聽,項王伏弩射中漢王。漢王傷,走入成皋。
\\\\
  項王聞淮陰侯已舉河北,破齊、趙,且欲擊楚,乃使龍且往擊之。淮陰侯與戰,騎將灌嬰擊之,大破楚軍,殺龍且。韓信因自立為齊王。項王聞龍且軍破,則恐,使盱臺人武涉往說淮陰侯。淮陰侯弗聽。是時,彭越復反,下梁地,絕楚糧。項王乃謂海春侯大司馬曹咎等曰:「謹守成皋,則漢欲挑戰,慎勿與戰,毋令得東而已。我十五日必誅彭越,定梁地,復從將軍。」乃東,行擊陳留、外黃。
\\\\
  外黃不下。數日,已降,項王怒,悉令男子年十五已上詣城東,欲阬之。外黃令舍人兒年十三,往說項王曰:「彭越彊劫外黃,外黃恐,故且降,待大王。大王至,又皆阬之,百姓豈有歸心?從此以東,梁地十餘城皆恐,莫肯下矣。」項王然其言,乃赦外黃當阬者。東至睢陽,聞之皆爭下項王。
\\\\
  漢果數挑楚軍戰,楚軍不出。使人辱之,五六日,大司馬怒,渡兵汜水。士卒半渡,漢擊之,大破楚軍,盡得楚國貨賂。大司馬咎、長史翳、塞王欣皆自剄汜水上。大司馬咎者,故蘄獄掾,長史欣亦故櫟陽獄吏,兩人嘗有德於項梁,是以項王信任之。當是時,項王在睢陽,聞海春侯軍敗,則引兵還。漢軍方圍鐘離眛於滎陽東,項王至,漢軍畏楚,盡走險阻。
\\\\
  是時,漢兵盛食多,項王兵罷食絕。漢遣陸賈說項王,請太公,項王弗聽。漢王復使侯公往說項王,項王乃與漢約,中分天下,割鴻溝以西者為漢,鴻溝而東者為楚。項王許之,即歸漢王父母妻子。軍皆呼萬歲。漢王乃封侯公為平國君。匿弗肯復見。曰:「此天下辯士,所居傾國,故號為平國君。」項王已約,乃引兵解而東歸。
\\\\
  漢欲西歸,張良、陳平說曰:「漢有天下太半,而諸侯皆附之。楚兵罷食盡,此天亡楚之時也,不如因其機而遂取之。今釋弗擊,此所謂『養虎自遺患』也。」漢王聽之。漢五年,漢王乃追項王至陽夏南,止軍,與淮陰侯韓信、建成侯彭越期會而擊楚軍。至固陵,而信、越之兵不會。楚擊漢軍,大破之。漢王復入壁,深塹而自守。謂張子房曰:「諸侯不從約,為之柰何?」對曰:「楚兵且破,信、越未有分地,其不至固宜。君王能與共分天下,今可立致也。即不能,事未可知也。君王能自陳以東傅海,盡與韓信;睢陽以北至穀城,以與彭越:使各自為戰,則楚易敗也。」漢王曰:「善。」於是乃發使者告韓信、彭越曰:「并力擊楚。楚破,自陳以東傅海與齊王,睢陽以北至穀城與彭相國。」使者至,韓信、彭越皆報曰:「請今進兵。」韓信乃從齊往,劉賈軍從壽春并行,屠城父,至垓下。大司馬周殷叛楚,以舒屠六,舉九江兵,隨劉賈、彭越皆會垓下,詣項王。
\\\\
  項王軍壁垓下,兵少食盡,漢軍及諸侯兵圍之數重。夜聞漢軍四面皆楚歌,項王乃大驚曰:「漢皆已得楚乎?是何楚人之多也!」項王則夜起,飲帳中。有美人名虞,常幸從;駿馬名騅,常騎之。於是項王乃悲歌慨,自為詩曰:「力拔山兮氣蓋世,時不利兮騅不逝。騅不逝兮可柰何,虞兮虞兮柰若何!」歌數闋,美人和之。項王泣數行下,左右皆泣,莫能仰視。
\\\\
  於是項王乃上馬騎,麾下壯士騎從者八百餘人,直夜潰圍南出,馳走。平明,漢軍乃覺之,令騎將灌嬰以五千騎追之。項王渡淮,騎能屬者百餘人耳。項王至陰陵,迷失道,問一田父,田父紿曰「左」。左,乃陷大澤中。以故漢追及之。項王乃復引兵而東,至東城,乃有二十八騎。漢騎追者數千人。項王自度不得脫。謂其騎曰:「吾起兵至今八歲矣,身七十餘戰,所當者破,所擊者服,未嘗敗北,遂霸有天下。然今卒困於此,此天之亡我,非戰之罪也。今日固決死,願為諸君快戰,必三勝之,為諸君潰圍,斬將,刈旗,令諸君知天亡我,非戰之罪也。」乃分其騎以為四隊,四向。漢軍圍之數重。項王謂其騎曰:「吾為公取彼一將。」令四面騎馳下,期山東為三處。於是項王大呼馳下,漢軍皆披靡,遂斬漢一將。是時,赤泉侯為騎將,追項王,項王瞋目而叱之,赤泉侯人馬俱驚,辟易數里與其騎會為三處。漢軍不知項王所在,乃分軍為三,復圍之。項王乃馳,復斬漢一都尉,殺數十百人,復聚其騎,亡其兩騎耳。乃謂其騎曰:「何如?」騎皆伏曰:「如大王言。」
\\\\
  於是項王乃欲東渡烏江。烏江亭長檥船待,謂項王曰:「江東雖小,地方千里,眾數十萬人,亦足王也。願大王急渡。今獨臣有船,漢軍至,無以渡。」項王笑曰:「天之亡我,我何渡為!且籍與江東子弟八千人渡江而西,今無一人還,縱江東父兄憐而王我,我何面目見之?縱彼不言,籍獨不愧於心乎?」乃謂亭長曰:「吾知公長者。吾騎此馬五歲,所當無敵,嘗一日行千里,不忍殺之,以賜公。」乃令騎皆下馬步行,持短兵接戰。獨籍所殺漢軍數百人。項王身亦被十餘創。顧見漢騎司馬呂馬童,曰:「若非吾故人乎?」馬童面之,指王翳曰:「此項王也。」項王乃曰:「吾聞漢購我頭千金,邑萬戶,吾為若德。」乃自刎而死。王翳取其頭,餘騎相蹂踐爭項王,相殺者數十人。最其後,郎中騎楊喜,騎司馬呂馬童,郎中呂勝、楊武各得其一體。五人共會其體,皆是。故分其地為五:封呂馬童為中水侯,封王翳為杜衍侯,封楊喜為赤泉侯,封楊武為吳防侯,封呂勝為涅陽侯。
\\\\
  項王已死,楚地皆降漢,獨魯不下。漢乃引天下兵欲屠之,為其守禮義,為主死節,乃持項王頭視魯,魯父兄乃降。始,楚懷王初封項籍為魯公,及其死,魯最後下,故以魯公禮葬項王穀城。漢王為發哀,泣之而去。
\\\\
  諸項氏枝屬,漢王皆不誅。乃封項伯為射陽侯。桃侯、平皋侯、玄武侯皆項氏,賜姓劉。
\\\\
  太史公曰:吾聞之周生曰「舜目蓋重瞳子」,又聞項羽亦重瞳子。羽豈其苗裔邪?何興之暴也!夫秦失其政,陳涉首難,豪傑蜂起,相與并爭,不可勝數。然羽非有尺寸,乘執起隴畝之中,三年,遂將五諸侯滅秦,分裂天下,而封王侯,政由羽出,號為「霸王」,位雖不終,近古以來未嘗有也。及羽背關懷楚,放逐義帝而自立,怨王侯叛己,難矣。自矜功伐,奮其私智而不師古,謂霸王之業,欲以力征經營天下,五年卒亡其國,身死東城,尚不覺寤而不自責,過矣。乃引「天亡我,非用兵之罪也」,豈不謬哉!
\section{高祖本紀}
  高祖,沛豐邑中陽裏人,姓劉氏,字季。父曰太公,母曰劉媼。其先劉媼嘗息大澤之陂,夢與神遇。是時雷電晦冥,太公往視,則見蛟龍於其上。已而有身,遂產高祖。
\\\\
  高祖為人,隆準而龍顏,美須髯,左股有七十二黑子。仁而愛人,喜施,意豁如也。常有大度,不事家人生產作業。及壯,試為吏,為泗水亭長,廷中吏無所不狎侮,好酒及色。常從王媼、武負貰酒,醉臥,武負、王媼見其上常有龍,怪之。高祖每酤留飲,酒讎數倍。及見怪,歲竟,此兩家常折券棄責。
\\\\
  高祖常繇咸陽,縱觀,觀秦皇帝,喟然太息曰:「嗟乎,大丈夫當如此也!」
\\\\
  單父人呂公善沛令,避仇從之客,因家沛焉。沛中豪桀吏聞令有重客,皆往賀。蕭何為主吏,主進,令諸大夫曰:「進不滿千錢,坐之堂下。」高祖為亭長,素易諸吏,乃紿為謁曰「賀錢萬」,實不持一錢。謁入,呂公大驚,起,迎之門。呂公者,好相人,見高祖狀貌,因重敬之,引入坐。蕭何曰:「劉季固多大言,少成事。」高祖因狎侮諸客,遂坐上坐,無所詘。酒闌,呂公因目固留高祖。高祖竟酒,後。呂公曰:「臣少好相人,相人多矣,無如季相,願季自愛。臣有息女,願為季箕帚妾。」酒罷,呂媼怒呂公曰:「公始常欲奇此女,與貴人。沛令善公,求之不與,何自妄許與劉季?」呂公曰:「此非兒女子所知也。」卒與劉季。呂公女乃呂后也,生孝惠帝、魯元公主。
\\\\
  高祖為亭長時,常告歸之田。呂后與兩子居田中耨,有一老父過請飲,呂后因餔之。老父相呂后曰:「夫人天下貴人。」令相兩子,見孝惠,曰:「夫人所以貴者,乃此男也。」相魯元,亦皆貴。老父已去,高祖適從旁舍來,呂后具言客有過,相我子母皆大貴。高祖問,曰:「未遠。」乃追及,問老父。老父曰:「鄉者夫人嬰兒皆似君,君相貴不可言。」高祖乃謝曰:「誠如父言,不敢忘德。」及高祖貴,遂不知老父處。
\\\\
  高祖為亭長,乃以竹皮為冠,令求盜之薛治之,時時冠之,及貴常冠,所謂「劉氏冠」乃是也。
\\\\
  高祖以亭長為縣送徒酈山,徒多道亡。自度比至皆亡之,到豐西澤中,止飲,夜乃解縱所送徒。曰:「公等皆去,吾亦從此逝矣!」徒中壯士願從者十餘人。高祖被酒,夜徑澤中,令一人行前。行前者還報曰:「前有大蛇當徑,願還。」高祖醉,曰:「壯士行,何畏!」乃前,拔劍擊斬蛇。蛇遂分為兩,徑開。行數里,醉,因臥。後人來至蛇所,有一老嫗夜哭。人問何哭,嫗曰:「人殺吾子,故哭之。」人曰:「嫗子何為見殺?」嫗曰:「吾,白帝子也,化為蛇,當道,今為赤帝子斬之,故哭。」人乃以嫗為不誠,欲告之,嫗因忽不見。後人至,高祖覺。後人告高祖,高祖乃心獨喜,自負。諸從者日益畏之。
\\\\
  秦始皇帝常曰「東南有天子氣」,於是因東游以厭之。高祖即自疑,亡匿,隱於芒、碭山澤巖石之閒。呂后與人俱求,常得之。高祖怪問之。呂后曰:「季所居上常有雲氣,故從往常得季。」高祖心喜。沛中子弟或聞之,多欲附者矣。
\\\\
  秦二世元年秋,陳勝等起蘄,至陳而王,號為「張楚」。諸郡縣皆多殺其長吏以應陳涉。沛令恐,欲以沛應涉。掾、主吏蕭何、曹參乃曰:「君為秦吏,今欲背之,率沛子弟,恐不聽。願君召諸亡在外者,可得數百人,因劫眾,眾不敢不聽。」乃令樊噲召劉季。劉季之眾已數十百人矣。
\\\\
  於是樊噲從劉季來。沛令後悔,恐其有變,乃閉城城守,欲誅蕭、曹。蕭、曹恐,踰城保劉季。劉季乃書帛射城上,謂沛父老曰:「天下苦秦久矣。今父老雖為沛令守,諸侯并起,今屠沛。沛今共誅令,擇子弟可立者立之,以應諸侯,則家室完。不然,父子俱屠,無為也。」父老乃率子弟共殺沛令,開城門迎劉季,欲以為沛令。劉季曰:「天下方擾,諸侯并起,今置將不善,壹敗涂地。吾非敢自愛,恐能薄,不能完父兄子弟。此大事,願更相推擇可者。」蕭、曹等皆文吏,自愛,恐事不就,後秦種族其家,盡讓劉季。諸父老皆曰:「平生所聞劉季諸珍怪,當貴,且卜筮之,莫如劉季最吉。」於是劉季數讓。眾莫敢為,乃立季為沛公。祠黃帝,祭蚩尤於沛庭,而釁鼓旗,幟皆赤。由所殺蛇白帝子,殺者赤帝子,故上赤。於是少年豪吏如蕭、曹、樊噲等皆為收沛子弟二三千人,攻胡陵、方與,還守豐。
\\\\
  秦二世二年,陳涉之將周章軍西至戲而還。燕、趙、齊、魏皆自立為王。項氏起吳。秦泗川監平將兵圍豐,二日,出與戰,破之。命雍齒守豐,引兵之薛。泗州守壯敗於薛,走至戚,沛公左司馬得泗川守壯,殺之。沛公還軍亢父,至方與,(周市來攻方與)未戰。陳王使魏人周市略地。周市使人謂雍齒曰:「豐,故梁徙也。今魏地已定者數十城。齒今下魏,魏以齒為侯守豐。不下,且屠豐。」雍齒雅不欲屬沛公,及魏招之,即反為魏守豐。沛公引兵攻豐,不能取。沛公病,還之沛。沛公怨雍齒與豐子弟叛之,聞東陽甯君、秦嘉立景駒為假王,在留,乃往從之,欲請兵以攻豐。是時秦將章邯從陳,別將司馬夷將兵北定楚地,屠相,至碭。東陽甯君、沛公引兵西,與戰蕭西,不利。還收兵聚留,引兵攻碭,三日乃取碭。因收碭兵,得五六千人。攻下邑,拔之。還軍豐。聞項梁在薛,從騎百餘往見之。項梁益沛公卒五千人,五大夫將十人。沛公還,引兵攻豐。
\\\\
  從項梁月餘,項羽已拔襄城還。項梁盡召別將居薛。聞陳王定死,因立楚後懷王孫心為楚王,治盱臺。項梁號武信君。居數月,北攻亢父,救東阿,破秦軍。齊軍歸,楚獨追北,使沛公、項羽別攻城陽,屠之。軍濮陽之東,與秦軍戰,破之。
\\\\
  秦軍復振,守濮陽,環水。楚軍去而攻定陶,定陶未下。沛公與項羽西略地至雍丘之下,與秦軍戰,大破之,斬李由。還攻外黃,外黃未下。
\\\\
  項梁再破秦軍,有驕色。宋義諫,不聽。秦益章邯兵,夜銜枚擊項梁,大破之定陶,項梁死。沛公與項羽方攻陳留,聞項梁死,引兵與呂將軍俱東。呂臣軍彭城東,項羽軍彭城西,沛公軍碭。
\\\\
  章邯已破項梁軍,則以為楚地兵不足憂,乃渡河,北擊趙,大破之。當是之時,趙歇為王,秦將王離圍之鉅鹿城,此所謂河北之軍也。
\\\\
  秦二世三年,楚懷王見項梁軍破,恐,徙盱臺都彭城,并呂臣、項羽軍自將之。以沛公為碭郡長,封為武安侯,將碭郡兵。封項羽為長安侯,號為魯公。呂臣為司徒,其父呂青為令尹。
\\\\
  趙數請救,懷王乃以宋義為上將軍,項羽為次將,范增為末將,北救趙。令沛公西略地入關。與諸將約,先入定關中者王之。
\\\\
  當是時,秦兵彊,常乘勝逐北,諸將莫利先入關。獨項羽怨秦破項梁軍,奮,願與沛公西入關。懷王諸老將皆曰:「項羽為人彊悍猾賊。項羽嘗攻襄城,襄城無遺類,皆阬之,諸所過無不殘滅。且楚數進取,前陳王、項梁皆敗。不如更遣長者扶義而西,告諭秦父兄。秦父兄苦其主久矣,今誠得長者往,毋侵暴,宜可下。今項羽彊悍,今不可遣。獨沛公素寬大長者,可遣。」卒不許項羽,而遣沛公西略地,收陳王、項梁散卒。乃道碭至成陽,與杠裏秦軍夾壁,破(魏)[秦]二軍。楚軍出兵擊王離,大破之。
\\\\
  沛公引兵西,遇彭越昌邑,因與俱攻秦軍,戰不利。還至栗,遇剛武侯,奪其軍,可四千餘人,并之。與魏將皇欣、魏申徒武蒲之軍并攻昌邑,昌邑未拔。西過高陽。酈食其(謂)[為]監門,曰:「諸將過此者多,吾視沛公大人長者。」乃求見說沛公。沛公方踞床,使兩女子洗足。酈生不拜,長揖,曰:「足下必欲誅無道秦,不宜踞見長者。」於是沛公起,攝衣謝之,延上坐。食其說沛公襲陳留,得秦積粟。乃以酈食其為廣野君,酈商為將,將陳留兵,與偕攻開封,開封未拔。西與秦將楊熊戰白馬,又戰曲遇東,大破之。楊熊走之滎陽,二世使使者斬以徇。南攻潁陽,屠之。因張良遂略韓地轘轅。
\\\\
  當是時,趙別將司馬卬方欲渡河入關,沛公乃北攻平陰,絕河津。南,戰雒陽東,軍不利,還至陽城,收軍中馬騎,與南陽守齮戰犨東,破之。略南陽郡,南陽守齮走,保城守宛。沛公引兵過而西。張良諫曰:「沛公雖欲急入關,秦兵尚眾,距險。今不下宛,宛從後擊,彊秦在前,此危道也。」於是沛公乃夜引兵從他道還,更旗幟,黎明,圍宛城三匝。南陽守欲自剄。其舍人陳恢曰:「死未晚也。」乃踰城見沛公,曰:「臣聞足下約,先入咸陽者王之。今足下留守宛。宛,大郡之都也,連城數十,人民眾,積蓄多,吏人自以為降必死,故皆堅守乘城。今足下盡日止攻,士死傷者必多;引兵去宛,宛必隨足下後:足下前則失咸陽之約,後又有彊宛之患。為足下計,莫若約降,封其守,因使止守,引其甲卒與之西。諸城未下者,聞聲爭開門而待,足下通行無所累。」沛公曰:「善。」乃以宛守為殷侯,封陳恢千戶。引兵西,無不下者。至丹水,高武侯鰓、襄侯王陵降西陵。還攻胡陽,遇番君別將梅鋗,與皆,降析、酈。遣魏人甯昌使秦,使者未來。是時章邯已以軍降項羽於趙矣。
\\\\
  初,項羽與宋義北救趙,及項羽殺宋義,代為上將軍,諸將黥布皆屬,破秦將王離軍,降章邯,諸侯皆附。及趙高已殺二世,使人來,欲約分王關中。沛公以為詐,乃用張良計,使酈生、陸賈往說秦將,啗以利,因襲攻武關,破之。又與秦軍戰於藍田南,益張疑兵旗幟,諸所過毋得掠鹵,秦人喜,秦軍解,因大破之。又戰其北,大破之。乘勝,遂破之。
\\\\
  漢元年十月,沛公兵遂先諸侯至霸上。秦王子嬰素車白馬,系頸以組,封皇帝璽符節,降軹道旁。諸將或言誅秦王。沛公曰:「始懷王遣我,固以能寬容;且人已服降,又殺之,不祥。」乃以秦王屬吏,遂西入咸陽。欲止宮休舍,樊噲、張良諫,乃封秦重寶財物府庫,還軍霸上。召諸縣父老豪桀曰:「父老苦秦苛法久矣,誹謗者族,偶語者棄市。吾與諸侯約,先入關者王之,吾當王關中。與父老約,法三章耳:殺人者死,傷人及盜抵罪。餘悉除去秦法。諸吏人皆案堵如故。凡吾所以來,為父老除害,非有所侵暴,無恐!且吾所以還軍霸上,待諸侯至而定約束耳。」乃使人與秦吏行縣鄉邑,告諭之。秦人大喜,爭持牛羊酒食獻饗軍士。沛公又讓不受,曰:「倉粟多,非乏,不欲費人。」人又益喜,唯恐沛公不為秦王。
\\\\
  或說沛公曰:「秦富十倍天下,地形彊。今聞章邯降項羽,項羽乃號為雍王,王關中。今則來,沛公恐不得有此。可急使兵守函谷關,無內諸侯軍,稍徵關中兵以自益,距之。」沛公然其計,從之。十一月中,項羽果率諸侯兵西,欲入關,關門閉。聞沛公已定關中,大怒,使黥布等攻破函谷關。十二月中,遂至戲。沛公左司馬曹無傷聞項王怒,欲攻沛公,使人言項羽曰:「沛公欲王關中,令子嬰為相,珍寶盡有之。」欲以求封。亞父勸項羽擊沛公。方饗士,旦日合戰。是時項羽兵四十萬,號百萬。沛公兵十萬,號二十萬,力不敵。會項伯欲活張良,夜往見良,因以文諭項羽,項羽乃止。沛公從百餘騎,驅之鴻門,見謝項羽。項羽曰:「此沛公左司馬曹無傷言之。不然,籍何以生此!」沛公以樊噲、張良故,得解歸。歸,立誅曹無傷。
\\\\
  項羽遂西,屠燒咸陽秦宮室,所過無不殘破。秦人大失望,然恐,不敢不服耳。
\\\\
  項羽使人還報懷王。懷王曰:「如約。」項羽怨懷王不肯令與沛公俱西入關,而北救趙,後天下約。乃曰:「懷王者,吾家項梁所立耳,非有功伐,何以得主約!本定天下,諸將及籍也。」乃詳尊懷王為義帝,實不用其命。
\\\\
  正月,項羽自立為西楚霸王,王梁、楚地九郡,都彭城。負約,更立沛公為漢王,王巴、蜀、漢中,都南鄭。三分關中,立秦三將:章邯為雍王,都廢丘;司馬欣為塞王,都櫟陽;董翳為翟王,都高奴。楚將瑕丘申陽為河南王,都洛陽。趙將司馬卬為殷王,都朝歌。趙王歇徙王代。趙相張耳為常山王,都襄國。當陽君黥布為九江王,都六。懷王柱國共敖為臨江王,都江陵。番君吳芮為衡山王,都邾。燕將臧荼為燕王,都薊。故燕王韓廣徙王遼東。廣不聽,臧荼攻殺之無終。封成安君陳餘河閒三縣,居南皮。封梅鋗十萬戶。
\\\\
  四月,兵罷戲下,諸侯各就國。漢王之國,項王使卒三萬人從,楚與諸侯之慕從者數萬人,從杜南入蝕中。去輒燒絕棧道,以備諸侯盜兵襲之,亦示項羽無東意。至南鄭,諸將及士卒多道亡歸,士卒皆歌思東歸。韓信說漢王曰:「項羽王諸將之有功者,而王獨居南鄭,是遷也。軍吏士卒皆山東之人也,日夜跂而望歸,及其鋒而用之,可以有大功。天下已定,人皆自寧,不可復用。不如決策東鄉,爭權天下。」
\\\\
  項羽出關,使人徙義帝。曰:「古之帝者地方千里,必居上游。」乃使使徙義帝長沙郴縣,趣義帝行,群臣稍倍叛之,乃陰令衡山王、臨江王擊之,殺義帝江南。項羽怨田榮,立齊將田都為齊王。田榮怒,因自立為齊王,殺田都而反楚;予彭越將軍印,令反梁地。楚令蕭公角擊彭越,彭越大破之。陳餘怨項羽之弗王己也,令夏說說田榮,請兵擊張耳。齊予陳餘兵,擊破常山王張耳,張耳亡歸漢。迎趙王歇於代,復立為趙王。趙王因立陳餘為代王。項羽大怒,北擊齊。
\\\\
  八月,漢王用韓信之計,從故道還,襲雍王章邯。邯迎擊漢陳倉,雍兵敗,還走;止戰好畤,又復敗,走廢丘。漢王遂定雍地。東至咸陽,引兵圍雍王廢丘,而遣諸將略定隴西、北地、上郡。令將軍薛歐、王吸出武關,因王陵兵南陽,以迎太公、呂後於沛。楚聞之,發兵距之陽夏,不得前。令故吳令鄭昌為韓王,距漢兵。
\\\\
  二年,漢王東略地,塞王欣、翟王翳、河南王申陽皆降。韓王昌不聽,使韓信擊破之。於是置隴西、北地、上郡、渭南、河上、中地郡;關外置河南郡。更立韓太尉信為韓王。諸將以萬人若以一郡降者,封萬戶。繕治河上塞。諸故秦苑囿園池,皆令人得田之,正月,虜雍王弟章平。大赦罪人。
\\\\
  漢王之出關至陜,撫關外父老,還,張耳來見,漢王厚遇之。
\\\\
  二月,令除秦社稷,更立漢社稷。
\\\\
  三月,漢王從臨晉渡,魏王豹將兵從。下河內,虜殷王,置河內郡。南渡平陰津,至雒陽。新城三老董公遮說漢王以義帝死故。漢王聞之,袒而大哭。遂為義帝發喪,臨三日。發使者告諸侯曰:「天下共立義帝,北面事之。今項羽放殺義帝於江南,大逆無道。寡人親為發喪,諸侯皆縞素。悉發關內兵,收三河士,南浮江漢以下,願從諸侯王擊楚之殺義帝者。」
\\\\
  是時項王北擊齊,田榮與戰城陽。田榮敗,走平原,平原民殺之。齊皆降楚。楚因焚燒其城郭,系虜其子女。齊人叛之。田榮弟橫立榮子廣為齊王,齊王反楚城陽。項羽雖聞漢東,既已連齊兵,欲遂破之而擊漢。漢王以故得劫五諸侯兵,遂入彭城。項羽聞之,乃引兵去齊,從魯出胡陵,至蕭,與漢大戰彭城靈壁東睢水上,大破漢軍,多殺士卒,睢水為之不流。乃取漢王父母妻子於沛,置之軍中以為質。當是時,諸侯見楚彊漢敗,還皆去漢復為楚。塞王欣亡入楚。
\\\\
  呂后兄周呂侯為漢將兵,居下邑。漢王從之,稍收士卒,軍碭。漢王乃西過梁地,至虞。使謁者隨何之九江王布所,曰:「公能令布舉兵叛楚,項羽必留擊之。得留數月,吾取天下必矣。」隨何往說九江王布,布果背楚。楚使龍且往擊之。
\\\\
  漢王之敗彭城而西,行使人求家室,家室亦亡,不相得。敗後乃獨得孝惠,六月,立為太子,大赦罪人。令太子守櫟陽,諸侯子在關中者皆集櫟陽為衛。引水灌廢丘,廢丘降,章邯自殺。更名廢丘為槐里。於是令祠官祀天地四方上帝山川,以時祀之。興關內卒乘塞。
\\\\
  是時九江王布與龍且戰,不勝,與隨何閒行歸漢。漢王稍收士卒,與諸將及關中卒益出,是以兵大振滎陽,破楚京、索閒。
\\\\
  三年,魏王豹謁歸視親疾,至即絕河津,反為楚。漢王使酈生說豹,豹不聽。漢王遣將軍韓信擊,大破之,虜豹。遂定魏地,置三郡,曰河東、太原、上黨。漢王乃令張耳與韓信遂東下井陘擊趙,斬陳餘、趙王歇。其明年,立張耳為趙王。
\\\\
  漢王軍滎陽南,筑甬道屬之河,以取敖倉。與項羽相距歲餘。項羽數侵奪漢甬道,漢軍乏食,遂圍漢王。漢王請和,割滎陽以西者為漢。項王不聽。漢王患之,乃用陳平之計,予陳平金四萬斤,以閒疏楚君臣。於是項羽乃疑亞父。亞父是時勸項羽遂下滎陽,及其見疑,乃怒,辭老,願賜骸骨歸卒伍,未至彭城而死。
\\\\
  漢軍絕食,乃夜出女子東門二千餘人,被甲,楚因四面擊之。將軍紀信乃乘王駕,詐為漢王,誑楚,楚皆呼萬歲,之城東觀,以故漢王得與數十騎出西門遁。令御史大夫周苛、魏豹、樅公守滎陽。諸將卒不能從者,盡在城中。周苛、樅公相謂曰:「反國之王,難與守城。」因殺魏豹。
\\\\
  漢王之出滎陽入關,收兵欲復東。袁生說漢王曰:「漢與楚相距滎陽數歲,漢常困。願君王出武關,項羽必引兵南走,王深壁,令滎陽成皋閒且得休。使韓信等輯河北趙地,連燕齊,君王乃復走滎陽,未晚也。如此,則楚所備者多,力分,漢得休,復與之戰,破楚必矣。」漢王從其計,出軍宛葉閒,與黥布行收兵。
\\\\
  項羽聞漢王在宛,果引兵南。漢王堅壁不與戰。是時彭越渡睢水,與項聲、薛公戰下邳,彭越大破楚軍。項羽乃引兵東擊彭越。漢王亦引兵北軍成皋。項羽已破走彭越,聞漢王復軍成皋,乃復引兵西,拔滎陽,誅周苛、樅公,而虜韓王信,遂圍成皋。
\\\\
  漢王跳,獨與滕公共車出成皋玉門,北渡河,馳宿修武。自稱使者,晨馳入張耳、韓信壁,而奪之軍。乃使張耳北益收兵趙地,使韓信東擊齊。漢王得韓信軍,則復振。引兵臨河,南饗軍小修武南,欲復戰。郎中鄭忠乃說止漢王,使高壘深塹,勿與戰。漢王聽其計,使盧綰、劉賈將卒二萬人,騎數百,渡白馬津,入楚地,與彭越復擊破楚軍燕郭西,遂復下梁地十餘城。
\\\\
  淮陰已受命東,未渡平原。漢王使酈生往說齊王田廣,廣叛楚,與漢和,共擊項羽。韓信用蒯通計,遂襲破齊。齊王烹酈生,東走高密。項羽聞韓信已舉河北兵破齊、趙,且欲擊楚,則使龍且、周蘭往擊之。韓信與戰,騎將灌嬰擊,大破楚軍,殺龍且。齊王廣奔彭越。當此時,彭越將兵居梁地,往來苦楚兵,絕其糧食。
\\\\
  四年,項羽乃謂海春侯大司馬曹咎曰:「謹守成皋。若漢挑戰,慎勿與戰,無令得東而已。我十五日必定梁地,復從將軍。」乃行擊陳留、外黃、睢陽,下之。漢果數挑楚軍,楚軍不出,使人辱之五六日,大司馬怒,度兵汜水。士卒半渡,漢擊之,大破楚軍,盡得楚國金玉貨賂。大司馬咎、長史欣皆自剄汜水上。項羽至睢陽,聞海春侯破,乃引兵還。漢軍方圍鐘離眛於滎陽東,項羽至,盡走險阻。
\\\\
  韓信已破齊,使人言曰:「齊邊楚,權輕,不為假王,恐不能安齊。」漢王欲攻之。留侯曰:「不如因而立之,使自為守。」乃遣張良操印綬立韓信為齊王。
\\\\
  項羽聞龍且軍破,則恐,使盱臺人武涉往說韓信。韓信不聽。
\\\\
  楚漢久相持未決,丁壯苦軍旅,老弱罷轉馕。漢王項羽相與臨廣武之閒而語。項羽欲與漢王獨身挑戰。漢王數項羽曰:「始與項羽俱受命懷王,曰先入定關中者王之,項羽負約,王我於蜀漢,罪一。秦項羽矯殺卿子冠軍而自尊,罪二。項羽已救趙,當還報,而擅劫諸侯兵入關,罪三。懷王約入秦無暴掠,項羽燒秦宮室,掘始皇帝冢,私收其財物,罪四。又彊殺秦降王子嬰,罪五。詐阬秦子弟新安二十萬,王其將,罪六。項羽皆王諸將善地,而徙逐故主,令臣下爭叛逆,罪七。項羽出逐義帝彭城,自都之,奪韓王地,并王梁楚,多自予,罪八。項羽使人陰弒義帝江南,罪九。夫為人臣而弒其主,殺已降,為政不平,主約不信,天下所不容,大逆無道,罪十也。吾以義兵從諸侯誅殘賊,使刑餘罪人擊殺項羽,何苦乃與公挑戰!」項羽大怒,伏弩射中漢王。漢王傷匈,乃捫足曰:「虜中吾指!」漢王病創臥,張良彊請漢王起行勞軍,以安士卒,毋令楚乘勝於漢。漢王出行軍,病甚,因馳入成皋。
\\\\
  病愈,西入關,至櫟陽,存問父老,置酒,梟故塞王欣頭櫟陽市。留四日,復如軍,軍廣武。關中兵益出。
\\\\
  當此時,彭越將兵居梁地,往來苦楚兵,絕其糧食。田橫往從之。項羽數擊彭越等,齊王信又進擊楚。項羽恐,乃與漢王約,中分天下,割鴻溝而西者為漢,鴻溝而東者為楚。項王歸漢王父母妻子,軍中皆呼萬歲,乃歸而別去。
\\\\
  項羽解而東歸。漢王欲引而西歸,用留侯、陳平計,乃進兵追項羽,至陽夏南止軍,與齊王信、建成侯彭越期會而擊楚軍。至固陵,不會。楚擊漢軍,大破之。漢王復入壁,深塹而守之。用張良計,於是韓信、彭越皆往。及劉賈入楚地,圍壽春,漢王敗碧陵,乃使使者召大司馬周殷舉九江兵而迎(之)武王,行屠城父,隨(何)劉賈、齊梁諸侯皆大會垓下。立武王布為淮南王。
\\\\
  五年,高祖與諸侯兵共擊楚軍,與項羽決勝垓下。淮陰侯將三十萬自當之,孔將軍居左,費將軍居右,皇帝在後,絳侯、柴將軍在皇帝後。項羽之卒可十萬。淮陰先合,不利,卻。孔將軍、費將軍縱,楚兵不利,淮陰侯復乘之,大敗垓下。項羽卒聞漢軍之楚歌,以為漢盡得楚地,項羽乃敗而走,是以兵大敗。使騎將灌嬰追殺項羽東城,斬首八萬,遂略定楚地。魯為楚堅守不下。漢王引諸侯兵北,示魯父老項羽頭,魯乃降。遂以魯公號葬項羽穀城。還至定陶,馳入齊王壁,奪其軍。
\\\\
  正月,諸侯及將相相與共請尊漢王為皇帝。漢王曰:「吾聞帝賢者有也,空言虛語,非所守也,吾不敢當帝位。」群臣皆曰:「大王起微細,誅暴逆,平定四海,有功者輒裂地而封為王侯。大王不尊號,皆疑不信。臣等以死守之。」漢王三讓,不得已,曰:「諸君必以為便,便國家。」甲午,乃即皇帝位汜水之陽。
\\\\
  皇帝曰義帝無後。齊王韓信習楚風俗,徙為楚王,都下邳。立建成侯彭越為梁王,都定陶。故韓王信為韓王,都陽翟。徙衡山王吳芮為長沙王,都臨湘。番君之將梅鋗有功,從入武關,故德番君。淮南王布、燕王臧荼、趙王敖皆如故。
\\\\
  天下大定。高祖都雒陽,諸侯皆臣屬。故臨江王驩為項羽叛漢,令盧綰、劉賈圍之,不下。數月而降,殺之雒陽。
\\\\
  五月,兵皆罷歸家。諸侯子在關中者復之十二歲,其歸者復之六歲,食之一歲。
\\\\
  高祖置酒雒陽南宮。高祖曰:「列侯諸將無敢隱朕,皆言其情。吾所以有天下者何?項氏之所以失天下者何?」高起、王陵對曰:「陛下慢而侮人,項羽仁而愛人。然陛下使人攻城略地,所降下者因以予之,與天下同利也。項羽妒賢嫉能,有功者害之,賢者疑之,戰勝而不予人功,得地而不予人利,此所以失天下也。」高祖曰:「公知其一,未知其二。夫運籌策帷帳之中,決勝於千里之外,吾不如子房。鎮國家,撫百姓,給餽馕,不絕糧道,吾不如蕭何。連百萬之軍,戰必勝,攻必取,吾不如韓信。此三者,皆人傑也,吾能用之,此吾所以取天下也。項羽有一范增而不能用,此其所以為我擒也。」
\\\\
  高祖欲長都雒陽,齊人劉敬說,乃留侯勸上入都關中,高祖是日駕,入都關中。六月,大赦天下。
\\\\
  十月,燕王臧荼反,攻下代地。高祖自將擊之,得燕王臧荼。即立太尉盧綰為燕王。使丞相噲將兵攻代。
\\\\
  其秋,利幾反,高祖自將兵擊之,利幾走。利幾者,項氏之將。項氏敗,利幾為陳公,不隨項羽,亡降高祖,高祖侯之潁川。高祖至雒陽,舉通侯籍召之,而利幾恐,故反。
\\\\
  六年,高祖五日一朝太公,如家人父子禮。太公家令說太公曰:「天無二日,土無二王。今高祖雖子,人主也;太公雖父,人臣也。柰何令人主拜人臣!如此,則威重不行。」後高祖朝,太公擁篲,迎門卻行。高祖大驚,下扶太公。太公曰:「帝,人主也,柰何以我亂天下法!」於是高祖乃尊太公為太上皇。心善家令言,賜金五百斤。
\\\\
  十二月,人有上變事告楚王信謀反,上問左右,左右爭欲擊之。用陳平計,乃偽遊雲夢,會諸侯於陳,楚王信迎,即因執之。是日,大赦天下。田肯賀,因說高祖曰:「陛下得韓信,又治秦中。秦,形勝之國,帶河山之險,縣隔千里,持戟百萬,秦得百二焉。地勢便利,其以下兵於諸侯,譬猶居高屋之上建瓴水也。夫齊,東有瑯邪、即墨之饒,南有泰山之固,西有濁河之限,北有勃海之利。地方二千里,持戟百萬,縣隔千里之外,齊得十二焉。故此東西秦也。非親子弟,莫可使王齊矣。」高祖曰:「善。」賜黃金五百斤。
\\\\
  後十餘日,封韓信為淮陰侯,分其地為二國。高祖曰將軍劉賈數有功,以為荊王,王淮東。弟交為楚王,王淮西。子肥為齊王,王七十餘城,民能齊言者皆屬齊。乃論功,與諸列侯剖符行封。徙韓王信太原。
\\\\
  七年,匈奴攻韓王信馬邑,信因與謀反太原。白土曼丘臣、王黃立故趙將趙利為王以反,高祖自往擊之。會天寒,士卒墮指者什二三,遂至平城。匈奴圍我平城,七日而後罷去。令樊噲止定代地。立兄劉仲為代王。
\\\\
  二月,高祖自平城過趙、雒陽,至長安。長樂宮成,丞相已下徙治長安。
\\\\
  八年,高祖東擊韓王信餘反寇於東垣。
\\\\
  蕭丞相營作未央宮,立東闕、北闕、前殿、武庫、太倉。高祖還,見宮闕壯甚,怒,謂蕭何曰:「天下匈匈苦戰數歲,成敗未可知,是何治宮室過度也?」蕭何曰:「天下方未定,故可因遂就宮室。且夫天子四海為家,非壯麗無以重威,且無令後世有以加也。」高祖乃說。
\\\\
  高祖之東垣,過柏人,趙相貫高等謀弒高祖,高祖心動,因不留。代王劉仲棄國亡,自歸雒陽,廢以為合陽侯。
\\\\
  九年,趙相貫高等事發覺,夷三族。廢趙王敖為宣平侯。是歲,徙貴族楚昭、屈、景、懷、齊田氏關中。
\\\\
  未央宮成。高祖大朝諸侯群臣,置酒未央前殿。高祖奉玉卮,起為太上皇壽,曰:「始大人常以臣無賴,不能治產業,不如仲力。今某之業所就孰與仲多?」殿上群臣皆呼萬歲,大笑為樂。
\\\\
  十年十月,淮南王黥布、梁王彭越、燕王盧綰、荊王劉賈、楚王劉交、齊王劉肥、長沙王吳芮皆來朝長樂宮。春夏無事。
\\\\
  七月,太上皇崩櫟陽宮。楚王、梁王皆來送葬。赦櫟陽囚。更命酈邑曰新豐。
\\\\
  八月,趙相國陳豨反代地。上曰:「豨嘗為吾使,甚有信。代地吾所急也,故封豨為列侯,以相國守代,今乃與王黃等劫掠代地!代地吏民非有罪也。其赦代吏民。」九月,上自東往擊之。至邯鄲,上喜曰:「豨不南據邯鄲而阻漳水,吾知其無能為也。」聞豨將皆故賈人也,上曰:「吾知所以與之。」乃多以金啗豨將,豨將多降者。
\\\\
  十一年,高祖在邯鄲誅豨等未畢,豨將侯敞將萬餘人游行,王黃軍曲逆,張春渡河擊聊城。漢使將軍郭蒙與齊將擊,大破之。太尉周勃道太原入,定代地。至馬邑,馬邑不下,即攻殘之。
\\\\
  豨將趙利守東垣,高祖攻之,不下。月餘,卒罵高祖,高祖怒。城降,令出罵者斬之,不罵者原之。於是乃分趙山北,立子恒以為代王,都晉陽。
\\\\
  春,淮陰侯韓信謀反關中,夷三族。
\\\\
  夏,梁王彭越謀反,廢遷蜀;復欲反,遂夷三族。立子恢為梁王,子友為淮陽王。
\\\\
  秋七月,淮南王黥布反,東并荊王劉賈地,北渡淮,楚王交走入薛。高祖自往擊之。立子長為淮南王。
\\\\
  十二年,十月,高祖已擊布軍會甀,布走,令別將追之。
\\\\
  高祖還歸,過沛,留。置酒沛宮,悉召故人父老子弟縱酒,發沛中兒得百二十人,教之歌。酒酣,高祖擊筑,自為歌詩曰:「大風起兮雲飛揚,威加海內兮歸故鄉,安得猛士兮守四方!」令兒皆和習之。高祖乃起舞,慷慨傷懷,泣數行下。謂沛父兄曰:「游子悲故鄉。吾雖都關中,萬歲後吾魂魄猶樂思沛。且朕自沛公以誅暴逆,遂有天下,其以沛為朕湯沐邑,復其民,世世無有所與。」沛父兄諸母故人日樂飲極驩,道舊故為笑樂。十餘日,高祖欲去,沛父兄固請留高祖。高祖曰:「吾人眾多,父兄不能給。」乃去。沛中空縣皆之邑西獻。高祖復留止,張飲三日。沛父兄皆頓首曰:「沛幸得復,豐未復,唯陛下哀憐之。」高祖曰:「豐吾所生長,極不忘耳,吾特為其以雍齒故反我為魏。」沛父兄固請,乃并復豐,比沛。於是拜沛侯劉濞為吳王。
\\\\
  漢將別擊布軍洮水南北,皆大破之,追得斬布鄱陽。
\\\\
  樊噲別將兵定代,斬陳豨當城。
\\\\
  十一月,高祖自布軍至長安。十二月,高祖曰:「秦始皇帝、楚隱王陳涉、魏安釐王、齊緡王、趙悼襄王皆絕無後,予守冢各十家,秦皇帝二十家,魏公子無忌五家。」赦代地吏民為陳豨、趙利所劫掠者,皆赦之。陳豨降將言豨反時,燕王盧綰使人之豨所,與陰謀。上使辟陽侯迎綰,綰稱病。辟陽侯歸,具言綰反有端矣。二月,使樊噲、周勃將兵擊燕王綰,赦燕吏民與反者。立皇子建為燕王。
\\\\
  高祖擊布時,為流矢所中,行道病。病甚,呂后迎良醫,醫入見,高祖問醫,醫曰:「病可治。」於是高祖嫚罵之曰:「吾以布衣提三尺劍取天下,此非天命乎?命乃在天,雖扁鵲何益!」遂不使治病,賜金五十斤罷之。已而呂后問:「陛下百歲後,蕭相國即死,令誰代之?」上曰:「曹參可。」問其次,上曰:「王陵可。然陵少憨,陳平可以助之。陳平智有餘,然難以獨任。周勃重厚少文,然安劉氏者必勃也,可令為太尉。」呂后復問其次,上曰:「此後亦非而所知也。」
\\\\
  盧綰與數千騎居塞下候伺,幸上病愈自入謝。
\\\\
  四月甲辰,高祖崩長樂宮。四日不發喪。呂后與審食其謀曰:「諸將與帝為編戶民,今北面為臣,此常怏怏,今乃事少主,非盡族是,天下不安。」人或聞之,語酈將軍。酈將軍往見審食其,曰:「吾聞帝已崩,四日不發喪,欲誅諸將。誠如此,天下危矣。陳平、灌嬰將十萬守滎陽,樊噲、周勃將二十萬定燕、代,此聞帝崩,諸將皆誅,必連兵還鄉以攻關中。大臣內叛,諸侯外反,亡可翹足而待也。」審食其入言之,乃以丁未發喪,大赦天下。
\\\\
  盧綰聞高祖崩,遂亡入匈奴。
\\\\
  丙寅,葬。己巳,立太子,至太上皇廟。群臣皆曰:「高祖起微細,撥亂世反之正,平定天下,為漢太祖,功最高。」上尊號為高皇帝。太子襲號為皇帝,孝惠帝也。令郡國諸侯各立高祖廟,以歲時祠。
\\\\
  及孝惠五年,思高祖之悲樂沛,以沛宮為高祖原廟。高祖所教歌兒百二十人,皆令為吹樂,後有缺,輒補之。
\\\\
  高帝八男:長庶齊悼惠王肥;次孝惠,呂后子;次戚夫人子趙隱王如意;次代王恒,已立為孝文帝,薄太后子;次梁王恢,呂太后時徙為趙共王;次淮陽王友,呂太后時徙為趙幽王;次淮南厲王長;次燕王建。
\\\\
  太史公曰:夏之政忠。忠之敝,小人以野,故殷人承之以敬。敬之敝,小人以鬼,故周人承之以文。文之敝,小人以僿,故救僿莫若以忠。三王之道若循環,終而復始。周秦之閒,可謂文敝矣。秦政不改,反酷刑法,豈不繆乎?故漢興,承敝易變,使人不倦,得天統矣。朝以十月。車服黃屋左纛。葬長陵。

\part{書}
\LARGE

\section{天官書}
  中宮天極星,其一明者,太一常居也;旁三星三公,或曰子屬。後句四星,末大星正妃,餘三星後宮之屬也。環之匡衛十二星,藩臣。皆曰紫宮。
\\\\
  前列直斗口三星,隨北端兌,若見若不,曰陰德,或曰天一。紫宮左三星曰天槍,右五星曰天棓,後六星絕漢抵營室,曰閣道。
\\\\
  北斗七星,所謂「旋、璣、玉衡以齊七政」。杓攜龍角,衡殷南斗,魁枕參首。用昏建者杓;杓,自華以西南。夜半建者衡;衡,殷中州河、濟之閒。平旦建者魁;魁,海岱以東北也。斗為帝車,運于中央,臨制四鄉。分陰陽,建四時,均五行,移節度,定諸紀,皆系於斗。
\\\\
  斗魁戴匡六星曰文昌宮:一曰上將,二曰次將,三曰貴相,四曰司命,五曰司中,六曰司祿。在斗魁中,貴人之牢。魁下六星,兩兩相比者,名曰三能。三能色齊,君臣和;不齊,為乖戾。輔星明近,輔臣親彊;斥小,疏弱。
\\\\
  杓端有兩星:一內為矛,招搖;一外為盾,天鋒。有句圜十五星,屬杓,曰賤人之牢。其牢中星實則囚多,虛則開出。
\\\\
  天一、槍、棓、矛、盾動搖,角大,兵起。
\\\\
  東宮蒼龍,房、心。心為明堂,大星天王,前後星子屬。不欲直,直則天王失計。房為府,曰天駟。其陰,右驂。旁有兩星曰衿;北一星曰舝。東北曲十二星曰旗。旗中四星天市;中六星曰市樓。市中星眾者實;其虛則秏。房南眾星曰騎官。
\\\\
  左角,李;右角,將。大角者,天王帝廷。其兩旁各有三星,鼎足句之,曰攝提。攝提者,直斗杓所指,以建時節,故曰「攝提格」。亢為疏廟,主疾。其南北兩大星,曰南門。氐為天根,主疫。
\\\\
  尾為九子,曰君臣;斥絕,不和。箕為敖客,曰口舌。
\\\\
  火犯守角,則有戰。房、心,王者惡之也。
\\\\
  南宮朱鳥,權、衡。衡,太微,三光之廷。匡衛十二星,藩臣:西,將;東,相;南四星,執法;中,端門;門左右,掖門。門內六星,諸侯。其內五星,五帝坐。後聚一十五星,蔚然,曰郎位;傍一大星,將位也。月、五星順入,軌道,司其出,所守,天子所誅也。其逆入,若不軌道,以所犯命之;中坐,成形,皆群下從謀也。金、火尤甚。廷藩西有隋星五,曰少微,士大夫。權,軒轅。軒轅,黃龍體。前大星,女主象;旁小星,御者後宮屬。月、五星守犯者,如衡占。
\\\\
  東井為水事。其西曲星曰鉞。鉞北,北河;南,南河;兩河、天闕閒為關梁。輿鬼,鬼祠事;中白者為質。火守南北河,兵起,穀不登。故德成衡,觀成潢,傷成鉞,禍成井,誅成質。
\\\\
  柳為鳥注,主木草。七星,頸,為員官。主急事。張,素,為廚,主觴客。翼為羽翮,主遠客。
\\\\
  軫為車,主風。其旁有一小星,曰長沙,星星不欲明;明與四星等,若五星入軫中,兵大起。軫南眾星曰天庫樓;庫有五車。車星角若益眾,及不具,無處車馬。
\\\\
  西宮咸池,曰天五潢。五潢,五帝車舍。火入,旱;金,兵;水,水。中有三柱;柱不具,兵起。
\\\\
  奎曰封豕,為溝瀆。婁為聚眾。胃為天倉。其南眾星曰廥積。
\\\\
  昴曰髦頭,胡星也,為白衣會。畢曰罕車,為邊兵,主弋獵。其大星旁小星為附耳。附耳搖動,有讒亂臣在側。昴、畢閒為天街。其陰,陰國;陽,陽國。
\\\\
  參為白虎。三星直者,是為衡石。下有三星,兌,曰罰,為斬艾事。其外四星,左右肩股也。小三星隅置,曰觜觿,為虎首,主葆旅事。其南有四星,曰天廁。廁下一星,曰天矢。矢黃則吉;青、白、黑,凶。其西有句曲九星,三處羅:一曰天旗,二曰天苑,三曰九游。其東有大星曰狼。狼角變色,多盜賊。下有四星曰弧,直狼。狼比地有大星,曰南極老人。老人見,治安;不見,兵起。常以秋分時候之于南郊。
\\\\
  附耳入畢中,兵起。
\\\\
  北宮玄武,虛、危。危為蓋屋;虛為哭泣之事。
\\\\
  其南有眾星,曰羽林天軍。軍西為壘,或曰鉞。旁有一大星為北落。北落若微亡,軍星動角益希,及五星犯北落,入軍,軍起。火、金、水尤甚:火,軍憂;水,[水]患;木、土,軍吉。危東六星,兩兩相比,曰司空。
\\\\
  營室為清廟,曰離宮、閣道。漢中四星,曰天駟。旁一星,曰王良。王良策馬,車騎滿野。旁有八星,絕漢,曰天潢。天潢旁,江星。江星動,人涉水。
\\\\
  杵、臼四星,在危南。匏瓜,有青黑星守之,魚鹽貴。
\\\\
  南斗為廟,其北建星。建星者,旗也。牽牛為犧牲。其北河鼓。河鼓大星,上將;左右,左右將。婺女,其北織女。織女,天女孫也。
\\\\
  察日、月之行以揆歲星順逆。曰東方木,主春,日甲乙。義失者,罰出歲星。歲星贏縮,以其捨命國。所在國不可伐,可以罰人。其趨舍而前曰贏,退舍曰縮。贏,其國有兵不復;縮,其國有憂,將亡,國傾敗。其所在,五星皆從而聚於一舍,其下之國可以義致天下。
\\\\
  以攝提格歲:歲陰左行在寅,歲星右轉居丑。正月,與斗、牽牛晨出東方,名曰監德。色蒼蒼有光。其失次,有應見柳。歲早,水;晚,旱。
\\\\
  歲星出,東行十二度,百日而止,反逆行;逆行八度,百日,復東行。歲行三十度十六分度之七,率日行十二分度之一,十二歲而周天。出常東方,以晨;入於西方,用昏。
\\\\
  單閼歲:歲陰在卯,星居子。以二月與婺女、虛、危晨出,曰降入。大有光。其失次,有應見張。[名曰降入]其歲大水。
\\\\
  執徐歲:歲陰在辰,星居亥。以三月[居]與營室、東壁晨出,曰青章。青青甚章。其失次;有應見軫。[曰青章]歲早,旱;晚,水。
\\\\
  大荒駱歲:歲陰在巳,星居戌。以四月與奎、婁[胃昴]晨出,曰跰踵。熊熊赤色,有光。其失次,有應見亢。
\\\\
  敦牂歲:歲陰在午,星居酉。以五月與胃、昴、畢晨出,曰開明。炎炎有光。偃兵;唯利公王,不利治兵。其失次,有應見房。歲早,旱;晚,水。
\\\\
  協洽歲:歲陰在未,星居申。以六月與觜觿、參晨出,曰長列。昭昭有光。利行兵。其失次,有應見箕。
\\\\
  涒灘歲:歲陰在申,星居未。以七月與東井、輿鬼晨出,曰大音。昭昭白。其失次,有應見牽牛。
\\\\
  作鄂歲:歲陰在酉,星居午。以八月與柳、七星、張晨出,曰(為)長王。作作有芒。國其昌,熟穀。其失次,有應見危。[曰大章]有旱而昌,有女喪,民疾。
\\\\
  閹茂歲:歲陰在戌,星居巳。以九月與翼、軫晨出,曰天睢。白色大明。其失次,有應見東壁。歲水,女喪。
\\\\
  大淵獻歲:歲陰在亥,星居辰。以十月與角、亢晨出,曰大章。蒼蒼然,星若躍而陰出旦,是謂「正平」。起師旅,其率必武;其國有德,將有四海。其失次,有應見婁。
\\\\
  困敦歲:歲陰在子,星居卯。以十一月與氐、房、心晨出,曰天泉。玄色甚明。江池其昌,不利起兵。其失次,有應(在)[見]昴。
\\\\
  赤奮若歲:歲陰在丑,星居寅,以十二月與尾、箕晨出,曰天皓。黫然黑色甚明。其失次,有應見參。
\\\\
  當居不居,居之又左右搖,未當去去之,與他星會,其國凶。所居久,國有德厚。其角動,乍小乍大,若色數變,人主有憂。
\\\\
  其失次舍以下,進而東北,三月生天棓,長四丈,末兌,進而東南,三月生彗星,長二丈,類彗。退而西北,三月生天欃,長四丈,末兌。退而西南,三月生天槍,長數丈,兩頭兌。謹視其所見之國,不可舉事用兵。其出如浮如沈,其國有土功;如沈如浮,其野亡。色赤而有角,其所居國昌。迎角而戰者,不勝。星色赤黃而沈,所居野大穰。色青白而赤灰,所居野有憂。歲星入月,其野有逐相;與太白斗,其野有破軍。
\\\\
  歲星一曰攝提,曰重華,曰應星,曰紀星。營室為清廟,歲星廟也。
\\\\
  察剛氣以處熒惑。曰南方火,主夏,日丙、丁。禮失,罰出熒惑,熒惑失行是也。出則有兵,入則兵散。以其捨命國。(熒惑)熒惑為勃亂,殘賊、疾、喪、饑、兵。反道二舍以上,居之,三月有殃,五月受兵,七月半亡地,九月太半亡地。因與俱出入,國絕祀。居之,殃還至,雖大當小;久而至,當小反大。其南為丈夫[喪],北為女子喪。若角動繞環之,及乍前乍後,左右,殃益大。與他星斗,光相逮,為害;不相逮,不害。五星皆從而聚于一舍,其下國可以禮致天下。
\\\\
  法,出東行十六舍而止;逆行二舍;六旬,復東行,自所止數十舍,十月而入西方;伏行五月,出東方。其出西方曰「反明」,主命者惡之。東行急,一日行一度半。
\\\\
  其行東、西、南、北疾也。兵各聚其下;用戰,順之勝,逆之敗。熒惑從太白,軍憂;離之,軍卻。出太白陰,有分軍;行其陽,有偏將戰。當其行,太白逮之,破軍殺將。其入守犯太微、軒轅、營室,主命惡之。心為明堂,熒惑廟也。謹候此。
\\\\
  歷斗之會以定填星之位。曰中央土,主季夏,日戊、己,黃帝,主德,女主象也。歲填一宿,其所居國吉。未當居而居,若已去而復還,還居之,其國得土,不乃得女。若當居而不居,既已居之,又西東去,其國失土,不乃失女,不可舉事用兵。其居久,其國福厚;易,福薄。
\\\\
  其一名曰地侯,主歲。歲行十(二)[三]度百十二分度之五,日行二十八分度之一,二十八歲周天。其所居,五星皆從而聚于一舍,其下之國,可[以]重致天下。禮、德、義、殺、刑盡失,而填星乃為之動搖。
\\\\
  贏,為王不寧;其縮,有軍不復。填星,其色黃,九芒,音曰黃鐘宮。其失次上二三宿曰贏,有主命不成,不乃大水。失次下二三宿曰縮,有后戚,其歲不復,不乃天裂若地動。
\\\\
  斗為文太室,填星廟,天子之星也。
\\\\
  木星與土合,為內亂。饑,主勿用戰,敗;水則變謀而更事;火為旱;金為白衣會若水。金在南曰牝牡,年穀熟,金在北,歲偏無。火與水合為焠,與金合為鑠,為喪,皆不可舉事,用兵大敗。土為憂,主孽卿;大饑,戰敗,為北軍,軍困,舉事大敗。土與水合,穰而擁閼,有覆軍,其國不可舉事。出,亡地;入,得地。金為疾,為內兵,亡地。三星若合,其宿地國外內有兵與喪,改立公王。四星合,兵喪并起,君子憂,小人流。五星合,是為易行,有德,受慶,改立大人,掩有四方,子孫蕃昌;無德,受殃若亡。五星皆大,其事亦大;皆小,事亦小。
\\\\
  蚤出者為贏,贏者為客。晚出者為縮,縮者為主人。必有天應見於杓星。同舍為合。相陵為斗,七寸以內必之矣。
\\\\
  五星色白圜,為喪旱;赤圜,則中不平,為兵;青圜,為憂水;黑圜,為疾,多死;黃圜,則吉。赤角犯我城,黃角地之爭,白角哭泣之聲,青角有兵憂,黑角則水。意,行窮兵之所終。五星同色,天下偃兵,百姓寧昌。春風秋雨,冬寒夏暑,動搖常以此。
\\\\
  填星出百二十日而逆西行,西行百二十日反東行。見三百三十日而入,入三十日復出東方。太歲在甲寅,鎮星在東壁,故在營室。
\\\\
  察日行以處位太白。曰西方,秋,(司兵月行及天矢)日庚、辛,主殺。殺失者,罰出太白。太白失行,以其捨命國。其出行十八舍二百四十日而入。入東方,伏行十一舍百三十日;其入西方,伏行三舍十六日而出。當出不出,當入不入,是謂失舍,不有破軍,必有國君之篡。
\\\\
  其紀上元,以攝提格之歲,與營室晨出東方,至角而入;與營室夕出西方,至角而入;與角晨出,入畢;與角夕出,入畢;與畢晨出,入箕;與畢夕出,入箕;與箕晨出,入柳;與箕夕出,入柳;與柳晨出,入營室;與柳夕出,入營室。凡出入東西各五,為八歲,二百二十日,復與營室晨出東方。其大率,歲一周天。其始出東方,行遲,率日半度,一百二十日,必逆行一二舍;上極而反,東行,行日一度半,一百二十日入。其庳,近日,曰明星,柔;高,遠日,曰大囂,剛。其始出西[方],行疾,率日一度半,百二十日;上極而行遲,日半度,百二十日,旦入,必逆行一二舍而入。其庳,近日,曰大白,柔;高,遠日,曰大相,剛。出以辰、戌,入以丑、未。
\\\\
  當出不出,未當入而入,天下偃兵,兵在外,入。未當出而出,當入而不入,[天]下起兵,有破國。其當期出也,其國昌。其出東為東,入東為北方;出西為西,入西為南方。所居久,其鄉利;(疾)[易],其鄉凶。
\\\\
  出西(逆行)至東,正西國吉。出東至西,正東國吉。其出不經天;經天,天下革政。
\\\\
  小以角動,兵起。始出大,後小,兵弱;出小,後大,兵彊。出高,用兵深吉,淺凶;庳,淺吉,深凶。日方南金居其南,日方北金居其北,曰贏,侯王不寧,用兵進吉退凶。日方南金居其北,日方北金居其南,曰縮,侯王有憂,用兵退吉進凶。用兵象太白:太白行疾,疾行;遲,遲行。角,敢戰。動搖躁,躁。圜以靜,靜。順角所指,吉;反之,皆凶。出則出兵,入則入兵。赤角,有戰;白角,有喪;黑圜角,憂,有水事;青圜小角,憂,有木事;黃圜和角,有土事,有年。其已出三日而復,有微入,入三日乃復盛出,是謂耎,其下國有軍敗將北。其已入三日又復微出,出三日而復盛入,其下國有憂;師有糧食兵革,遺人用之;卒雖眾,將為人虜。其出西失行,外國敗;其出東失行,中國敗。其色大圜黃滜,可為好事;其圜大赤,兵盛不戰。
\\\\
  太白白,比狼;赤,比心;黃,比參左肩;蒼,比參右肩;黑,比奎大星。五星皆從太白而聚乎一舍,其下之國可以兵從天下。居實,有得也;居虛,無得也。行勝色,色勝位,有位勝無位,有色勝無色,行得盡勝之。出而留桑榆閒,疾其下國。上而疾,未盡其日,過參天,疾其對國。上復下,下復上,有反將。其入月,將僇。金、木星合,光,其下戰不合,兵雖起而不鬬;合相毀,野有破軍。出西方,昏而出陰,陰兵彊;暮食出,小弱;夜半出,中弱;雞鳴出,大弱:是謂陰陷於陽。其在東方,乘明而出陽,陽兵之彊,雞鳴出,小弱;夜半出,中弱;昏出,大弱:是謂陽陷於陰。太白伏也,以出兵,兵有殃。其出卯南,南勝北方;出卯北,北勝南方;正在卯,東國利。出酉北,北勝南方;出酉南,南勝北方;正在酉,西國勝。
\\\\
  其與列星相犯,小戰;五星,大戰。其相犯,太白出其南,南國敗;出其北,北國敗。行疾,武;不行,文。色白五芒,出蚤為月蝕,晚為天夭及彗星,將發其國。出東為德,舉事左之迎之,吉。出西為刑,舉事右之背之,吉。反之皆凶。太白光見景,戰勝。晝見而經天,是謂爭明,彊國弱,小國彊,女主昌。
\\\\
  亢為疏廟,太白廟也。太白,大臣也,其號上公。其他名殷星、太正、營星、觀星、宮星、明星、大衰、大澤、終星、大相、天浩、序星、月緯。大司馬位謹候此。
\\\\
  察日辰之會,以治辰星之位。曰北方水,太陰之精,主冬,日壬、癸。刑失者,罰出辰星,以其宿命國。
\\\\
  是正四時:仲春春分,夕出郊奎、婁、胃東五舍,為齊;仲夏夏至,夕出郊東井、輿鬼、柳東七舍,為楚;仲秋秋分,夕出郊角、亢、氐、房東四舍,為漢;仲冬冬至,晨出郊東方,與尾、箕、斗、牽牛俱西,為中國。其出入常以辰、戌、丑、未。
\\\\
  其蚤,為月蝕;晚,為彗星及天夭。其時宜效不效為失,追兵在外不戰。一時不出,其時不和;四時不出,天下大饑。其當效而出也,色白為旱,黃為五穀熟,赤為兵,黑為水。出東方,大而白,有兵於外,解。常在東方,其赤,中國勝;其西而赤,外國利。無兵於外而赤,兵起。其與太白俱出東方,皆赤而角,外國大敗,中國勝;其與太白俱出西方,皆赤而角,外國利。五星分天之中,積于東方,中國利;積于西方,外國用[兵]者利。五星皆從辰星而聚于一舍,其所捨之國可以法致天下。辰星不出,太白為客;其出,太白為主。出而與太白不相從,野雖有軍,不戰。出東方,太白出西方;若出西方,太白出東方,為格,野雖有兵不戰。失其時而出,為當寒反溫,當溫反寒。當出不出,是謂擊卒,兵大起。其入太白中而上出,破軍殺將,客軍勝;下出,客亡地。辰星來抵太白,太白不去,將死。正旗上出,破軍殺將,客勝;下出,客亡地。視旗所指,以命破軍。其繞環太白,若與鬬,大戰,客勝。兔過太白,閒可椷劍,小戰,客勝。兔居太白前,軍罷;出太白左,小戰;摩太白,有數萬人戰,主人吏死;出太白右,去三尺,軍急約戰。青角,兵憂;黑角,水。赤行窮兵之所終。
\\\\
  兔七命,曰小正、辰星、天欃、安周星、細爽、能星、鉤星。其色黃而小,出而易處,天下之文變而不善矣。兔五色,青圜憂,白圜喪,赤圜中不平,黑圜吉。赤角犯我城,黃角地之爭,白角號泣之聲。
\\\\
  其出東方,行四舍四十八日,其數二十日,而反入于東方;其出西方,行四舍四十八日,其數二十日,而反入于西方。其一候之營室、角、畢、箕、柳。出房、心閒,地動。
\\\\
  辰星之色:春,青黃;夏,赤白;秋,青白,而歲熟;冬,黃而不明。即變其色,其時不昌。春不見,大風,秋則不實。夏不見,有六十日之旱,月蝕。秋不見,有兵,春則不生。冬不見,陰雨六十日,有流邑,夏則不長。
\\\\
  角、亢、氐,兗州。房、心,豫州。尾、箕,幽州。斗,江、湖。牽牛、婺女,楊州。虛、危,青州。營室至東壁,并州。奎、婁、胃,徐州。昴、畢,冀州。觜觿、參,益州。東井、輿鬼,雍州。柳、七星、張,三河。翼、軫,荊州。
\\\\
  七星為員官,辰星廟,蠻夷星也。
\\\\
  兩軍相當,日暈;暈等,力鈞;厚長大,有勝;薄短小,無勝。重抱大破無。抱為和,背[為]不和,為分離相去。直為自立,立侯王;(指暈)[破軍](若曰)殺將。負且戴,有喜。圍在中,中勝;在外,外勝。青外赤中,以和相去;赤外青中,以惡相去。氣暈先至而後去,居軍勝。先至先去,前利後病;後至後去,前病後利;後至先去,前後皆病,居軍不勝。見而去,其發疾,雖勝無功。見半日以上,功大。白虹屈短,上下兌,有者下大流血。日暈制勝,近期三十日,遠期六十日。
\\\\
  其食,食所不利;復生,生所利;而食益盡,為主位。以其直及日所宿,加以日時,用命其國也。
\\\\
  月行中道,安寧和平。陰閒,多水,陰事。外北三尺,陰星。北三尺,太陰,大水,兵。陽閒,驕恣。陽星,多暴獄。太陽,大旱喪也。角、天門,十月為四月,十一月為五月,十二月為六月,水發,近三尺,遠五尺。犯四輔,輔臣誅。行南北河,以陰陽言,旱水兵喪。
\\\\
  月蝕歲星,其宿地,饑若亡。熒惑也亂,填星也下犯上,太白也彊國以戰敗,辰星也女亂。(食)[蝕]大角,主命者惡之;心,則為內賊亂也;列星,其宿地憂。
\\\\
  月食始日,五月者六,六月者五,五月復六,六月者一,而五月者五,凡百一十三月而復始。故月蝕,常也;日蝕,為不臧也。甲、乙,四海之外,日月不占。丙、丁,江、淮、海岱也。戊、己,中州、河、濟也。庚、辛,華山以西。壬、癸,恒山以北。日蝕,國君;月蝕,將相當之。
\\\\
  國皇星,大而赤,狀類南極。所出,其下起兵,兵彊;其沖不利。
\\\\
  昭明星,大而白,無角,乍上乍下。所出國,起兵,多變。
\\\\
  五殘星,出正東東方之野。其星狀類辰星,去地可六丈。
\\\\
  大賊星,出正南南方之野。星去地可六丈,大而赤,數動,有光。
\\\\
  司危星,出正西西方之野。星去地可六丈,大而白,類太白。
\\\\
  獄漢星,出正北北方之野。星去地可六丈,大而赤,數動,察之中青。此四野星所出,出非其方,其下有兵,沖不利。
\\\\
  四填星,所出四隅,去地可四丈。
\\\\
  地維咸光,亦出四隅,去地可三丈,若月始出。所見,下有亂;亂者亡,有德者昌。
\\\\
  燭星,狀如太白,其出也不行。見則滅。所燭者,城邑亂。
\\\\
  如星非星,如雲非雲,命曰歸邪。歸邪出,必有歸國者。
\\\\
  星者,金之散氣,[其]本曰火。星眾,國吉;少則凶。
\\\\
  漢者,亦金之散氣,其本曰水。漢,星多,多水,少則旱,其大經也。
\\\\
  天鼓,有音如雷非雷,音在地而下及地。其所往者,兵發其下。
\\\\
  天狗,狀如大奔星,有聲,其下止地,類狗。所墮及,望之如火光炎炎沖天。其下圜如數頃田處,上兌者則有黃色,千里破軍殺將。
\\\\
  格澤星者,如炎火之狀。黃白,起地而上。下大,上兌。其見也,不種而穫;不有土功,必有大害。
\\\\
  蚩尤之旗,類彗而後曲,象旗。見則王者征伐四方。
\\\\
  旬始,出於北斗旁,狀如雄雞。其怒,青黑,象伏鱉。
\\\\
  枉矢,類大流星,蛇行而倉黑,望之如有毛羽然。
\\\\
  長庚,如一匹布著天。此星見,兵起。
\\\\
  星墜至地,則石也。河、濟之閒,時有墜星。
\\\\
  天精而見景星。景星者,德星也。其狀無常,常出於有道之國。
\\\\
  凡望雲氣,仰而望之,三四百里;平望,在桑榆上,千餘[里]二千里;登高而望之,下屬地者三千里。雲氣有獸居上者,勝。
\\\\
  自華以南,氣下黑上赤。嵩高、三河之郊,氣正赤。恒山之北,氣下黑下青。勃、碣、海、岱之閒,氣皆黑。江、淮之閒,氣皆白。
\\\\
  徒氣白。土功氣黃。車氣乍高乍下,往往而聚。騎氣卑而布。卒氣摶。前卑而後高者,疾;前方而後高者,兌;後兌而卑者,卻。其氣平者其行徐。前高而後卑者,不止而反。氣相遇者,卑勝高,兌勝方。氣來卑而循車通者,不過三四日,去之五六里見。氣來高七八尺者,不過五六日,去之十餘里見。氣來高丈餘二丈者,不過三四十日,去之五六十里見。
\\\\
  稍云精白者,其將悍,其士怯。其大根而前絕遠者,當戰。青白,其前低者,戰勝;其前赤而仰者,戰不勝。陣雲如立垣。杼雲類杼。軸雲摶兩端兌。杓雲如繩者,居前亙天,其半半天。其蛪者類闕旗故。鉤雲句曲。諸此雲見,以五色合占。而澤摶密,其見動人,乃有占;兵必起,合斗其直。
\\\\
  王朔所候,決於日旁。日旁雲氣,人主象。皆如其形以占。
\\\\
  故北夷之氣如群畜穹閭,南夷之氣類舟船幡旗。大水處,敗軍場,破國之虛,下有積錢,金寶之上,皆有氣,不可不察。海旁蜄氣象樓臺;廣野氣成宮闕然。雲氣各象其山川人民所聚積。
\\\\
  故候息秏者,入國邑,視封疆田疇之正治,城郭室屋門戶之潤澤,次至車服畜產精華。實息者,吉;虛秏者,凶。
\\\\
  若煙非煙,若雲非雲,郁郁紛紛,蕭索輪囷,是謂卿雲。卿雲[見],喜氣也。若霧非霧,衣冠而不濡,見則其域被甲而趨。
\\\\
  (天)[夫]雷電、蝦虹、辟歷、夜明者,陽氣之動者也,春夏則發,秋冬則藏,故候者無不司之。
\\\\
  天開縣物,地動坼絕。山崩及徙,川塞谿垘;水澹(澤竭)地長,[澤竭]見象。城郭門閭,閨臬[枯槁]槁枯;宮廟邸第,人民所次。謠俗車服,觀民飲食。五穀草木,觀其所屬。倉府廄庫,四通之路。六畜禽獸,所產去就;魚鱉鳥鼠,觀其所處。鬼哭若呼,其人逢俉。化言,誠然。
\\\\
  凡候歲美惡,謹候歲始。歲始或冬至日,產氣始萌。臘明日,人眾卒歲,一會飲食,發陽氣,故曰初歲。正月旦,王者歲首;立春日,四時之(卒)始也。四始者,候之日。
\\\\
  而漢魏鮮集臘明正月旦決八風。風從南方來,大旱;西南,小旱;西方,有兵;西北,戎菽為,小雨,趣兵;北方,為中歲;東北,為上歲;東方,大水;東南,民有疾疫,歲惡。故八風各與其沖對,課多者為勝。多勝少,久勝亟,疾勝徐。旦至食,為麥;食至日昳,為稷;昳至餔,為黍;餔至下餔,為菽;下餔至日入,為麻。欲終日(有雨)有雲,有風,有日。日當其時者,深而多實;無雲有風日,當其時,淺而多實;有雲風,無日,當其時,深而少實;有日,無雲,不風,當其時者稼有敗。如食頃,小敗;熟五斗米頃,大敗。則風復起,有雲,其稼復起。各以其時用雲色占種(其)所宜。其雨雪若寒,歲惡。
\\\\
  是日光明,聽都邑人民之聲。聲宮,則歲善,吉;商,則有兵;徵,旱;羽,水;角,歲惡。
\\\\
  或從正月旦比數雨。率日食一升,至七升而極;過之,不占。數至十二日,日直其月,占水旱。為其環(城)[域]千里內占,則(其)為天下候,竟正月。月所離列宿,日、風、雲,占其國。然必察太歲所在。在金,穰;水,毀;木,饑;火,旱。此其大經也。
\\\\
  正月上甲,風從東方,宜蠶;風從西方,若旦黃雲,惡。
\\\\
  冬至短極,縣土炭,炭動,鹿解角,蘭根出,泉水躍,略以知日至,要決晷景。歲星所在,五穀逢昌。其對為沖,歲乃有殃。
\\\\
  太史公曰:自初生民以來,世主曷嘗不歷日月星辰?及至五家、三代,紹而明之,內冠帶,外夷狄,分中國為十有二州,仰則觀象於天,俯則法類於地。天則有日月,地則有陰陽。天有五星,地有五行。天則有列宿,地則有州域。三光者,陰陽之精,氣本在地,而聖人統理之。
\\\\
  幽厲以往,尚矣。所見天變,皆國殊窟穴,家占物怪,以合時應,其文圖籍禨祥不法。是以孔子論六經,紀異而說不書。至天道命,不傳;傳其人,不待告;告非其人,雖言不著。
\\\\
  昔之傳天數者:高辛之前,重、黎;於唐、虞,羲、和;有夏,昆吾;殷商,巫咸;周室,史佚、萇弘;於宋,子韋;鄭則裨灶;在齊,甘公;楚,唐眛;趙,尹皋;魏,石申。
\\\\
  夫天運,三十歲一小變,百年中變,五百載大變;三大變一紀,三紀而大備:此其大數也。為國者必貴三五。上下各千歲,然後天人之際續備。
\\\\
  太史公推古天變,未有可考于今者。蓋略以春秋二百四十二年之閒,日蝕三十六,彗星三見,宋襄公時星隕如雨。天子微,諸侯力政,五伯代興,更為主命,自是之後,眾暴寡,大并小。秦、楚、吳、越,夷狄也,為彊伯。田氏篡齊,三家分晉,并為戰國。爭於攻取,兵革更起,城邑數屠,因以饑饉疾疫焦苦,臣主共憂患,其察禨祥候星氣尤急。近世十二諸侯七國相王,言從衡者繼踵,而皋、唐、甘、石因時務論其書傳,故其占驗淩雜米鹽。
\\\\
  二十八舍主十二州,斗秉兼之,所從來久矣。秦之疆也,候在太白,占於狼、弧。吳、楚之疆,候在熒惑,占於鳥衡。燕、齊之疆,候在辰星,占於虛、危。宋、鄭之疆,候在歲星,占於房、心。晉之疆,亦候在辰星,占於參罰。
\\\\
  及秦并吞三晉、燕、代,自河山以南者中國。中國於四海內則在東南,為陽;陽則日、歲星、熒惑、填星;占於街南,畢主之。其西北則胡、貉、月氏諸衣旃裘引弓之民,為陰;陰則月、太白、辰星;占於街北,昴主之。故中國山川東北流,其維,首在隴、蜀,尾沒于勃、碣。是以秦、晉好用兵,復占太白,太白主中國;而胡、貉數侵掠,獨占辰星,辰星出入躁疾,常主夷狄:其大經也。此更為客主人。熒惑為孛,外則理兵,內則理政。故曰「雖有明天子,必視熒惑所在」。諸侯更彊,時菑異記,無可錄者。
\\\\
  秦始皇之時,十五年彗星四見,久者八十日,長或竟天。其後秦遂以兵滅六王,并中國,外攘四夷,死人如亂麻,因以張楚并起,三十年之閒兵相駘藉,不可勝數。自蚩尤以來,未嘗若斯也。
\\\\
  項羽救鉅鹿,枉矢西流,山東遂合從諸侯,西坑秦人,誅屠咸陽。
\\\\
  漢之興,五星聚于東井。平城之圍,月暈參、畢七重。諸呂作亂,日蝕,晝晦。吳楚七國叛逆,彗星數丈,天狗過梁野;及兵起,遂伏尸流血其下。元光、元狩,蚩尤之旗再見,長則半天。其後京師師四出,誅夷狄者數十年,而伐胡尤甚。越之亡,熒惑守鬬;朝鮮之拔,星茀于河戍;兵征大宛,星茀招搖:此其犖犖大者。若至委曲小變,不可勝道。由是觀之,未有不先形見而應隨之者也。
\\\\
  夫自漢之為天數者,星則唐都,氣則王朔,占歲則魏鮮。故甘、石歷五星法,唯獨熒惑有反逆行;逆行所守,及他星逆行,日月薄蝕,皆以為占。
\\\\
  余觀史記,考行事,百年之中,五星無出而不反逆行,反逆行,嘗盛大而變色;日月薄蝕,行南北有時:此其大度也。故紫宮、房心、權衡、咸池、虛危列宿部星,此天之五官坐位也,為經,不移徙,大小有差,闊狹有常。水、火、金、木、填星,此五星者,天之五佐,為[經]緯,見伏有時,所過行贏縮有度。
\\\\
  日變修德,月變省刑,星變結和。凡天變,過度乃占。國君彊大,有德者昌;羽小,飾詐者亡。太上修德,其次修政,其次修救,其次修禳,正下無之。夫常星之變希見,而三光之占亟用。日月暈適,雲風,此天之客氣,其發見亦有大運。然其與政事俯仰,最近(大)[天]人之符。此五者,天之感動。為天數者,必通三五。終始古今,深觀時變,察其精粗,則天官備矣。
\\\\
  蒼帝行德,天門為之開。赤帝行德,天牢為之空。黃帝行德,天夭為之起。風從西北來,必以庚、辛。一秋中,五至,大赦;三至,小赦。白帝行德,以正月二十日、二十一日,月暈圍,常大赦載,謂有太陽也。一曰:白帝行德,畢、昴為之圍。圍三暮,德乃成;不三暮,及圍不合,德不成。二曰:以辰圍,不出其旬。黑帝行德,天關為之動。天行德,天子更立年;不德,風雨破石。三能、三衡者,天廷也。客星出天廷,有奇令。

\part{世家}
\LARGE
\section{孔子世家}
  孔子生魯昌平鄉陬邑。其先宋人也,曰孔防叔。防叔生伯夏,伯夏生叔梁紇。紇與顏氏女野合而生孔子,禱於尼丘得孔子。魯襄公二十二年而孔子生。生而首上圩頂,故因名曰丘云。字仲尼,姓孔氏。
\\\\
  丘生而叔梁紇死,葬於防山。防山在魯東,由是孔子疑其父墓處,母諱之也。孔子為兒嬉戲,常陳俎豆,設禮容。孔子母死,乃殯五父之衢,蓋其慎也。郰人輓父之母誨孔子父墓,然後往合葬於防焉。
\\\\
  孔子要絰,季氏饗士,孔子與往。陽虎絀曰:「季氏饗士,非敢饗子也。」孔子由是退。
\\\\
  孔子年十七,魯大夫孟釐子病且死,誡其嗣懿子曰:「孔丘,聖人之後,滅於宋。其祖弗父何始有宋而嗣讓厲公。及正考父佐戴、武、宣公,三命茲益恭,故鼎銘云:『一命而僂,再命而傴,三命而俯,循墻而走,亦莫敢余侮。饘於是,粥於是,以餬余口。』其恭如是。吾聞聖人之後,雖不當世,必有達者。今孔丘年少好禮,其達者歟?吾即沒,若必師之。」及釐子卒,懿子與魯人南宮敬叔往學禮焉。是歲,季武子卒,平子代立。
\\\\
  孔子貧且賤。及長,嘗為季氏史,料量平;嘗為司職吏而畜蕃息。由是為司空。已而去魯,斥乎齊,逐乎宋、衛,困於陳蔡之間,於是反魯。孔子長九尺有六寸,人皆謂之「長人」而異之。魯復善待,由是反魯。
\\\\
  魯南宮敬叔言魯君曰:「請與孔子適周。」魯君與之一乘車,兩馬,一豎子俱,適周問禮,蓋見老子云。辭去,而老子送之曰:「吾聞富貴者送人以財,仁人者送人以言。吾不能富貴,竊仁人之號,送子以言,曰:『聰明深察而近於死者,好議人者也。博辯廣大危其身者,發人之惡者也。為人子者毋以有己,為人臣者毋以有己。』」孔子自周反于魯,弟子稍益進焉。
\\\\
  是時也,晉平公淫,六卿擅權,東伐諸侯;楚靈王兵彊,陵轢中國;齊大而近於魯。魯小弱,附於楚則晉怒;附於晉則楚來伐;不備於齊,齊師侵魯。
\\\\
  魯昭公之二十年,而孔子蓋年三十矣。齊景公與晏嬰來適魯,景公問孔子曰:「昔秦穆公國小處辟,其霸何也?」對曰:「秦,國雖小,其志大;處雖辟,行中正。身舉五羖,爵之大夫,起纍紲之中,與語三日,授之以政。以此取之,雖王可也,其霸小矣。」景公說。
\\\\
  孔子年三十五,而季平子與郈昭伯以鬬雞故得罪魯昭公,昭公率師擊平子,平子與孟氏、叔孫氏三家共攻昭公,昭公師敗,奔於齊,齊處昭公乾侯。其後頃之,魯亂。孔子適齊,為高昭子家臣,欲以通乎景公。與齊太師語樂,聞韶音,學之,三月不知肉味,齊人稱之。
\\\\
  景公問政孔子,孔子曰:「君君,臣臣,父父,子子。」景公曰:「善哉!信如君不君,臣不臣,父不父,子不子,雖有粟,吾豈得而食諸!」他日又復問政於孔子,孔子曰:「政在節財。」景公說,將欲以尼谿田封孔子。晏嬰進曰:「夫儒者滑稽而不可軌法;倨傲自順,不可以為下;崇喪遂哀,破產厚葬,不可以為俗;游說乞貸,不可以為國。自大賢之息,周室既衰,禮樂缺有間。今孔子盛容飾,繁登降之禮,趨詳之節,累世不能殫其學,當年不能究其禮。君欲用之以移齊俗,非所以先細民也。」後景公敬見孔子,不問其禮。異日,景公止孔子曰:「奉子以季氏,吾不能。」以季孟之間待之。齊大夫欲害孔子,孔子聞之。景公曰:「吾老矣,弗能用也。」孔子遂行,反乎魯。
\\\\
  孔子年四十二,魯昭公卒於乾侯,定公立。定公立五年,夏,季平子卒,桓子嗣立。季桓子穿井得土缶,中若羊,問仲尼云「得狗」。仲尼曰:「以丘所聞,羊也。丘聞之,木石之怪夔、罔閬,水之怪龍、罔象,土之怪墳羊。」
\\\\
  吳伐越,墮會稽,得骨節專車。吳使使問仲尼:「骨何者最大?」仲尼曰:「禹致群神於會稽山,防風氏後至,禹殺而戮之,其節專車,此為大矣。」吳客曰:「誰為神?」仲尼曰:「山川之神足以綱紀天下,其守為神,社稷為公侯,皆屬於王者。」客曰:「防風何守?」仲尼曰:「汪罔氏之君守封、禺之山,為釐姓。在虞、夏、商為汪罔,於周為長翟,今謂之大人。」客曰:「人長幾何?」仲尼曰:「僬僥氏三尺,短之至也。長者不過十之,數之極也。」於是吳客曰:「善哉聖人!」
\\\\
  桓子嬖臣曰仲梁懷,與陽虎有隙。陽虎欲逐懷,公山不狃止之。其秋,懷益驕,陽虎執懷。桓子怒,陽虎因囚桓子,與盟而醳之。陽虎由此益輕季氏。季氏亦僭於公室,陪臣執國政,是以魯自大夫以下皆僭離於正道。故孔子不仕,退而脩詩書禮樂,弟子彌眾,至自遠方,莫不受業焉。
\\\\
  定公八年,公山不狃不得意於季氏,因陽虎為亂,欲廢三桓之適,更立其庶孽陽虎素所善者,遂執季桓子。桓子詐之,得脫。定公九年,陽虎不勝,奔于齊。是時孔子年五十。
\\\\
  公山不狃以費畔季氏,使人召孔子。孔子循道彌久,溫溫無所試,莫能己用,曰:「蓋周文武起豐鎬而王,今費雖小,儻庶幾乎!」欲往。子路不說,止孔子。孔子曰:「夫召我者豈徒哉?如用我,其為東周乎!」然亦卒不行。
\\\\
  其後定公以孔子為中都宰,一年,四方皆則之。由中都宰為司空,由司空為大司寇。
\\\\
  定公十年春,及齊平。夏,齊大夫黎鉏言於景公曰:「魯用孔丘,其勢危齊。」乃使使告魯為好會,會於夾谷。魯定公且以乘車好往。孔子攝相事,曰:「臣聞有文事者必有武備,有武事者必有文備。古者諸侯出疆,必具官以從。請具左右司馬。」定公曰:「諾。」具左右司馬。會齊侯夾谷,為壇位,土階三等,以會遇之禮相見,揖讓而登。獻酬之禮畢,齊有司趨而進曰:「請奏四方之樂。」景公曰:「諾。」於是旍旄羽袚矛戟劍撥鼓噪而至。孔子趨而進,歷階而登,不盡一等,舉袂而言曰:「吾兩君為好會,夷狄之樂何為於此!請命有司!」有司卻之,不去,則左右視晏子與景公。景公心怍,麾而去之。有頃,齊有司趨而進曰:「請奏宮中之樂。」景公曰:「諾。」優倡侏儒為戲而前。孔子趨而進,歷階而登,不盡一等,曰:「匹夫而營惑諸侯者罪當誅!請命有司!」有司加法焉,手足異處。景公懼而動,知義不若,歸而大恐,告其群臣曰:「魯以君子之道輔其君,而子獨以夷狄之道教寡人,使得罪於魯君,為之奈何?」有司進對曰:「君子有過則謝以質,小人有過則謝以文。君若悼之,則謝以質。」於是齊侯乃歸所侵魯之鄆、汶陽、龜陰之田以謝過。
\\\\
  定公十三年夏,孔子言於定公曰:「臣無藏甲,大夫毋百雉之城。」使仲由為季氏宰,將墮三都。於是叔孫氏先墮郈。季氏將墮費,公山不狃、叔孫輒率費人襲魯。公與三子入于季氏之宮,登武子之臺。費人攻之,弗克,入及公側。孔子命申句須、樂頎下伐之,費人北。國人追之,敗諸姑蔑。二子奔齊,遂墮費。將墮成,公斂處父謂孟孫曰:「墮成,齊人必至于北門。且成,孟氏之保鄣,無成是無孟氏也。我將弗墮。」十二月,公圍成,弗克。
\\\\
  定公十四年,孔子年五十六,由大司寇行攝相事,有喜色。門人曰:「聞君子禍至不懼,福至不喜。」孔子曰:「有是言也。不曰『樂其以貴下人』乎?」於是誅魯大夫亂政者少正卯。與聞國政三月,粥羔豚者弗飾賈;男女行者別於塗;塗不拾遺;四方之客至乎邑者不求有司,皆予之以歸。
\\\\
  齊人聞而懼,曰:「孔子為政必霸,霸則吾地近焉,我之為先并矣。盍致地焉?」黎鉏曰:「請先嘗沮之;沮之而不可則致地,庸遲乎!」於是選齊國中女子好者八十人,皆衣文衣而舞康樂,文馬三十駟,遺魯君。陳女樂文馬於魯城南高門外,季桓子微服往觀再三,將受,乃語魯君為周道游,往觀終日,怠於政事。子路曰:「夫子可以行矣。」孔子曰:「魯今且郊,如致膰乎大夫,則吾猶可以止。」桓子卒受齊女樂,三日不聽政;郊,又不致膰俎於大夫。孔子遂行,宿乎屯。而師己送,曰:「夫子則非罪。」孔子曰:「吾歌可夫?」歌曰:「彼婦之口,可以出走;彼婦之謁,可以死敗。蓋優哉游哉,維以卒歲!」師己反,桓子曰:「孔子亦何言?」師己以實告。桓子喟然嘆曰:「夫子罪我以群婢故也夫!」
\\\\
  孔子遂適衛,主於子路妻兄顏濁鄒家。衛靈公問孔子:「居魯得祿幾何?」對曰:「奉粟六萬。」衛人亦致粟六萬。居頃之,或譖孔子於衛靈公。靈公使公孫余假一出一入。孔子恐獲罪焉,居十月,去衛。
\\\\
  將適陳,過匡,顏刻為僕,以其策指之曰:「昔吾入此,由彼缺也。」匡人聞之,以為魯之陽虎。陽虎嘗暴匡人,匡人於是遂止孔子。孔子狀類陽虎,拘焉五日,顏淵後,子曰:「吾以汝為死矣。」顏淵曰:「子在,回何敢死!」匡人拘孔子益急,弟子懼。孔子曰:「文王既沒,文不在茲乎?天之將喪斯文也,後死者不得與于斯文也。天之未喪斯文也,匡人其如予何!」孔子使從者為甯武子臣於衛,然後得去。
\\\\
  去即過蒲。月餘,反乎衛,主蘧伯玉家。靈公夫人有南子者,使人謂孔子曰:「四方之君子不辱欲與寡君為兄弟者,必見寡小君。寡小君願見。」孔子辭謝,不得已而見之。夫人在絺帷中。孔子入門,北面稽首。夫人自帷中再拜,環珮玉聲璆然。孔子曰:「吾鄉為弗見,見之禮答焉。」子路不說。孔子矢之曰:「予所不者,天厭之!天厭之!」居衛月餘,靈公與夫人同車,宦者雍渠參乘,出,使孔子為次乘,招搖市過之。孔子曰:「吾未見好德如好色者也。」於是醜之,去衛,過曹。是歲,魯定公卒。
\\\\
  孔子去曹適宋,與弟子習禮大樹下。宋司馬桓魋欲殺孔子,拔其樹。孔子去。弟子曰:「可以速矣。」孔子曰:「天生德於予,桓魋其如予何!」
\\\\
  孔子適鄭,與弟子相失,孔子獨立郭東門。鄭人或謂子貢曰:「東門有人,其顙似堯,其項類皋陶,其肩類子產,然自要以下不及禹三寸。纍纍若喪家之狗。」子貢以實告孔子。孔子欣然笑曰:「形狀,末也。而謂似喪家之狗,然哉!然哉!」
\\\\
  孔子遂至陳,主於司城貞子家。歲餘,吳王夫差伐陳,取三邑而去。趙鞅伐朝歌。楚圍蔡,蔡遷于吳。吳敗越王句踐會稽。
\\\\
  有隼集于陳廷而死,楛矢貫之,石砮,矢長尺有咫。陳湣公使使問仲尼。仲尼曰:「隼來遠矣,此肅慎之矢也。昔武王克商,通道九夷百蠻,使各以其方賄來貢,使無忘職業。於是肅慎貢楛矢石砮,長尺有咫。先王欲昭其令德,以肅慎矢分大姬,配虞胡公而封諸陳。分同姓以珍玉,展親;分異姓以遠職,使無忘服。故分陳以肅慎矢。」試求之故府,果得之。
\\\\
  孔子居陳三歲,會晉楚爭彊,更伐陳,及吳侵陳,陳常被寇。孔子曰:「歸與!歸與!吾黨之小子狂簡,進取不忘其初。」於是孔子去陳。
\\\\
  過蒲,會公叔氏以蒲畔,蒲人止孔子。弟子有公良孺者,以私車五乘從孔子。其為人長賢,有勇力,謂曰:「吾昔從夫子遇難於匡,今又遇難於此,命也已。吾與夫子再罹難,寧鬬而死。」鬬甚疾。蒲人懼,謂孔子曰:「苟毋適衛,吾出子。」與之盟,出孔子東門。孔子遂適衛。子貢曰:「盟可負耶?」孔子曰:「要盟也,神不聽。」
\\\\
  衛靈公聞孔子來,喜,郊迎。問曰:「蒲可伐乎?」對曰:「可。」靈公曰:「吾大夫以為不可。今蒲,衛之所以待晉楚也,以衛伐之,無乃不可乎?」孔子曰:「其男子有死之志,婦人有保西河之志。吾所伐者不過四五人。」靈公曰:「善。」然不伐蒲。
\\\\
  靈公老,怠於政,不用孔子。孔子喟然歎曰:「苟有用我者,朞月而已,三年有成。」孔子行。
\\\\
  佛肸為中牟宰。趙簡子攻范、中行,伐中牟。佛肸畔,使人召孔子。孔子欲往。子路曰:「由聞諸夫子,『其身親為不善者,君子不入也』。今佛肸親以中牟畔,子欲往,如之何?」孔子曰:「有是言也。不曰堅乎,磨而不磷;不曰白乎,涅而不淄。我豈匏瓜也哉,焉能系而不食?」
\\\\
  孔子擊磬。有荷蕢而過門者,曰:「有心哉,擊磬乎!硜硜乎,莫己知也夫而已矣!」
\\\\
  孔子學鼓琴師襄子,十日不進。師襄子曰:「可以益矣。」孔子曰:「丘已習其曲矣,未得其數也。」有間,曰:「已習其數,可以益矣。」孔子曰:「丘未得其志也。」有間,曰:「已習其志,可以益矣。」孔子曰:「丘未得其為人也。」有間,[曰]有所穆然深思焉,有所怡然高望而遠志焉。曰:「丘得其為人,黯然而黑,幾然而長,眼如望羊,如王四國,非文王其誰能為此也!」師襄子辟席再拜,曰:「師蓋云文王操也。」
\\\\
  孔子既不得用於衛,將西見趙簡子。至於河而聞竇鳴犢、舜華之死也,臨河而嘆曰:「美哉水,洋洋乎!丘之不濟此,命也夫!」子貢趨而進曰:「敢問何謂也?」孔子曰:「竇鳴犢,舜華,晉國之賢大夫也。趙簡子未得志之時,須此兩人而后從政;及其已得志,殺之乃從政。丘聞之也:刳胎殺夭,則麒麟不至郊;竭澤涸漁,則蛟龍不合陰陽;覆巢毀卵,則鳳皇不翔。何則?君子諱傷其類也。夫鳥獸之於不義也尚知辟之,而況乎丘哉!」乃還息乎陬鄉,作為陬操以哀之。而反乎衛,入主蘧伯玉家。
\\\\
  他日,靈公問兵陳。孔子曰:「俎豆之事則嘗聞之,軍旅之事未之學也。」明日,與孔子語,見蜚鴈,仰視之,色不在孔子。孔子遂行,復如陳。
\\\\
  夏,衛靈公卒,立孫輒,是為衛出公。六月,趙鞅內太子蒯聵于戚。陽虎使太子絻,八人衰絰,偽自衛迎者,哭而入,遂居焉。冬,蔡遷于州來。是歲魯哀公三年,而孔子年六十矣。齊助衛圍戚,以衛太子蒯聵在故也。
\\\\
  夏,魯桓釐廟燔,南宮敬叔救火。孔子在陳,聞之,曰:「災必於桓釐廟乎?」已而果然。
\\\\
  秋,季桓子病,輦而見魯城,喟然嘆曰:「昔此國幾興矣,以吾獲罪於孔子,故不興也。」顧謂其嗣康子曰:「我即死,若必相魯;相魯,必召仲尼。」後數日,桓子卒,康子代立。已葬,欲召仲尼。公之魚曰:「昔吾先君用之不終,終為諸侯笑。今又用之,不能終,是再為諸侯笑。」康子曰:「則誰召而可?」曰:「必召冉求。」於是使使召冉求。冉求將行,孔子曰:「魯人召求,非小用之,將大用之也。」是日,孔子曰:「歸乎歸乎!吾黨之小子狂簡,斐然成章,吾不知所以裁之。」子貢知孔子思歸,送冉求,因誡曰「即用,以孔子為招」云。
\\\\
  冉求既去,明年,孔子自陳遷于蔡。蔡昭公將如吳,吳召之也。前昭公欺其臣遷州來,後將往,大夫懼復遷,公孫翩射殺昭公。楚侵蔡。秋,齊景公卒。
\\\\
  明年,孔子自蔡如葉。葉公問政,孔子曰:「政在來遠附邇。」他日,葉公問孔子於子路,子路不對。孔子聞之,曰:「由,爾何不對曰『其為人也,學道不倦,誨人不厭,發憤忘食,樂以忘憂,不知老之將至』云爾。」
\\\\
  去葉,反于蔡。長沮、桀溺耦而耕,孔子以為隱者,使子路問津焉。長沮曰:「彼執輿者為誰?」子路曰:「為孔丘。」曰:「是魯孔丘與?」曰:「然。」曰:「是知津矣。」桀溺謂子路曰:「子為誰?」曰:「為仲由。」曰:「子,孔丘之徒與?」曰:「然。」桀溺曰:「悠悠者天下皆是也,而誰以易之?且與其從辟人之士,豈若從辟世之士哉!」耰而不輟。子路以告孔子,孔子憮然曰:「鳥獸不可與同群。天下有道,丘不與易也。」
\\\\
  他日,子路行,遇荷蓧丈人,曰:「子見夫子乎?」丈人曰:「四體不勤,五穀不分,孰為夫子!」植其杖而芸。子路以告,孔子曰:「隱者也。」復往,則亡。
\\\\
  孔子遷于蔡三歲,吳伐陳。楚救陳,軍于城父。聞孔子在陳蔡之間,楚使人聘孔子。孔子將往拜禮,陳蔡大夫謀曰:「孔子賢者,所刺譏皆中諸侯之疾。今者久留陳蔡之間,諸大夫所設行皆非仲尼之意。今楚,大國也,來聘孔子。孔子用於楚,則陳蔡用事大夫危矣。」於是乃相與發徒役圍孔子於野。不得行,絕糧。從者病,莫能興。孔子講誦弦歌不衰。子路慍見曰:「君子亦有窮乎?」孔子曰:「君子固窮,小人窮斯濫矣。」
\\\\
  子貢色作。孔子曰:「賜,爾以予為多學而識之者與?」曰:「然。非與?」孔子曰:「非也。予一以貫之。」
\\\\
  孔子知弟子有慍心,乃召子路而問曰:「《詩》云『匪兕匪虎,率彼曠野』。吾道非邪?吾何為於此?」子路曰:「意者吾未仁邪?人之不我信也。意者吾未知邪?人之不我行也。」孔子曰:「有是乎!由,譬使仁者而必信,安有伯夷、叔齊?使知者而必行,安有王子比干?」
\\\\
  子路出,子貢入見。孔子曰:「賜,《詩》云『匪兕匪虎,率彼曠野』。吾道非邪?吾何為於此?」子貢曰:「夫子之道至大也,故天下莫能容夫子。夫子蓋少貶焉?」孔子曰:「賜,良農能稼而不能為穡,良工能巧而不能為順。君子能脩其道,綱而紀之,統而理之,而不能為容。今爾不脩爾道而求為容。賜,而志不遠矣!」
\\\\
  子貢出,顏回入見。孔子曰:「回,《詩》云『匪兕匪虎,率彼曠野』。吾道非邪?吾何為於此?」顏回曰:「夫子之道至大,故天下莫能容。雖然,夫子推而行之,不容何病,不容然後見君子!夫道之不修也,是吾醜也。夫道既已大修而不用,是有國者之醜也。不容何病,不容然後見君子!」孔子欣然而笑曰:「有是哉顏氏之子!使爾多財,吾為爾宰。」
\\\\
  於是使子貢至楚。楚昭王興師迎孔子,然後得免。
\\\\
  昭王將以書社地七百里封孔子。楚令尹子西曰:「王之使使諸侯有如子貢者乎?」曰:「無有。」「王之輔相有如顏回者乎?」曰:「無有。」「王之將率有如子路者乎?」曰:「無有。」「王之官尹有如宰予者乎?」曰:「無有。」「且楚之祖封於周,號為子男五十里。今孔丘述三五之法,明周召之業,王若用之,則楚安得世世堂堂方數千里乎?夫文王在豐,武王在鎬,百里之君卒王天下。今孔丘得據土壤,賢弟子為佐,非楚之福也。」昭王乃止。其秋,楚昭王卒于城父。
\\\\
  楚狂接輿歌而過孔子曰:「鳳兮!鳳兮!何德之衰?往者不可諫兮,來者猶可追也!已而,已而!今之從政者殆而!」孔子下,欲與之言。趨而去,弗得與之言。
\\\\
  於是孔子自楚反乎衛。是歲也,孔子年六十三,而魯哀公六年也。
\\\\
  其明年,吳與魯會繒,徵百牢。太宰嚭召季康子。康子使子貢往,然後得已。
\\\\
  孔子曰:「魯衛之政,兄弟也。」是時,衛君輒父不得立,在外,諸侯數以為讓。而孔子弟子多仕於衛,衛君欲得孔子為政。子路曰:「衛君待子而為政,子將奚先?」孔子曰:「必也正名乎!」子路曰:「有是哉,子之迂也!何其正也?」孔子曰:「野哉由也!夫名不正則言不順,言不順則事不成,事不成則禮樂不興,禮樂不興則刑罰不中,刑罰不中則民無所錯手足矣。夫君子為之必可名,言之必可行。君子於其言,無所苟而已矣。」
\\\\
  其明年,冉有為季氏將師,與齊戰於郎,克之。季康子曰:「子之於軍旅,學之乎?性之乎?」冉有曰:「學之於孔子。」季康子曰:「孔子何如人哉?」對曰:「用之有名;播之百姓,質諸鬼神而無憾。求之至於此道,雖累千社,夫子不利也。」康子曰:「我欲召之,可乎?」對曰:「欲召之,則毋以小人固之,則可矣。」而衛孔文子將攻太叔,問策於仲尼。仲尼辭不知,退而命載而行,曰:「鳥能擇木,木豈能擇鳥乎!」文子固止。會季康子逐公華、公賓、公林,以幣迎孔子,孔子歸魯。
\\\\
  孔子之去魯凡十四歲而反乎魯。
\\\\
  魯哀公問政,對曰:「政在選臣。」季康子問政,曰:「舉直錯諸枉,則枉者直。」康子患盜,孔子曰:「苟子之不欲,雖賞之不竊。」然魯終不能用孔子,孔子亦不求仕。
\\\\
  孔子之時,周室微而禮樂廢,詩書缺。追跡三代之禮,序書傳,上紀唐虞之際,下至秦繆,編次其事。曰:「夏禮吾能言之,杞不足徵也。殷禮吾能言之,宋不足徵也。足,則吾能徵之矣。」觀殷夏所損益,曰:「後雖百世可知也,以一文一質。周監二代,郁郁乎文哉。吾從周。」故《書傳》、《禮記》自孔氏。
\\\\
  孔子語魯大師:「樂其可知也。始作翕如,縱之純如,皦如,繹如也,以成。」「吾自衛反魯,然後樂正,雅頌各得其所。」
\\\\
  古者詩三千餘篇,及至孔子,去其重,取可施於禮義,上采契后稷,中述殷周之盛,至幽厲之缺,始於衽席,故曰「關雎之亂以為風始,鹿鳴為小雅始,文王為大雅始,清廟為頌始」。三百五篇孔子皆弦歌之,以求合韶武雅頌之音。禮樂自此可得而述,以備王道,成六藝。
\\\\
  孔子晚而喜易,序彖、繫、象、說卦、文言。讀易,韋編三絕。曰:「假我數年,若是,我於易則彬彬矣。」
\\\\
  孔子以詩書禮樂教,弟子蓋三千焉,身通六藝者七十有二人。如顏濁鄒之徒,頗受業者甚眾。
\\\\
  孔子以四教:文,行,忠,信。絕四:毋意,毋必,毋固,毋我。所慎:齊,戰,疾。子罕言利與命與仁。不憤不啟,舉一隅不以三隅反,則弗復也。
\\\\
  其於鄉黨,恂恂似不能言者。其於宗廟朝廷,辯辯言,唯謹爾。朝,與上大夫言,誾誾如也;與下大夫言,侃侃如也。
\\\\
  入公門,鞠躬如也;趨進,翼如也。君召使儐,色勃如也。君命召,不俟駕行矣。
\\\\
  魚餒,肉敗,割不正,不食。席不正,不坐。食於有喪者之側,未嘗飽也。
\\\\
  是日哭,則不歌。見齊衰、瞽者,雖童子必變。
\\\\
  「三人行,必得我師。」「德之不脩,學之不講,聞義不能徙,不善不能改,是吾憂也。」使人歌,善,則使復之,然后和之。
\\\\
  子不語:怪,力,亂,神。
\\\\
  子貢曰:「夫子之文章,可得聞也。夫子言天道與性命,弗可得聞也已。」顏淵喟然嘆曰:「仰之彌高,鑽之彌堅。瞻之在前,忽焉在後。夫子循循然善誘人,博我以文,約我以禮,欲罷不能。既竭我才,如有所立,卓爾。雖欲從之,蔑由也已。」達巷黨人[童子]曰:「大哉孔子,博學而無所成名。」子聞之曰:「我何執?執御乎?執射乎?我執御矣。」牢曰:「子云『不試,故藝』。」
\\\\
  魯哀公十四年春,狩大野。叔孫氏車子鉏商獲獸,以為不祥。仲尼視之,曰:「麟也。」取之。曰:「河不出圖,雒不出書,吾已矣夫!」顏淵死,孔子曰:「天喪予!」及西狩見麟,曰:「吾道窮矣!」喟然嘆曰:「莫知我夫!」子貢曰:「何為莫知子?」子曰:「不怨天,不尤人,下學而上達,知我者其天乎!」
\\\\
  「不降其志,不辱其身,伯夷、叔齊乎!」謂「柳下惠、少連降志辱身矣」。謂「虞仲、夷逸隱居放言,行中清,廢中權」。「我則異於是,無可無不可。」
\\\\
  子曰:「弗乎弗乎,君子病沒世而名不稱焉。吾道不行矣,吾何以自見於後世哉?」乃因史記作春秋,上至隱公,下訖哀公十四年,十二公。據魯,親周,故殷,運之三代。約其文辭而指博。故吳楚之君自稱王,而春秋貶之曰「子」;踐土之會實召周天子,而春秋諱之曰「天王狩於河陽」:推此類以繩當世。貶損之義,後有王者舉而開之。春秋之義行,則天下亂臣賊子懼焉。
\\\\
  孔子在位聽訟,文辭有可與人共者,弗獨有也。至於為春秋,筆則筆,削則削,子夏之徒不能贊一辭。弟子受春秋,孔子曰:「後世知丘者以春秋,而罪丘者亦以春秋。」
\\\\
  明歲,子路死於衛。孔子病,子貢請見。孔子方負杖逍遙於門,曰:「賜,汝來何其晚也?」孔子因歎,歌曰:「太山壞乎!梁柱摧乎!哲人萎乎!」因以涕下。謂子貢曰:「天下無道久矣,莫能宗予。夏人殯於東階,周人於西階,殷人兩柱閒。昨暮予夢坐奠兩柱之閒,予始殷人也。」後七日卒。
\\\\
  孔子年七十三,以魯哀公十六年四月己丑卒。
\\\\
  哀公誄之曰:「旻天不弔,不愸遺一老,俾屏余一人以在位,煢煢余在疚。嗚呼哀哉!尼父,毋自律!」子貢曰:「君其不沒於魯乎!夫子之言曰:『禮失則昬,名失則愆。失志為昬,失所為愆。』生不能用,死而誄之,非禮也。稱『余一人』,非名也。」
\\\\
  孔子葬魯城北泗上,弟子皆服三年。三年心喪畢,相訣而去,則哭,各復盡哀;或復留。唯子貢廬於冢上,凡六年,然後去。弟子及魯人往從冢而家者百有餘室,因命曰孔里。魯世世相傳以歲時奉祠孔子冢,而諸儒亦講禮鄉飲大射於孔子冢。孔子冢大一頃。故所居堂弟子內,後世因廟藏孔子衣冠琴車書,至于漢二百餘年不絕。高皇帝過魯,以太牢祠焉。諸侯卿相至,常先謁然後從政。
\\\\
  孔子生鯉,字伯魚。伯魚年五十,先孔子死。
\\\\
  伯魚生伋,字子思,年六十二。嘗困於宋。子思作中庸。
\\\\
  子思生白,字子上,年四十七。子上生求,字子家,年四十五。子家生箕,字子京,年四十六。子京生穿,字子高,年五十一。子高生子慎,年五十七,嘗為魏相。
\\\\
  子慎生鮒,年五十七,為陳王涉博士,死於陳下。
\\\\
  鮒弟子襄,年五十七。嘗為孝惠皇帝博士,遷為長沙太守。長九尺六寸。
\\\\
  子襄生忠,年五十七。忠生武,武生延年及安國。安國為今皇帝博士,至臨淮太守,蚤卒。安國生卬,卬生驩。
\\\\
  太史公曰:《詩》有之:「高山仰止,景行行止。」雖不能至,然心鄉往之。余讀孔氏書,想見其為人。適魯,觀仲尼廟堂車服禮器,諸生以時習禮其家,余祗回留之不能去云。天下君王至於賢人眾矣,當時則榮,沒則已焉。孔子布衣,傳十餘世,學者宗之。自天子王侯,中國言六藝者折中於夫子,可謂至聖矣!


\section{留侯世家}
  留侯張良者,其先韓人也。大父開地,相韓昭侯、宣惠王、襄哀王。父平,相釐王、悼惠王。悼惠王二十三年,平卒。卒二十歲,秦滅韓。良年少,未宦事韓。韓破,良家僮三百人,弟死不葬,悉以家財求客刺秦王,為韓報仇,以大父、父五世相韓故。
\\\\
  良嘗學禮淮陽。東見倉海君。得力士,為鐵椎重百二十斤。秦皇帝東游,良與客狙擊秦皇帝博浪沙中,誤中副車。秦皇帝大怒,大索天下,求賊甚急,為張良故也。良乃更名姓,亡匿下邳。
\\\\
  良嘗閒從容步游下邳圯上,有一老父,衣褐,至良所,直墮其履圯下,顧謂良曰:「孺子,下取履!」良鄂然,欲毆之。為其老,彊忍,下取履。父曰:「履我!」良業為取履,因長跪履之。父以足受,笑而去。良殊大驚,隨目之。父去里所,復還,曰:「孺子可教矣。後五日平明,與我會此。」良因怪之,跪曰:「諾。」五日平明,良往。父已先在,怒曰:「與老人期,後,何也?」去,曰:「後五日早會。」五日雞鳴,良往。父又先在,復怒曰:「後,何也?」去,曰:「後五日復早來。」五日,良夜未半往。有頃,父亦來,喜曰:「當如是。」出一編書,曰:「讀此則為王者師矣。後十年興。十三年孺子見我濟北,穀城山下黃石即我矣。」遂去,無他言,不復見。旦日視其書,乃太公兵法也。良因異之,常習誦讀之。
\\\\
  居下邳,為任俠。項伯常殺人,從良匿。
\\\\
  後十年,陳涉等起兵,良亦聚少年百餘人。景駒自立為楚假王,在留。良欲往從之,道還沛公。沛公將數千人,略地下邳西,遂屬焉。沛公拜良為廄將。良數以太公兵法說沛公,沛公善之,常用其策。良為他人者,皆不省。良曰:「沛公殆天授。」故遂從之,不去見景駒。
\\\\
  及沛公之薛,見項梁。項梁立楚懷王。良乃說項梁曰:「君已立楚後,而韓諸公子橫陽君成賢,可立為王,益樹黨。」項梁使良求韓成,立以為韓王。以良為韓申徒,與韓王將千餘人西略韓地,得數城,秦輒復取之,往來為游兵潁川。
\\\\
  沛公之從雒陽南出轘轅,良引兵從沛公,下韓十餘城,擊破楊熊軍。沛公乃令韓王成留守陽翟,與良俱南,攻下宛,西入武關。沛公欲以兵二萬人擊秦嶢下軍,良說曰:「秦兵尚彊,未可輕。臣聞其將屠者子,賈豎易動以利。願沛公且留壁,使人先行,為五萬人具食,益為張旗幟諸山上,為疑兵,令酈食其持重寶啗秦將。」秦將果畔,欲連和俱西襲咸陽,沛公欲聽之。良曰:「此獨其將欲叛耳,恐士卒不從。不從必危,不如因其解擊之。」沛公乃引兵擊秦軍,大破之。(遂)[逐]北至藍田,再戰,秦兵竟敗。遂至咸陽,秦王子嬰降沛公。
\\\\
  沛公入秦宮,宮室帷帳狗馬重寶婦女以千數,意欲留居之。樊噲諫沛公出舍,沛公不聽。良曰:「夫秦為無道,故沛公得至此。夫為天下除殘賊,宜縞素為資。今始入秦,即安其樂,此所謂『助桀為虐』。且『忠言逆耳利於行,毒藥苦口利於病』,願沛公聽樊噲言。」沛公乃還軍霸上。
\\\\
  項羽至鴻門下,欲擊沛公,項伯乃夜馳入沛公軍,私見張良,欲與俱去。良曰:「臣為韓王送沛公,今事有急,亡去不義。」乃具以語沛公。沛公大驚,曰:「為將奈何?」良曰:「沛公誠欲倍項羽邪?」沛公曰:「鯫生教我距關無內諸侯,秦地可盡王,故聽之。」良曰:「沛公自度能卻項羽乎?」沛公默然良久,曰:「固不能也。今為奈何?」良乃固要項伯。項伯見沛公。沛公與飲為壽,結賓婚。令項伯具言沛公不敢倍項羽,所以距關者,備他盜也。及見項羽後解,語在項羽事中。
\\\\
  漢元年正月,沛公為漢王,王巴蜀。漢王賜良金百鎰,珠二斗,良具以獻項伯。漢王亦因令良厚遺項伯,使請漢中地。項王乃許之,遂得漢中地。漢王之國,良送至襃中,遣良歸韓。良因說漢王曰:「王何不燒絕所過棧道,示天下無還心,以固項王意。」乃使良還。行,燒絕棧道。
\\\\
  良至韓,韓王成以良從漢王故,項王不遣成之國,從與俱東。良說項王曰:「漢王燒絕棧道,無還心矣。」乃以齊王田榮反,書告項王。項王以此無西憂漢心,而發兵北擊齊。
\\\\
  項王竟不肯遣韓王,乃以為侯,又殺之彭城。良亡,間行歸漢王,漢王亦已還定三秦矣。復以良為成信侯,從東擊楚。至彭城,漢敗而還。至下邑,漢王下馬踞鞍而問曰:「吾欲捐關以東等棄之,誰可與共功者?」良進曰:「九江王黥布,楚梟將,與項王有郄;彭越與齊王田榮反梁地:此兩人可急使。而漢王之將獨韓信可屬大事,當一面。即欲捐之,捐之此三人,則楚可破也。」漢王乃遣隨何說九江王布,而使人連彭越。及魏王豹反,使韓信將兵擊之,因舉燕、代、齊、趙。然卒破楚者,此三人力也。
\\\\
  張良多病,未嘗特將也,常為畫策,時時從漢王。
\\\\
  漢三年,項羽急圍漢王滎陽,漢王恐憂,與酈食其謀橈楚權。食其曰:「昔湯伐桀,封其後於杞。武王伐紂,封其後於宋。今秦失德棄義,侵伐諸侯社稷,滅六國之後,使無立錐之地。陛下誠能復立六國後世,畢已受印,此其君臣百姓必皆戴陛下之德,莫不鄉風慕義,願為臣妾。德義已行,陛下南鄉稱霸,楚必斂衽而朝。」漢王曰:「善。趣刻印,先生因行佩之矣。」
\\\\
  食其未行,張良從外來謁。漢王方食,曰:「子房前!客有為我計橈楚權者。」其以酈生語告,曰:「於子房何如?」良曰:「誰為陛下畫此計者?陛下事去矣。」漢王曰:「何哉?」張良對曰:「臣請藉前箸為大王籌之。」曰:「昔者湯伐桀而封其後於杞者,度能制桀之死命也。今陛下能制項籍之死命乎?」曰:「未能也。」「其不可一也。武王伐紂封其後於宋者,度能得紂之頭也。今陛下能得項籍之頭乎?」曰:「未能也。」「其不可二也。武王入殷,表商容之閭,釋箕子之拘,封比干之墓。今陛下能封聖人之墓,表賢者之閭,式智者之門乎?」曰:「未能也。」「其不可三也。發鉅橋之粟,散鹿臺之錢,以賜貧窮。今陛下能散府庫以賜貧窮乎?」曰:「未能也。」「其不可四矣。殷事已畢,偃革為軒,倒置干戈,覆以虎皮,以示天下不復用兵。今陛下能偃武行文,不復用兵乎?」曰:「未能也。」「其不可五矣。休馬華山之陽,示以無所為。今陛下能休馬無所用乎?」曰:「未能也。」「其不可六矣。放牛桃林之陰,以示不復輸積。今陛下能放牛不復輸積乎?」曰:「未能也。」「其不可七矣。且天下游士離其親戚,棄墳墓,去故舊,從陛下游者,徒欲日夜望咫尺之地。今復六國,立韓、魏、燕、趙、齊、楚之後,天下游士各歸事其主,從其親戚,反其故舊墳墓,陛下與誰取天下乎?其不可八矣。且夫楚唯無彊,六國立者復橈而從之,陛下焉得而臣之?誠用客之謀,陛下事去矣。」漢王輟食吐哺,罵曰:「豎儒,幾敗而公事!」令趣銷印。
\\\\
  漢四年,韓信破齊而欲自立為齊王,漢王怒。張良說漢王,漢王使良授齊王信印,語在淮陰事中。
\\\\
  其秋,漢王追楚至陽夏南,戰不利而壁固陵,諸侯期不至。良說漢王,漢王用其計,諸侯皆至。語在項籍事中。
\\\\
  漢六年正月,封功臣。良未嘗有戰鬬功,高帝曰:「運籌策帷帳中,決勝千里外,子房功也。自擇齊三萬戶。」良曰:「始臣起下邳,與上會留,此天以臣授陛下。陛下用臣計,幸而時中,臣願封留足矣,不敢當三萬戶。」乃封張良為留侯,與蕭何等俱封。
\\\\
  [六年]上已封大功臣二十餘人,其餘日夜爭功不決,未得行封。上在雒陽南宮,從複道望見諸將往往相與坐沙中語。上曰:「此何語?」留侯曰:「陛下不知乎?此謀反耳。」上曰:「天下屬安定,何故反乎?」留侯曰:「陛下起布衣,以此屬取天下,今陛下為天子,而所封皆蕭、曹故人所親愛,而所誅者皆生平所仇怨。今軍吏計功,以天下不足遍封,此屬畏陛下不能盡封,恐又見疑平生過失及誅,故即相聚謀反耳。」上乃憂曰:「為之奈何?」留侯曰:「上平生所憎,群臣所共知,誰最甚者?」上曰:「雍齒與我故,數嘗窘辱我。我欲殺之,為其功多,故不忍。」留侯曰:「今急先封雍齒以示群臣,群臣見雍齒封,則人人自堅矣。」於是上乃置酒,封雍齒為什方侯,而急趣丞相、御史定功行封。群臣罷酒,皆喜曰:「雍齒尚為侯,我屬無患矣。」
\\\\
  劉敬說高帝曰:「都關中。」上疑之。左右大臣皆山東人,多勸上都雒陽:「雒陽東有成皋,西有殽黽,倍河,向伊雒,其固亦足恃。」留侯曰:「雒陽雖有此固,其中小,不過數百里,田地薄,四面受敵,此非用武之國也。夫關中左殽函,右隴蜀,沃野千里,南有巴蜀之饒,北有胡苑之利,阻三面而守,獨以一面東制諸侯。諸侯安定,河渭漕輓天下,西給京師;諸侯有變,順流而下,足以委輸。此所謂金城千里,天府之國也,劉敬說是也。」於是高帝即日駕,西都關中。
\\\\
  留侯從入關。留侯性多病,即道引不食穀,杜門不出歲餘。
\\\\
  上欲廢太子,立戚夫人子趙王如意。大臣多諫爭,未能得堅決者也。呂后恐,不知所為。人或謂呂后曰:「留侯善畫計筴,上信用之。」呂后乃使建成侯呂澤劫留侯,曰:「君常為上謀臣,今上欲易太子,君安得高枕而臥乎?」留侯曰:「始上數在困急之中,幸用臣筴。今天下安定,以愛欲易太子,骨肉之間,雖臣等百餘人何益。」呂澤彊要曰:「為我畫計。」留侯曰:「此難以口舌爭也。顧上有不能致者,天下有四人。四人者年老矣,皆以為上慢侮人,故逃匿山中,義不為漢臣。然上高此四人。今公誠能無愛金玉璧帛,令太子為書,卑辭安車,因使辯士固請,宜來。來,以為客,時時從入朝,令上見之,則必異而問之。問之,上知此四人賢,則一助也。」於是呂后令呂澤使人奉太子書,卑辭厚禮,迎此四人。四人至,客建成侯所。
\\\\
  漢十一年,黥布反,上病,欲使太子將,往擊之。四人相謂曰:「凡來者,將以存太子。太子將兵,事危矣。」乃說建成侯曰:「太子將兵,有功則位不益太子;無功還,則從此受禍矣。且太子所與俱諸將,皆嘗與上定天下梟將也,今使太子將之,此無異使羊將狼也,皆不肯為盡力,其無功必矣。臣聞『母愛者子抱』,今戚夫人日夜待御,趙王如意常抱居前,上曰『終不使不肖子居愛子之上』,明乎其代太子位必矣。君何不急請呂后承間為上泣言:『黥布,天下猛將也,善用兵,今諸將皆陛下故等夷,乃令太子將此屬,無異使羊將狼,莫肯為用,且使布聞之,則鼓行而西耳。上雖病,彊載輜車,臥而護之,諸將不敢不盡力。上雖苦,為妻子自彊。』」於是呂澤立夜見呂后,呂后承間為上泣涕而言,如四人意。上曰:「吾惟豎子固不足遣,而公自行耳。」於是上自將兵而東,群臣居守,皆送至灞上。留侯病,自彊起,至曲郵,見上曰:「臣宜從,病甚。楚人剽疾,願上無與楚人爭鋒。」因說上曰:「令太子為將軍,監關中兵。」上曰:「子房雖病,彊臥而傅太子。」是時叔孫通為太傅,留侯行少傅事。
\\\\
  漢十二年,上從擊破布軍歸,疾益甚,愈欲易太子。留侯諫,不聽,因疾不視事。叔孫太傅稱說引古今,以死爭太子。上詳許之,猶欲易之。及燕,置酒,太子侍。四人從太子,年皆八十有餘,鬚眉皓白,衣冠甚偉。上怪之,問曰:「彼何為者?」四人前對,各言名姓,曰東園公,角里先生,綺里季,夏黃公。上乃大驚,曰:「吾求公數歲,公辟逃我,今公何自從吾兒游乎?」四人皆曰:「陛下輕士善罵,臣等義不受辱,故恐而亡匿。竊聞太子為人仁孝,恭敬愛士,天下莫不延頸欲為太子死者,故臣等來耳。」上曰:「煩公幸卒調護太子。」
\\\\
  四人為壽已畢,趨去。上目送之,召戚夫人指示四人者曰:「我欲易之,彼四人輔之,羽翼已成,難動矣。呂后真而主矣。」戚夫人泣,上曰:「為我楚舞,吾為若楚歌。」歌曰:「鴻鴈高飛,一舉千里。羽翮已就,橫絕四海。橫絕四海,當可奈何!雖有矰繳,尚安所施!」歌數闋,戚夫人噓唏流涕,上起去,罷酒。竟不易太子者,留侯本招此四人之力也。
\\\\
  留侯從上擊代,出奇計馬邑下,及立蕭何相國,所與上從容言天下事甚眾,非天下所以存亡,故不著。留侯乃稱曰:「家世相韓,及韓滅,不愛萬金之資,為韓報讐彊秦,天下振動。今以三寸舌為帝者師,封萬戶,位列侯,此布衣之極,於良足矣。願棄人間事,欲從赤松子游耳。」乃學辟穀,道引輕身。會高帝崩,呂后德留侯,乃彊食之,曰:「人生一世間,如白駒過隙,何至自苦如此乎!」留侯不得已,彊聽而食。
\\\\
  後八年卒,謚為文成侯。子不疑代侯。
\\\\
  子房始所見下邳圯上老父與太公書者,後十三年從高帝過濟北,果見穀城山下黃石,取而葆祠之。留侯死,并葬黃石(冢)。每上冢伏臘,祠黃石。
\\\\
  留侯不疑,孝文帝五年坐不敬,國除。
\\\\
  太史公曰:學者多言無鬼神,然言有物。至如留侯所見老父予書,亦可怪矣。高祖離困者數矣,而留侯常有功力焉,豈可謂非天乎?上曰:「夫運籌筴帷帳之中,決勝千里外,吾不如子房。」余以為其人計魁梧奇偉,至見其圖,狀貌如婦人好女。蓋孔子曰:「以貌取人,失之子羽。」留侯亦云。


\section{陳丞相世家}

  陳丞相平者,陽武戶牖鄉人也。少時家貧,好讀書,有田三十畝,獨與兄伯居。伯常耕田,縱平使游學。平為人長[大]美色。人或謂陳平曰:「貧何食而肥若是?」其嫂嫉平之不視家生產,曰:「亦食糠覈耳。有叔如此,不如無有。」伯聞之,逐其婦而棄之。
\\\\
  及平長,可娶妻,富人莫肯與者,貧者平亦恥之。久之,戶牖富人有張負,張負女孫五嫁而夫輒死,人莫敢娶。平欲得之。邑中有喪,平貧,侍喪,以先往後罷為助。張負既見之喪所,獨視偉平,平亦以故後去。負隨平至其家,家乃負郭窮巷,以獘席為門,然門外多有長者車轍。張負歸,謂其子仲曰:「吾欲以女孫予陳平。」張仲曰:「平貧不事事,一縣中盡笑其所為,獨柰何予女乎?」負曰:「人固有好美如陳平而長貧賤者乎?」卒與女。為平貧,乃假貸幣以聘,予酒肉之資以內婦。負誡其孫曰:「毋以貧故,事人不謹。事兄伯如事父,事嫂如母。」平既娶張氏女,齎用益饒,游道日廣。
\\\\
  裏中社,平為宰,分肉食甚均。父老曰:「善,陳孺子之為宰!」平曰:「嗟乎,使平得宰天下,亦如是肉矣!」
\\\\
  陳涉起而王陳,使周市略定魏地,立魏咎為魏王,與秦軍相攻於臨濟。陳平固已前謝其兄伯,從少年往事魏王咎於臨濟。魏王以為太仆。說魏王不聽,人或讒之,陳平亡去。
\\\\
  久之,項羽略地至河上,陳平往歸之,從入破秦,賜平爵卿。項羽之東王彭城也,漢王還定三秦而東,殷王反楚。項羽乃以平為信武君,將魏王咎客在楚者以往,擊降殷王而還。項王使項悍拜平為都尉,賜金二十溢。居無何,漢王攻下殷(王)。項王怒,將誅定殷者將吏。陳平懼誅,乃封其金與印,使使歸項王,而平身閒行杖劍亡。渡河,船人見其美丈夫獨行,疑其亡將,要中當有金玉寶器,目之,欲殺平。平恐,乃解衣躶而佐刺船,船人知其無有,乃止。
\\\\
  平遂至修武降漢,因魏無知求見漢王,漢王召入。是時萬石君奮為漢王中涓,受平謁,入見平。平等七人俱進,賜食。王曰:「罷,就舍矣。」平曰:「臣為事來,所言不可以過今日。」於是漢王與語而說之,問曰:「子之居楚何官?」曰:「為都尉。」是日乃拜平為都尉,使為參乘,典護軍。諸將盡讙,曰:「大王一日得楚之亡卒,未知其高下,而即與同載,反使監護軍長者!」漢王聞之,愈益幸平。遂與東伐項王。至彭城,為楚所敗。引而還,收散兵至滎陽,以平為亞將,屬於韓王信,軍廣武。
\\\\
  絳侯、灌嬰等咸讒陳平曰:「平雖美丈夫,如冠玉耳,其中未必有也。臣聞平居家時,盜其嫂;事魏不容,亡歸楚;歸楚不中,又亡歸漢。今日大王尊官之,令護軍。臣聞平受諸將金,金多者得善處,金少者得惡處。平,反覆亂臣也,願王察之。」漢王疑之,召讓魏無知。無知曰:「臣所言者,能也;陛下所問者,行也。今有尾生、孝己之行而無益處於勝負之數,陛下何暇用之乎?楚漢相距,臣進奇謀之士,顧其計誠足以利國家不耳。且盜嫂受金又何足疑乎?」漢王召讓平曰:「先生事魏不中,遂事楚而去,今又從吾游,信者固多心乎?」平曰:「臣事魏王,魏王不能用臣說,故去事項王。項王不能信人,其所任愛,非諸項即妻之昆弟,雖有奇士不能用,平乃去楚。聞漢王之能用人,故歸大王。臣躶身來,不受金無以為資。誠臣計畫有可采者,(顧)[願]大王用之;使無可用者,金具在,請封輸官,得請骸骨。」漢王乃謝,厚賜,拜為護軍中尉,盡護諸將。諸將乃不敢復言。
\\\\
  其後,楚急攻,絕漢甬道,圍漢王於滎陽城。久之,漢王患之,請割滎陽以西以和。項王不聽。漢王謂陳平曰:「天下紛紛,何時定乎?」陳平曰:「項王為人,恭敬愛人,士之廉節好禮者多歸之。至於行功爵邑,重之,士亦以此不附。今大王慢而少禮,士廉節者不來;然大王能饒人以爵邑,士之頑鈍嗜利無恥者亦多歸漢。誠各去其兩短,襲其兩長,天下指麾則定矣。然大王恣侮人,不能得廉節之士。顧楚有可亂者,彼項王骨鯁之臣亞父、鐘離眛、龍且、周殷之屬,不過數人耳。大王誠能出捐數萬斤金,行反閒,閒其君臣,以疑其心,項王為人意忌信讒,必內相誅。漢因舉兵而攻之,破楚必矣。」漢王以為然,乃出黃金四萬斤,與陳平,恣所為,不問其出入。
\\\\
  陳平既多以金縱反閒於楚軍,宣言諸將鐘離眛等為項王將,功多矣,然而終不得裂地而王,欲與漢為一,以滅項氏而分王其地。項羽果意不信鐘離眛等。項王既疑之,使使至漢。漢王為太牢具,舉進。見楚使,即詳驚曰:「吾以為亞父使,乃項王使!」復持去,更以惡草具進楚使。楚使歸,具以報項王。項王果大疑亞父。亞父欲急攻下滎陽城,項王不信,不肯聽。亞父聞項王疑之,乃怒曰:「天下事大定矣,君王自為之!願請骸骨歸!」歸未至彭城,疽發背而死。陳平乃夜出女子二千人滎陽城東門,楚因擊之,陳平乃與漢王從城西門夜出去。遂入關,收散兵復東。
\\\\
  其明年,淮陰侯破齊,自立為齊王,使使言之漢王。漢王大怒而罵,陳平躡漢王。漢王亦悟,乃厚遇齊使,使張子房卒立信為齊王。封平以戶牖鄉。用其奇計策,卒滅楚。常以護軍中尉從定燕王臧荼。
\\\\
  漢六年,人有上書告楚王韓信反。高帝問諸將,諸將曰:「亟發兵阬豎子耳。」高帝默然。問陳平,平固辭謝,曰:「諸將云何?」上具告之。陳平曰:「人之上書言信反,有知之者乎?」曰:「未有。」曰:「信知之乎?」曰:「不知。」陳平曰:「陛下精兵孰與楚?」上曰:「不能過。」平曰:「陛下將用兵有能過韓信者乎?」上曰:「莫及也。」平曰:「今兵不如楚精,而將不能及,而舉兵攻之,是趣之戰也,竊為陛下危之。」上曰:「為之柰何?」平曰:「古者天子巡狩,會諸侯。南方有雲夢,陛下弟出偽游雲夢,會諸侯於陳。陳,楚之西界,信聞天子以好出游,其勢必無事而郊迎謁。謁,而陛下因禽之,此特一力士之事耳。」高帝以為然,乃發使告諸侯會陳,「吾將南游雲夢」。上因隨以行。行未至陳,楚王信果郊迎道中。高帝豫具武士,見信至,即執縛之,載後車。信呼曰:「天下已定,我固當烹!」高帝顧謂信曰:「若毋聲!而反,明矣!」武士反接之。遂會諸侯于陳,盡定楚地。還至雒陽,赦信以為淮陰侯,而與功臣剖符定封。
\\\\
  於是與平剖符,世世勿絕,為戶牖侯。平辭曰:「此非臣之功也。」上曰:「吾用先生謀計,戰勝剋敵,非功而何?」平曰:「非魏無知臣安得進?」上曰;「若子可謂不背本矣。」乃復賞魏無知。其明年,以護軍中尉從攻反者韓王信於代。卒至平城,為匈奴所圍,七日不得食。高帝用陳平奇計,使單于閼氏,圍以得開。高帝既出,其計祕,世莫得聞。
\\\\
  高帝南過曲逆,上其城,望見其屋室甚大,曰:「壯哉縣!吾行天下,獨見洛陽與是耳。」顧問御史曰:「曲逆戶口幾何?」對曰:「始秦時三萬餘戶,閒者兵數起,多亡匿,今見五千戶。」於是乃詔御史,更以陳平為曲逆侯,盡食之,除前所食戶牖。
\\\\
  其後常以護軍中尉從攻陳豨及黥布。凡六出奇計,輒益邑,凡六益封。奇計或頗祕,世莫能聞也。
\\\\
  高帝從破布軍還,病創,徐行至長安。燕王盧綰反,上使樊噲以相國將兵攻之。既行,人有短惡噲者。高帝怒曰:「噲見吾病,乃冀我死也。」用陳平謀而召絳侯周勃受詔床下,曰:「陳平亟馳傳載勃代噲將,平至軍中即斬噲頭!」二人既受詔,馳傳未至軍,行計之曰:「樊噲,帝之故人也,功多,且又乃呂后弟呂媭之夫,有親且貴,帝以忿怒故,欲斬之,則恐後悔。寧囚而致上,上自誅之。」未至軍,為壇,以節召樊噲。噲受詔,即反接載檻車,傳詣長安,而令絳侯勃代將,將兵定燕反縣。
\\\\
  平行聞高帝崩,平恐呂太后及呂媭讒怒,乃馳傳先去。逢使者詔平與灌嬰屯於滎陽。平受詔,立復馳至宮,哭甚哀,因奏事喪前。呂太后哀之,曰:「君勞,出休矣。」平畏讒之就,因固請得宿衛中。太后乃以為郎中令,曰:「傅教孝惠。」是後呂媭讒乃不得行。樊噲至,則赦復爵邑。
\\\\
  孝惠帝六年,相國曹參卒,以安國侯王陵為右丞相,陳平為左丞相。
\\\\
  王陵者,故沛人,始為縣豪,高祖微時,兄事陵。陵少文,任氣,好直言。及高祖起沛,入至咸陽,陵亦自聚黨數千人,居南陽,不肯從沛公。及漢王之還攻項籍,陵乃以兵屬漢。項羽取陵母置軍中,陵使至,則東鄉坐陵母,欲以招陵。陵母既私送使者,泣曰:「為老妾語陵,謹事漢王。漢王,長者也,無以老妾故,持二心。妾以死送使者。」遂伏劍而死。項王怒,烹陵母。陵卒從漢王定天下。以善雍齒,雍齒,高帝之仇,而陵本無意從高帝,以故晚封,為安國侯。
\\\\
  安國侯既為右丞相,二歲,孝惠帝崩。高后欲立諸呂為王,問王陵,王陵曰:「不可。」問陳平,陳平曰:「可。」呂太后怒,乃詳遷陵為帝太傅,實不用陵。陵怒,謝疾免,杜門竟不朝請,七年而卒。
\\\\
  陵之免丞相,呂太后乃徙平為右丞相,以辟陽侯審食其為左丞相。左丞相不治,常給事於中。
\\\\
  食其亦沛人。漢王之敗彭城西,楚取太上皇、呂后為質,食其以舍人侍呂后。其後從破項籍為侯,幸於呂太后。及為相,居中,百官皆因決事。
\\\\
  呂媭常以前陳平為高帝謀執樊噲,數讒曰:「陳平為相非治事,日飲醇酒,戲婦女。」陳平聞,日益甚。呂太后聞之,私獨喜。面質呂媭於陳平曰:「鄙語曰『兒婦人口不可用』,顧君與我何如耳。無畏呂媭之讒也。」
\\\\
  呂太后立諸呂為王,陳平偽聽之。及呂太后崩,平與太尉勃合謀,卒誅諸呂,立孝文皇帝,陳平本謀也。審食其免相。
\\\\
  孝文帝立,以為太尉勃親以兵誅呂氏,功多;陳平欲讓勃尊位,乃謝病。孝文帝初立,怪平病,問之。平曰:「高祖時,勃功不如臣平。及誅諸呂,臣功亦不如勃。願以右丞相讓勃。」於是孝文帝乃以絳侯勃為右丞相,位次第一;平徙為左丞相,位次第二。賜平金千斤,益封三千戶。
\\\\
  居頃之,孝文皇帝既益明習國家事,朝而問右丞相勃曰:「天下一歲決獄幾何?」勃謝曰:「不知。」問:「天下一歲錢穀出入幾何?」勃又謝不知,汗出沾背,愧不能對。於是上亦問左丞相平。平曰:「有主者。」上曰:「主者謂誰?」平曰:「陛下即問決獄,責廷尉;問錢穀,責治粟內史。」上曰:「茍各有主者,而君所主者何事也?」平謝曰:「主臣!陛下不知其駑下,使待罪宰相。宰相者,上佐天子理陰陽,順四時,下育萬物之宜,外鎮撫四夷諸侯,內親附百姓,使卿大夫各得任其職焉。」孝文帝乃稱善。右丞相大慚,出而讓陳平曰:「君獨不素教我對!」陳平笑曰:「君居其位,不知其任邪?且陛下即問長安中盜賊數,君欲彊對邪?」於是絳侯自知其能不如平遠矣。居頃之,絳侯謝病請免相,陳平專為一丞相。
\\\\
  孝文帝二年,丞相陳平卒,謚為獻侯。子共侯買代侯。二年卒,子簡侯恢代侯。二十三年卒,子何代侯。二十三年,何坐略人妻,棄市,國除。
\\\\
  始陳平曰:「我多陰謀,是道家之所禁。吾世即廢,亦已矣,終不能復起,以吾多陰禍也。」然其後曾孫陳掌以衛氏親貴戚,願得續封陳氏,然終不得。
\\\\
  太史公曰:陳丞相平少時,本好黃帝、老子之術。方其割肉俎上之時,其意固已遠矣。傾側擾攘楚魏之閒,卒歸高帝。常出奇計,救紛糾之難,振國家之患。及呂后時,事多故矣,然平竟自脫,定宗廟,以榮名終,稱賢相,豈不善始善終哉!非知謀孰能當此者乎?


\part{列傳}
\LARGE

\section{老子韓非列傳}
  老子者,楚苦縣厲鄉曲仁里人也,姓李氏,名耳,字耼,周守藏室之史也。
\\\\
  孔子適周,將問禮於老子。老子曰:「子所言者,其人與骨皆已朽矣,獨其言在耳。且君子得其時則駕,不得其時則蓬累而行。吾聞之,良賈深藏若虛,君子盛德容貌若愚。去子之驕氣與多欲,態色與淫志,是皆無益於子之身。吾所以告子,若是而已。」孔子去,謂弟子曰:「鳥,吾知其能飛;魚,吾知其能游;獸,吾知其能走。走者可以為罔,游者可以為綸,飛者可以為矰。至於龍,吾不能知其乘風雲而上天。吾今日見老子,其猶龍邪!」
\\\\
  老子修道德,其學以自隱無名為務。居周久之,見周之衰,乃遂去。至關,關令尹喜曰:「子將隱矣,彊為我著書。」於是老子乃著書上下篇,言道德之意五千餘言而去,莫知其所終。
\\\\
  或曰:老萊子亦楚人也,著書十五篇,言道家之用,與孔子同時云。
\\\\
  蓋老子百有六十餘歲,或言二百餘歲,以其修道而養壽也。
\\\\
  自孔子死之後百二十九年,而史記周太史儋見秦獻公曰:「始秦與周合,合五百歲而離,離七十歲而霸王者出焉。」或曰儋即老子,或曰非也,世莫知其然否。老子,隱君子也。
\\\\
  老子之子名宗,宗為魏將,封於段干。宗子注,注子宮,宮玄孫假,假仕於漢孝文帝。而假之子解為膠西王卬太傅,因家于齊焉。
\\\\
  世之學老子者則絀儒學,儒學亦絀老子。「道不同不相為謀」,豈謂是邪?李耳無為自化,清靜自正。
\\\\
  莊子者,蒙人也,名周。周嘗為蒙漆園吏,與梁惠王、齊宣王同時。其學無所不闚,然其要本歸於老子之言。故其著書十餘萬言,大抵率寓言也。作漁父、盜跖、胠篋,以詆訿孔子之徒,以明老子之術。畏累虛、亢桑子之屬,皆空語無事實。然善屬書離辭,指事類情,用剽剝儒、墨,雖當世宿學不能自解免也。其言洸洋自恣以適己,故自王公大人不能器之。
\\\\
  楚威王聞莊周賢,使使厚幣迎之,許以為相。莊周笑謂楚使者曰:「千金,重利;卿相,尊位也。子獨不見郊祭之犧牛乎?養食之數歲,衣以文繡,以入大廟。當是之時,雖欲為孤豚,豈可得乎?子亟去,無污我。我寧游戲污瀆之中自快,無為有國者所羈,終身不仕,以快吾志焉。」
\\\\
  申不害者,京人也,故鄭之賤臣。學術以干韓昭侯,昭侯用為相。內修政教,外應諸侯,十五年。終申子之身,國治兵彊,無侵韓者。申子之學本於黃老而主刑名。著書二篇,號曰申子。
\\\\
  韓非者,韓之諸公子也。喜刑名法術之學,而其歸本於黃老。非為人口吃,不能道說,而善著書。與李斯俱事荀卿,斯自以為不如非。
\\\\
  非見韓之削弱,數以書諫韓王,韓王不能用。於是韓非疾治國不務修明其法制,執勢以御其臣下,富國彊兵而以求人任賢,反舉浮淫之蠹而加之於功實之上。以為儒者用文亂法,而俠者以武犯禁。寬則寵名譽之人,急則用介胄之士。今者所養非所用,所用非所養。悲廉直不容於邪枉之臣,觀往者得失之變,故作孤憤、五蠹、內外儲、說林、說難十餘萬言。
\\\\
  然韓非知說之難,為說難書甚具,終死於秦,不能自脫。
\\\\
  說難曰:
\\\\
  凡說之難,非吾知之有以說之難也;又非吾辯之難能明吾意之難也;又非吾敢橫失能盡之難也。凡說之難,在知所說之心,可以吾說當之。
\\\\
  所說出於為名高者也,而說之以厚利,則見下節而遇卑賤,必棄遠矣。所說出於厚利者也。而說之以名高,則見無心而遠事情,必不收矣。所說實為厚利而顯為名高者也,而說之以名高,則陽收其身而實疏之;若說之以厚利,則陰用其言而顯棄其身。此之不可不知也。
\\\\
  夫事以密成,語以泄敗。未必其身泄之也,而語及其所匿之事,如是者身危。貴人有過端,而說者明言善議以推其惡者,則身危。周澤未渥也而語極知,說行而有功則德亡,說不行而有敗則見疑,如是者身危。夫貴人得計而欲自以為功,說者與知焉,則身危。彼顯有所出事,迺自以為也故,說者與知焉,則身危。彊之以其所必不為,止之以其所不能已者,身危。故曰:與之論大人,則以為閒己;與之論細人,則以為粥權。論其所愛,則以為借資;論其所憎,則以為嘗己。徑省其辭,則不知而屈之;汎濫博文,則多而久之。順事陳意,則曰怯懦而不盡;慮事廣肆,則曰草野而倨侮。此說之難,不可不知也。
\\\\
  凡說之務,在知飾所說之所敬,而滅其所醜。彼自知其計,則毋以其失窮之;自勇其斷,則毋以其敵怒之;自多其力,則毋以其難概之。規異事與同計,譽異人與同行者,則以飾之無傷也。有與同失者,則明飾其無失也。大忠無所拂悟,辭言無所擊排,迺後申其辯知焉。此所以親近不疑,知盡之難也。得曠日彌久,而周澤既渥,深計而不疑,交爭而不罪,迺明計利害以致其功,直指是非以飾其身,以此相持,此說之成也。
\\\\
  伊尹為庖,百里奚為虜,皆所由干其上也。故此二子者,皆聖人也,猶不能無役身而涉世如此其汙也,則非能仕之所設也。
\\\\
  宋有富人,天雨牆壞。其子曰「不築且有盜」,其鄰人之父亦云,暮而果大亡其財,其家甚知其子而疑鄰人之父。昔者鄭武公欲伐胡,迺以其子妻之。因問群臣曰: 「吾欲用兵,誰可伐者?」關其思曰:「胡可伐。」迺戮關其思,曰:「胡,兄弟之國也,子言伐之,何也?」胡君聞之,以鄭為親己而不備鄭。鄭人襲胡,取之。此二說者,其知皆當矣,然而甚者為戮,薄者見疑。非知之難也,處知則難矣。
\\\\
  昔者彌子瑕見愛於衛君。衛國之法,竊駕君車者罪至刖。既而彌子之母病,人聞,往夜告之,彌子矯駕君車而出。君聞之而賢之曰:「孝哉,為母之故而犯刖罪!」與君游果園,彌子食桃而甘,不盡而奉君。君曰:「愛我哉,忘其口而念我!」及彌子色衰而愛弛,得罪於君。君曰:「是嘗矯駕吾車,又嘗食我以其餘桃。」故彌子之行未變於初也,前見賢而後獲罪者,愛憎之至變也。故有愛於主,則知當而加親;見憎於主,則罪當而加疏。故諫說之士不可不察愛憎之主而後說之矣。夫龍之為蟲也,可擾狎而騎也。然其喉下有逆鱗徑尺,人有嬰之,則必殺人。人主亦有逆鱗,說之者能無嬰人主之逆鱗,則幾矣。
\\\\
  人或傳其書至秦。秦王見孤憤、五蠹之書,曰:「嗟乎,寡人得見此人與之游,死不恨矣!」李斯曰:「此韓非之所著書也。」秦因急攻韓。韓王始不用非,及急,乃遣非使秦。秦王悅之,未信用。李斯、姚賈害之,毀之曰:「韓非,韓之諸公子也。今王欲并諸侯,非終為韓不為秦,此人之情也。今王不用,久留而歸之,此自遺患也,不如以過法誅之。」秦王以為然,下吏治非。李斯使人遺非藥,使自殺。韓非欲自陳,不得見。秦王後悔之,使人赦之,非已死矣。申子、韓子皆著書,傳於後世,學者多有。余獨悲韓子為說難而不能自脫耳。
\\\\
  太史公曰:老子所貴道,虛無,因應變化於無為,故著書辭稱微妙難識。莊子散道德,放論,要亦歸之自然。申子卑卑,施之於名實。韓子引繩墨,切事情,明是非,其極慘礉少恩。皆原於道德之意,而老子深遠矣。

\section{滑稽列傳}
  孔子曰:「六藝於治一也。禮以節人,樂以發和,書以道事,詩以達意,易以神化,春秋以義。」太史公曰:天道恢恢,豈不大哉!談言微中,亦可以解紛。
\\\\
  淳于髡者,齊之贅婿也。長不滿七尺,滑稽多辯,數使諸侯,未嘗屈辱。齊威王之時喜隱,好為淫樂長夜之飲,沈湎不治,委政卿大夫。百官荒亂,諸侯并侵,國且危亡,在於旦暮,左右莫敢諫。淳于髡說之以隱曰:「國中有大鳥,止王之庭,三年不蜚又不鳴,不知此鳥何也?」王曰:「此鳥不飛則已,一飛沖天;不鳴則已,一鳴驚人。」於是乃朝諸縣令長七十二人,賞一人,誅一人,奮兵而出。諸侯振驚,皆還齊侵地。威行三十六年。語在田完世家中。
\\\\
  威王八年,楚大發兵加齊。齊王使淳于髡之趙請救兵,齎金百斤,車馬十駟。淳于髡仰天大笑,冠纓索絕。王曰:「先生少之乎?」髡曰:「何敢!」王曰:「笑豈有說乎?」髡曰:「今者臣從東方來,見道傍有禳田者,操一豚蹄,酒一盂,祝曰:『甌窶滿篝,汙邪滿車,五穀蕃熟,穰穰滿家。』臣見其所持者狹而所欲者奢,故笑之。」於是齊威王乃益齎黃金千溢,白璧十雙,車馬百駟。髡辭而行,至趙。趙王與之精兵十萬,革車千乘。楚聞之,夜引兵而去。
\\\\
  威王大說,置酒後宮,召髡賜之酒。問曰:「先生能飲幾何而醉?」對曰:「臣飲一斗亦醉,一石亦醉。」威王曰:「先生飲一斗而醉,惡能飲一石哉!其說可得聞乎?」髡曰:「賜酒大王之前,執法在傍,御史在後,髡恐懼俯伏而飲,不過一斗徑醉矣。若親有嚴客,髡帣韝鞠跽,待酒於前,時賜餘瀝,奉觴上壽,數起,飲不過二斗徑醉矣。若朋友交遊,久不相見,卒然相睹,歡然道故,私情相語,飲可五六斗徑醉矣。若乃州閭之會,男女雜坐,行酒稽留,六博投壺,相引為曹,握手無罰,目眙不禁,前有墮珥,后有遺簪,髡竊樂此,飲可八斗而醉二參。日暮酒闌,合尊促坐,男女同席,履舄交錯,杯盤狼藉,堂上燭滅,主人留髡而送客,羅襦襟解,微聞薌澤,當此之時,髡心最歡,能飲一石。故曰酒極則亂,樂極則悲;萬事盡然,言不可極,極之而衰。」以諷諫焉。齊王曰:「善。」乃罷長夜之飲,以髡為諸侯主客。宗室置酒,髡嘗在側。
\\\\
  其後百餘年,楚有優孟。
\\\\
  優孟,故楚之樂人也。長八尺,多辯,常以談笑諷諫。楚莊王之時,有所愛馬,衣以文繡,置之華屋之下,席以露床,啗以棗脯。馬病肥死,使群臣喪之,欲以棺槨大夫禮葬之。左右爭之,以為不可。王下令曰:「有敢以馬諫者,罪至死。」優孟聞之,入殿門。仰天大哭。王驚而問其故。優孟曰:「馬者王之所愛也,以楚國堂堂之大,何求不得,而以大夫禮葬之,薄,請以人君禮葬之。」王曰:「何如?」對曰:「臣請以彫玉為棺,文梓為槨,楩楓豫章為題湊,發甲卒為穿壙,老弱負土,齊趙陪位於前,韓魏翼衛其后,廟食太牢,奉以萬戶之邑。諸侯聞之,皆知大王賤人而貴馬也。」王曰:「寡人之過一至此乎!為之柰何?」優孟曰:「請為大王六畜葬之。以壟灶為槨,銅歷為棺,齎以薑棗,薦以木蘭,祭以糧稻,衣以火光,葬之於人腹腸。」於是王乃使以馬屬太官,無令天下久聞也。
\\\\
  楚相孫叔敖知其賢人也,善待之。病且死,屬其子曰:「我死,汝必貧困。若往見優孟,言我孫叔敖之子也。」居數年,其子窮困負薪,逢優孟,與言曰:「我,孫叔敖子也。父且死時,屬我貧困往見優孟。」優孟曰:「若無遠有所之。」即為孫叔敖衣冠,抵掌談語。歲餘,像孫叔敖,楚王及左右不能別也。莊王置酒,優孟前為壽。莊王大驚,以為孫叔敖復生也,欲以為相。優孟曰:「請歸與婦計之,三日而為相。」莊王許之。三日後,優孟復來。王曰:「婦言謂何?」孟曰:「婦言慎無為,楚相不足為也。如孫叔敖之為楚相,盡忠為廉以治楚,楚王得以霸。今死,其子無立錐之地,貧困負薪以自飲食。必如孫叔敖,不如自殺。」因歌曰:「山居耕田苦,難以得食。起而為吏,身貪鄙者餘財,不顧恥辱。身死家室富,又恐受賕枉法,為姦觸大罪,身死而家滅。貪吏安可為也!念為廉吏,奉法守職,竟死不敢為非。廉吏安可為也!楚相孫叔敖持廉至死,方今妻子窮困負薪而食,不足為也!」於是莊王謝優孟,乃召孫叔敖子,封之寢丘四百戶,以奉其祀。后十世不絕。此知可以言時矣。
\\\\
  其後二百餘年,秦有優旃。
\\\\
  優旃者,秦倡侏儒也。善為笑言,然合於大道,秦始皇時,置酒而天雨,陛楯者皆沾寒。優旃見而哀之,謂之曰:「汝欲休乎?」陛楯者皆曰:「幸甚。」優旃曰:「我即呼汝,汝疾應曰諾。」居有頃,殿上上壽呼萬歲。優旃臨檻大呼曰:「陛楯郎!」郎曰:「諾。」優旃曰:「汝雖長,何益,幸雨立。我雖短也,幸休居。」於是始皇使陛楯者得半相代。
\\\\
  始皇嘗議欲大苑囿,東至函谷關,西至雍、陳倉。優旃曰:「善。多縱禽獸於其中,寇從東方來,令麋鹿觸之足矣。」始皇以故輟止。
\\\\
  二世立,又欲漆其城。優旃曰:「善。主上雖無言,臣固將請之。漆城雖於百姓愁費,然佳哉!漆城蕩蕩,寇來不能上。即欲就之,易為漆耳,顧難為蔭室。」於是二世笑之,以其故止。居無何,二世殺死,優旃歸漢,數年而卒。
\\\\
  太史公曰:淳于髡仰天大笑,齊威王橫行。優孟搖頭而歌,負薪者以封。優旃臨檻疾呼,陛楯得以半更。豈不亦偉哉!
\\\\
  褚先生曰:臣幸得以經術為郎,而好讀外家傳語。竊不遜讓,復作故事滑稽之語六章,編之於左。可以覽觀揚意,以示後世好事者讀之,以游心駭耳,以附益上方太史公之三章。
\\\\
  武帝時有所幸倡郭舍人者,發言陳辭雖不合大道,然令人主和說。武帝少時,東武侯母常養帝,帝壯時,號之曰「大乳母」。率一月再朝。朝奏入,有詔使幸臣馬游卿以帛五十匹賜乳母,又奉飲糒飱養乳母。乳母上書曰:「某所有公田,願得假倩之。」帝曰:「乳母欲得之乎?」以賜乳母。乳母所言,未嘗不聽。有詔得令乳母乘車行馳道中。當此之時,公卿大臣皆敬重乳母。乳母家子孫奴從者橫暴長安中,當道掣頓人車馬,奪人衣服。聞於中,不忍致之法。有司請徙乳母家室,處之於邊。奏可。乳母當入至前,面見辭。乳母先見郭舍人,為下泣。舍人曰:「即入見辭去,疾步數還顧。」乳母如其言,謝去,疾步數還顧。郭舍人疾言罵之曰:「咄!老女子!何不疾行!陛下已壯矣,寧尚須汝乳而活邪?尚何還顧!」於是人主憐焉悲之,乃下詔止無徙乳母,罰謫譖之者。
\\\\
  武帝時,齊人有東方生名朔,以好古傳書,愛經術,多所博觀外家之語。朔初入長安,至公車上書,凡用三千奏牘。公車令兩人共持舉其書,僅然能勝之。人主從上方讀之,止,輒乙其處,讀之二月乃盡。詔拜以為郎,常在側侍中。數召至前談語,人主未嘗不說也。時詔賜之食於前。飯已,盡懷其餘肉持去,衣盡汙。數賜縑帛,檐揭而去。徒用所賜錢帛,取少婦於長安中好女。率取婦一歲所者即棄去,更取婦。所賜錢財盡索之於女子。人主左右諸郎半呼之「狂人」。人主聞之,曰:「令朔在事無為是行者,若等安能及之哉!」朔任其子為郎,又為侍謁者,常持節出使。朔行殿中,郎謂之曰:「人皆以先生為狂。」朔曰:「如朔等,所謂避世於朝廷閒者也。古之人,乃避世於深山中。」時坐席中,酒酣,據地歌曰:「陸沈於俗,避世金馬門。宮殿中可以避世全身,何必深山之中,蒿廬之下。」金馬門者,宦[者]署門也,門傍有銅馬,故謂之曰「金馬門」。
\\\\
  時會聚宮下博士諸先生與論議,共難之曰:「蘇秦、張儀一當萬乘之主,而都卿相之位,澤及後世。今子大夫修先王之術,慕聖人之義,諷誦詩書百家之言,不可勝數。著於竹帛,自以為海內無雙,即可謂博聞辯智矣。然悉力盡忠以事聖帝,曠日持久,積數十年,官不過侍郎,位不過執戟,意者尚有遺行邪?其故何也?」東方生曰:「是固非子所能備也。彼一時也,此一時也,豈可同哉!夫張儀、蘇秦之時,周室大壞,諸侯不朝,力政爭權,相禽以兵,并為十二國,未有雌雄,得士者彊,失士者亡,故說聽行通,身處尊位,澤及後世,子孫長榮。今非然也。聖帝在上,德流天下,諸侯賓服,威振四夷,連四海之外以為席,安於覆盂,天下平均,合為一家,動發舉事,猶如運之掌中。賢與不肖,何以異哉?方今以天下之大,士民之眾,竭精馳說,并進輻湊者,不可勝數。悉力慕義,困於衣食,或失門戶。使張儀、蘇秦與仆并生於今之世,曾不能得掌故,安敢望常侍侍郎乎!傳曰:『天下無害菑,雖有聖人,無所施其才;上下和同,雖有賢者,無所立功。』故曰時異則事異。雖然,安可以不務修身乎?《詩》曰:『鼓鐘于宮,聲聞于外。鶴鳴九皋,聲聞于天。』。茍能修身,何患不榮!太公躬行仁義七十二年,逢文王,得行其說,封於齊,七百歲而不絕。此士之所以日夜孜孜,修學行道,不敢止也。今世之處士,時雖不用,崛然獨立,塊然獨處,上觀許由,下察接輿,策同范蠡,忠合子胥,天下和平,與義相扶,寡偶少徒,固其常也。子何疑於余哉!」於是諸先生默然無以應也。
\\\\
  建章宮後閤重櫟中有物出焉,其狀似麋。以聞,武帝往臨視之。問左右群臣習事通經術者,莫能知。詔東方朔視之。朔曰:「臣知之,願賜美酒粱飯大飱臣,臣乃言。」詔曰:「可。」已又曰:「某所有公田魚池蒲葦數頃,陛下以賜臣,臣朔乃言。」詔曰:「可。」於是朔乃肯言,曰:「所謂騶牙者也。遠方當來歸義,而騶牙先見。其齒前后若一,齊等無牙,故謂之騶牙。」其後一歲所,匈奴混邪王果將十萬眾來降漢。乃復賜東方生錢財甚多。
\\\\
  至老,朔且死時,諫曰:「《詩》云『營營青蠅,止于蕃。愷悌君子,無信讒言。讒言罔極,交亂四國』。願陛下遠巧佞,退讒言。」帝曰:「今顧東方朔多善言?」怪之。居無幾何,朔果病死。傳曰:「鳥之將死,其鳴也哀;人之將死,其言也善。」此之謂也。
\\\\
  武帝時,大將軍衛青者,衛后兄也,封為長平侯。從軍擊匈奴,至余吾水上而還,斬首捕虜,有功來歸,詔賜金千斤。將軍出宮門,齊人東郭先生以方士待詔公車,當道遮衛將軍車,拜謁曰:「願白事。」將軍止車前,東郭先生旁車言曰:「王夫人新得幸於上,家貧。今將軍得金千斤,誠以其半賜王夫人之親,人主聞之必喜。此所謂奇策便計也。」衛將軍謝之曰:「先生幸告之以便計,請奉教。」於是衛將軍乃以五百金為王夫人之親壽。王夫人以聞武帝。帝曰:「大將軍不知為此。」問之安所受計策,對曰:「受之待詔者東郭先生。」詔召東郭先生,拜以為郡都尉。東郭先生久待詔公車,貧困饑寒,衣敝,履不完。行雪中,履有上無下,足盡踐地。道中人笑之,東郭先生應之曰:「誰能履行雪中,令人視之,其上履也,其履下處乃似人足者乎?」及其拜為二千石,佩青緺出宮門,行謝主人。故所以同官待詔者,等比祖道於都門外。榮華道路,立名當世。此所謂衣褐懷寶者也。當其貧困時,人莫省視;至其貴也,乃爭附之。諺曰:「相馬失之瘦,相士失之貧。」其此之謂邪?
\\\\
  王夫人病甚,人主至自往問之曰:「子當為王,欲安所置之?」對曰:「願居洛陽。」人主曰:「不可。洛陽有武庫、敖倉,當關口,天下咽喉。自先帝以來,傳不為置王。然關東國莫大於齊,可以為齊王。」王夫人以手擊頭,呼「幸甚」。王夫人死,號曰「齊王太后薨」。
\\\\
  昔者,齊王使淳于髡獻鵠於楚。出邑門,道飛其鵠,徒揭空籠,造詐成辭,往見楚王曰:「齊王使臣來獻鵠,過於水上,不忍鵠之渴,出而飲之,去我飛亡。吾欲刺腹絞頸而死。恐人之議吾王以鳥獸之故令士自傷殺也。鵠,毛物,多相類者,吾欲買而代之,是不信而欺吾王也。欲赴佗國奔亡,痛吾兩主使不通。故來服過,叩頭受罪大王。」楚王曰:「善,齊王有信士若此哉!」厚賜之,財倍鵠在也。
\\\\
  武帝時,徵北海太守詣行在所。有文學卒史王先生者,自請與太守俱,「吾有益於君」,君許之。諸府掾功曹白云:「王先生嗜酒,多言少實,恐不可與俱。」太守曰:「先生意欲行,不可逆。」遂與俱。行至宮下,待詔宮府門。王先生徒懷錢沽酒,與衛卒仆射飲,日醉,不視其太守。太守入跪拜。王先生謂戶郎曰:「幸為我呼吾君至門內遙語。」戶郎為呼太守。太守來,望見王先生。王先生曰:「天子即問君何以治北海令無盜賊,君對曰何哉?」對曰:「選擇賢材,各任之以其能,賞異等,罰不肖。」王先生曰:「對如是,是自譽自伐功,不可也。願君對言,非臣之力,盡陛下神靈威武所變化也。」太守曰:「諾。」召入,至于殿下,有詔問之曰:「何於治北海,令盜賊不起?」叩頭對言:「非臣之力,盡陛下神靈威武之所變化也。」武帝大笑,曰:「於呼!安得長者之語而稱之!安所受之?」對曰:「受之文學卒史。」帝曰:「今安在?」對曰:「在宮府門外。」有詔召拜王先生為水衡丞,以北海太守為水衡都尉。傳曰:「美言可以市,尊行可以加人。君子相送以言,小人相送以財。」
\\\\
  魏文侯時,西門豹為鄴令。豹往到鄴,會長老,問之民所疾苦。長老曰:「苦為河伯娶婦,以故貧。」豹問其故,對曰:「鄴三老、廷掾常歲賦斂百姓,收取其錢得數百萬,用其二三十萬為河伯娶婦,與祝巫共分其餘錢持歸。當其時,巫行視小家女好者,云是當為河伯婦,即娉取。洗沐之,為治新繒綺縠衣,閒居齋戒;為治齋宮河上,張緹絳帷,女居其中。為具牛酒飯食,行十餘日。共粉飾之,如嫁女床席,令女居其上,浮之河中。始浮,行數十里乃沒。其人家有好女者,恐大巫祝為河伯取之,以故多持女遠逃亡。以故城中益空無人,又困貧,所從來久遠矣。民人俗語曰『即不為河伯娶婦,水來漂沒,溺其人民』云。」西門豹曰:「至為河伯娶婦時,願三老、巫祝、父老送女河上,幸來告語之,吾亦往送女。」皆曰:「諾。」
\\\\
  至其時,西門豹往會之河上。三老、官屬、豪長者、裏父老皆會,以人民往觀之者三二千人。其巫,老女子也,已年七十。從弟子女十人所,皆衣繒單衣,立大巫后。西門豹曰:「呼河伯婦來,視其好醜。」即將女出帷中,來至前。豹視之,顧謂三老、巫祝、父老曰:「是女子不好,煩大巫嫗為入報河伯,得更求好女,后日送之。」即使吏卒共抱大巫嫗投之河中。有頃,曰:「巫嫗何久也?弟子趣之!」復以弟子一人投河中。有頃,曰:「弟子何久也?復使一人趣之!」復投一弟子河中。凡投三弟子。西門豹曰:「巫嫗弟子是女子也,不能白事,煩三老為入白之。」復投三老河中。西門豹簪筆磬折,向河立待良久。長老、吏傍觀者皆驚恐。西門豹顧曰:「巫嫗、三老不來還,柰之何?」欲復使廷掾與豪長者一人入趣之。皆叩頭,叩頭且破,額血流地,色如死灰。西門豹曰:「諾,且留待之須臾。」須臾,豹曰:「廷掾起矣。狀河伯留客之久,若皆罷去歸矣。」鄴吏民大驚恐,從是以後,不敢復言為河伯娶婦。
\\\\
  西門豹即發民鑿十二渠,引河水灌民田,田皆溉。當其時,民治渠少煩苦,不欲也。豹曰:「民可以樂成,不可與慮始。今父老子弟雖患苦我,然百歲後期令父老子孫思我言。」至今皆得水利,民人以給足富。十二渠經絕馳道,到漢之立,而長吏以為十二渠橋絕馳道,相比近,不可。欲合渠水,且至馳道合三渠為一橋。鄴民人父老不肯聽長吏,以為西門君所為也,賢君之法式不可更也。長吏終聽置之。故西門豹為鄴令,名聞天下,澤流後世,無絕已時,幾可謂非賢大夫哉!
\\\\
  傳曰:「子產治鄭,民不能欺;子賤治單父,民不忍欺;西門豹治鄴,民不敢欺。」三子之才能誰最賢哉?辨治者當能別之。

\section{貨殖列傳}
  老子曰:「至治之極,鄰國相望,雞狗之聲相聞,民各甘其食,美其服,安其俗,樂其業,至老死,不相往來。」必用此為務,輓近世涂民耳目,則幾無行矣。
\\\\
  太史公曰:夫神農以前,吾不知已。至若詩書所述虞夏以來,耳目欲極聲色之好,口欲窮芻豢之味,身安逸樂,而心誇矜輓能之榮使。俗之漸民久矣,雖戶說以眇論,終不能化。故善者因之,其次利道之,其次教誨之,其次整齊之,最下者與之爭。
\\\\
  夫山西饒材、竹、穀、纑、旄、玉石;山東多魚、鹽、漆、絲、聲色;江南出枏、梓、薑、桂、金、錫、連、丹沙、犀、瑁、珠璣、齒革;龍門、碣石北多馬、牛、羊、旃裘、筋角;銅、鐵則千里往往山出棋置:此其大較也。皆中國人民所喜好謠俗被服飲食奉生送死之具也。故待農而食之,虞而出之,工而成之,商而通之。此寧有政教發徵期會哉?人各任其能,竭其力,以得所欲。故物賤之徵貴,貴之徵賤,各勸其業,樂其事,若水之趨下,日夜無休時,不召而自來,不求而民出之。豈非道之所符,而自然之驗邪?
\\\\
  《周書》曰:「農不出則乏其食,工不出則乏其事,商不出則三寶絕,虞不出則財匱少。」財匱少而山澤不辟矣。此四者,民所衣食之原也。原大則饒,原小則鮮。上則富國,下則富家。貧富之道,莫之奪予,而巧者有餘,拙者不足。故太公望封於營丘,地澙鹵,人民寡,於是太公勸其女功,極技巧,通魚鹽,則人物歸之,繦至而輻湊。故齊冠帶衣履天下,海岱之閒斂袂而往朝焉。其後齊中衰,管子修之,設輕重九府,則桓公以霸,九合諸侯,一匡天下;而管氏亦有三歸,位在陪臣,富於列國之君。是以齊富彊至於威、宣也。
\\\\
  故曰:「倉廩實而知禮節,衣食足而知榮辱。」禮生於有而廢於無。故君子富,好行其德;小人富,以適其力。淵深而魚生之,山深而獸往之,人富而仁義附焉。富者得埶益彰,失埶則客無所之,以而不樂。夷狄益甚。諺曰:「千金之子,不死於市。」此非空言也。故曰:「天下熙熙,皆為利來;天下壤壤,皆為利往。」夫千乘之王,萬家之侯,百室之君,尚猶患貧,而況匹夫編戶之民乎!
\\\\
  昔者越王句踐困於會稽之上,乃用范蠡、計然。計然曰:「知斗則修備,時用則知物,二者形則萬貨之情可得而觀已。故歲在金,穰;水,毀;木,饑;火,旱。旱則資舟,水則資車,物之理也。六歲穰,六歲旱,十二歲一大饑。夫糶,二十病農,九十病末。末病則財不出,農病則草不辟矣。上不過八十,下不減三十,則農末俱利,平糶齊物,關市不乏,治國之道也。積著之理,務完物,無息幣。以物相貿易,腐敗而食之貨勿留,無敢居貴。論其有餘不足,則知貴賤。貴上極則反賤,賤下極則反貴。貴出如糞土,賤取如珠玉。財幣欲其行如流水。」修之十年,國富,厚賂戰士,士赴矢石,如渴得飲,遂報彊吳,觀兵中國,稱號「五霸」。
\\\\
  范蠡既雪會稽之恥,乃喟然而嘆曰:「計然之策七,越用其五而得意。既已施於國,吾欲用之家。」乃乘扁舟浮於江湖,變名易姓,適齊為鴟夷子皮,之陶為朱公。朱公以為陶天下之中,諸侯四通,貨物所交易也。乃治產積居。與時逐而不責於人。故善治生者,能擇人而任時。十九年之中三致千金,再分散與貧交疏昆弟。此所謂富好行其德者也。後年衰老而聽子孫,子孫修業而息之,遂至巨萬。故言富者皆稱陶朱公。
\\\\
  子贛既學於仲尼,退而仕於衛,廢著鬻財於曹、魯之閒,七十子之徒,賜最為饒益。原憲不厭糟糠,匿於窮巷。子貢結駟連騎,束帛之幣以聘享諸侯,所至,國君無不分庭與之抗禮。夫使孔子名布揚於天下者,子貢先後之也。此所謂得埶而益彰者乎?
\\\\
  白圭,周人也。當魏文侯時,李克務盡地力,而白圭樂觀時變,故人棄我取,人取我與。夫歲孰取穀,予之絲漆;繭出取帛絮,予之食。太陰在卯,穰;明歲衰惡。至午,旱;明歲美。至酉,穰;明歲衰惡。至子,大旱;明歲美,有水。至卯,積著率歲倍。欲長錢,取下穀;長石斗,取上種。能薄飲食,忍嗜欲,節衣服,與用事僮仆同苦樂,趨時若猛獸摯鳥之發。故曰:「吾治生產,猶伊尹、呂尚之謀,孫吳用兵,商鞅行法是也。是故其智不足與權變,勇不足以決斷,仁不能以取予,彊不能有所守,雖欲學吾術,終不告之矣。」蓋天下言治生祖白圭。白圭其有所試矣,能試有所長,非茍而已也。
\\\\
  猗頓用盬鹽起。而邯鄲郭縱以鐵冶成業,與王者埒富。
\\\\
  烏氏倮牧,及眾,斥賣,求奇繒物,閒獻遺戎王。戎王什倍其償,與之畜,畜至用谷量馬牛。秦始皇帝令倮比封君,以時與列臣朝請。而巴[蜀]寡婦清,其先得丹穴,而擅其利數世,家亦不訾。清,寡婦也,能守其業,用財自衛,不見侵犯。秦皇帝以為貞婦而客之,為筑女懷清臺。夫倮鄙人牧長,清窮鄉寡婦,禮抗萬乘,名顯天下,豈非以富邪?
\\\\
  漢興,海內為一,開關梁,弛山澤之禁,是以富商大賈周流天下,交易之物莫不通,得其所欲,而徙豪傑諸侯彊族於京師。
\\\\
  關中自汧、雍以東至河、華,膏壤沃野千里,自虞夏之貢以為上田,而公劉適邠,大王、王季在岐,文王作豐,武王治鎬,故其民猶有先王之遺風,好稼穡,殖五穀,地重,重為邪。及秦文、(孝)[德]、繆居雍,隙隴蜀之貨物而多賈。獻(孝)公徙櫟邑,櫟邑北卻戎翟,東通三晉,亦多大賈。(武)[孝]、昭治咸陽,因以漢都,長安諸陵,四方輻湊并至而會,地小人眾,故其民益玩巧而事末也。南則巴蜀。巴蜀亦沃野,地饒炧、薑、丹沙、石、銅、鐵、竹、木之器。南御滇僰,僰僮。西近邛笮,笮馬、旄牛。然四塞,棧道千里,無所不通,唯褒斜綰轂其口,以所多易所鮮。天水、隴西、北地、上郡與關中同俗,然西有羌中之利,北有戎翟之畜,畜牧為天下饒。然地亦窮險,唯京師要其道。故關中之地,於天下三分之一,而人眾不過什三;然量其富,什居其六。
\\\\
  昔唐人都河東,殷人都河內,周人都河南。夫三河在天下之中,若鼎足,王者所更居也,建國各數百千歲,土地小狹,民人眾,都國諸侯所聚會,故其俗纖儉習事。楊、平陽陳西賈秦、翟,北賈種、代。種、代,石北也,地邊胡,數被寇。人民矜懻忮,好氣,任俠為姦,不事農商。然迫近北夷,師旅亟往,中國委輸時有奇羨。其民羯羠不均,自全晉之時固已患其僄悍,而武靈王益厲之,其謠俗猶有趙之風也。故楊、平陽陳掾其閒,得所欲。溫、軹西賈上黨,北賈趙、中山。中山地薄人眾,猶有沙丘紂淫地餘民,民俗懁急,仰機利而食。丈夫相聚游戲,悲歌慨,起則相隨椎剽,休則掘冢作巧姦冶,多美物,為倡優。女子則鼓鳴瑟,跕屣,游媚貴富,入後宮,遍諸侯。
\\\\
  然邯鄲亦漳、河之閒一都會也。北通燕、涿,南有鄭、衛。鄭、衛俗與趙相類,然近梁、魯,微重而矜節。濮上之邑徙野王,野王好氣任俠,衛之風也。
\\\\
  夫燕亦勃、碣之閒一都會也。南通齊、趙,東北邊胡。上谷至遼東,地踔遠,人民希,數被寇,大與趙、代俗相類,而民雕捍少慮,有魚鹽棗栗之饒。北鄰烏桓、夫餘,東綰穢貉、朝鮮、真番之利。
\\\\
  洛陽東賈齊、魯,南賈梁、楚。故泰山之陽則魯,其陰則齊。
\\\\
  齊帶山海,膏壤千里,宜桑麻,人民多文綵布帛魚鹽。臨菑亦海岱之閒一都會也。其俗寬緩闊達,而足智,好議論,地重,難動搖,怯於眾鬬,勇於持刺,故多劫人者,大國之風也。其中具五民。
\\\\
  而鄒、魯濱洙、泗,猶有周公遺風,俗好儒,備於禮,故其民齪齪。頗有桑麻之業,無林澤之饒。地小人眾,儉嗇,畏罪遠邪。及其衰,好賈趨利,甚於周人。
\\\\
  夫自鴻溝以東,芒、碭以北,屬巨野,此梁、宋也。陶、睢陽亦一都會也。昔堯作(游)[於]成陽,舜漁於雷澤,湯止于亳。其俗猶有先王遺風,重厚多君子,好稼穡,雖無山川之饒,能惡衣食,致其蓄藏。
\\\\
  越、楚則有三俗。夫自淮北沛、陳、汝南、南郡,此西楚也。其俗剽輕,易發怒,地薄,寡於積聚。江陵故郢都,西通巫、巴,東有雲夢之饒。陳在楚夏之交,通魚鹽之貨,其民多賈。徐、僮、取慮,則清刻,矜己諾。
\\\\
  彭城以東,東海、吳、廣陵,此東楚也。其俗類徐、僮。朐、繒以北,俗則齊。浙江南則越。夫吳自闔廬、春申、王濞三人招致天下之喜游子弟,東有海鹽之饒,章山之銅,三江、五湖之利,亦江東一都會也。
\\\\
  衡山、九江、江南、豫章、長沙,是南楚也,其俗大類西楚。郢之後徙壽春,亦一都會也。而合肥受南北潮,皮革、鮑、木輸會也。與閩中、干越雜俗,故南楚好辭,巧說少信。江南卑溼,丈夫早夭。多竹木。豫章出黃金,長沙出連、錫,然堇堇物之所有,取之不足以更費。九疑、蒼梧以南至儋耳者,與江南大同俗,而楊越多焉。番禺亦其一都會也,珠璣、犀、瑁、果、布之湊。
\\\\
  潁川、南陽,夏人之居也。夏人政尚忠樸,猶有先王之遺風。潁川敦願。秦末世,遷不軌之民於南陽。南陽西通武關、鄖關,東南受漢、江、淮。宛亦一都會也。俗雜好事,業多賈。其任俠,交通潁川,故至今謂之「夏人」。
\\\\
  夫天下物所鮮所多,人民謠俗,山東食海鹽,山西食鹽鹵,領南、沙北固往往出鹽,大體如此矣。
\\\\
  總之,楚越之地,地廣人希,飯稻羹魚,或火耕而水耨,果隋蠃蛤,不待賈而足,地埶饒食,無饑饉之患,以故呰窳偷生,無積聚而多貧。是故江淮以南,無凍餓之人,亦無千金之家。沂、泗水以北,宜五穀桑麻六畜,地小人眾,數被水旱之害,民好畜藏,故秦、夏、梁、魯好農而重民。三河、宛、陳亦然,加以商賈。齊、趙設智巧,仰機利。燕、代田畜而事蠶。
\\\\
  由此觀之,賢人深謀於廊廟,論議朝廷,守信死節隱居巖穴之士設為名高者安歸乎?歸於富厚也。是以廉吏久,久更富,廉賈歸富。富者,人之情性,所不學而俱欲者也。故壯士在軍,攻城先登,陷陣卻敵,斬將搴旗,前蒙矢石,不避湯火之難者,為重賞使也。其在閭巷少年,攻剽椎埋,劫人作姦,掘冢鑄幣,任俠并兼,借交報仇,篡逐幽隱,不避法禁,走死地如騖者,其實皆為財用耳。今夫趙女鄭姬,設形容,揳鳴琴,揄長袂,躡利屣,目挑心招,出不遠千里,不擇老少者,奔富厚也。游閒公子,飾冠劍,連車騎,亦為富貴容也。弋射漁獵,犯晨夜,冒霜雪,馳阬谷,不避猛獸之害,為得味也。博戲馳逐,鬬雞走狗,作色相矜,必爭勝者,重失負也。醫方諸食技術之人,焦神極能,為重糈也。吏士舞文弄法,刻章偽書,不避刀鋸之誅者,沒於賂遺也。農工商賈畜長,固求富益貨也。此有知盡能索耳,終不餘力而讓財矣。
\\\\
  諺曰:「百里不販樵,千里不販糴。」居之一歲,種之以穀;十歲,樹之以木;百歲,來之以德。德者,人物之謂也。今有無秩祿之奉,爵邑之入,而樂與之比者。命曰「素封」。封者食租稅,歲率戶二百。千戶之君則二十萬,朝覲聘享出其中。庶民農工商賈,率亦歲萬息二千,百萬之家則二十萬,而更傜租賦出其中。衣食之欲,恣所好美矣。故曰陸地牧馬二百蹄,牛蹄角千,千足羊,澤中千足彘,水居千石魚陂,山居千章之材。安邑千樹棗;燕、秦千樹栗;蜀、漢、江陵千樹橘;淮北、常山已南,河濟之閒千樹萩;陳、夏千畝漆;齊、魯千畝桑麻;渭川千畝竹;及名國萬家之城,帶郭千畝畝鐘之田,若千畝卮茜,千畦薑韭:此其人皆與千戶侯等。然是富給之資也,不窺市井,不行異邑,坐而待收,身有處士之義而取給焉。若至家貧親老,妻子軟弱,歲時無以祭祀進醵,飲食被服不足以自通,如此不慚恥,則無所比矣。是以無財作力,少有鬬智,既饒爭時,此其大經也。今治生不待危身取給,則賢人勉焉。是故本富為上,末富次之,姦富最下。無巖處奇士之行,而長貧賤,好語仁義,亦足羞也。
\\\\
  凡編戶之民,富相什則卑下之,伯則畏憚之,千則役,萬則仆,物之理也。夫用貧求富,農不如工,工不如商,刺繡文不如倚市門,此言末業,貧者之資也。通邑大都,酤一歲千釀,醯醬千瓨,漿千甔,屠牛羊彘千皮,販穀糶千鐘,薪槁千車,船長千丈,木千章,竹竿萬,其軺車百乘,牛車千兩,木器髤者千枚,銅器千鈞,素木鐵器若炧茜千石,馬蹄蹾千,牛千足,羊彘千雙,僮手指千,筋角丹沙千斤,其帛絮細布千鈞,文采千匹,榻布皮革千石,漆千斗,糱麹鹽豉千荅,鮐鮆千斤,鯫千石,鮑千鈞,棗栗千石者三之,狐鼦裘千皮,羔羊裘千石,旃席千具,佗果菜千鐘,子貸金錢千貫,節駔會,貪賈三之,廉賈五之,此亦比千乘之家,其大率也。佗雜業不中什二,則非吾財也。
\\\\
  請略道當世千里之中,賢人所以富者,令後世得以觀擇焉。
\\\\
  蜀卓氏之先,趙人也,用鐵冶富。秦破趙,遷卓氏。卓氏見虜略,獨夫妻推輦,行詣遷處。諸遷虜少有餘財,爭與吏,求近處,處葭萌。唯卓氏曰:「此地狹薄。吾聞汶山之下,沃野,下有蹲鴟,至死不饑。民工於市,易賈。」乃求遠遷。致之臨邛,大喜,即鐵山鼓鑄,運籌策,傾滇蜀之民,富至僮千人。田池射獵之樂,擬於人君。
\\\\
  程鄭,山東遷虜也,亦冶鑄,賈椎髻之民,富埒卓氏,俱居臨邛。
\\\\
  宛孔氏之先,梁人也,用鐵冶為業。秦伐魏,遷孔氏南陽。大鼓鑄,規陂池,連車騎,游諸侯,因通商賈之利,有游閒公子之賜與名。然其贏得過當,愈於纖嗇,家致富數千金,故南陽行賈盡法孔氏之雍容。
\\\\
  魯人俗儉嗇,而曹邴氏尤甚,以鐵冶起,富至巨萬。然家自父兄子孫約,俛有拾,仰有取,貰貸行賈遍郡國。鄒、魯以其故多去文學而趨利者,以曹邴氏也。
\\\\
  齊俗賤奴虜,而刀閒獨愛貴之。桀黠奴,人之所患也,唯刀閒收取,使之逐漁鹽商賈之利,或連車騎,交守相,然愈益任之。終得其力,起富數千萬。故曰「寧爵毋刀」,言其能使豪奴自饒而盡其力。
\\\\
  周人既纖,而師史尤甚,轉轂以百數,賈郡國,無所不至。洛陽街居在齊秦楚趙之中,貧人學事富家,相矜以久賈,數過邑不入門,設任此等,故師史能致七千萬。
\\\\
  宣曲任氏之先,為督道倉吏。秦之敗也,豪傑皆爭取金玉,而任氏獨窖倉粟。楚漢相距滎陽也,民不得耕種,米石至萬,而豪傑金玉盡歸任氏,任氏以此起富。富人爭奢侈,而任氏折節為儉,力田畜。田畜人爭取賤賈,任氏獨取貴善。富者數世。然任公家約,非田畜所出弗衣食,公事不畢則身不得飲酒食肉。以此為閭里率,故富而主上重之。
\\\\
  塞之斥也,唯橋姚已致馬千匹,牛倍之,羊萬頭,粟以萬鐘計。吳楚七國兵起時,長安中列侯封君行從軍旅,齎貸子錢,子錢家以為侯邑國在關東,關東成敗未決,莫肯與。唯無鹽氏出捐千金貸,其息什之。三月,吳楚平,一歲之中,則無鹽氏之息什倍,用此富埒關中。
\\\\
  關中富商大賈,大抵盡諸田,田嗇、田蘭。韋家栗氏,安陵、杜杜氏,亦巨萬。
\\\\
  此其章章尤異者也。皆非有爵邑奉祿弄法犯姦而富,盡椎埋去就,與時俯仰,獲其贏利,以末致財,用本守之,以武一切,用文持之,變化有概,故足術也。若至力農畜,工虞商賈,為權利以成富,大者傾郡,中者傾縣,下者傾鄉里者,不可勝數。
\\\\
  夫纖嗇筋力,治生之正道也,而富者必用奇勝。田農,掘業,而秦揚以蓋一州。掘冢,姦事也,而田叔以起。博戲,惡業也,而桓發用富。行賈,丈夫賤行也,而雍樂成以饒。販脂,辱處也,而雍伯千金。賣漿,小業也,而張氏千萬。灑削,薄技也,而郅氏鼎食。胃脯,簡微耳,濁氏連騎。馬醫,淺方,張裏擊鐘。此皆誠壹之所致。由是觀之,富無經業,則貨無常主,能者輻湊,不肖者瓦解。千金之家比一都之君,巨萬者乃與王者同樂。豈所謂「素封」者邪?非也?

《太史公自序》

提到《太史公自序》的書籍 電子圖書館
\\\\
  昔在顓頊,命南正重以司天,北正黎以司地。唐虞之際,紹重黎之後,使復典之,至于夏商,故重黎氏世序天地。其在周,程伯休甫其後也。當周宣王時,失其守而為司馬氏。司馬氏世典周史。惠襄之閒,司馬氏去周適晉。晉中軍隨會奔秦,而司馬氏入少梁。
\\\\
  自司馬氏去周適晉,分散,或在衛,或在趙,或在秦。其在衛者,相中山。在趙者,以傳劍論顯,蒯聵其後也。在秦者名錯,與張儀爭論,於是惠王使錯將伐蜀,遂拔,因而守之。錯孫靳,事武安君白起。而少梁更名曰夏陽。靳與武安君阬趙長平軍,還而與之俱賜死杜郵,葬於華池。靳孫昌,昌為秦主鐵官,當始皇之時。蒯聵玄孫卬為武信君將而徇朝歌。諸侯之相王,王卬於殷。漢之伐楚,卬歸漢,以其地為河內郡。昌生無澤,無澤為漢市長。無澤生喜,喜為五大夫,卒,皆葬高門。喜生談,談為太史公。
\\\\
  太史公學天官於唐都,受易於楊何,習道論於黃子。太史公仕於建元元封之閒,愍學者之不達其意而師悖,乃論六家之要指曰:
\\\\
  易大傳:「天下一致而百慮,同歸而殊涂。」夫陰陽、儒、墨、名、法、道德,此務為治者也,直所從言之異路,有省不省耳。嘗竊觀陰陽之術,大祥而眾忌諱,使人拘而多所畏;然其序四時之大順,不可失也。儒者博而寡要,勞而少功,是以其事難盡從;然其序君臣父子之禮,列夫婦長幼之別,不可易也。墨者儉而難遵,是以其事不可遍循;然其彊本節用,不可廢也。法家嚴而少恩;然其正君臣上下之分,不可改矣。名家使人儉而善失真;然其正名實,不可不察也。道家使人精神專一,動合無形,贍足萬物。其為術也,因陰陽之大順,采儒墨之善,撮名法之要,與時遷移,應物變化,立俗施事,無所不宜,指約而易操,事少而功多。儒者則不然。以為人主天下之儀表也,主倡而臣和,主先而臣隨。如此則主勞而臣逸。至於大道之要,去健羨,絀聰明,釋此而任術。夫神大用則竭,形大勞則敝。形神騷動,欲與天地長久,非所聞也。
\\\\
  夫陰陽四時、八位、十二度、二十四節各有教令,順之者昌,逆之者不死則亡,未必然也,故曰「使人拘而多畏」。夫春生夏長,秋收冬藏,此天道之大經也,弗順則無以為天下綱紀,故曰「四時之大順,不可失也」。
\\\\
  夫儒者以六藝為法。六藝經傳以千萬數,累世不能通其學,當年不能究其禮,故曰「博而寡要,勞而少功」。若夫列君臣父子之禮,序夫婦長幼之別,雖百家弗能易也。
\\\\
  墨者亦尚堯舜道,言其德行曰:「堂高三尺,土階三等,茅茨不翦,采椽不刮。食土簋,啜土刑,糲粱之食,藜霍之羹。夏日葛衣,冬日鹿裘。」其送死,桐棺三寸,舉音不盡其哀。教喪禮,必以此為萬民之率。使天下法若此,則尊卑無別也。夫世異時移,事業不必同,故曰「儉而難遵」。要曰彊本節用,則人給家足之道也。此墨子之所長,雖百長弗能廢也。
\\\\
  法家不別親疏,不殊貴賤,一斷於法,則親親尊尊之恩絕矣。可以行一時之計,而不可長用也,故曰「嚴而少恩」。若尊主卑臣,明分職不得相踰越,雖百家弗能改也。
\\\\
  名家苛察繳繞,使人不得反其意,專決於名而失人情,故曰「使人儉而善失真」。若夫控名責實,參伍不失,此不可不察也。
\\\\
  道家無為,又曰無不為,其實易行,其辭難知。其術以虛無為本,以因循為用。無成埶,無常形,故能究萬物之情。不為物先,不為物後,故能為萬物主。有法無法,因時為業;有度無度,因物與合。故曰「聖人不朽,時變是守。虛者道之常也,因者君之綱」也。群臣并至,使各自明也。其實中其聲者謂之端,實不中其聲者謂之窾。窾言不聽,姦乃不生,賢不肖自分,白黑乃形。在所欲用耳,何事不成。乃合大道,混混冥冥。光燿天下,復反無名。凡人所生者神也,所託者形也。神大用則竭,形大勞則敝,形神離則死。死者不可復生,離者不可復反,故聖人重之。由是觀之,神者生之本也,形者生之具也。不先定其神[形],而曰「我有以治天下」,何由哉?
\\\\
  太史公既掌天官,不治民。有子曰遷。
\\\\
  遷生龍門,耕牧河山之陽。年十歲則誦古文。二十而南游江、淮,上會稽,探禹穴,闚九疑,浮於沅、湘;北涉汶、泗,講業齊、魯之都,觀孔子之遺風,鄉射鄒、嶧;戹困鄱、薛、彭城,過梁、楚以歸。於是遷仕為郎中,奉使西征巴、蜀以南,南略邛、笮、昆明,還報命。
\\\\
  是歲天子始建漢家之封,而太史公留滯周南,不得與從事,故發憤且卒。而子遷適使反,見父於河洛之閒。太史公執遷手而泣曰:「余先周室之太史也。自上世嘗顯功名於虞夏,典天官事。後世中衰,絕於予乎?汝復為太史,則續吾祖矣。今天子接千歲之統,封泰山,而余不得從行,是命也夫,命也夫!余死,汝必為太史;為太史,無忘吾所欲論著矣。且夫孝始於事親,中於事君,終於立身。揚名於後世,以顯父母,此孝之大者。夫天下稱誦周公,言其能論歌文武之德,宣周邵之風,達太王王季之思慮,爰及公劉,以尊后稷也。幽厲之後,王道缺,禮樂衰,孔子修舊起廢,論詩書,作春秋,則學者至今則之。自獲麟以來四百有餘歲,而諸侯相兼,史記放絕。今漢興,海內一統,明主賢君忠臣死義之士,余為太史而弗論載,廢天下之史文,余甚懼焉,汝其念哉!」遷俯首流涕曰:「小子不敏,請悉論先人所次舊聞,弗敢闕。」
\\\\
  卒三歲而遷為太史令,紬史記石室金匱之書。五年而當太初元年,十一月甲子朔旦冬至,天歷始改,建於明堂,諸神受紀。
\\\\
  太史公曰:「先人有言:『自周公卒五百歲而有孔子。孔子卒後至於今五百歲,有能紹明世,正易傳,繼春秋,本詩書禮樂之際?』意在斯乎!意在斯乎!小子何敢讓焉。」
\\\\
  上大夫壺遂曰:「昔孔子何為而作春秋哉?」太史公曰:「余聞董生曰:『周道衰廢,孔子為魯司寇,諸侯害之,大夫壅之。孔子知言之不用,道之不行也,是非二百四十二年之中,以為天下儀表,貶天子,退諸侯,討大夫,以達王事而已矣。』子曰:『我欲載之空言,不如見之於行事之深切著明也。』夫春秋,上明三王之道,下辨人事之紀,別嫌疑,明是非,定猶豫,善善惡惡,賢賢賤不肖,存亡國,繼絕世,補敝起廢,王道之大者也。易著天地陰陽四時五行,故長於變;禮經紀人倫,故長於行;書記先王之事,故長於政;詩記山川谿谷禽獸草木牝牡雌雄,故長於風;樂樂所以立,故長於和;春秋辯是非,故長於治人。是故禮以節人,樂以發和,書以道事,詩以達意,易以道化,春秋以道義。撥亂世反之正,莫近於春秋。春秋文成數萬,其指數千。萬物之散聚皆在春秋。春秋之中,弒君三十六,亡國五十二,諸侯奔走不得保其社稷者不可勝數。察其所以,皆失其本已。故易曰『失之豪釐,差以千里』。故曰『臣弒君,子弒父,非一旦一夕之故也,其漸久矣』。故有國者不可以不知春秋,前有讒而弗見,後有賊而不知。為人臣者不可以不知春秋,守經事而不知其宜,遭變事而不知其權。為人君父而不通於春秋之義者,必蒙首惡之名。為人臣子而不通於春秋之義者,必陷篡弒之誅,死罪之名。其實皆以為善,為之不知其義,被之空言而不敢辭。夫不通禮義之旨,至於君不君,臣不臣,父不父,子不子。夫君不君則犯,臣不臣則誅,父不父則無道,子不子則不孝。此四行者,天下之大過也。以天下之大過予之,則受而弗敢辭。故春秋者,禮義之大宗也。夫禮禁未然之前,法施已然之後;法之所為用者易見,而禮之所為禁者難知。」
\\\\
  壺遂曰:「孔子之時,上無明君,下不得任用,故作春秋,垂空文以斷禮義,當一王之法。今夫子上遇明天子,下得守職,萬事既具,咸各序其宜,夫子所論,欲以何明?」
\\\\
  太史公曰:「唯唯,否否,不然。余聞之先人曰:『伏羲至純厚,作易八卦。堯舜之盛,尚書載之,禮樂作焉。湯武之隆,詩人歌之。春秋采善貶惡,推三代之德,褒周室,非獨刺譏而已也。』漢興以來,至明天子,獲符瑞,封禪,改正朔,易服色,受命於穆清,澤流罔極,海外殊俗,重譯款塞,請來獻見者,不可勝道。臣下百官力誦聖德,猶不能宣盡其意。且士賢能而不用,有國者之恥;主上明聖而德不布聞,有司之過也。且余嘗掌其官,廢明聖盛德不載,滅功臣世家賢大夫之業不述,墮先人所言,罪莫大焉。余所謂述故事,整齊其世傳,非所謂作也,而君比之於春秋,謬矣。」
\\\\
  於是論次其文。七年而太史公遭李陵之禍,幽於縲紲。乃喟然而嘆曰:「是余之罪也夫!是余之罪也夫!身毀不用矣。」退而深惟曰:「夫詩書隱約者,欲遂其志之思也。昔西伯拘羑里,演周易;孔子戹陳蔡,作春秋;屈原放逐,著離騷;左丘失明,厥有國語;孫子臏腳,而論兵法;不韋遷蜀,世傳呂覽;韓非囚秦,說難、孤憤;詩三百篇,大抵賢聖發憤之所為作也。此人皆意有所郁結,不得通其道也,故述往事,思來者。」於是卒述陶唐以來,至于麟止,自黃帝始。
\\\\
  維昔黃帝,法天則地,四聖遵序,各成法度;唐堯遜位,虞舜不台;厥美帝功,萬世載之。作五帝本紀第一。
\\\\
  維禹之功,九州攸同,光唐虞際,德流苗裔;夏桀淫驕,乃放鳴條。作夏本紀第二。
\\\\
  維契作商,爰及成湯;太甲居桐,德盛阿衡;武丁得說,乃稱高宗;帝辛湛湎,諸侯不享。作殷本紀第三。
\\\\
  維棄作稷,德盛西伯;武王牧野,實撫天下;幽厲昏亂,既喪酆鎬;陵遲至赧;洛邑不祀。作周本紀第四。
\\\\
  維秦之先,伯翳佐禹;穆公思義,悼豪之旅;以人為殉,詩歌黃鳥;昭襄業帝。作秦本紀第五。
\\\\
  始皇既立,并兼六國,銷鋒鑄鐻,維偃干革,尊號稱帝,矜武任力;二世受運,子嬰降虜。作始皇本紀第六。
\\\\
  秦失其道,豪桀并擾;項梁業之,子羽接之;殺慶救趙,諸侯立之;誅嬰背懷,天下非之。作項羽本紀第七。
\\\\
  子羽暴虐,漢行功德;憤發蜀漢,還定三秦;誅籍業帝,天下惟寧,改制易俗。作高祖本紀第八。
\\\\
  惠之早霣,諸呂不台;崇彊祿、產,諸侯謀之;殺隱幽友,大臣洞疑,遂及宗禍。作呂太后本紀第九。
\\\\
  漢既初興,繼嗣不明,迎王踐祚,天下歸心;蠲除肉刑,開通關梁,廣恩博施,厥稱太宗。作孝文本紀第十。
\\\\
  諸侯驕恣,吳首為亂,京師行誅,七國伏辜,天下翕然,大安殷富。作孝景本紀第十一。
\\\\
  漢興五世,隆在建元,外攘夷狄,內修法度,封禪,改正朔,易服色。作今上本紀第十二。
\\\\
  維三代尚矣,年紀不可考,蓋取之譜牒舊聞,本于茲,於是略推,作三代世表第一。
\\\\
  幽厲之後,周室衰微,諸侯專政,春秋有所不紀;而譜牒經略,五霸更盛衰,欲睹周世相先後之意,作十二諸侯年表第二。
\\\\
  春秋之後,陪臣秉政,彊國相王;以至于秦,卒并諸夏,滅封地,擅其號。作六國年表第三。
\\\\
  秦既暴虐,楚人發難,項氏遂亂,漢乃扶義征伐;八年之閒,天下三嬗,事繁變眾,故詳著秦楚之際月表第四。
\\\\
  漢興已來,至于太初百年,諸侯廢立分削,譜紀不明,有司靡踵,彊弱之原云以世。作漢興已來諸侯年表第五。
\\\\
  維高祖元功,輔臣股肱,剖符而爵,澤流苗裔,忘其昭穆,或殺身隕國。作高祖功臣侯者年表第六。
\\\\
  惠景之閒,維申功臣宗屬爵邑,作惠景閒侯者年表第七。
\\\\
  北討彊胡,南誅勁越,征伐夷蠻,武功爰列。作建元以來侯者年表第八。
\\\\
  諸侯既彊,七國為從,子弟眾多,無爵封邑,推恩行義,其埶銷弱,德歸京師。作王子侯者年表第九。
\\\\
  國有賢相良將,民之師表也。維見漢興以來將相名臣年表,賢者記其治,不賢者彰其事。作漢興以來將相名臣年表第十。
\\\\
  維三代之禮,所損益各殊務,然要以近性情,通王道,故禮因人質為之節文,略協古今之變。作禮書第一。
\\\\
  樂者,所以移風易俗也。自雅頌聲興,則已好鄭衛之音,鄭衛之音所從來久矣。人情之所感,遠俗則懷。比樂書以述來古,作樂書第二。
\\\\
  非兵不彊,非德不昌,黃帝、湯、武以興,桀、紂、二世以崩,可不慎歟?司馬法所從來尚矣,太公、孫、吳、王子能紹而明之,切近世,極人變。作律書第三。
\\\\
  律居陰而治陽,歷居陽而治陰,律歷更相治,閒不容翲忽。五家之文怫異,維太初之元論。作歷書第四。
\\\\
  星氣之書,多雜禨祥,不經;推其文,考其應,不殊。比集論其行事,驗于軌度以次,作天官書第五。
\\\\
  受命而王,封禪之符罕用,用則萬靈罔不禋祀。追本諸神名山大川禮,作封禪書第六。
\\\\
  維禹浚川,九州攸寧;爰及宣防,決瀆通溝。作河渠書第七。
\\\\
  維幣之行,以通農商;其極則玩巧,并兼茲殖,爭於機利,去本趨末。作平準書以觀事變,第八。
\\\\
  太伯避歷,江蠻是適;文武攸興,古公王跡。闔廬弒僚,賓服荊楚;夫差克齊,子胥鴟夷;信嚭親越,吳國既滅。嘉伯之讓,作吳世家第一。
\\\\
  申、呂肖矣,尚父側微,卒歸西伯,文武是師;功冠群公,繆權于幽;番番黃髪,爰饗營丘。不背柯盟,桓公以昌,九合諸侯,霸功顯彰。田闞爭寵,姜姓解亡。嘉父之謀,作齊太公世家第二。
\\\\
  依之違之,周公綏之;憤發文德,天下和之;輔翼成王,諸侯宗周。隱桓之際,是獨何哉?三桓爭彊,魯乃不昌。嘉旦金縢,作周公世家第三。
\\\\
  武王克紂,天下未協而崩。成王既幼,管蔡疑之,淮夷叛之,於是召公率德,安集王室,以寧東土。燕(易)[噲]之禪,乃成禍亂。嘉甘棠之詩,作燕世家第四。
\\\\
  管蔡相武庚,將寧舊商;及旦攝政,二叔不饗;殺鮮放度,周公為盟;大任十子,周以宗彊。嘉仲悔過,作管蔡世家第五。
\\\\
  王后不絕,舜禹是說;維德休明,苗裔蒙烈。百世享祀,爰周陳杞,楚實滅之。齊田既起,舜何人哉?作陳杞世家第六。
\\\\
  收殷餘民,叔封始邑,申以商亂,酒材是告,及朔之生,衛頃不寧;南子惡蒯聵,子父易名。周德卑微,戰國既彊,衛以小弱,角獨後亡。喜彼康誥,作衛世家第七。
\\\\
  嗟箕子乎!嗟箕子乎!正言不用,乃反為奴。武庚既死,周封微子。襄公傷於泓,君子孰稱。景公謙德,熒惑退行。剔成暴虐,宋乃滅亡。喜微子問太師,作宋世家第八。
\\\\
  武王既崩,叔虞邑唐。君子譏名,卒滅武公。驪姬之愛,亂者五世;重耳不得意,乃能成霸。六卿專權,晉國以秏。嘉文公錫珪鬯,作晉世家第九。
\\\\
  重黎業之,吳回接之;殷之季世,粥子牒之。周用熊繹,熊渠是續。莊王之賢,乃復國陳;既赦鄭伯,班師華元。懷王客死,蘭咎屈原;好諛信讒,楚并於秦。嘉莊王之義,作楚世家第十。
\\\\
  少康之子,實賓南海,文身斷發,黿鱓與處,既守封禺,奉禹之祀。句踐困彼,乃用種、蠡。嘉句踐夷蠻能修其德,滅彊吳以尊周室,作越王句踐世家第十一。
\\\\
  桓公之東,太史是庸。及侵周禾,王人是議。祭仲要盟,鄭久不昌。子產之仁,紹世稱賢。三晉侵伐,鄭納於韓。嘉厲公納惠王,作鄭世家第十二。
\\\\
  維驥騄耳,乃章造父。趙夙事獻,衰續厥緒。佐文尊王,卒為晉輔。襄子困辱,乃禽智伯。主父生縛,餓死探爵。王遷辟淫,良將是斥。嘉鞅討周亂,作趙世家第十三。
\\\\
  畢萬爵魏,卜人知之。及絳戮干,戎翟和之。文侯慕義,子夏師之。惠王自矜,齊秦攻之。既疑信陵,諸侯罷之。卒亡大梁,王假廝之。嘉武佐晉文申霸道,作魏世家第十四。
\\\\
  韓厥陰德,趙武攸興。紹絕立廢,晉人宗之。昭侯顯列,申子庸之。疑非不信,秦人襲之。嘉厥輔晉匡周天子之賦,作韓世家第十五。
\\\\
  完子避難,適齊為援,陰施五世,齊人歌之。成子得政,田和為侯。王建動心,乃遷于共。嘉威、宣能撥濁世而獨宗周,作田敬仲完世家第十六。
\\\\
  周室既衰,諸侯恣行。仲尼悼禮廢樂崩,追修經術,以達王道,匡亂世反之於正,見其文辭,為天下制儀法,垂六藝之統紀於後世。作孔子世家第十七。
\\\\
  桀、紂失其道而湯、武作,周失其道而春秋作。秦失其政,而陳涉發跡,諸侯作難,風起雲蒸,卒亡秦族。天下之端,自涉發難。作陳涉世家第十八。
\\\\
  成皋之臺,薄氏始基。詘意適代,厥崇諸竇。栗姬偩貴,王氏乃遂。陳後太驕,卒尊子夫。嘉夫德若斯,作外戚世家十九。
\\\\
  漢既譎謀,禽信於陳;越荊剽輕,乃封弟交為楚王,爰都彭城,以彊淮泗,為漢宗藩。戊溺於邪,禮復紹之。嘉游輔祖,作楚元王世家二十。
\\\\
  維祖師旅,劉賈是與;為布所襲,喪其荊、吳。營陵激呂,乃王瑯邪;怵午信齊,往而不歸,遂西入關,遭立孝文,獲復王燕。天下未集,賈、澤以族,為漢藩輔。作荊燕世家第二十一。
\\\\
  天下已平,親屬既寡;悼惠先壯,實鎮東土。哀王擅興,發怒諸呂,駟鈞暴戾,京師弗許。厲之內淫,禍成主父。嘉肥股肱,作齊悼惠王世家第二十二。
\\\\
  楚人圍我滎陽,相守三年;蕭何填撫山西,推計踵兵,給糧食不絕,使百姓愛漢,不樂為楚。作蕭相國世家第二十三。
\\\\
  與信定魏,破趙拔齊,遂弱楚人。續何相國,不變不革,黎庶攸寧。嘉參不伐功矜能,作曹相國世家第二十四。
\\\\
  運籌帷幄之中,制勝於無形,子房計謀其事,無知名,無勇功,圖難於易,為大於細。作留侯世家第二十五。
\\\\
  六奇既用,諸侯賓從於漢;呂氏之事,平為本謀,終安宗廟,定社稷。作陳丞相世家第二十六。
\\\\
  諸呂為從,謀弱京師,而勃反經合於權;吳楚之兵,亞夫駐於昌邑,以戹齊趙,而出委以梁。作絳侯世家第二十七。
\\\\
  七國叛逆,蕃屏京師,唯梁為捍;偩愛矜功,幾獲于禍。嘉其能距吳楚,作梁孝王世家第二十八。
\\\\
  五宗既王,親屬洽和,諸侯大小為藩,爰得其宜,僭擬之事稍衰貶矣。作五宗世家第二十九。
\\\\
  三子之王,文辭可觀。作三王世家第三十。
\\\\
  末世爭利,維彼奔義;讓國餓死,天下稱之。作伯夷列傳第一。
\\\\
  晏子儉矣,夷吾則奢;齊桓以霸,景公以治。作管晏列傳第二。
\\\\
  李耳無為自化,清凈自正;韓非揣事情,循埶理。作老子韓非列傳第三。
\\\\
  自古王者而有司馬法,穰苴能申明之。作司馬穰苴列傳第四。
\\\\
  非信廉仁勇不能傳兵論劍,與道同符,內可以治身,外可以應變,君子比德焉。作孫子吳起列傳第五。
\\\\
  維建遇讒,爰及子奢,尚既匡父,伍員奔吳。作伍子胥列傳第六。
\\\\
  孔氏述文,弟子興業,咸為師傅,崇仁厲義。作仲尼弟子列傳第七。
\\\\
  鞅去衛適秦,能明其術,彊霸孝公,後世遵其法。作商君列傳第八。
\\\\
  天下患衡秦毋饜,而蘇子能存諸侯,約從以抑貪彊。作蘇秦列傳第九。
\\\\
  六國既從親,而張儀能明其說,復散解諸侯。作張儀列傳第十。
\\\\
  秦所以東攘雄諸侯,樗裏、甘茂之策。作樗裏甘茂列傳第十一。
\\\\
  苞河山,圍大梁,使諸侯斂手而事秦者,魏冉之功。作穰侯列傳第十二。
\\\\
  南拔鄢郢,北摧長平,遂圍邯鄲,武安為率;破荊滅趙,王翦之計。作白起王翦列傳第十三。
\\\\
  獵儒墨之遺文,明禮義之統紀,絕惠王利端,列往世興衰。作孟子荀卿列傳第十四。
\\\\
  好客喜士,士歸于薛,為齊捍楚魏。作孟嘗君列傳第十五。
\\\\
  爭馮亭以權,如楚以救邯鄲之圍,使其君復稱於諸侯。作平原君虞卿列傳第十六。
\\\\
  能以富貴下貧賤,賢能詘於不肖,唯信陵君為能行之。作魏公子列傳第十七。
\\\\
  以身徇君,遂脫彊秦,使馳說之士南鄉走楚者,黃歇之義。作春申君列傳第十八
\\\\
  能忍於魏齊,而信威於彊秦,推賢讓位,二子有之。作范睢蔡澤列傳第十九。
\\\\
  率行其謀,連五國兵,為弱燕報彊齊之讎,雪其先君之恥。作樂毅列傳第二十。
\\\\
  能信意彊秦,而屈體廉子,用徇其君,俱重於諸侯。作廉頗藺相如列傳第二十一。
\\\\
  湣王既失臨淄而奔莒,唯田單用即墨破走騎劫,遂存齊社稷。作田單列傳第二十二。
\\\\
  能設詭說解患於圍城,輕爵祿,樂肆志。作魯仲連鄒陽列傳第二十三。
\\\\
  作辭以諷諫,連類以爭義,離騷有之。作屈原賈生列傳第二十四。
\\\\
  結子楚親,使諸侯之士斐然爭入事秦。作呂不韋列傳第二十五。
\\\\
  曹子匕首,魯獲其田,齊明其信;豫讓義不為二心。作刺客列傳第二十六。
\\\\
  能明其畫,因時推秦,遂得意於海內,斯為謀首。作李斯列傳第二十七。
\\\\
  為秦開地益眾,北靡匈奴,據河為塞,因山為固,建榆中。作蒙恬列傳第二十八。
\\\\
  填趙塞常山以廣河內,弱楚權,明漢王之信於天下。作張耳陳餘列傳第二十九。
\\\\
  收西河、上黨之兵,從至彭城;越之侵掠梁地以苦項羽。作魏豹彭越列傳第三十。
\\\\
  以淮南叛楚歸漢,漢用得大司馬殷,卒破子羽于垓下。作黥布列傳第三十一。
\\\\
  楚人迫我京索,而信拔魏趙,定燕齊,使漢三分天下有其二,以滅項籍。作淮陰侯列傳第三十二。
\\\\
  楚漢相距鞏洛,而韓信為填潁川,盧綰絕籍糧餉。作韓信盧綰列傳第三十三。
\\\\
  諸侯畔項王,唯齊連子羽城陽,漢得以閒遂入彭城。作田儋列傳第三十四。
\\\\
  攻城野戰,獲功歸報,噲、商有力焉,非獨鞭策,又與之脫難。作樊酈列傳第三十五。
\\\\
  漢既初定,文理未明,蒼為主計,整齊度量,序律歷。作張丞相列傳第三十六。
\\\\
  結言通使,約懷諸侯;諸侯咸親,歸漢為藩輔。作酈生陸賈列傳第三十七。
\\\\
  欲詳知秦楚之事,維周∴從高祖,平定諸侯。作傅靳蒯成列傳第三十八。
\\\\
  徙彊族,都關中,和約匈奴;明朝廷禮,次宗廟儀法。作劉敬叔孫通列傳第三十九。
\\\\
  能摧剛作柔,卒為列臣;欒公不劫於埶而倍死。作季布欒布列傳第四十。
\\\\
  敢犯顏色以達主義,不顧其身,為國家樹長畫。作袁盎晁錯列傳第四十一。
\\\\
  守法不失大理,言古賢人,增主之明。作張釋之馮唐列傳第四十二。
\\\\
  敦厚慈孝,訥於言,敏於行,務在鞠躬,君子長者。作萬石張叔列傳第四十三。
\\\\
  守節切直,義足以言廉,行足以厲賢,任重權不可以非理撓。作田叔列傳第四十四。
\\\\
  扁鵲言醫,為方者宗,守數精明;後世(修)[循]序,弗能易也,而倉公可謂近之矣。作扁鵲倉公列傳第四十五。
\\\\
  維仲之省,厥濞王吳,遭漢初定,以填撫江淮之閒。作吳王濞列傳第四十六。
\\\\
  吳楚為亂,宗屬唯嬰賢而喜士,士鄉之,率師抗山東滎陽。作魏其武安列傳第四十七。
\\\\
  智足以應近世之變,寬足用得人。作韓長孺列傳第四十八。
\\\\
  勇於當敵,仁愛士卒,號令不煩,師徒鄉之。作李將軍列傳第四十九。
\\\\
  自三代以來,匈奴常為中國患害;欲知彊弱之時,設備征討,作匈奴列傳第五十。
\\\\
  直曲塞,廣河南,破祁連,通西國,靡北胡。作衛將軍驃騎列傳第五十一。
\\\\
  大臣宗室以侈靡相高,唯弘用節衣食為百吏先。作平津侯列傳第五十二。
\\\\
  漢既平中國,而佗能集楊越以保南藩,納貢職。作南越列傳第五十三。
\\\\
  吳之叛逆,甌人斬濞,葆守封禺為臣。作東越列傳第五十四。
\\\\
  燕丹散亂遼閒,滿收其亡民,厥聚海東,以集真藩,葆塞為外臣。作朝鮮列傳第五十五。
\\\\
  唐蒙使略通夜郎,而邛笮之君請為內臣受吏。作西南夷列傳第五十六。
\\\\
  子虛之事,大人賦說,靡麗多誇,然其指風諫,歸於無為。作司馬相如列傳第五十七。
\\\\
  黥布叛逆,子長國之,以填江淮之南,安剽楚庶民。作淮南衡山列傳第五十八。
\\\\
  奉法循理之吏,不伐功矜能,百姓無稱,亦無過行。作循吏列傳第五十九。
\\\\
  正衣冠立於朝廷,而群臣莫敢言浮說,長孺矜焉;好薦人,稱長者,壯有溉。作汲鄭列傳第六十。
\\\\
  自孔子卒,京師莫崇庠序,唯建元元狩之閒,文辭粲如也。作儒林列傳第六十一。
\\\\
  民倍本多巧,姦軌弄法,善人不能化,唯一切嚴削為能齊之。作酷吏列傳第六十二。
\\\\
  漢既通使大夏,而西極遠蠻,引領內鄉,欲觀中國。作大宛列傳第六十三。
\\\\
  救人於緦振人不贍,仁者有乎;不既信,不倍言,義者有取焉。作游俠列傳第六十四。
\\\\
  夫事人君能說主耳目,和主顏色,而獲親近,非獨色愛,能亦各有所長。作佞幸列傳第六十五。
\\\\
  不流世俗,不爭埶利,上下無所凝滯,人莫之害,以道之用。作滑稽列傳第六十六。
\\\\
  齊、楚、秦、趙為日者,各有俗所用。欲循觀其大旨,作日者列傳第六十七。
\\\\
  三王不同龜,四夷各異卜,然各以決吉凶。略闚其要,作龜策列傳第六十八。
\\\\
  布衣匹夫之人,不害於政,不妨百姓,取與以時而息財富,智者有采焉。作貨殖列傳第六十九。
\\\\
  維我漢繼五帝末流,接三代(統)[絕]業。周道廢,秦撥去古文,焚滅詩書,故明堂石室金匱玉版圖籍散亂。於是漢興,蕭何次律令,韓信申軍法,張蒼為章程,叔孫通定禮儀,則文學彬彬稍進,詩書往往閒出矣。自曹參薦蓋公言黃老,而賈生、晁錯明申、商,公孫弘以儒顯,百年之閒,天下遺文古事靡不畢集太史公。太史公仍父子相續纂其職。曰:「於戲!余維先人嘗掌斯事,顯於唐虞,至于周,復典之,故司馬氏世主天官。至於余乎,欽念哉!欽念哉!」罔羅天下放失舊聞,王跡所興,原始察終,見盛觀衰,論考之行事,略推三代,錄秦漢,上記軒轅,下至于茲,著十二本紀,既科條之矣。并時異世,年差不明,作十表。禮樂損益,律歷改易,兵權山川鬼神,天人之際,承敝通變,作八書。二十八宿環北辰,三十輻共一轂,運行無窮,輔拂股肱之臣配焉,忠信行道,以奉主上,作三十世家。扶義俶儻,不令己失時,立功名於天下,作七十列傳。凡百三十篇,五十二萬六千五百字,為太史公書。序略,以拾遺補闕,成一家之言,厥協六經異傳,整齊百家雜語,藏之名山,副在京師,俟後世聖人君子。第七十。
\\\\
  太史公曰:余述歷黃帝以來至太初而訖,百三十篇。

\end{document}
