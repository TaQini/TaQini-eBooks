
\section{老子韓非列傳}
  老子者,楚苦縣厲鄉曲仁里人也,姓李氏,名耳,字耼,周守藏室之史也。
\\\\
  孔子適周,將問禮於老子。老子曰:「子所言者,其人與骨皆已朽矣,獨其言在耳。且君子得其時則駕,不得其時則蓬累而行。吾聞之,良賈深藏若虛,君子盛德容貌若愚。去子之驕氣與多欲,態色與淫志,是皆無益於子之身。吾所以告子,若是而已。」孔子去,謂弟子曰:「鳥,吾知其能飛;魚,吾知其能游;獸,吾知其能走。走者可以為罔,游者可以為綸,飛者可以為矰。至於龍,吾不能知其乘風雲而上天。吾今日見老子,其猶龍邪!」
\\\\
  老子修道德,其學以自隱無名為務。居周久之,見周之衰,乃遂去。至關,關令尹喜曰:「子將隱矣,彊為我著書。」於是老子乃著書上下篇,言道德之意五千餘言而去,莫知其所終。
\\\\
  或曰:老萊子亦楚人也,著書十五篇,言道家之用,與孔子同時云。
\\\\
  蓋老子百有六十餘歲,或言二百餘歲,以其修道而養壽也。
\\\\
  自孔子死之後百二十九年,而史記周太史儋見秦獻公曰:「始秦與周合,合五百歲而離,離七十歲而霸王者出焉。」或曰儋即老子,或曰非也,世莫知其然否。老子,隱君子也。
\\\\
  老子之子名宗,宗為魏將,封於段干。宗子注,注子宮,宮玄孫假,假仕於漢孝文帝。而假之子解為膠西王卬太傅,因家于齊焉。
\\\\
  世之學老子者則絀儒學,儒學亦絀老子。「道不同不相為謀」,豈謂是邪?李耳無為自化,清靜自正。
\\\\
  莊子者,蒙人也,名周。周嘗為蒙漆園吏,與梁惠王、齊宣王同時。其學無所不闚,然其要本歸於老子之言。故其著書十餘萬言,大抵率寓言也。作漁父、盜跖、胠篋,以詆訿孔子之徒,以明老子之術。畏累虛、亢桑子之屬,皆空語無事實。然善屬書離辭,指事類情,用剽剝儒、墨,雖當世宿學不能自解免也。其言洸洋自恣以適己,故自王公大人不能器之。
\\\\
  楚威王聞莊周賢,使使厚幣迎之,許以為相。莊周笑謂楚使者曰:「千金,重利;卿相,尊位也。子獨不見郊祭之犧牛乎?養食之數歲,衣以文繡,以入大廟。當是之時,雖欲為孤豚,豈可得乎?子亟去,無污我。我寧游戲污瀆之中自快,無為有國者所羈,終身不仕,以快吾志焉。」
\\\\
  申不害者,京人也,故鄭之賤臣。學術以干韓昭侯,昭侯用為相。內修政教,外應諸侯,十五年。終申子之身,國治兵彊,無侵韓者。申子之學本於黃老而主刑名。著書二篇,號曰申子。
\\\\
  韓非者,韓之諸公子也。喜刑名法術之學,而其歸本於黃老。非為人口吃,不能道說,而善著書。與李斯俱事荀卿,斯自以為不如非。
\\\\
  非見韓之削弱,數以書諫韓王,韓王不能用。於是韓非疾治國不務修明其法制,執勢以御其臣下,富國彊兵而以求人任賢,反舉浮淫之蠹而加之於功實之上。以為儒者用文亂法,而俠者以武犯禁。寬則寵名譽之人,急則用介胄之士。今者所養非所用,所用非所養。悲廉直不容於邪枉之臣,觀往者得失之變,故作孤憤、五蠹、內外儲、說林、說難十餘萬言。
\\\\
  然韓非知說之難,為說難書甚具,終死於秦,不能自脫。
\\\\
  說難曰:
\\\\
  凡說之難,非吾知之有以說之難也;又非吾辯之難能明吾意之難也;又非吾敢橫失能盡之難也。凡說之難,在知所說之心,可以吾說當之。
\\\\
  所說出於為名高者也,而說之以厚利,則見下節而遇卑賤,必棄遠矣。所說出於厚利者也。而說之以名高,則見無心而遠事情,必不收矣。所說實為厚利而顯為名高者也,而說之以名高,則陽收其身而實疏之;若說之以厚利,則陰用其言而顯棄其身。此之不可不知也。
\\\\
  夫事以密成,語以泄敗。未必其身泄之也,而語及其所匿之事,如是者身危。貴人有過端,而說者明言善議以推其惡者,則身危。周澤未渥也而語極知,說行而有功則德亡,說不行而有敗則見疑,如是者身危。夫貴人得計而欲自以為功,說者與知焉,則身危。彼顯有所出事,迺自以為也故,說者與知焉,則身危。彊之以其所必不為,止之以其所不能已者,身危。故曰:與之論大人,則以為閒己;與之論細人,則以為粥權。論其所愛,則以為借資;論其所憎,則以為嘗己。徑省其辭,則不知而屈之;汎濫博文,則多而久之。順事陳意,則曰怯懦而不盡;慮事廣肆,則曰草野而倨侮。此說之難,不可不知也。
\\\\
  凡說之務,在知飾所說之所敬,而滅其所醜。彼自知其計,則毋以其失窮之;自勇其斷,則毋以其敵怒之;自多其力,則毋以其難概之。規異事與同計,譽異人與同行者,則以飾之無傷也。有與同失者,則明飾其無失也。大忠無所拂悟,辭言無所擊排,迺後申其辯知焉。此所以親近不疑,知盡之難也。得曠日彌久,而周澤既渥,深計而不疑,交爭而不罪,迺明計利害以致其功,直指是非以飾其身,以此相持,此說之成也。
\\\\
  伊尹為庖,百里奚為虜,皆所由干其上也。故此二子者,皆聖人也,猶不能無役身而涉世如此其汙也,則非能仕之所設也。
\\\\
  宋有富人,天雨牆壞。其子曰「不築且有盜」,其鄰人之父亦云,暮而果大亡其財,其家甚知其子而疑鄰人之父。昔者鄭武公欲伐胡,迺以其子妻之。因問群臣曰: 「吾欲用兵,誰可伐者?」關其思曰:「胡可伐。」迺戮關其思,曰:「胡,兄弟之國也,子言伐之,何也?」胡君聞之,以鄭為親己而不備鄭。鄭人襲胡,取之。此二說者,其知皆當矣,然而甚者為戮,薄者見疑。非知之難也,處知則難矣。
\\\\
  昔者彌子瑕見愛於衛君。衛國之法,竊駕君車者罪至刖。既而彌子之母病,人聞,往夜告之,彌子矯駕君車而出。君聞之而賢之曰:「孝哉,為母之故而犯刖罪!」與君游果園,彌子食桃而甘,不盡而奉君。君曰:「愛我哉,忘其口而念我!」及彌子色衰而愛弛,得罪於君。君曰:「是嘗矯駕吾車,又嘗食我以其餘桃。」故彌子之行未變於初也,前見賢而後獲罪者,愛憎之至變也。故有愛於主,則知當而加親;見憎於主,則罪當而加疏。故諫說之士不可不察愛憎之主而後說之矣。夫龍之為蟲也,可擾狎而騎也。然其喉下有逆鱗徑尺,人有嬰之,則必殺人。人主亦有逆鱗,說之者能無嬰人主之逆鱗,則幾矣。
\\\\
  人或傳其書至秦。秦王見孤憤、五蠹之書,曰:「嗟乎,寡人得見此人與之游,死不恨矣!」李斯曰:「此韓非之所著書也。」秦因急攻韓。韓王始不用非,及急,乃遣非使秦。秦王悅之,未信用。李斯、姚賈害之,毀之曰:「韓非,韓之諸公子也。今王欲并諸侯,非終為韓不為秦,此人之情也。今王不用,久留而歸之,此自遺患也,不如以過法誅之。」秦王以為然,下吏治非。李斯使人遺非藥,使自殺。韓非欲自陳,不得見。秦王後悔之,使人赦之,非已死矣。申子、韓子皆著書,傳於後世,學者多有。余獨悲韓子為說難而不能自脫耳。
\\\\
  太史公曰:老子所貴道,虛無,因應變化於無為,故著書辭稱微妙難識。莊子散道德,放論,要亦歸之自然。申子卑卑,施之於名實。韓子引繩墨,切事情,明是非,其極慘礉少恩。皆原於道德之意,而老子深遠矣。

\section{滑稽列傳}
  孔子曰:「六藝於治一也。禮以節人,樂以發和,書以道事,詩以達意,易以神化,春秋以義。」太史公曰:天道恢恢,豈不大哉!談言微中,亦可以解紛。
\\\\
  淳于髡者,齊之贅婿也。長不滿七尺,滑稽多辯,數使諸侯,未嘗屈辱。齊威王之時喜隱,好為淫樂長夜之飲,沈湎不治,委政卿大夫。百官荒亂,諸侯并侵,國且危亡,在於旦暮,左右莫敢諫。淳于髡說之以隱曰:「國中有大鳥,止王之庭,三年不蜚又不鳴,不知此鳥何也?」王曰:「此鳥不飛則已,一飛沖天;不鳴則已,一鳴驚人。」於是乃朝諸縣令長七十二人,賞一人,誅一人,奮兵而出。諸侯振驚,皆還齊侵地。威行三十六年。語在田完世家中。
\\\\
  威王八年,楚大發兵加齊。齊王使淳于髡之趙請救兵,齎金百斤,車馬十駟。淳于髡仰天大笑,冠纓索絕。王曰:「先生少之乎?」髡曰:「何敢!」王曰:「笑豈有說乎?」髡曰:「今者臣從東方來,見道傍有禳田者,操一豚蹄,酒一盂,祝曰:『甌窶滿篝,汙邪滿車,五穀蕃熟,穰穰滿家。』臣見其所持者狹而所欲者奢,故笑之。」於是齊威王乃益齎黃金千溢,白璧十雙,車馬百駟。髡辭而行,至趙。趙王與之精兵十萬,革車千乘。楚聞之,夜引兵而去。
\\\\
  威王大說,置酒後宮,召髡賜之酒。問曰:「先生能飲幾何而醉?」對曰:「臣飲一斗亦醉,一石亦醉。」威王曰:「先生飲一斗而醉,惡能飲一石哉!其說可得聞乎?」髡曰:「賜酒大王之前,執法在傍,御史在後,髡恐懼俯伏而飲,不過一斗徑醉矣。若親有嚴客,髡帣韝鞠跽,待酒於前,時賜餘瀝,奉觴上壽,數起,飲不過二斗徑醉矣。若朋友交遊,久不相見,卒然相睹,歡然道故,私情相語,飲可五六斗徑醉矣。若乃州閭之會,男女雜坐,行酒稽留,六博投壺,相引為曹,握手無罰,目眙不禁,前有墮珥,后有遺簪,髡竊樂此,飲可八斗而醉二參。日暮酒闌,合尊促坐,男女同席,履舄交錯,杯盤狼藉,堂上燭滅,主人留髡而送客,羅襦襟解,微聞薌澤,當此之時,髡心最歡,能飲一石。故曰酒極則亂,樂極則悲;萬事盡然,言不可極,極之而衰。」以諷諫焉。齊王曰:「善。」乃罷長夜之飲,以髡為諸侯主客。宗室置酒,髡嘗在側。
\\\\
  其後百餘年,楚有優孟。
\\\\
  優孟,故楚之樂人也。長八尺,多辯,常以談笑諷諫。楚莊王之時,有所愛馬,衣以文繡,置之華屋之下,席以露床,啗以棗脯。馬病肥死,使群臣喪之,欲以棺槨大夫禮葬之。左右爭之,以為不可。王下令曰:「有敢以馬諫者,罪至死。」優孟聞之,入殿門。仰天大哭。王驚而問其故。優孟曰:「馬者王之所愛也,以楚國堂堂之大,何求不得,而以大夫禮葬之,薄,請以人君禮葬之。」王曰:「何如?」對曰:「臣請以彫玉為棺,文梓為槨,楩楓豫章為題湊,發甲卒為穿壙,老弱負土,齊趙陪位於前,韓魏翼衛其后,廟食太牢,奉以萬戶之邑。諸侯聞之,皆知大王賤人而貴馬也。」王曰:「寡人之過一至此乎!為之柰何?」優孟曰:「請為大王六畜葬之。以壟灶為槨,銅歷為棺,齎以薑棗,薦以木蘭,祭以糧稻,衣以火光,葬之於人腹腸。」於是王乃使以馬屬太官,無令天下久聞也。
\\\\
  楚相孫叔敖知其賢人也,善待之。病且死,屬其子曰:「我死,汝必貧困。若往見優孟,言我孫叔敖之子也。」居數年,其子窮困負薪,逢優孟,與言曰:「我,孫叔敖子也。父且死時,屬我貧困往見優孟。」優孟曰:「若無遠有所之。」即為孫叔敖衣冠,抵掌談語。歲餘,像孫叔敖,楚王及左右不能別也。莊王置酒,優孟前為壽。莊王大驚,以為孫叔敖復生也,欲以為相。優孟曰:「請歸與婦計之,三日而為相。」莊王許之。三日後,優孟復來。王曰:「婦言謂何?」孟曰:「婦言慎無為,楚相不足為也。如孫叔敖之為楚相,盡忠為廉以治楚,楚王得以霸。今死,其子無立錐之地,貧困負薪以自飲食。必如孫叔敖,不如自殺。」因歌曰:「山居耕田苦,難以得食。起而為吏,身貪鄙者餘財,不顧恥辱。身死家室富,又恐受賕枉法,為姦觸大罪,身死而家滅。貪吏安可為也!念為廉吏,奉法守職,竟死不敢為非。廉吏安可為也!楚相孫叔敖持廉至死,方今妻子窮困負薪而食,不足為也!」於是莊王謝優孟,乃召孫叔敖子,封之寢丘四百戶,以奉其祀。后十世不絕。此知可以言時矣。
\\\\
  其後二百餘年,秦有優旃。
\\\\
  優旃者,秦倡侏儒也。善為笑言,然合於大道,秦始皇時,置酒而天雨,陛楯者皆沾寒。優旃見而哀之,謂之曰:「汝欲休乎?」陛楯者皆曰:「幸甚。」優旃曰:「我即呼汝,汝疾應曰諾。」居有頃,殿上上壽呼萬歲。優旃臨檻大呼曰:「陛楯郎!」郎曰:「諾。」優旃曰:「汝雖長,何益,幸雨立。我雖短也,幸休居。」於是始皇使陛楯者得半相代。
\\\\
  始皇嘗議欲大苑囿,東至函谷關,西至雍、陳倉。優旃曰:「善。多縱禽獸於其中,寇從東方來,令麋鹿觸之足矣。」始皇以故輟止。
\\\\
  二世立,又欲漆其城。優旃曰:「善。主上雖無言,臣固將請之。漆城雖於百姓愁費,然佳哉!漆城蕩蕩,寇來不能上。即欲就之,易為漆耳,顧難為蔭室。」於是二世笑之,以其故止。居無何,二世殺死,優旃歸漢,數年而卒。
\\\\
  太史公曰:淳于髡仰天大笑,齊威王橫行。優孟搖頭而歌,負薪者以封。優旃臨檻疾呼,陛楯得以半更。豈不亦偉哉!
\\\\
  褚先生曰:臣幸得以經術為郎,而好讀外家傳語。竊不遜讓,復作故事滑稽之語六章,編之於左。可以覽觀揚意,以示後世好事者讀之,以游心駭耳,以附益上方太史公之三章。
\\\\
  武帝時有所幸倡郭舍人者,發言陳辭雖不合大道,然令人主和說。武帝少時,東武侯母常養帝,帝壯時,號之曰「大乳母」。率一月再朝。朝奏入,有詔使幸臣馬游卿以帛五十匹賜乳母,又奉飲糒飱養乳母。乳母上書曰:「某所有公田,願得假倩之。」帝曰:「乳母欲得之乎?」以賜乳母。乳母所言,未嘗不聽。有詔得令乳母乘車行馳道中。當此之時,公卿大臣皆敬重乳母。乳母家子孫奴從者橫暴長安中,當道掣頓人車馬,奪人衣服。聞於中,不忍致之法。有司請徙乳母家室,處之於邊。奏可。乳母當入至前,面見辭。乳母先見郭舍人,為下泣。舍人曰:「即入見辭去,疾步數還顧。」乳母如其言,謝去,疾步數還顧。郭舍人疾言罵之曰:「咄!老女子!何不疾行!陛下已壯矣,寧尚須汝乳而活邪?尚何還顧!」於是人主憐焉悲之,乃下詔止無徙乳母,罰謫譖之者。
\\\\
  武帝時,齊人有東方生名朔,以好古傳書,愛經術,多所博觀外家之語。朔初入長安,至公車上書,凡用三千奏牘。公車令兩人共持舉其書,僅然能勝之。人主從上方讀之,止,輒乙其處,讀之二月乃盡。詔拜以為郎,常在側侍中。數召至前談語,人主未嘗不說也。時詔賜之食於前。飯已,盡懷其餘肉持去,衣盡汙。數賜縑帛,檐揭而去。徒用所賜錢帛,取少婦於長安中好女。率取婦一歲所者即棄去,更取婦。所賜錢財盡索之於女子。人主左右諸郎半呼之「狂人」。人主聞之,曰:「令朔在事無為是行者,若等安能及之哉!」朔任其子為郎,又為侍謁者,常持節出使。朔行殿中,郎謂之曰:「人皆以先生為狂。」朔曰:「如朔等,所謂避世於朝廷閒者也。古之人,乃避世於深山中。」時坐席中,酒酣,據地歌曰:「陸沈於俗,避世金馬門。宮殿中可以避世全身,何必深山之中,蒿廬之下。」金馬門者,宦[者]署門也,門傍有銅馬,故謂之曰「金馬門」。
\\\\
  時會聚宮下博士諸先生與論議,共難之曰:「蘇秦、張儀一當萬乘之主,而都卿相之位,澤及後世。今子大夫修先王之術,慕聖人之義,諷誦詩書百家之言,不可勝數。著於竹帛,自以為海內無雙,即可謂博聞辯智矣。然悉力盡忠以事聖帝,曠日持久,積數十年,官不過侍郎,位不過執戟,意者尚有遺行邪?其故何也?」東方生曰:「是固非子所能備也。彼一時也,此一時也,豈可同哉!夫張儀、蘇秦之時,周室大壞,諸侯不朝,力政爭權,相禽以兵,并為十二國,未有雌雄,得士者彊,失士者亡,故說聽行通,身處尊位,澤及後世,子孫長榮。今非然也。聖帝在上,德流天下,諸侯賓服,威振四夷,連四海之外以為席,安於覆盂,天下平均,合為一家,動發舉事,猶如運之掌中。賢與不肖,何以異哉?方今以天下之大,士民之眾,竭精馳說,并進輻湊者,不可勝數。悉力慕義,困於衣食,或失門戶。使張儀、蘇秦與仆并生於今之世,曾不能得掌故,安敢望常侍侍郎乎!傳曰:『天下無害菑,雖有聖人,無所施其才;上下和同,雖有賢者,無所立功。』故曰時異則事異。雖然,安可以不務修身乎?《詩》曰:『鼓鐘于宮,聲聞于外。鶴鳴九皋,聲聞于天。』。茍能修身,何患不榮!太公躬行仁義七十二年,逢文王,得行其說,封於齊,七百歲而不絕。此士之所以日夜孜孜,修學行道,不敢止也。今世之處士,時雖不用,崛然獨立,塊然獨處,上觀許由,下察接輿,策同范蠡,忠合子胥,天下和平,與義相扶,寡偶少徒,固其常也。子何疑於余哉!」於是諸先生默然無以應也。
\\\\
  建章宮後閤重櫟中有物出焉,其狀似麋。以聞,武帝往臨視之。問左右群臣習事通經術者,莫能知。詔東方朔視之。朔曰:「臣知之,願賜美酒粱飯大飱臣,臣乃言。」詔曰:「可。」已又曰:「某所有公田魚池蒲葦數頃,陛下以賜臣,臣朔乃言。」詔曰:「可。」於是朔乃肯言,曰:「所謂騶牙者也。遠方當來歸義,而騶牙先見。其齒前后若一,齊等無牙,故謂之騶牙。」其後一歲所,匈奴混邪王果將十萬眾來降漢。乃復賜東方生錢財甚多。
\\\\
  至老,朔且死時,諫曰:「《詩》云『營營青蠅,止于蕃。愷悌君子,無信讒言。讒言罔極,交亂四國』。願陛下遠巧佞,退讒言。」帝曰:「今顧東方朔多善言?」怪之。居無幾何,朔果病死。傳曰:「鳥之將死,其鳴也哀;人之將死,其言也善。」此之謂也。
\\\\
  武帝時,大將軍衛青者,衛后兄也,封為長平侯。從軍擊匈奴,至余吾水上而還,斬首捕虜,有功來歸,詔賜金千斤。將軍出宮門,齊人東郭先生以方士待詔公車,當道遮衛將軍車,拜謁曰:「願白事。」將軍止車前,東郭先生旁車言曰:「王夫人新得幸於上,家貧。今將軍得金千斤,誠以其半賜王夫人之親,人主聞之必喜。此所謂奇策便計也。」衛將軍謝之曰:「先生幸告之以便計,請奉教。」於是衛將軍乃以五百金為王夫人之親壽。王夫人以聞武帝。帝曰:「大將軍不知為此。」問之安所受計策,對曰:「受之待詔者東郭先生。」詔召東郭先生,拜以為郡都尉。東郭先生久待詔公車,貧困饑寒,衣敝,履不完。行雪中,履有上無下,足盡踐地。道中人笑之,東郭先生應之曰:「誰能履行雪中,令人視之,其上履也,其履下處乃似人足者乎?」及其拜為二千石,佩青緺出宮門,行謝主人。故所以同官待詔者,等比祖道於都門外。榮華道路,立名當世。此所謂衣褐懷寶者也。當其貧困時,人莫省視;至其貴也,乃爭附之。諺曰:「相馬失之瘦,相士失之貧。」其此之謂邪?
\\\\
  王夫人病甚,人主至自往問之曰:「子當為王,欲安所置之?」對曰:「願居洛陽。」人主曰:「不可。洛陽有武庫、敖倉,當關口,天下咽喉。自先帝以來,傳不為置王。然關東國莫大於齊,可以為齊王。」王夫人以手擊頭,呼「幸甚」。王夫人死,號曰「齊王太后薨」。
\\\\
  昔者,齊王使淳于髡獻鵠於楚。出邑門,道飛其鵠,徒揭空籠,造詐成辭,往見楚王曰:「齊王使臣來獻鵠,過於水上,不忍鵠之渴,出而飲之,去我飛亡。吾欲刺腹絞頸而死。恐人之議吾王以鳥獸之故令士自傷殺也。鵠,毛物,多相類者,吾欲買而代之,是不信而欺吾王也。欲赴佗國奔亡,痛吾兩主使不通。故來服過,叩頭受罪大王。」楚王曰:「善,齊王有信士若此哉!」厚賜之,財倍鵠在也。
\\\\
  武帝時,徵北海太守詣行在所。有文學卒史王先生者,自請與太守俱,「吾有益於君」,君許之。諸府掾功曹白云:「王先生嗜酒,多言少實,恐不可與俱。」太守曰:「先生意欲行,不可逆。」遂與俱。行至宮下,待詔宮府門。王先生徒懷錢沽酒,與衛卒仆射飲,日醉,不視其太守。太守入跪拜。王先生謂戶郎曰:「幸為我呼吾君至門內遙語。」戶郎為呼太守。太守來,望見王先生。王先生曰:「天子即問君何以治北海令無盜賊,君對曰何哉?」對曰:「選擇賢材,各任之以其能,賞異等,罰不肖。」王先生曰:「對如是,是自譽自伐功,不可也。願君對言,非臣之力,盡陛下神靈威武所變化也。」太守曰:「諾。」召入,至于殿下,有詔問之曰:「何於治北海,令盜賊不起?」叩頭對言:「非臣之力,盡陛下神靈威武之所變化也。」武帝大笑,曰:「於呼!安得長者之語而稱之!安所受之?」對曰:「受之文學卒史。」帝曰:「今安在?」對曰:「在宮府門外。」有詔召拜王先生為水衡丞,以北海太守為水衡都尉。傳曰:「美言可以市,尊行可以加人。君子相送以言,小人相送以財。」
\\\\
  魏文侯時,西門豹為鄴令。豹往到鄴,會長老,問之民所疾苦。長老曰:「苦為河伯娶婦,以故貧。」豹問其故,對曰:「鄴三老、廷掾常歲賦斂百姓,收取其錢得數百萬,用其二三十萬為河伯娶婦,與祝巫共分其餘錢持歸。當其時,巫行視小家女好者,云是當為河伯婦,即娉取。洗沐之,為治新繒綺縠衣,閒居齋戒;為治齋宮河上,張緹絳帷,女居其中。為具牛酒飯食,行十餘日。共粉飾之,如嫁女床席,令女居其上,浮之河中。始浮,行數十里乃沒。其人家有好女者,恐大巫祝為河伯取之,以故多持女遠逃亡。以故城中益空無人,又困貧,所從來久遠矣。民人俗語曰『即不為河伯娶婦,水來漂沒,溺其人民』云。」西門豹曰:「至為河伯娶婦時,願三老、巫祝、父老送女河上,幸來告語之,吾亦往送女。」皆曰:「諾。」
\\\\
  至其時,西門豹往會之河上。三老、官屬、豪長者、裏父老皆會,以人民往觀之者三二千人。其巫,老女子也,已年七十。從弟子女十人所,皆衣繒單衣,立大巫后。西門豹曰:「呼河伯婦來,視其好醜。」即將女出帷中,來至前。豹視之,顧謂三老、巫祝、父老曰:「是女子不好,煩大巫嫗為入報河伯,得更求好女,后日送之。」即使吏卒共抱大巫嫗投之河中。有頃,曰:「巫嫗何久也?弟子趣之!」復以弟子一人投河中。有頃,曰:「弟子何久也?復使一人趣之!」復投一弟子河中。凡投三弟子。西門豹曰:「巫嫗弟子是女子也,不能白事,煩三老為入白之。」復投三老河中。西門豹簪筆磬折,向河立待良久。長老、吏傍觀者皆驚恐。西門豹顧曰:「巫嫗、三老不來還,柰之何?」欲復使廷掾與豪長者一人入趣之。皆叩頭,叩頭且破,額血流地,色如死灰。西門豹曰:「諾,且留待之須臾。」須臾,豹曰:「廷掾起矣。狀河伯留客之久,若皆罷去歸矣。」鄴吏民大驚恐,從是以後,不敢復言為河伯娶婦。
\\\\
  西門豹即發民鑿十二渠,引河水灌民田,田皆溉。當其時,民治渠少煩苦,不欲也。豹曰:「民可以樂成,不可與慮始。今父老子弟雖患苦我,然百歲後期令父老子孫思我言。」至今皆得水利,民人以給足富。十二渠經絕馳道,到漢之立,而長吏以為十二渠橋絕馳道,相比近,不可。欲合渠水,且至馳道合三渠為一橋。鄴民人父老不肯聽長吏,以為西門君所為也,賢君之法式不可更也。長吏終聽置之。故西門豹為鄴令,名聞天下,澤流後世,無絕已時,幾可謂非賢大夫哉!
\\\\
  傳曰:「子產治鄭,民不能欺;子賤治單父,民不忍欺;西門豹治鄴,民不敢欺。」三子之才能誰最賢哉?辨治者當能別之。

\section{貨殖列傳}
  老子曰:「至治之極,鄰國相望,雞狗之聲相聞,民各甘其食,美其服,安其俗,樂其業,至老死,不相往來。」必用此為務,輓近世涂民耳目,則幾無行矣。
\\\\
  太史公曰:夫神農以前,吾不知已。至若詩書所述虞夏以來,耳目欲極聲色之好,口欲窮芻豢之味,身安逸樂,而心誇矜輓能之榮使。俗之漸民久矣,雖戶說以眇論,終不能化。故善者因之,其次利道之,其次教誨之,其次整齊之,最下者與之爭。
\\\\
  夫山西饒材、竹、穀、纑、旄、玉石;山東多魚、鹽、漆、絲、聲色;江南出枏、梓、薑、桂、金、錫、連、丹沙、犀、瑁、珠璣、齒革;龍門、碣石北多馬、牛、羊、旃裘、筋角;銅、鐵則千里往往山出棋置:此其大較也。皆中國人民所喜好謠俗被服飲食奉生送死之具也。故待農而食之,虞而出之,工而成之,商而通之。此寧有政教發徵期會哉?人各任其能,竭其力,以得所欲。故物賤之徵貴,貴之徵賤,各勸其業,樂其事,若水之趨下,日夜無休時,不召而自來,不求而民出之。豈非道之所符,而自然之驗邪?
\\\\
  《周書》曰:「農不出則乏其食,工不出則乏其事,商不出則三寶絕,虞不出則財匱少。」財匱少而山澤不辟矣。此四者,民所衣食之原也。原大則饒,原小則鮮。上則富國,下則富家。貧富之道,莫之奪予,而巧者有餘,拙者不足。故太公望封於營丘,地澙鹵,人民寡,於是太公勸其女功,極技巧,通魚鹽,則人物歸之,繦至而輻湊。故齊冠帶衣履天下,海岱之閒斂袂而往朝焉。其後齊中衰,管子修之,設輕重九府,則桓公以霸,九合諸侯,一匡天下;而管氏亦有三歸,位在陪臣,富於列國之君。是以齊富彊至於威、宣也。
\\\\
  故曰:「倉廩實而知禮節,衣食足而知榮辱。」禮生於有而廢於無。故君子富,好行其德;小人富,以適其力。淵深而魚生之,山深而獸往之,人富而仁義附焉。富者得埶益彰,失埶則客無所之,以而不樂。夷狄益甚。諺曰:「千金之子,不死於市。」此非空言也。故曰:「天下熙熙,皆為利來;天下壤壤,皆為利往。」夫千乘之王,萬家之侯,百室之君,尚猶患貧,而況匹夫編戶之民乎!
\\\\
  昔者越王句踐困於會稽之上,乃用范蠡、計然。計然曰:「知斗則修備,時用則知物,二者形則萬貨之情可得而觀已。故歲在金,穰;水,毀;木,饑;火,旱。旱則資舟,水則資車,物之理也。六歲穰,六歲旱,十二歲一大饑。夫糶,二十病農,九十病末。末病則財不出,農病則草不辟矣。上不過八十,下不減三十,則農末俱利,平糶齊物,關市不乏,治國之道也。積著之理,務完物,無息幣。以物相貿易,腐敗而食之貨勿留,無敢居貴。論其有餘不足,則知貴賤。貴上極則反賤,賤下極則反貴。貴出如糞土,賤取如珠玉。財幣欲其行如流水。」修之十年,國富,厚賂戰士,士赴矢石,如渴得飲,遂報彊吳,觀兵中國,稱號「五霸」。
\\\\
  范蠡既雪會稽之恥,乃喟然而嘆曰:「計然之策七,越用其五而得意。既已施於國,吾欲用之家。」乃乘扁舟浮於江湖,變名易姓,適齊為鴟夷子皮,之陶為朱公。朱公以為陶天下之中,諸侯四通,貨物所交易也。乃治產積居。與時逐而不責於人。故善治生者,能擇人而任時。十九年之中三致千金,再分散與貧交疏昆弟。此所謂富好行其德者也。後年衰老而聽子孫,子孫修業而息之,遂至巨萬。故言富者皆稱陶朱公。
\\\\
  子贛既學於仲尼,退而仕於衛,廢著鬻財於曹、魯之閒,七十子之徒,賜最為饒益。原憲不厭糟糠,匿於窮巷。子貢結駟連騎,束帛之幣以聘享諸侯,所至,國君無不分庭與之抗禮。夫使孔子名布揚於天下者,子貢先後之也。此所謂得埶而益彰者乎?
\\\\
  白圭,周人也。當魏文侯時,李克務盡地力,而白圭樂觀時變,故人棄我取,人取我與。夫歲孰取穀,予之絲漆;繭出取帛絮,予之食。太陰在卯,穰;明歲衰惡。至午,旱;明歲美。至酉,穰;明歲衰惡。至子,大旱;明歲美,有水。至卯,積著率歲倍。欲長錢,取下穀;長石斗,取上種。能薄飲食,忍嗜欲,節衣服,與用事僮仆同苦樂,趨時若猛獸摯鳥之發。故曰:「吾治生產,猶伊尹、呂尚之謀,孫吳用兵,商鞅行法是也。是故其智不足與權變,勇不足以決斷,仁不能以取予,彊不能有所守,雖欲學吾術,終不告之矣。」蓋天下言治生祖白圭。白圭其有所試矣,能試有所長,非茍而已也。
\\\\
  猗頓用盬鹽起。而邯鄲郭縱以鐵冶成業,與王者埒富。
\\\\
  烏氏倮牧,及眾,斥賣,求奇繒物,閒獻遺戎王。戎王什倍其償,與之畜,畜至用谷量馬牛。秦始皇帝令倮比封君,以時與列臣朝請。而巴[蜀]寡婦清,其先得丹穴,而擅其利數世,家亦不訾。清,寡婦也,能守其業,用財自衛,不見侵犯。秦皇帝以為貞婦而客之,為筑女懷清臺。夫倮鄙人牧長,清窮鄉寡婦,禮抗萬乘,名顯天下,豈非以富邪?
\\\\
  漢興,海內為一,開關梁,弛山澤之禁,是以富商大賈周流天下,交易之物莫不通,得其所欲,而徙豪傑諸侯彊族於京師。
\\\\
  關中自汧、雍以東至河、華,膏壤沃野千里,自虞夏之貢以為上田,而公劉適邠,大王、王季在岐,文王作豐,武王治鎬,故其民猶有先王之遺風,好稼穡,殖五穀,地重,重為邪。及秦文、(孝)[德]、繆居雍,隙隴蜀之貨物而多賈。獻(孝)公徙櫟邑,櫟邑北卻戎翟,東通三晉,亦多大賈。(武)[孝]、昭治咸陽,因以漢都,長安諸陵,四方輻湊并至而會,地小人眾,故其民益玩巧而事末也。南則巴蜀。巴蜀亦沃野,地饒炧、薑、丹沙、石、銅、鐵、竹、木之器。南御滇僰,僰僮。西近邛笮,笮馬、旄牛。然四塞,棧道千里,無所不通,唯褒斜綰轂其口,以所多易所鮮。天水、隴西、北地、上郡與關中同俗,然西有羌中之利,北有戎翟之畜,畜牧為天下饒。然地亦窮險,唯京師要其道。故關中之地,於天下三分之一,而人眾不過什三;然量其富,什居其六。
\\\\
  昔唐人都河東,殷人都河內,周人都河南。夫三河在天下之中,若鼎足,王者所更居也,建國各數百千歲,土地小狹,民人眾,都國諸侯所聚會,故其俗纖儉習事。楊、平陽陳西賈秦、翟,北賈種、代。種、代,石北也,地邊胡,數被寇。人民矜懻忮,好氣,任俠為姦,不事農商。然迫近北夷,師旅亟往,中國委輸時有奇羨。其民羯羠不均,自全晉之時固已患其僄悍,而武靈王益厲之,其謠俗猶有趙之風也。故楊、平陽陳掾其閒,得所欲。溫、軹西賈上黨,北賈趙、中山。中山地薄人眾,猶有沙丘紂淫地餘民,民俗懁急,仰機利而食。丈夫相聚游戲,悲歌慨,起則相隨椎剽,休則掘冢作巧姦冶,多美物,為倡優。女子則鼓鳴瑟,跕屣,游媚貴富,入後宮,遍諸侯。
\\\\
  然邯鄲亦漳、河之閒一都會也。北通燕、涿,南有鄭、衛。鄭、衛俗與趙相類,然近梁、魯,微重而矜節。濮上之邑徙野王,野王好氣任俠,衛之風也。
\\\\
  夫燕亦勃、碣之閒一都會也。南通齊、趙,東北邊胡。上谷至遼東,地踔遠,人民希,數被寇,大與趙、代俗相類,而民雕捍少慮,有魚鹽棗栗之饒。北鄰烏桓、夫餘,東綰穢貉、朝鮮、真番之利。
\\\\
  洛陽東賈齊、魯,南賈梁、楚。故泰山之陽則魯,其陰則齊。
\\\\
  齊帶山海,膏壤千里,宜桑麻,人民多文綵布帛魚鹽。臨菑亦海岱之閒一都會也。其俗寬緩闊達,而足智,好議論,地重,難動搖,怯於眾鬬,勇於持刺,故多劫人者,大國之風也。其中具五民。
\\\\
  而鄒、魯濱洙、泗,猶有周公遺風,俗好儒,備於禮,故其民齪齪。頗有桑麻之業,無林澤之饒。地小人眾,儉嗇,畏罪遠邪。及其衰,好賈趨利,甚於周人。
\\\\
  夫自鴻溝以東,芒、碭以北,屬巨野,此梁、宋也。陶、睢陽亦一都會也。昔堯作(游)[於]成陽,舜漁於雷澤,湯止于亳。其俗猶有先王遺風,重厚多君子,好稼穡,雖無山川之饒,能惡衣食,致其蓄藏。
\\\\
  越、楚則有三俗。夫自淮北沛、陳、汝南、南郡,此西楚也。其俗剽輕,易發怒,地薄,寡於積聚。江陵故郢都,西通巫、巴,東有雲夢之饒。陳在楚夏之交,通魚鹽之貨,其民多賈。徐、僮、取慮,則清刻,矜己諾。
\\\\
  彭城以東,東海、吳、廣陵,此東楚也。其俗類徐、僮。朐、繒以北,俗則齊。浙江南則越。夫吳自闔廬、春申、王濞三人招致天下之喜游子弟,東有海鹽之饒,章山之銅,三江、五湖之利,亦江東一都會也。
\\\\
  衡山、九江、江南、豫章、長沙,是南楚也,其俗大類西楚。郢之後徙壽春,亦一都會也。而合肥受南北潮,皮革、鮑、木輸會也。與閩中、干越雜俗,故南楚好辭,巧說少信。江南卑溼,丈夫早夭。多竹木。豫章出黃金,長沙出連、錫,然堇堇物之所有,取之不足以更費。九疑、蒼梧以南至儋耳者,與江南大同俗,而楊越多焉。番禺亦其一都會也,珠璣、犀、瑁、果、布之湊。
\\\\
  潁川、南陽,夏人之居也。夏人政尚忠樸,猶有先王之遺風。潁川敦願。秦末世,遷不軌之民於南陽。南陽西通武關、鄖關,東南受漢、江、淮。宛亦一都會也。俗雜好事,業多賈。其任俠,交通潁川,故至今謂之「夏人」。
\\\\
  夫天下物所鮮所多,人民謠俗,山東食海鹽,山西食鹽鹵,領南、沙北固往往出鹽,大體如此矣。
\\\\
  總之,楚越之地,地廣人希,飯稻羹魚,或火耕而水耨,果隋蠃蛤,不待賈而足,地埶饒食,無饑饉之患,以故呰窳偷生,無積聚而多貧。是故江淮以南,無凍餓之人,亦無千金之家。沂、泗水以北,宜五穀桑麻六畜,地小人眾,數被水旱之害,民好畜藏,故秦、夏、梁、魯好農而重民。三河、宛、陳亦然,加以商賈。齊、趙設智巧,仰機利。燕、代田畜而事蠶。
\\\\
  由此觀之,賢人深謀於廊廟,論議朝廷,守信死節隱居巖穴之士設為名高者安歸乎?歸於富厚也。是以廉吏久,久更富,廉賈歸富。富者,人之情性,所不學而俱欲者也。故壯士在軍,攻城先登,陷陣卻敵,斬將搴旗,前蒙矢石,不避湯火之難者,為重賞使也。其在閭巷少年,攻剽椎埋,劫人作姦,掘冢鑄幣,任俠并兼,借交報仇,篡逐幽隱,不避法禁,走死地如騖者,其實皆為財用耳。今夫趙女鄭姬,設形容,揳鳴琴,揄長袂,躡利屣,目挑心招,出不遠千里,不擇老少者,奔富厚也。游閒公子,飾冠劍,連車騎,亦為富貴容也。弋射漁獵,犯晨夜,冒霜雪,馳阬谷,不避猛獸之害,為得味也。博戲馳逐,鬬雞走狗,作色相矜,必爭勝者,重失負也。醫方諸食技術之人,焦神極能,為重糈也。吏士舞文弄法,刻章偽書,不避刀鋸之誅者,沒於賂遺也。農工商賈畜長,固求富益貨也。此有知盡能索耳,終不餘力而讓財矣。
\\\\
  諺曰:「百里不販樵,千里不販糴。」居之一歲,種之以穀;十歲,樹之以木;百歲,來之以德。德者,人物之謂也。今有無秩祿之奉,爵邑之入,而樂與之比者。命曰「素封」。封者食租稅,歲率戶二百。千戶之君則二十萬,朝覲聘享出其中。庶民農工商賈,率亦歲萬息二千,百萬之家則二十萬,而更傜租賦出其中。衣食之欲,恣所好美矣。故曰陸地牧馬二百蹄,牛蹄角千,千足羊,澤中千足彘,水居千石魚陂,山居千章之材。安邑千樹棗;燕、秦千樹栗;蜀、漢、江陵千樹橘;淮北、常山已南,河濟之閒千樹萩;陳、夏千畝漆;齊、魯千畝桑麻;渭川千畝竹;及名國萬家之城,帶郭千畝畝鐘之田,若千畝卮茜,千畦薑韭:此其人皆與千戶侯等。然是富給之資也,不窺市井,不行異邑,坐而待收,身有處士之義而取給焉。若至家貧親老,妻子軟弱,歲時無以祭祀進醵,飲食被服不足以自通,如此不慚恥,則無所比矣。是以無財作力,少有鬬智,既饒爭時,此其大經也。今治生不待危身取給,則賢人勉焉。是故本富為上,末富次之,姦富最下。無巖處奇士之行,而長貧賤,好語仁義,亦足羞也。
\\\\
  凡編戶之民,富相什則卑下之,伯則畏憚之,千則役,萬則仆,物之理也。夫用貧求富,農不如工,工不如商,刺繡文不如倚市門,此言末業,貧者之資也。通邑大都,酤一歲千釀,醯醬千瓨,漿千甔,屠牛羊彘千皮,販穀糶千鐘,薪槁千車,船長千丈,木千章,竹竿萬,其軺車百乘,牛車千兩,木器髤者千枚,銅器千鈞,素木鐵器若炧茜千石,馬蹄蹾千,牛千足,羊彘千雙,僮手指千,筋角丹沙千斤,其帛絮細布千鈞,文采千匹,榻布皮革千石,漆千斗,糱麹鹽豉千荅,鮐鮆千斤,鯫千石,鮑千鈞,棗栗千石者三之,狐鼦裘千皮,羔羊裘千石,旃席千具,佗果菜千鐘,子貸金錢千貫,節駔會,貪賈三之,廉賈五之,此亦比千乘之家,其大率也。佗雜業不中什二,則非吾財也。
\\\\
  請略道當世千里之中,賢人所以富者,令後世得以觀擇焉。
\\\\
  蜀卓氏之先,趙人也,用鐵冶富。秦破趙,遷卓氏。卓氏見虜略,獨夫妻推輦,行詣遷處。諸遷虜少有餘財,爭與吏,求近處,處葭萌。唯卓氏曰:「此地狹薄。吾聞汶山之下,沃野,下有蹲鴟,至死不饑。民工於市,易賈。」乃求遠遷。致之臨邛,大喜,即鐵山鼓鑄,運籌策,傾滇蜀之民,富至僮千人。田池射獵之樂,擬於人君。
\\\\
  程鄭,山東遷虜也,亦冶鑄,賈椎髻之民,富埒卓氏,俱居臨邛。
\\\\
  宛孔氏之先,梁人也,用鐵冶為業。秦伐魏,遷孔氏南陽。大鼓鑄,規陂池,連車騎,游諸侯,因通商賈之利,有游閒公子之賜與名。然其贏得過當,愈於纖嗇,家致富數千金,故南陽行賈盡法孔氏之雍容。
\\\\
  魯人俗儉嗇,而曹邴氏尤甚,以鐵冶起,富至巨萬。然家自父兄子孫約,俛有拾,仰有取,貰貸行賈遍郡國。鄒、魯以其故多去文學而趨利者,以曹邴氏也。
\\\\
  齊俗賤奴虜,而刀閒獨愛貴之。桀黠奴,人之所患也,唯刀閒收取,使之逐漁鹽商賈之利,或連車騎,交守相,然愈益任之。終得其力,起富數千萬。故曰「寧爵毋刀」,言其能使豪奴自饒而盡其力。
\\\\
  周人既纖,而師史尤甚,轉轂以百數,賈郡國,無所不至。洛陽街居在齊秦楚趙之中,貧人學事富家,相矜以久賈,數過邑不入門,設任此等,故師史能致七千萬。
\\\\
  宣曲任氏之先,為督道倉吏。秦之敗也,豪傑皆爭取金玉,而任氏獨窖倉粟。楚漢相距滎陽也,民不得耕種,米石至萬,而豪傑金玉盡歸任氏,任氏以此起富。富人爭奢侈,而任氏折節為儉,力田畜。田畜人爭取賤賈,任氏獨取貴善。富者數世。然任公家約,非田畜所出弗衣食,公事不畢則身不得飲酒食肉。以此為閭里率,故富而主上重之。
\\\\
  塞之斥也,唯橋姚已致馬千匹,牛倍之,羊萬頭,粟以萬鐘計。吳楚七國兵起時,長安中列侯封君行從軍旅,齎貸子錢,子錢家以為侯邑國在關東,關東成敗未決,莫肯與。唯無鹽氏出捐千金貸,其息什之。三月,吳楚平,一歲之中,則無鹽氏之息什倍,用此富埒關中。
\\\\
  關中富商大賈,大抵盡諸田,田嗇、田蘭。韋家栗氏,安陵、杜杜氏,亦巨萬。
\\\\
  此其章章尤異者也。皆非有爵邑奉祿弄法犯姦而富,盡椎埋去就,與時俯仰,獲其贏利,以末致財,用本守之,以武一切,用文持之,變化有概,故足術也。若至力農畜,工虞商賈,為權利以成富,大者傾郡,中者傾縣,下者傾鄉里者,不可勝數。
\\\\
  夫纖嗇筋力,治生之正道也,而富者必用奇勝。田農,掘業,而秦揚以蓋一州。掘冢,姦事也,而田叔以起。博戲,惡業也,而桓發用富。行賈,丈夫賤行也,而雍樂成以饒。販脂,辱處也,而雍伯千金。賣漿,小業也,而張氏千萬。灑削,薄技也,而郅氏鼎食。胃脯,簡微耳,濁氏連騎。馬醫,淺方,張裏擊鐘。此皆誠壹之所致。由是觀之,富無經業,則貨無常主,能者輻湊,不肖者瓦解。千金之家比一都之君,巨萬者乃與王者同樂。豈所謂「素封」者邪?非也?

《太史公自序》

提到《太史公自序》的書籍 電子圖書館
\\\\
  昔在顓頊,命南正重以司天,北正黎以司地。唐虞之際,紹重黎之後,使復典之,至于夏商,故重黎氏世序天地。其在周,程伯休甫其後也。當周宣王時,失其守而為司馬氏。司馬氏世典周史。惠襄之閒,司馬氏去周適晉。晉中軍隨會奔秦,而司馬氏入少梁。
\\\\
  自司馬氏去周適晉,分散,或在衛,或在趙,或在秦。其在衛者,相中山。在趙者,以傳劍論顯,蒯聵其後也。在秦者名錯,與張儀爭論,於是惠王使錯將伐蜀,遂拔,因而守之。錯孫靳,事武安君白起。而少梁更名曰夏陽。靳與武安君阬趙長平軍,還而與之俱賜死杜郵,葬於華池。靳孫昌,昌為秦主鐵官,當始皇之時。蒯聵玄孫卬為武信君將而徇朝歌。諸侯之相王,王卬於殷。漢之伐楚,卬歸漢,以其地為河內郡。昌生無澤,無澤為漢市長。無澤生喜,喜為五大夫,卒,皆葬高門。喜生談,談為太史公。
\\\\
  太史公學天官於唐都,受易於楊何,習道論於黃子。太史公仕於建元元封之閒,愍學者之不達其意而師悖,乃論六家之要指曰:
\\\\
  易大傳:「天下一致而百慮,同歸而殊涂。」夫陰陽、儒、墨、名、法、道德,此務為治者也,直所從言之異路,有省不省耳。嘗竊觀陰陽之術,大祥而眾忌諱,使人拘而多所畏;然其序四時之大順,不可失也。儒者博而寡要,勞而少功,是以其事難盡從;然其序君臣父子之禮,列夫婦長幼之別,不可易也。墨者儉而難遵,是以其事不可遍循;然其彊本節用,不可廢也。法家嚴而少恩;然其正君臣上下之分,不可改矣。名家使人儉而善失真;然其正名實,不可不察也。道家使人精神專一,動合無形,贍足萬物。其為術也,因陰陽之大順,采儒墨之善,撮名法之要,與時遷移,應物變化,立俗施事,無所不宜,指約而易操,事少而功多。儒者則不然。以為人主天下之儀表也,主倡而臣和,主先而臣隨。如此則主勞而臣逸。至於大道之要,去健羨,絀聰明,釋此而任術。夫神大用則竭,形大勞則敝。形神騷動,欲與天地長久,非所聞也。
\\\\
  夫陰陽四時、八位、十二度、二十四節各有教令,順之者昌,逆之者不死則亡,未必然也,故曰「使人拘而多畏」。夫春生夏長,秋收冬藏,此天道之大經也,弗順則無以為天下綱紀,故曰「四時之大順,不可失也」。
\\\\
  夫儒者以六藝為法。六藝經傳以千萬數,累世不能通其學,當年不能究其禮,故曰「博而寡要,勞而少功」。若夫列君臣父子之禮,序夫婦長幼之別,雖百家弗能易也。
\\\\
  墨者亦尚堯舜道,言其德行曰:「堂高三尺,土階三等,茅茨不翦,采椽不刮。食土簋,啜土刑,糲粱之食,藜霍之羹。夏日葛衣,冬日鹿裘。」其送死,桐棺三寸,舉音不盡其哀。教喪禮,必以此為萬民之率。使天下法若此,則尊卑無別也。夫世異時移,事業不必同,故曰「儉而難遵」。要曰彊本節用,則人給家足之道也。此墨子之所長,雖百長弗能廢也。
\\\\
  法家不別親疏,不殊貴賤,一斷於法,則親親尊尊之恩絕矣。可以行一時之計,而不可長用也,故曰「嚴而少恩」。若尊主卑臣,明分職不得相踰越,雖百家弗能改也。
\\\\
  名家苛察繳繞,使人不得反其意,專決於名而失人情,故曰「使人儉而善失真」。若夫控名責實,參伍不失,此不可不察也。
\\\\
  道家無為,又曰無不為,其實易行,其辭難知。其術以虛無為本,以因循為用。無成埶,無常形,故能究萬物之情。不為物先,不為物後,故能為萬物主。有法無法,因時為業;有度無度,因物與合。故曰「聖人不朽,時變是守。虛者道之常也,因者君之綱」也。群臣并至,使各自明也。其實中其聲者謂之端,實不中其聲者謂之窾。窾言不聽,姦乃不生,賢不肖自分,白黑乃形。在所欲用耳,何事不成。乃合大道,混混冥冥。光燿天下,復反無名。凡人所生者神也,所託者形也。神大用則竭,形大勞則敝,形神離則死。死者不可復生,離者不可復反,故聖人重之。由是觀之,神者生之本也,形者生之具也。不先定其神[形],而曰「我有以治天下」,何由哉?
\\\\
  太史公既掌天官,不治民。有子曰遷。
\\\\
  遷生龍門,耕牧河山之陽。年十歲則誦古文。二十而南游江、淮,上會稽,探禹穴,闚九疑,浮於沅、湘;北涉汶、泗,講業齊、魯之都,觀孔子之遺風,鄉射鄒、嶧;戹困鄱、薛、彭城,過梁、楚以歸。於是遷仕為郎中,奉使西征巴、蜀以南,南略邛、笮、昆明,還報命。
\\\\
  是歲天子始建漢家之封,而太史公留滯周南,不得與從事,故發憤且卒。而子遷適使反,見父於河洛之閒。太史公執遷手而泣曰:「余先周室之太史也。自上世嘗顯功名於虞夏,典天官事。後世中衰,絕於予乎?汝復為太史,則續吾祖矣。今天子接千歲之統,封泰山,而余不得從行,是命也夫,命也夫!余死,汝必為太史;為太史,無忘吾所欲論著矣。且夫孝始於事親,中於事君,終於立身。揚名於後世,以顯父母,此孝之大者。夫天下稱誦周公,言其能論歌文武之德,宣周邵之風,達太王王季之思慮,爰及公劉,以尊后稷也。幽厲之後,王道缺,禮樂衰,孔子修舊起廢,論詩書,作春秋,則學者至今則之。自獲麟以來四百有餘歲,而諸侯相兼,史記放絕。今漢興,海內一統,明主賢君忠臣死義之士,余為太史而弗論載,廢天下之史文,余甚懼焉,汝其念哉!」遷俯首流涕曰:「小子不敏,請悉論先人所次舊聞,弗敢闕。」
\\\\
  卒三歲而遷為太史令,紬史記石室金匱之書。五年而當太初元年,十一月甲子朔旦冬至,天歷始改,建於明堂,諸神受紀。
\\\\
  太史公曰:「先人有言:『自周公卒五百歲而有孔子。孔子卒後至於今五百歲,有能紹明世,正易傳,繼春秋,本詩書禮樂之際?』意在斯乎!意在斯乎!小子何敢讓焉。」
\\\\
  上大夫壺遂曰:「昔孔子何為而作春秋哉?」太史公曰:「余聞董生曰:『周道衰廢,孔子為魯司寇,諸侯害之,大夫壅之。孔子知言之不用,道之不行也,是非二百四十二年之中,以為天下儀表,貶天子,退諸侯,討大夫,以達王事而已矣。』子曰:『我欲載之空言,不如見之於行事之深切著明也。』夫春秋,上明三王之道,下辨人事之紀,別嫌疑,明是非,定猶豫,善善惡惡,賢賢賤不肖,存亡國,繼絕世,補敝起廢,王道之大者也。易著天地陰陽四時五行,故長於變;禮經紀人倫,故長於行;書記先王之事,故長於政;詩記山川谿谷禽獸草木牝牡雌雄,故長於風;樂樂所以立,故長於和;春秋辯是非,故長於治人。是故禮以節人,樂以發和,書以道事,詩以達意,易以道化,春秋以道義。撥亂世反之正,莫近於春秋。春秋文成數萬,其指數千。萬物之散聚皆在春秋。春秋之中,弒君三十六,亡國五十二,諸侯奔走不得保其社稷者不可勝數。察其所以,皆失其本已。故易曰『失之豪釐,差以千里』。故曰『臣弒君,子弒父,非一旦一夕之故也,其漸久矣』。故有國者不可以不知春秋,前有讒而弗見,後有賊而不知。為人臣者不可以不知春秋,守經事而不知其宜,遭變事而不知其權。為人君父而不通於春秋之義者,必蒙首惡之名。為人臣子而不通於春秋之義者,必陷篡弒之誅,死罪之名。其實皆以為善,為之不知其義,被之空言而不敢辭。夫不通禮義之旨,至於君不君,臣不臣,父不父,子不子。夫君不君則犯,臣不臣則誅,父不父則無道,子不子則不孝。此四行者,天下之大過也。以天下之大過予之,則受而弗敢辭。故春秋者,禮義之大宗也。夫禮禁未然之前,法施已然之後;法之所為用者易見,而禮之所為禁者難知。」
\\\\
  壺遂曰:「孔子之時,上無明君,下不得任用,故作春秋,垂空文以斷禮義,當一王之法。今夫子上遇明天子,下得守職,萬事既具,咸各序其宜,夫子所論,欲以何明?」
\\\\
  太史公曰:「唯唯,否否,不然。余聞之先人曰:『伏羲至純厚,作易八卦。堯舜之盛,尚書載之,禮樂作焉。湯武之隆,詩人歌之。春秋采善貶惡,推三代之德,褒周室,非獨刺譏而已也。』漢興以來,至明天子,獲符瑞,封禪,改正朔,易服色,受命於穆清,澤流罔極,海外殊俗,重譯款塞,請來獻見者,不可勝道。臣下百官力誦聖德,猶不能宣盡其意。且士賢能而不用,有國者之恥;主上明聖而德不布聞,有司之過也。且余嘗掌其官,廢明聖盛德不載,滅功臣世家賢大夫之業不述,墮先人所言,罪莫大焉。余所謂述故事,整齊其世傳,非所謂作也,而君比之於春秋,謬矣。」
\\\\
  於是論次其文。七年而太史公遭李陵之禍,幽於縲紲。乃喟然而嘆曰:「是余之罪也夫!是余之罪也夫!身毀不用矣。」退而深惟曰:「夫詩書隱約者,欲遂其志之思也。昔西伯拘羑里,演周易;孔子戹陳蔡,作春秋;屈原放逐,著離騷;左丘失明,厥有國語;孫子臏腳,而論兵法;不韋遷蜀,世傳呂覽;韓非囚秦,說難、孤憤;詩三百篇,大抵賢聖發憤之所為作也。此人皆意有所郁結,不得通其道也,故述往事,思來者。」於是卒述陶唐以來,至于麟止,自黃帝始。
\\\\
  維昔黃帝,法天則地,四聖遵序,各成法度;唐堯遜位,虞舜不台;厥美帝功,萬世載之。作五帝本紀第一。
\\\\
  維禹之功,九州攸同,光唐虞際,德流苗裔;夏桀淫驕,乃放鳴條。作夏本紀第二。
\\\\
  維契作商,爰及成湯;太甲居桐,德盛阿衡;武丁得說,乃稱高宗;帝辛湛湎,諸侯不享。作殷本紀第三。
\\\\
  維棄作稷,德盛西伯;武王牧野,實撫天下;幽厲昏亂,既喪酆鎬;陵遲至赧;洛邑不祀。作周本紀第四。
\\\\
  維秦之先,伯翳佐禹;穆公思義,悼豪之旅;以人為殉,詩歌黃鳥;昭襄業帝。作秦本紀第五。
\\\\
  始皇既立,并兼六國,銷鋒鑄鐻,維偃干革,尊號稱帝,矜武任力;二世受運,子嬰降虜。作始皇本紀第六。
\\\\
  秦失其道,豪桀并擾;項梁業之,子羽接之;殺慶救趙,諸侯立之;誅嬰背懷,天下非之。作項羽本紀第七。
\\\\
  子羽暴虐,漢行功德;憤發蜀漢,還定三秦;誅籍業帝,天下惟寧,改制易俗。作高祖本紀第八。
\\\\
  惠之早霣,諸呂不台;崇彊祿、產,諸侯謀之;殺隱幽友,大臣洞疑,遂及宗禍。作呂太后本紀第九。
\\\\
  漢既初興,繼嗣不明,迎王踐祚,天下歸心;蠲除肉刑,開通關梁,廣恩博施,厥稱太宗。作孝文本紀第十。
\\\\
  諸侯驕恣,吳首為亂,京師行誅,七國伏辜,天下翕然,大安殷富。作孝景本紀第十一。
\\\\
  漢興五世,隆在建元,外攘夷狄,內修法度,封禪,改正朔,易服色。作今上本紀第十二。
\\\\
  維三代尚矣,年紀不可考,蓋取之譜牒舊聞,本于茲,於是略推,作三代世表第一。
\\\\
  幽厲之後,周室衰微,諸侯專政,春秋有所不紀;而譜牒經略,五霸更盛衰,欲睹周世相先後之意,作十二諸侯年表第二。
\\\\
  春秋之後,陪臣秉政,彊國相王;以至于秦,卒并諸夏,滅封地,擅其號。作六國年表第三。
\\\\
  秦既暴虐,楚人發難,項氏遂亂,漢乃扶義征伐;八年之閒,天下三嬗,事繁變眾,故詳著秦楚之際月表第四。
\\\\
  漢興已來,至于太初百年,諸侯廢立分削,譜紀不明,有司靡踵,彊弱之原云以世。作漢興已來諸侯年表第五。
\\\\
  維高祖元功,輔臣股肱,剖符而爵,澤流苗裔,忘其昭穆,或殺身隕國。作高祖功臣侯者年表第六。
\\\\
  惠景之閒,維申功臣宗屬爵邑,作惠景閒侯者年表第七。
\\\\
  北討彊胡,南誅勁越,征伐夷蠻,武功爰列。作建元以來侯者年表第八。
\\\\
  諸侯既彊,七國為從,子弟眾多,無爵封邑,推恩行義,其埶銷弱,德歸京師。作王子侯者年表第九。
\\\\
  國有賢相良將,民之師表也。維見漢興以來將相名臣年表,賢者記其治,不賢者彰其事。作漢興以來將相名臣年表第十。
\\\\
  維三代之禮,所損益各殊務,然要以近性情,通王道,故禮因人質為之節文,略協古今之變。作禮書第一。
\\\\
  樂者,所以移風易俗也。自雅頌聲興,則已好鄭衛之音,鄭衛之音所從來久矣。人情之所感,遠俗則懷。比樂書以述來古,作樂書第二。
\\\\
  非兵不彊,非德不昌,黃帝、湯、武以興,桀、紂、二世以崩,可不慎歟?司馬法所從來尚矣,太公、孫、吳、王子能紹而明之,切近世,極人變。作律書第三。
\\\\
  律居陰而治陽,歷居陽而治陰,律歷更相治,閒不容翲忽。五家之文怫異,維太初之元論。作歷書第四。
\\\\
  星氣之書,多雜禨祥,不經;推其文,考其應,不殊。比集論其行事,驗于軌度以次,作天官書第五。
\\\\
  受命而王,封禪之符罕用,用則萬靈罔不禋祀。追本諸神名山大川禮,作封禪書第六。
\\\\
  維禹浚川,九州攸寧;爰及宣防,決瀆通溝。作河渠書第七。
\\\\
  維幣之行,以通農商;其極則玩巧,并兼茲殖,爭於機利,去本趨末。作平準書以觀事變,第八。
\\\\
  太伯避歷,江蠻是適;文武攸興,古公王跡。闔廬弒僚,賓服荊楚;夫差克齊,子胥鴟夷;信嚭親越,吳國既滅。嘉伯之讓,作吳世家第一。
\\\\
  申、呂肖矣,尚父側微,卒歸西伯,文武是師;功冠群公,繆權于幽;番番黃髪,爰饗營丘。不背柯盟,桓公以昌,九合諸侯,霸功顯彰。田闞爭寵,姜姓解亡。嘉父之謀,作齊太公世家第二。
\\\\
  依之違之,周公綏之;憤發文德,天下和之;輔翼成王,諸侯宗周。隱桓之際,是獨何哉?三桓爭彊,魯乃不昌。嘉旦金縢,作周公世家第三。
\\\\
  武王克紂,天下未協而崩。成王既幼,管蔡疑之,淮夷叛之,於是召公率德,安集王室,以寧東土。燕(易)[噲]之禪,乃成禍亂。嘉甘棠之詩,作燕世家第四。
\\\\
  管蔡相武庚,將寧舊商;及旦攝政,二叔不饗;殺鮮放度,周公為盟;大任十子,周以宗彊。嘉仲悔過,作管蔡世家第五。
\\\\
  王后不絕,舜禹是說;維德休明,苗裔蒙烈。百世享祀,爰周陳杞,楚實滅之。齊田既起,舜何人哉?作陳杞世家第六。
\\\\
  收殷餘民,叔封始邑,申以商亂,酒材是告,及朔之生,衛頃不寧;南子惡蒯聵,子父易名。周德卑微,戰國既彊,衛以小弱,角獨後亡。喜彼康誥,作衛世家第七。
\\\\
  嗟箕子乎!嗟箕子乎!正言不用,乃反為奴。武庚既死,周封微子。襄公傷於泓,君子孰稱。景公謙德,熒惑退行。剔成暴虐,宋乃滅亡。喜微子問太師,作宋世家第八。
\\\\
  武王既崩,叔虞邑唐。君子譏名,卒滅武公。驪姬之愛,亂者五世;重耳不得意,乃能成霸。六卿專權,晉國以秏。嘉文公錫珪鬯,作晉世家第九。
\\\\
  重黎業之,吳回接之;殷之季世,粥子牒之。周用熊繹,熊渠是續。莊王之賢,乃復國陳;既赦鄭伯,班師華元。懷王客死,蘭咎屈原;好諛信讒,楚并於秦。嘉莊王之義,作楚世家第十。
\\\\
  少康之子,實賓南海,文身斷發,黿鱓與處,既守封禺,奉禹之祀。句踐困彼,乃用種、蠡。嘉句踐夷蠻能修其德,滅彊吳以尊周室,作越王句踐世家第十一。
\\\\
  桓公之東,太史是庸。及侵周禾,王人是議。祭仲要盟,鄭久不昌。子產之仁,紹世稱賢。三晉侵伐,鄭納於韓。嘉厲公納惠王,作鄭世家第十二。
\\\\
  維驥騄耳,乃章造父。趙夙事獻,衰續厥緒。佐文尊王,卒為晉輔。襄子困辱,乃禽智伯。主父生縛,餓死探爵。王遷辟淫,良將是斥。嘉鞅討周亂,作趙世家第十三。
\\\\
  畢萬爵魏,卜人知之。及絳戮干,戎翟和之。文侯慕義,子夏師之。惠王自矜,齊秦攻之。既疑信陵,諸侯罷之。卒亡大梁,王假廝之。嘉武佐晉文申霸道,作魏世家第十四。
\\\\
  韓厥陰德,趙武攸興。紹絕立廢,晉人宗之。昭侯顯列,申子庸之。疑非不信,秦人襲之。嘉厥輔晉匡周天子之賦,作韓世家第十五。
\\\\
  完子避難,適齊為援,陰施五世,齊人歌之。成子得政,田和為侯。王建動心,乃遷于共。嘉威、宣能撥濁世而獨宗周,作田敬仲完世家第十六。
\\\\
  周室既衰,諸侯恣行。仲尼悼禮廢樂崩,追修經術,以達王道,匡亂世反之於正,見其文辭,為天下制儀法,垂六藝之統紀於後世。作孔子世家第十七。
\\\\
  桀、紂失其道而湯、武作,周失其道而春秋作。秦失其政,而陳涉發跡,諸侯作難,風起雲蒸,卒亡秦族。天下之端,自涉發難。作陳涉世家第十八。
\\\\
  成皋之臺,薄氏始基。詘意適代,厥崇諸竇。栗姬偩貴,王氏乃遂。陳後太驕,卒尊子夫。嘉夫德若斯,作外戚世家十九。
\\\\
  漢既譎謀,禽信於陳;越荊剽輕,乃封弟交為楚王,爰都彭城,以彊淮泗,為漢宗藩。戊溺於邪,禮復紹之。嘉游輔祖,作楚元王世家二十。
\\\\
  維祖師旅,劉賈是與;為布所襲,喪其荊、吳。營陵激呂,乃王瑯邪;怵午信齊,往而不歸,遂西入關,遭立孝文,獲復王燕。天下未集,賈、澤以族,為漢藩輔。作荊燕世家第二十一。
\\\\
  天下已平,親屬既寡;悼惠先壯,實鎮東土。哀王擅興,發怒諸呂,駟鈞暴戾,京師弗許。厲之內淫,禍成主父。嘉肥股肱,作齊悼惠王世家第二十二。
\\\\
  楚人圍我滎陽,相守三年;蕭何填撫山西,推計踵兵,給糧食不絕,使百姓愛漢,不樂為楚。作蕭相國世家第二十三。
\\\\
  與信定魏,破趙拔齊,遂弱楚人。續何相國,不變不革,黎庶攸寧。嘉參不伐功矜能,作曹相國世家第二十四。
\\\\
  運籌帷幄之中,制勝於無形,子房計謀其事,無知名,無勇功,圖難於易,為大於細。作留侯世家第二十五。
\\\\
  六奇既用,諸侯賓從於漢;呂氏之事,平為本謀,終安宗廟,定社稷。作陳丞相世家第二十六。
\\\\
  諸呂為從,謀弱京師,而勃反經合於權;吳楚之兵,亞夫駐於昌邑,以戹齊趙,而出委以梁。作絳侯世家第二十七。
\\\\
  七國叛逆,蕃屏京師,唯梁為捍;偩愛矜功,幾獲于禍。嘉其能距吳楚,作梁孝王世家第二十八。
\\\\
  五宗既王,親屬洽和,諸侯大小為藩,爰得其宜,僭擬之事稍衰貶矣。作五宗世家第二十九。
\\\\
  三子之王,文辭可觀。作三王世家第三十。
\\\\
  末世爭利,維彼奔義;讓國餓死,天下稱之。作伯夷列傳第一。
\\\\
  晏子儉矣,夷吾則奢;齊桓以霸,景公以治。作管晏列傳第二。
\\\\
  李耳無為自化,清凈自正;韓非揣事情,循埶理。作老子韓非列傳第三。
\\\\
  自古王者而有司馬法,穰苴能申明之。作司馬穰苴列傳第四。
\\\\
  非信廉仁勇不能傳兵論劍,與道同符,內可以治身,外可以應變,君子比德焉。作孫子吳起列傳第五。
\\\\
  維建遇讒,爰及子奢,尚既匡父,伍員奔吳。作伍子胥列傳第六。
\\\\
  孔氏述文,弟子興業,咸為師傅,崇仁厲義。作仲尼弟子列傳第七。
\\\\
  鞅去衛適秦,能明其術,彊霸孝公,後世遵其法。作商君列傳第八。
\\\\
  天下患衡秦毋饜,而蘇子能存諸侯,約從以抑貪彊。作蘇秦列傳第九。
\\\\
  六國既從親,而張儀能明其說,復散解諸侯。作張儀列傳第十。
\\\\
  秦所以東攘雄諸侯,樗裏、甘茂之策。作樗裏甘茂列傳第十一。
\\\\
  苞河山,圍大梁,使諸侯斂手而事秦者,魏冉之功。作穰侯列傳第十二。
\\\\
  南拔鄢郢,北摧長平,遂圍邯鄲,武安為率;破荊滅趙,王翦之計。作白起王翦列傳第十三。
\\\\
  獵儒墨之遺文,明禮義之統紀,絕惠王利端,列往世興衰。作孟子荀卿列傳第十四。
\\\\
  好客喜士,士歸于薛,為齊捍楚魏。作孟嘗君列傳第十五。
\\\\
  爭馮亭以權,如楚以救邯鄲之圍,使其君復稱於諸侯。作平原君虞卿列傳第十六。
\\\\
  能以富貴下貧賤,賢能詘於不肖,唯信陵君為能行之。作魏公子列傳第十七。
\\\\
  以身徇君,遂脫彊秦,使馳說之士南鄉走楚者,黃歇之義。作春申君列傳第十八
\\\\
  能忍於魏齊,而信威於彊秦,推賢讓位,二子有之。作范睢蔡澤列傳第十九。
\\\\
  率行其謀,連五國兵,為弱燕報彊齊之讎,雪其先君之恥。作樂毅列傳第二十。
\\\\
  能信意彊秦,而屈體廉子,用徇其君,俱重於諸侯。作廉頗藺相如列傳第二十一。
\\\\
  湣王既失臨淄而奔莒,唯田單用即墨破走騎劫,遂存齊社稷。作田單列傳第二十二。
\\\\
  能設詭說解患於圍城,輕爵祿,樂肆志。作魯仲連鄒陽列傳第二十三。
\\\\
  作辭以諷諫,連類以爭義,離騷有之。作屈原賈生列傳第二十四。
\\\\
  結子楚親,使諸侯之士斐然爭入事秦。作呂不韋列傳第二十五。
\\\\
  曹子匕首,魯獲其田,齊明其信;豫讓義不為二心。作刺客列傳第二十六。
\\\\
  能明其畫,因時推秦,遂得意於海內,斯為謀首。作李斯列傳第二十七。
\\\\
  為秦開地益眾,北靡匈奴,據河為塞,因山為固,建榆中。作蒙恬列傳第二十八。
\\\\
  填趙塞常山以廣河內,弱楚權,明漢王之信於天下。作張耳陳餘列傳第二十九。
\\\\
  收西河、上黨之兵,從至彭城;越之侵掠梁地以苦項羽。作魏豹彭越列傳第三十。
\\\\
  以淮南叛楚歸漢,漢用得大司馬殷,卒破子羽于垓下。作黥布列傳第三十一。
\\\\
  楚人迫我京索,而信拔魏趙,定燕齊,使漢三分天下有其二,以滅項籍。作淮陰侯列傳第三十二。
\\\\
  楚漢相距鞏洛,而韓信為填潁川,盧綰絕籍糧餉。作韓信盧綰列傳第三十三。
\\\\
  諸侯畔項王,唯齊連子羽城陽,漢得以閒遂入彭城。作田儋列傳第三十四。
\\\\
  攻城野戰,獲功歸報,噲、商有力焉,非獨鞭策,又與之脫難。作樊酈列傳第三十五。
\\\\
  漢既初定,文理未明,蒼為主計,整齊度量,序律歷。作張丞相列傳第三十六。
\\\\
  結言通使,約懷諸侯;諸侯咸親,歸漢為藩輔。作酈生陸賈列傳第三十七。
\\\\
  欲詳知秦楚之事,維周∴從高祖,平定諸侯。作傅靳蒯成列傳第三十八。
\\\\
  徙彊族,都關中,和約匈奴;明朝廷禮,次宗廟儀法。作劉敬叔孫通列傳第三十九。
\\\\
  能摧剛作柔,卒為列臣;欒公不劫於埶而倍死。作季布欒布列傳第四十。
\\\\
  敢犯顏色以達主義,不顧其身,為國家樹長畫。作袁盎晁錯列傳第四十一。
\\\\
  守法不失大理,言古賢人,增主之明。作張釋之馮唐列傳第四十二。
\\\\
  敦厚慈孝,訥於言,敏於行,務在鞠躬,君子長者。作萬石張叔列傳第四十三。
\\\\
  守節切直,義足以言廉,行足以厲賢,任重權不可以非理撓。作田叔列傳第四十四。
\\\\
  扁鵲言醫,為方者宗,守數精明;後世(修)[循]序,弗能易也,而倉公可謂近之矣。作扁鵲倉公列傳第四十五。
\\\\
  維仲之省,厥濞王吳,遭漢初定,以填撫江淮之閒。作吳王濞列傳第四十六。
\\\\
  吳楚為亂,宗屬唯嬰賢而喜士,士鄉之,率師抗山東滎陽。作魏其武安列傳第四十七。
\\\\
  智足以應近世之變,寬足用得人。作韓長孺列傳第四十八。
\\\\
  勇於當敵,仁愛士卒,號令不煩,師徒鄉之。作李將軍列傳第四十九。
\\\\
  自三代以來,匈奴常為中國患害;欲知彊弱之時,設備征討,作匈奴列傳第五十。
\\\\
  直曲塞,廣河南,破祁連,通西國,靡北胡。作衛將軍驃騎列傳第五十一。
\\\\
  大臣宗室以侈靡相高,唯弘用節衣食為百吏先。作平津侯列傳第五十二。
\\\\
  漢既平中國,而佗能集楊越以保南藩,納貢職。作南越列傳第五十三。
\\\\
  吳之叛逆,甌人斬濞,葆守封禺為臣。作東越列傳第五十四。
\\\\
  燕丹散亂遼閒,滿收其亡民,厥聚海東,以集真藩,葆塞為外臣。作朝鮮列傳第五十五。
\\\\
  唐蒙使略通夜郎,而邛笮之君請為內臣受吏。作西南夷列傳第五十六。
\\\\
  子虛之事,大人賦說,靡麗多誇,然其指風諫,歸於無為。作司馬相如列傳第五十七。
\\\\
  黥布叛逆,子長國之,以填江淮之南,安剽楚庶民。作淮南衡山列傳第五十八。
\\\\
  奉法循理之吏,不伐功矜能,百姓無稱,亦無過行。作循吏列傳第五十九。
\\\\
  正衣冠立於朝廷,而群臣莫敢言浮說,長孺矜焉;好薦人,稱長者,壯有溉。作汲鄭列傳第六十。
\\\\
  自孔子卒,京師莫崇庠序,唯建元元狩之閒,文辭粲如也。作儒林列傳第六十一。
\\\\
  民倍本多巧,姦軌弄法,善人不能化,唯一切嚴削為能齊之。作酷吏列傳第六十二。
\\\\
  漢既通使大夏,而西極遠蠻,引領內鄉,欲觀中國。作大宛列傳第六十三。
\\\\
  救人於緦振人不贍,仁者有乎;不既信,不倍言,義者有取焉。作游俠列傳第六十四。
\\\\
  夫事人君能說主耳目,和主顏色,而獲親近,非獨色愛,能亦各有所長。作佞幸列傳第六十五。
\\\\
  不流世俗,不爭埶利,上下無所凝滯,人莫之害,以道之用。作滑稽列傳第六十六。
\\\\
  齊、楚、秦、趙為日者,各有俗所用。欲循觀其大旨,作日者列傳第六十七。
\\\\
  三王不同龜,四夷各異卜,然各以決吉凶。略闚其要,作龜策列傳第六十八。
\\\\
  布衣匹夫之人,不害於政,不妨百姓,取與以時而息財富,智者有采焉。作貨殖列傳第六十九。
\\\\
  維我漢繼五帝末流,接三代(統)[絕]業。周道廢,秦撥去古文,焚滅詩書,故明堂石室金匱玉版圖籍散亂。於是漢興,蕭何次律令,韓信申軍法,張蒼為章程,叔孫通定禮儀,則文學彬彬稍進,詩書往往閒出矣。自曹參薦蓋公言黃老,而賈生、晁錯明申、商,公孫弘以儒顯,百年之閒,天下遺文古事靡不畢集太史公。太史公仍父子相續纂其職。曰:「於戲!余維先人嘗掌斯事,顯於唐虞,至于周,復典之,故司馬氏世主天官。至於余乎,欽念哉!欽念哉!」罔羅天下放失舊聞,王跡所興,原始察終,見盛觀衰,論考之行事,略推三代,錄秦漢,上記軒轅,下至于茲,著十二本紀,既科條之矣。并時異世,年差不明,作十表。禮樂損益,律歷改易,兵權山川鬼神,天人之際,承敝通變,作八書。二十八宿環北辰,三十輻共一轂,運行無窮,輔拂股肱之臣配焉,忠信行道,以奉主上,作三十世家。扶義俶儻,不令己失時,立功名於天下,作七十列傳。凡百三十篇,五十二萬六千五百字,為太史公書。序略,以拾遺補闕,成一家之言,厥協六經異傳,整齊百家雜語,藏之名山,副在京師,俟後世聖人君子。第七十。
\\\\
  太史公曰:余述歷黃帝以來至太初而訖,百三十篇。